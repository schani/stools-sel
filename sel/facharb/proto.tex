\documentstyle[12pt,german]{article}

\frenchspacing
\sloppy
\addtolength{\topmargin}{-2cm}
\textheight237mm
\textwidth148mm
\evensidemargin9.6mm
\oddsidemargin9.6mm

\begin{document}
\thispagestyle{empty}
\addtolength{\topmargin}{15mm}
\section*{Sch�lerprotokoll}
% \bigskip
\begin{description}
\item[Pr�fungskandidat:]
     Mark Probst
\item[Thema:]
     Design einer Programmiersprache und theoretische Besprechung
     eines Interpreters anhand der Programmiersprache SEL
\item[Schuljahr:]
     1993/94
\item[Betreuer:] 
     Herr Professor Freiler
\end{description}
\bigskip
\setlength{\parskip}{5pt plus 2pt minus 1pt}
\renewcommand{\baselinestretch}{1.6}
\begin{tabular}{|c|p{11.7cm}|}
\hline
Termin & Text \\
\hline
Juni '93         & grobe Festlegung des Themas \\
3. August '93    & Beginn der Implementation des SEL-Interpreters f�r
                   MS-DOS und Windows NT \\
September '93    & Fixierung des Themas \\
16. Oktober '93  & Beginn der Arbeit an der SEL-Kurzreferenz \\
7. November '93  & Die "`T�rme von Hanoi"' laufen unter SEL f�r MS-DOS \\
10. Dezember '93 & Besprechung: Vorlage einer Vorversion der
                   Kurzreferenz und Kl�rung diverser Fragen zur
                   Fachbereichsarbeit \\
17. Dezember '93 & Beginn der Schreibarbeit an der
                   Fachbereichsarbeit \\
14. J�nner '94   & Besprechung: Vorlage einer Vorversion der
                   Fachbereichsarbeit \\
30. J�nner '94   & Beginn der Korrekturarbeiten \\
16. Feber '94    & Beendung der Arbeit \\
\hline
\end{tabular}

\bigskip

\section*{Erkl�rung:}

Ich erkl�re, da� die vorliegende Fachbereichsarbeit von mir selbst verfa�t 
ist und da� ich dazu keine anderen als die angef�hrten Behelfe verwendet habe.

Au�erdem habe ich die Reinschrift der Fachbereichsarbeit einer Korrektur 
unterzogen.

\end{document}
