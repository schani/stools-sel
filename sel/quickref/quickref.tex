% -*- LaTeX -*-

\documentstyle[german,makeidx]{report}

% \documentclass{report}
% \usepackage{german}
% \usepackage{makeidx}

\addtolength{\textwidth}{4cm}
\addtolength{\oddsidemargin}{-1.7cm}
\addtolength{\evensidemargin}{-1.7cm}
\addtolength{\topmargin}{-80pt}
\addtolength{\textheight}{150pt}
\addtolength{\marginparsep}{-5pt}

% \addtolength{\textwidth}{2cm}
% \addtolength{\topmargin}{-30pt}
% \addtolength{\textheight}{60pt}
% \addtolength{\marginparsep}{-5pt}

\setcounter{secnumdepth}{3}
\setcounter{tocdepth}{3}

\sloppy
\frenchspacing
% \pagestyle{headings}
\makeindex

% -*- LaTeX -*-

\documentstyle[german,makeidx]{report}

% \documentclass{report}
% \usepackage{german}
% \usepackage{makeidx}

\addtolength{\textwidth}{4cm}
\addtolength{\oddsidemargin}{-1.7cm}
\addtolength{\evensidemargin}{-1.7cm}
\addtolength{\topmargin}{-80pt}
\addtolength{\textheight}{150pt}
\addtolength{\marginparsep}{-5pt}

% \addtolength{\textwidth}{2cm}
% \addtolength{\topmargin}{-30pt}
% \addtolength{\textheight}{60pt}
% \addtolength{\marginparsep}{-5pt}

\setcounter{secnumdepth}{3}
\setcounter{tocdepth}{3}

\sloppy
\frenchspacing
% \pagestyle{headings}
\makeindex

% -*- LaTeX -*-

\documentstyle[german,makeidx]{report}

% \documentclass{report}
% \usepackage{german}
% \usepackage{makeidx}

\addtolength{\textwidth}{4cm}
\addtolength{\oddsidemargin}{-1.7cm}
\addtolength{\evensidemargin}{-1.7cm}
\addtolength{\topmargin}{-80pt}
\addtolength{\textheight}{150pt}
\addtolength{\marginparsep}{-5pt}

% \addtolength{\textwidth}{2cm}
% \addtolength{\topmargin}{-30pt}
% \addtolength{\textheight}{60pt}
% \addtolength{\marginparsep}{-5pt}

\setcounter{secnumdepth}{3}
\setcounter{tocdepth}{3}

\sloppy
\frenchspacing
% \pagestyle{headings}
\makeindex

% -*- LaTeX -*-

\documentstyle[german,makeidx]{report}

% \documentclass{report}
% \usepackage{german}
% \usepackage{makeidx}

\addtolength{\textwidth}{4cm}
\addtolength{\oddsidemargin}{-1.7cm}
\addtolength{\evensidemargin}{-1.7cm}
\addtolength{\topmargin}{-80pt}
\addtolength{\textheight}{150pt}
\addtolength{\marginparsep}{-5pt}

% \addtolength{\textwidth}{2cm}
% \addtolength{\topmargin}{-30pt}
% \addtolength{\textheight}{60pt}
% \addtolength{\marginparsep}{-5pt}

\setcounter{secnumdepth}{3}
\setcounter{tocdepth}{3}

\sloppy
\frenchspacing
% \pagestyle{headings}
\makeindex

\input{quickref.inc}

\begin{document}

\begin{titlepage}
\vspace*{2cm}
\begin{flushleft}
\inch SEL\rm \hfill Version 0.3
\end{flushleft}
\rule{\textwidth}{2mm}
\begin{flushright}
\huge{\bf Kurzreferenz}\\[0.5ex]
\end{flushright}
\vspace{13.5cm}
\begin{flushleft}
\large \bf Mark Probst\\
\rule[2mm]{\textwidth}{1mm}
\end{flushleft}
\end{titlepage}

\tableofcontents

\newpage

\chapter{\protect{"Uber} diesen Text}

Dieser Text hat zum Ziel, die Programmiersprache SEL (Schani
Electronics Language) in ihrer momentanen Version 0.3
vollst"andig zu beschreiben. Dabei ist er als Kurzreferenz
aufgebaut, kann aber auch zur Not verwendet werden, um die
Sprache SEL kennenzulernen, vor allem von Programmierern, die
bereits {\sc Pascal}
\index{Pascal@{\sc Pascal}}
oder \ind{C}
kennen, da SEL, verglichen mit diesen Sprachen, kaum etwas Neues
bietet.

Die einzigen Implementationen von SEL, die zur Zeit verf"ugbar
sind, sind die beiden \ind{Interpreter} unter MS-DOS und Windows
NT. Die Funktionsreferenz (Kapitel \ref{funcref}) spiegelt deren
momentanen Stand in Bezug auf verf"ugbare Funktionen wider.

\section{Typographie}
\index{Typographie|ii}

Um Mi"sverst"andnisse zu vermeiden, werden in diesem Text
verschiedene Schriftarten f"ur verschiedene Zwecke verwendet.
Durch eine {\em kursive Schriftart} werden neue Begriffe
gekennzeichnet und wichtige Textabschnitte hervorgehoben.

Schl"usselw"orter der Sprache SEL werden in {\bf fetter Schrift}
geschrieben. Eine Aufstellung aller Schl"usselw"orter finden Sie
in Kapitel \ref{identifiers}. Operatoren hingegen werden in
{\tt Schreibmaschinenschrift} geschrieben.

Typen werden nicht nur in fetter Schrift geschrieben, sondern
zus"atzlich in geschwungene \type{Klammern} gesetzt, um sie von
Schl"usselw"ortern unterscheiden zu k"onnen.

Bezeichner, Makros, Funktionsnamen und Nonterminals der
SEL-Grammatik werden ebenfalls in
{\tt Schreib"-maschi"-nen"-schrift} geschrieben.

Indexeintr"age
\index{Index|ii}
\index{Stichwortverzeichnis|see{Index}}
sind wie folgt zu interpretieren: Kursive
Eintr"age wie etwa {\em 23} deuten auf einen Haupteintrag, wo die
meisten Informationen zu dem jeweiligen Thema zu finden sind.
Unterstrichene Eintr"age wie z.~B. \underline{31} bedeuten, da"s
auf dieser Seite ein Beispiel zum Thema zu finden ist, das jedoch
nicht unbedingt n"aher erl"autert wird.

\section{Die Backus-Naur-Form}
\index{Backus-Naur-Form|ii}
\index{BNF|see{Backus-Naur-Form}}

Die Backus-Naur-Form (kurz BNF) wird verwendet, um Grammatiken
darzustellen. Dabei besteht jede \indi{Grammatik} aus mehreren
Symbolen, die ihrerseits jeweils aus mindestens einer
\indi{Produktion} bestehen. Symbole werden links angeschrieben.
Symbole rechts davon, durch einen Doppelpunkt oder ein anderes
Zeichen vom Grammatiksymbol getrennt, jeweils durch einen
vertikalen Strich (\fu{|}) voneinander getrennt. Auf die letzte
Produktion eines jeden Symbols folgt ein Semikolon (\fu{;}).

Produktionen sind Aneinanderreihungen von Terminal- und
Nichtterminalsymbolen. Nichtterminalsymbole sind jene, die auf
den linken Seiten stehen. Terminalsymbole sind solche, die in
Quelltexten der Sprache direkt vorkommen k"onnen, und solche,
die durch regul"are Ausdr"ucke definiert werden (siehe n"achstes
Unterkapitel). Erstere werden unter Hochkomma angef"uhrt (wie
etwa \fu{'new'}), w"ahrend letztere durch Gro"sschreibung
gekennzeichnet sind.

Ein Satz einer Sprache mu"s nun so konstruiert sein, da"s er
durch Ersetzen von Nichtterminalsymbolen durch bestimmte Produktionen
derselben hergeleitet werden kann.

\newpage
\example
\begin{verbatim}
satz:         subjekt praedikat
            ;
subjekt:      artikel substantiv
            ;
artikel:      `der'
            | `die'
            | `das'
            ;
substantiv:   `frau'
            | `mann'
            | `kind'
            ;
pradikat:     `geht'
            | `schlaeft'
            | `isst'
            ;
\end{verbatim}

Diese Grammatik hat f"unf Nonterminalsymbole und kann einige
primitive deutsche S"atze herleiten wie etwa "`der mann
schlaeft"'. Dabei geht man vom Symbol \fu{satz} aus und setzt
f"ur die Nonterminals \fu{subjekt} und \fu{praedikat} jeweils
eine ihrer Produktionen ein. F"ur \fu{subjekt} wird \fu{artikel}
und \fu{substantiv} eingesetzt und f"ur diese wiederum die beiden
Terminals \fu{`der'} und \fu{`mann'}. Dasselbe geschieht mit
\fu{pradikat}.

Dieses Beispiel demonstriert nicht nur, wie sich S"atze von einer
BNF herleiten lassen, sondern auch, da"s diese nur f"ur den
syntaktischen Teil der Grammatik eingesetzt werden kann. So
l"a"st sich aus obiger BNF auch der Satz "`der kind isst"'
herleiten, der semantisch inkorrekt ist. Deshalb wird der
semantische
\index{Semantik}
Teil in der Regel verbal formuliert, wie etwa in
dieser Kurzreferenz.

Zus"atzlich zu dieser Grundfunktionalit"at beherrscht die BNF
noch zwei zus"atzliche Hilfsmittel zur Formulierung von
Grammatiken: Wird ein Teil einer Produktion in geschwungene Klammern
eingeschlossen (\fu{\{\}}), dann darf dieser Teil im Quelltext
weggelassen oder beliebig oft aufgef"uhrt werden. Durch die Produktion
\fu{`a' \{`b'\} `a'} werden also alle S"atze \fu{aa},
\fu{aba}, \fu{abba}, \fu{abbba}, \ldots\ dargestellt.

Zus"atzlich k"onnen Teile von Produktionen in eckige Klammern
eingeschlossen werden, was bedeutet, da"s sie optional sind, also
im Quelltext weggelassen k"onnen, aber auf keinen Fall "ofter als
einmal auftreten d"urfen. Die Produktion \fu{`a' [`b'] `a'}
beschreibt also die S"atze \fu{aa} und \fu{aba}.

\section{Regul\protect{"are} Ausdr\protect{"ucke}}
\index{regul""arer Ausdruck|ii}

Regul"are Ausdr"ucke werden benutzt, um Zeichenfolgen
wie etwa Bezeichner oder Zahlen zu beschreiben. Dabei werden
aufeinanderfolgende Zeichen hintereinander angeschrieben.
Zus"atzlich gibt es Steuerzeichen, mit denen es m"oglich ist, die
beschriebene Zeichenfolge zu variieren. Es folgt eine
Aufstellung:

\begin{tabular}{lp{13cm}}
\fu{()} & Die beiden runden Klammern werden benutzt, um eine
          Folge von Zeichen zu klammern, sie also nur als ein Zeichen zu
          behandeln. Das macht es m"oglich, Steuerzeichen wie etwa \fu{+}
          auf eine ganze Zeichenfolge anstatt nur auf das letzte Zeichen
          anzuwenden. \\
\fu{[]} & Die beiden eckigen Klammern werden verwendet, um
          Zeichen zu gruppieren. Der ganze Klammerausdruck steht
          f"ur ein beliebiges Zeichen, das innerhalb der Klammer
          steht. Um auch gr"o"sere Gruppen anschreiben zu
          k"onnen, ist es m"oglich, innerhalb der Gruppe mittels
          \fu{-} von einem bestimmten bis zu einem anderen
          bestimmten Zeichen alle Zeichen einzubinden. So gibt
          z.~B. \fu{[0-9]} die Gruppe aller Ziffern an. Tritt
          diese Gruppe irgendwo auf, steht sie f"ur eine
          beliebige Ziffer. Wird als erstes Zeichen einer Gruppe
          \fu{\char94} angegeben, umfa"st die Gruppe all jene
          Zeichen, die nicht innerhalb der Klammern aufgef"uhrt
          sind. \\
\end{tabular}          

\begin{tabular}{lp{13cm}}
\fu{`'} & Sollen Zeichen verwendet werden, die schon als
          Steuerzeichen vergeben sind, werden sie in einfache
          Anf"uhrungszeichen eingeschlossen. \\
\fu{.}  & Der Punkt steht f"ur jedes beliebige Zeichen au"ser dem
          Zeilenwechsel \fu{\char92 n}. \\
\fu{|}  & Ein Ausdruck \fu{a|b} kann entweder durch \fu{a} oder
          durch \fu{b} ersetzt werden. Um zu vermeiden, da"s
          immer der gesamte Ausdruck mit ODER verkn"upft wird, werden
          Oder-Ausdr"ucke in Klammern gesetzt. \\
\fu{?}  & Ein Zeichen oder ein Ausdruck in Klammern, gefolgt von
          einem \fu{?}, ist optional, darf also auftreten, kann
          aber auch weggelassen werden. \\
\fu{+}  & Ein Zeichen oder ein Ausdruck in Klammern, gefolgt von
          einem \fu{+}, steht f"ur ein oder mehrere Vorkommnisse
          dieses Zeichens oder Ausdrucks. \\
\fu{*}  & Der Operator \fu{*} ist mit dem Operator \fu{+}
          identisch, au"ser in dem Fall, da"s das vorhergehende Zeichen
          bzw. der in Klammern stehende Ausdruck auch weggelassen
          werden kann. \\
\end{tabular}

\example
\begin{verbatim}
IDENTIFIER: [a-zA-Z][^\n ]*
\end{verbatim}

Dieser regul"are Ausdruck beschreibt all jene Bezeichner, die mit
einem Buchstaben beginnen, auf den beliebige Zeichen
(au"ser dem Zeilenwechsel \fu{\char92 n} und dem Leerzeichen)
folgen k"onnen. G"ultige \fu{IDENTIFIER} sind etwa \fu{ax*} und
\fu{ZX\#\%}.

\begin{verbatim}
HEX_NUMBER: 0x[0-9a-fA-F]+
\end{verbatim}

Der Ausdruck \fu{HEX\_NUMBER} beschreibt eine hexadezimale
Zahlenkonstante,
\index{Zahl!hexadezimale}
wie sie etwa in \ind{C} verwendet werden kann. Sie beginnt mit
den Zeichen \fu{0x}, auf die mindestens eine Hexadezimalziffer
(Dezimalziffer oder Gro"s- bzw. Kleinbuchstabe von \fu{A} bis
\fu{F}) folgen mu"s. Zu der unendlich gro"sen Menge der
beschriebenen Zahlen z"ahlen \fu{0xff} und \fu{0x123}.

\chapter{Sprachreferenz}

Dieses Kapitel ist eine Referenz "uber die komplette Grammatik
\index{Grammatik!von SEL|ii}
der Programmiersprache SEL. Der Aufbau ist dabei top-down-orientiert,
da es meines Erachtens nat"urlicher ist, vom Gro"sen ins Kleine
zu gehen, als umgekehrt.

Das erste Unterkapitel widmet sich deshalb der groben Struktur
eines SEL-Programms. Es wird die Modularit"at von SEL
besprochen sowie der grundlegende Aufbau einer Funktion. Weiters
wird in diesem Unterkapitel die wichtigste Funktion eines
SEL-Programms beschrieben, n"amlich die \fu{main}-Funktion.

Das zweite Unterkapitel bespricht die Kontrollstrukturen, also
die Verzweigungen und Schleifen, die die strukturierte
Programmiersprache ausmachen.

Das dritte Unterkapitel widmet sich den Ausdr"ucken oder
{\em expressions}, die den Hauptanteil an einem Programm haben. Zu
Ausdr"ucken z"ahlen zum Beispiel Zuweisungen, Vergleiche oder
arithmetische Operationen.

Im vierten Unterkapitel werden die Datentypen besprochen, die
mehr oder weniger Bestandteile der Ausdr"ucke sind.

\section{Struktur eines Programms}
SEL ist eine modulare
\index{Modul}
Programmiersprache, d.h. ein Programm kann sich (mu"s sich aber
nicht) aus mehreren Modulen---oder Quelldateien---zusammensetzen.
Zus"atzlich kann jedes Modul seinen internen Aufbau vor dem Rest
des Programms sch"utzen, um Implementationsdetails zu verbergen
(\indi{Kapselung}), damit diese jederzeit ge"andert werden
k"onnen, ohne da"s der aufrufende Programmteil dazu ver"andert
werden mu"s. Dies setzt jedoch auch ein vern"unftiges
"`\ind{Interface}"' (Schnittstelle zwischen dem Modul und dem
Programm) voraus. Es wird also in der Regel n"otig sein, vor der
Implementation einen Entwurf des Programms zu machen---einfaches
"`Drauflosprogrammieren"' wird sich schnell als Sackgasse
erweisen.

\subsection{Das Modul}
\index{Modul|ii}
\fuindexii{module}
\fuindexii{block}
\fuindex{function}
\fuindex{local}
\fuindex{global}

\begin{verbatim}
module:        {block}
             ;

block:         function
             | local
             | global
             ;
\end{verbatim}

Jedes Modul besteht aus einem oder mehreren Bl"ocken,
\index{Block|ii}
die in Funktionen, \rwi{local}- und \rwi{global}-Bl"ocke
eingeteilt werden (es ist theoretisch auch m"oglich, ein Modul zu
schreiben, das keinen einzigen Block enth"alt, doch dieses Modul
h"atte keinen anderen Zweck, als wertvollen Speicherplatz zu
verschwenden).

Funktionen sind Programmteile---oder Unterprogramme---, die von
bestimmten Stellen im Programm aufgerufen werden k"onnen. Da SEL
eine modulare Programmiersprache ist, ist es m"oglich, Funktionen
zu kapseln, sie also nur in dem Modul verf"ugbar zu machen, in
dem sie definiert wurden.

Die \rw{local}-Bl"ocke dienen dazu, Variablen,
\index{Variable!modulglobale}
Funktionen
\index{Funktion!modulglobale}
und Strukturen
\index{Struktur}
zu definieren, die nur innerhalb des jeweiligen Moduls
erreichbar sein sollen.

Die \rw{global}-Bl"ocke dienen einerseits dem Zweck, Variablen,
\index{Variable!programmglobale}
Funktionen
\index{Funktion!programmglobale}
und Strukturen zu definieren, die auch au"serhalb des Moduls
verf"ugbar sein sollen, und andererseits jene Variablen,
Strukturen und Funktionen zu deklarieren, die nicht in diesem
Modul definiert sind und trotzdem verwendet werden sollen.
Beachten Sie bitte, da"s diese in einem anderen Modul
als global definiert sein m"ussen.

\subsection{Funktionen}
\index{Funktion|ii}
\index{Block}
\fuindexii{function}
\fuindex{type}
\fuindex{statement}
\fuindex{IDENTIFIER}
\rwindex{begin}
\rwindex{end}
\fuindexii{formal\protect{\_params}}
\fuindexii{func\protect{\_var}}
\fuindex{expr}
\fuindexii{func\protect{\_proto}}
\rwindexii{function}

\begin{verbatim}
function:      `function' IDENTIFIER `(' [formal_params] `)' `:' type
                 {func_var}
               `begin'
                 {statement}
               `end'
             ;

formal_params: IDENTIFIER `:' [`&'] type [`,' formal_params]
             ;

func_var:      IDENTIFIER `:' type [`=' expr] `;'
             ;

func_proto:    `function' IDENTIFIER
               `(' [[`&'] type {`,' [`&'] type}] `)' `:' type
             ;
\end{verbatim}

Eine Funktion ist ein Unterprogramm,
\index{Unterprogramm|see{Funktion}}
also ein Programmteil, der
von anderen Stellen des Programms aufgerufen werden kann. Dabei
k"onnen einer Funktion ein oder mehrere \indi{Parameter} (deren
Typen bei der Deklaration der Funktion festgelegt werden)
"ubergeben werden. Innerhalb dieser Funktion k"onnen diese
Parameter als ganz normale Variablen angesprochen werden. Die
Namen dieser Variablen werden im Funktionskopf angegeben.

Funktionsdeklarationen k"onnen an zweierlei Stellen angetroffen
werden, n"amlich dort, wo die Funktionen definiert sind, und
in \rw{local}- und \rw{global}-Bl"ocken, wo allerdings nur der
Funktionskopf ohne Parameternamen angegeben wird. Zu beachten
ist, da"s Funktionen im Programm erst nach dem ersten Vorkommen
ihrer Deklarationen oder Definitionen bekannt sind. Eine Funktion
kann also nicht aufgerufen werden, wenn sie im Quellcode nicht
vorher deklariert oder definiert wurde.

Funktionen sind, solange sie nicht explizit in einem
\rw{global}-Block des jeweiligen Moduls aufgef"uhrt sind,
gekapselt, also auf das Modul beschr"ankt, in dem sie
definiert wurden. Trotzdem ist es m"oglich, Funktionen in einem
\rw{local}-Block des
Moduls aufzuf"uhren, f"ur den Fall, da"s eine Funktion an einer
Stelle aufgerufen werden mu"s, die vor ihrer Definition liegt.

Zur Semantik: Dem \rw{function}-Schl"usselwort folgt der Name der
Funktion, der wie jeder Bezeichner case-sensitiv ist. In Klammern
gesetzt folgen darauf die Namen und Typen der einzelnen Parameter.
Innerhalb der Funktion kann auf die Parameter mit Hilfe ihrer
Namen wie auf normale Variablen zugegriffen werden.

SEL unterst"utzt zwei Arten der Parameter"ubergabe:
\index{Parameter""ubergabe|ii}
\index{call-by-value|see{Parameter""ubergabe}}
\index{call-by-reference|see{Parameter""ubergabe}}
{\em call-by-value} und {\em call-by-reference}. Ein
Parameter, der als call-by-value deklariert ist, wird der
Funktion als Wert "ubergeben, es wird beim Funktionsaufruf also
zuerst der Wert des Parameters berechnet und dieser dann kopiert.
Die jeweilige Funktion greift dann auf die Kopie des Parameters
zu, kann also die Variable, die "ubergeben wurde, nicht
ver"andern. Von einem als call-by-reference deklarierten Parameter
(dieser ist daran zu erkennen, da"s seinem Typ im Funktionskopf
das \fu{\&}-Zeichen
\index{Referenztyp|ii}
vorangestellt ist) hingegen wird der Funktion intern die Adresse
der als Parameter "ubergebenen Variable zug"anglich gemacht,
wovon die Funktion allerdings nichts merkt. Dies wirkt sich
dahingehend aus, da"s die Funktion durch "Anderung des Wertes des
Parameters auch den Wert der Variable ver"andert, die ihr als
Parameter "ubergeben wurde. Beachten Sie allerdings, da"s es
nicht m"oglich ist, eine \ind{Konstante} f"ur einen
call-by-reference-Parameter zu "ubergeben.

Nach der Parameterliste folgt der R"uckgabetyp
\index{R""uckgabetyp|ii}
der Funktion. Das ist der Typ, den der Aufrufer der Funktion
zur"uckgeliefert bekommt. Der R"uckgabewert wird durch die
\rwi{return}-Anweisung bestimmt. Ist keine R"uckgabe n"otig, kann
der Typ \tvoid\
\typeindex{void}
verwendet werden.

Im Anschlu"s an den R"uckgabetyp k"onnen lokale
Variablen
\index{Variable!lokale|ii}
vereinbart werden. Diese Variablen sind---der Name
sagt es bereits---nur innerhalb der Funktion verf"ugbar, in der
sie deklariert wurden. Gleichzeitig k"onnen diesen Variablen
Werte zugewiesen werden. Geschieht dies nicht, sind die Werte
dieser Variablen undefiniert. Beachten Sie, da"s Variablen
auch mit berechneten Werten initialisiert werden k"onnen.

Nach der Vereinbarung der lokalen Variablen folgt in
\rwi{begin}--\rwi{end}
eingeschlossen der eigentliche Code der
Funktion. Von hier aus k"onnen wieder andere Funktionen
aufgerufen werden, die Funktion kann sich sogar selbst aufrufen
(\ind{Rekursion}).

\example
Die Funktion \fuii{Factorial} berechnet die Fakult"at
\index{Fakult""at|ii}
ihres Parameters \fu{i}
(\hbox{$\fu{i}! = \prod_{n=1}^{\fu{\footnotesize i}} n$}).
Hierzu verwendet sie zwei lokale Variablen \fu{c} und \fu{r}.

\rwindexuu{global}
\rwindexuu{for}
\begin{verbatim}
global
  function Factorial (int) : int
end

function Factorial (i : int) : int

  c : int;
  r : int = 1;

begin
  for (c = 2 to i)
    r = r * c;
  end
  return r;
end
\end{verbatim}

\subsubsection{Die main-Funktion}
Die wichtigste \ind{Funktion} in einem SEL-Programm ist die Funktion
\fuii{main}. Sie stellt gewisserma"sen das Hauptprogramm dar und
ist somit die erste Funktion, die in einem SEL-Programm aufgerufen
wird. Hat \fu{main} ihre Arbeit beendet, kehrt der Computer zum
aufrufenden Programm (in der Regel der Kommandozeileninterpreter)
zur"uck. Dabei kann letzterem ein sogenannter {\em Exit-Code}
"ubergeben werden. Dieser \indi{Exit-Code} hat den Typ \tint\
\typeindex{int}
und wird mit der \rwi{return}-Anweisung zur"uckgegeben. Ist der
R"uckgabetyp der Funktion \fu{main} \tvoid ,
\typeindex{void}
so ist der Exit-Code \fu{0}.

Zu beachten ist, da"s die \fu{main}-Funktion immer als global
deklariert werden mu"s, unerheblich, in welchem Modul sie
definiert ist.

\example
Dieses {\em hello-world}-Programm
\index{hello-world|uu}
besteht aus einem
einzigen Modul und hat die Aufgabe, den String "`\fu{hello, world!}"'
auszugeben. Zus"atzlich wird der Exit-Code \fu{1} zur"uckgeliefert.

\fuindexuu{WriteString}
\rwindexuu{global}
\begin{verbatim}
#include "conio.sh"

global
  function main () : int
end

function main () : int
begin
  WriteString("hello, world!\n");
  return 1;
end
\end{verbatim}

\subsubsection{Die return-Anweisung}
Die \rwii{return}-Anweisung hat den Zweck, aus einer \ind{Funktion} in
die "ubergeordnete Funktion (bzw. das Betriebssystem im Falle der
\fu{main}-Funktion) zur"uckzukehren. Ist der R"uckgabetyp der
jeweiligen Funktion nicht \tvoid ,
\typeindex{void}
so mu"s auch ein \ind{Wert}
zur"uckgegeben werden. Dieser wird der \rw{return}-Anweisung
nachgestellt. \rw{return} kann an einer beliebigen Stelle
innerhalb einer Funktion stehen, Schleifen
\index{Schleife} werden---falls n"otig---abgebrochen.

\example
Die Funktion \fuii{Square} gibt
das \ind{Quadrat} eines ihr "ubergebenen \tint\ \fu{i} zur"uck.

\begin{verbatim}
function Square (i : int) : int
begin
  return i * i;
end
\end{verbatim}

\subsection{local-Bl\protect{"ocke}}
\rwindexii{local}
\rwindex{end}
\index{Block}
\fuindexii{local}
\fuindexii{local\protect{\_decl}}
\fuindex{struct}
\fuindex{func\protect{\_proto}}
\fuindex{var}

\begin{verbatim}
local:         `local'
                 {local_decl}
               `end'
             ;

local_decl:    struct
             | func_proto
             | var
             ;
\end{verbatim}

Die \rw{local}-Bl"ocke dienen dazu, Variablen,
\index{Variable!modulglobale|ii}
Strukturen
\index{Struktur}
und Funktionen
\index{Funktion!modulglobale|ii}
zu definieren, die nur in dem jeweiligen Modul
bekannt sein sollen. Dieses Vorgehen nennt sich \ind{Kapselung} und hat den
Vorteil, da"s es m"oglich ist, die Implementationsdetails eines
Moduls vor dem Rest des Programms zu verstecken. Das macht es
einerseits m"oglich, das Modul anders zu implementieren, ohne
da"s das Restprogramm in irgendeiner Weise ge"andert werden mu"s
und andererseits, da"s der Code des Moduls in anderen Programmen
leichter wiederverwendet werden kann. F"ur eine solche Vorgangsweise
ist es jedoch unabdingbar, f"ur jedes Modul ein gut "uberdachtes
\ind{Interface} (Schnittstelle zwischen Modul und Restprogramm) zu
verwenden.

Variablen, Strukturen und Funktionen sind im Source erst ab der
Stelle ihrer Deklaration bekannt. Aus diesem Grunde ist es
ratsam, \rw{local}-Bl"ocke an den Anfang der Datei zu stellen,
was au"serdem die "Ubersichtlichkeit des Codes erh"oht.

\subsection{global-Bl\protect{"ocke}}
\rwindexii{global}
\rwindex{end}
\index{Block}
\fuindexii{global}
\fuindexii{global\protect{\_decl}}
\fuindex{struct}
\fuindex{func\protect{\_proto}}
\fuindex{var}
\fuindex{extern}

\begin{verbatim}
global:        `global'
                 {global_decl}
               `end'
             ;

global_decl:   struct
             | func_proto
             | var
             | extern
             ;
\end{verbatim}

Die in einem \rw{global}-Block definierten Strukturen,
\index{Struktur}
Funktionen
\index{Funktion!programmglobale|ii}
und Variablen
\index{Variable!programmglobale|ii}
sind dazu bestimmt, auch von anderen
Modulen verwendet zu werden. Zus"atzlich k"onnen in einem
\rw{global}-Block Variablen deklariert werden, die in diesem Modul
verwendet werden sollen, jedoch in einem anderen Modul als global
definiert worden sind. Variablen, Strukturen und Funktionen sind
erst ab ihrer Deklaration verf"ugbar, soda"s es am
praktikabelsten ist, die \rw{global}-Bl"ocke an den Beginn der
Datei zu setzen oder in einer \ind{Include-Datei} unterzubringen.

\section{Anweisungen}
\index{Anweisung|ii}
\fuindexii{statement}
\fuindex{expr}
\fuindex{if}
\fuindex{while}
\fuindex{for}
\fuindex{repeat}
\fuindex{switch}
\fuindex{prefix}
\rwindex{break}
\rwindex{return}

\begin{verbatim}
statement:     [expr] `;'
             | if
             | while
             | for
             | repeat
             | switch
             | prefix
             | `break' `;'
             | `return' [expr] `;'
             ;
\end{verbatim}

Anweisungen (oder {\em statements}) k"onnen innerhalb von Funktionen
\index{Funktion}
verwendet werden und machen das eigentliche Programm aus.
Anweisungen k"onnen grunds"atzlich in Ausdr"ucke ({\em expressions}),
\index{Ausdruck}
Kontrollanweisungen
\index{Kontrollanweisung}
und Schleifen
\index{Schleife}
eingeteilt werden. Erstere verrichten die
eigentliche Arbeit, und die beiden letzteren steuern den Programmflu"s. Wie
aus der Grammatik ersichtlich ist, kann eine Anweisung auch
einfach nur ein Semikolon enthalten. Solche Anweisungen haben die
Bezeichnung {\em \indi{Nullanweisung}} und haben---bis auf die
Verbesserung der "Ubersichtlichkeit in bestimmten Situationen---%
keine Aufgabe.

\subsection{Die if-Anweisung}
\rwindexii{if}
\fuindexii{if}
\fuindex{expr}
\fuindex{statement}
\rwindexii{else}
\rwindex{end}

\begin{verbatim}
if:            `if' `(' expr `)'
                 {statement}
               [`else'
                 {statement}]
               `end'
             ;
\end{verbatim}

Die \rw{if}-Anweisung ist eine Kontrollanweisung und dient der
bedingten Ausf"uhrung eines Programmteils. Dem \rw{if}
folgt ein in Klammern eingeschlossener \ind{Ausdruck}, der zuerst
evaluiert (berechnet) und dann nach \tint\
\typeindex{int}
gecastet wird. Ist dieser
Ausdruck ungleich \fu{0}, werden die darauffolgenden
Anweisungen ausgef"uhrt. Ist jedoch der Ausdruck \fu{0}, dann
werden die Anweisungen, die dem \rw{else} folgen, sofern ein
\rw{else} vorhanden ist, ausgef"uhrt, ansonsten geschieht gar
nichts.

\example
Die Funktion \fuii{GetInt} fragt den Benutzer nach einer Zahl.
Ist die Zahl gr"o"ser oder gleich  \fu{0}, so wird der Text
"`\fu{Gratulation, die Zahl ist positiv!}"'
auf dem Bildschirm dargestellt. Andernfalls wird die Zahl ohne
negatives Vorzeichen ausgegeben.

\fuindexuu{ReadInt}
\fuindexuu{WriteString}
\fuindexuu{WriteInt}
\begin{verbatim}
function GetInt () : void

  i : int;

begin
  i = ReadInt();
  if (i >= 0)
    WriteString("Gratulation, die Zahl ist positiv!\n");
  else
    WriteInt(-i);
    WriteString("\n");
  end
end
\end{verbatim}

\subsection{Die switch-Anweisung}
\rwindexii{switch}
\fuindexii{switch}
\fuindex{expr}
\fuindex{statement}
\rwindex{end}
\fuindexii{prefix}
\rwindex{case}
\fuindex{constant}
\rwindex{default}

\begin{verbatim}
switch:        `switch' `(' expr `)'
                 {statement}
               `end'
             ;

prefix:        `case' constant `:'
             | `default' `:'
             ;
\end{verbatim}

Die \rw{switch}-Anweisung wird verwendet, um f"ur mehrere
verschiedene Werte eines Ausdrucks
\index{Ausdruck}
verschiedene Aktionen
auszuf"uhren. Als erstes wird der Ausdruck evaluiert. Danach
wird er mit der \ind{Konstante}, die dem ersten im
\fu{\{statement\}} vorkommenden \rw{case} folgt, verglichen. Ist
der Vergleich positiv, so werden alle der \rw{case}-Anweisung
folgenden Anweisungen ausgef"uhrt, solange, bis ein \rwii{break}
gefunden wird, was den Programflu"s nach dem \rw{end} fortsetzt.
Ist der Vergleich jedoch negativ, wird das n"achste \rw{case}
gesucht u.s.w. Wird bei diesem Vorgang ein \rwii{default} gefunden,
dann wird die Ausf"uhrung {\em in jedem Fall} nach dieser
Anweisung fortgesetzt. Aus diesem Grunde sollte \rw{default}
immer am Schlu"s der \rw{case}-Kette stehen.
Findet sich weder ein \rw{case}, das dem
Wert des Ausdrucks entspricht, noch ein \rw{default}, geschieht
nichts und die Ausf"uhrung wird nach dem \rw{end} fortgesetzt.

\example
Die Funktion \fuii{SimpleMenu} stellt ein Men"u auf
dem Bildschirm dar und wartet auf eine Eingabe. F"ur die Ziffern
\fu{`1'} bis \fu{`3'} werden die entsprechenden Funktionen
\fuii{Choice1} bis \fuii{Choice3}
\fuindexii{Choice2}
ausgef"uhrt. Dieser Vorgang
wiederholt sich so lange, bis der Benutzer \fu{`x'} oder \fu{`X'}
eingibt, was die Funktion beendet. Bet"atigt der Benutzer eine
Taste, die nicht verarbeitet werden kann, wird eine
dementsprechende Meldung ausgegeben. F"ur die Ausgabe des Men"us
auf dem Bildschirm wird die Funktion \fui{DisplayMenu} verwendet.

\rwindexuu{while}
\fuindexuu{ReadChar}
\fuindexuu{WriteString}
\begin{verbatim}
function SimpleMenu () : void
begin
  while (1)
    DisplayMenu();
    switch (ReadChar())
      case '1' :
        Choice1();
        break;
      case '2' :
        Choice2();
        break;
      case '3' :
        Choice3();
        break;
      case 'x' :
      case 'X' :
        return;
      default :
        WriteString("Falsche Eingabe!\n");
    end
  end
end
\end{verbatim}
        
\subsection{Die while-Schleife}
\rwindexii{while}
\fuindexii{while}
\fuindex{expr}
\fuindex{statement}
\rwindex{end}

\begin{verbatim}
while:         `while' `(' expr `)'
                 {statement}
               `end'
             ;
\end{verbatim}

Die \rw{while}-Schleife f"uhrt die zwischen \rw{while} und
\rw{end} aufgef"uhrten Anweisungen sooft aus, solange der in Klammern
eingeschlossene \ind{Ausdruck} ungleich \fu{0} ist. Dabei wird wie
folgt vorgegangen: Als erstes wird der Ausdruck evaluiert und
nach \tint\
\typeindex{int}
gecastet. Ist er
gleich \fu{0}, so wird die Ausf"uhrung nach dem \rw{end}
fortgesetzt, andernfalls werden die Anweisungen, die dem Ausdruck
folgen, ausgef"uhrt. Hierauf wird wieder an den Anfang der
Schleife gesprungen und der Ausdruck abermals gepr"uft u.s.w.
Beachten Sie, da"s es au"ser mit Hilfe der \rwi{return}-Anweisung
nicht m"oglich ist, aus der Schleife auszusteigen, ohne da"s der
Ausdruck gleich \fu{0} ist.

\example
Die Funktion \fuii{OutputList} durchl"auft die ihr "ubergebene
\indu{Liste} des Typs \fu{List}
\index{List@{\tt List}~(\rw{struct})|ii}
und gibt dabei alle Datenelemente---in
diesem Beispiel der Einfachheit halber vom Typ \tint ---aus.
Anmerkung: Das letzte Element der Liste ist dadurch
gekennzeichnet, da"s der Zeiger, der auf das n"achste Element
zeigen sollte, ein \ind{Nullzeiger} ist.

\fuindexuu{WriteInt}
\rwindexuu{local}
\begin{verbatim}
local
  struct List
    i    : int;
    Next : *List;
  end
end

function OutputList (l : *List) : void
begin
  while (l != 0)
    WriteInt(l->i);
    l = l->Next;
  end
end
\end{verbatim}

\subsection{Die repeat-Schleife}
\rwindexii{repeat}
\fuindexii{repeat}
\rwindexii{until}
\fuindex{statement}
\fuindex{expr}

\begin{verbatim}
repeat:        `repeat'
                 {statement}
               `until' `(' expr `)'
             ;
\end{verbatim}

Die \rw{repeat}-Schleife wiederholt die zwischen \rw{repeat} und
\rw{until} aufgef"uhrten Anweisungen so lange, bis der auf
\rw{until} folgende \ind{Ausdruck} ungleich \fu{0} ist. Der Ablauf ist
hierbei folgenderma"sen: Zun"achst werden die Anweisungen
ausgef"uhrt, danach wird der Ausdruck evaluiert, nach \tint\
\typeindex{int}
gecastet und auf
Gleichheit mit \fu{0} gepr"uft. Ist diese gegeben, wird zu
\rw{repeat} zur"uckgesprungen und der Vorgang wiederholt sich.
Ist der Ausdruck hingegen ungleich \fu{0}, ist die Schleife
beendet. Beachten Sie bitte, da"s---im Gegensatz zur
\rwi{while}-Schleife---die Schleife {\em mindestens einmal}
ausgef"uhrt wird, unerheblich, ob der Ausdruck schon vor der
Ausf"uhrung ungleich \fu{0} w"are.

\example
Dieses Beispielprogramm liest so lange Zahlen vom Benutzer ein,
bis dieser \fu{0} eingibt. Anschlie"send wird der Mittelwert der
eingelesenen Zahlen ausgegeben. Wurden au"ser der \fu{0} keine
Zahlen eingegeben, erscheint eine dementsprechende Meldung.
    
\rwindexuu{global}
\fuindexuu{ReadInt}
\rwindexuu{if}
\fuindexuu{WriteInt}
\fuindexuu{WriteString}
\begin{verbatim}
#include "conio.sh"

global
  function main () : void
end

function main () : void

  i : int;
  a : int =  0;
  n : int = -1;

begin
  repeat
    a = a + (i = ReadInt());
    n = n + 1;
  until (i == 0)
  if (n)
    WriteInt(a / n);
  else
    WriteString("Es wurden keine Zahlen eingegeben!");
  end
  WriteString("\n");
end
\end{verbatim}

\subsection{Die for-Schleife}
\rwindexii{for}
\fuindexii{for}
\fuindex{primary}
\fuindex{expr}
\rwindexii{to}
\rwindexii{downto}
\rwindexii{step}
\fuindex{statement}
\rwindex{end}

\begin{verbatim}
for:           `for' `(' primary `=' expr (`to' | `downto') expr
               [`step' expr] `)'
                 {statement}
               `end'
             ;
\end{verbatim}

Die \rw{for}-Schleife ist eine spezielle Schleife, die es
erlaubt, Anweisungen eine bestimmte Anzahl mal zu wiederholen.
Hierzu ist nach der "offnenden Klammer die Variable anzugeben,
die als Z"ahler fungiert. Da hier ein \fu{primary} (siehe Kapitel
\ref{exprs}) erwartet wird,
ist es auch m"oglich, Elemente eines Arrays
\index{Array}
oder einer \ind{Struktur}
als Z"ahler zu verwenden. Nach dem Zuweisungsoperator wird der
Anfangsindex angegeben und danach entweder das Schl"usselwort
\rw{to} oder \rw{downto}, je nachdem, ob nach oben oder nach
unten gez"ahlt werden soll. Darauf folgt der Zielindex, also
jener Wert, den die Indexvariable maximal oder minimal annehmen
darf. Optional kann noch nach dem Schl"usselwort \rw{step} die
Schrittweite angeben werden. Zu beachten ist, da"s diese auch
dann, wenn nach unten gez"ahlt wird, positiv sein mu"s.

Bei der Ausf"uhrung wird folgenderma"sen vorgegangen: Zun"achst
werden alle Ausdr"ucke
\index{Ausdruck}
evaluiert und nach \tint\
\typeindex{int}
gecastet. Danach
wird "uberpr"uft, ob der Wert der Indexvariablen schon au"serhalb
der zul"assigen Grenze ist. Ist das nicht der Fall, werden die
Anweisungen das erste Mal ausgef"uhrt und der Z"ahler
entsprechend erh"oht bzw. erniedrigt. Danach wird wieder
"uberpr"uft, ob die Anweisungen nocheinmal ausgef"uhrt werden
sollen u.s.w.

\example
Die Funktion \fuii{Squares} gibt der Reihe nach die Quadrate
\index{Quadrat}
der ganzen Zahlen $i$ f"ur \hbox{$\fu{a} \le i \le \fu{b}$} aus.
Der Funktion werden die \tint s \fu{a} und \fu{b} "ubergeben.

\fuindexuu{WriteInt}
\fuindexuu{WriteString}
\begin{verbatim}
function Squares (a : int, b : int) : void

  i : int;

begin
  for (i = a to b)
    WriteInt(i * i);
    WriteString("\n");
  end
end
\end{verbatim}

\section{Ausdr\protect{"ucke}}
\label{exprs}
\index{Ausdruck|ii}
\fuindexii{expr}
\rwindex{delete}
\fuindexii{primary}
\fuindex{IDENTIFIER}
\fuindex{INT\protect{\_CONSTANT}}
\fuindex{CHAR\protect{\_CONSTANT}}
\fuindex{FLOAT\protect{\_CONSTANT}}
\fuindex{STRING\protect{\_CONSTANT}}
\fuindex{type}
\rwindex{sizeof}
\rwindex{new}

\begin{verbatim}
expr:          expr (`=' | `&&' | `||' | `^^' | `==' | `<' | `>'
                   | `<=' | `>=' | `!=' | `&' | `|' | `^' | `<<'
                   | `>>' | `+' | `-' | `*' | `/' | `%') expr
             | (`!' | `~' | `-' | `*' | `&') expr
             | `delete' expr
             | primary
             ;

primary:       IDENTIFIER
             | `(' expr `)'
             | INT_CONSTANT
             | CHAR_CONSTANT
             | FLOAT_CONSTANT
             | STRING_CONSTANT
             | primary `->' IDENTIFIER
             | primary `:' type
             | primary `[' expr `]'
             | IDENTIFIER `(' [expr {`,' expr}] `)'
             | `sizeof' type
             | `new' type
             ;
\end{verbatim}

Ausdr"ucke sind sozusagen das "`Herz"' von SEL, denn sie
erledigen die eigentliche Arbeit, w"ahrend Kontrollanweisungen
und Schleifen nur kontrollieren, was passieren soll. Ausdr"ucke k"onnen die
verschiedensten Aufgaben erledigen: Sie k"onnen zum Vergleich
zweier Werte benutzt werden, zur Wertzuweisung oder zum
Funktionsaufruf. Die gemeinsamen Merkmale der Ausdr"ucke sind:
Jeder Ausdruck besitzt sowohl einen \ind{Typ} als auch einen
\ind{Wert} (bis auf die Ausdr"ucke, die vom Typ \tvoid\
\typeindex{void}
sind).

\subsection{Konstanten}
\index{Konstante}
\fuindexii{constant}
\fuindex{INT\protect{\_CONSTANT}}
\fuindex{CHAR\protect{\_CONSTANT}}
\fuindex{FLOAT\protect{\_CONSTANT}}
\fuindex{STRING\protect{\_CONSTANT}}

\begin{verbatim}
constant:      [`-'] INT_CONSTANT
             | CHAR_CONSTANT
             | [`-'] FLOAT_CONSTANT
             | STRING_CONSTANT
             ;
\end{verbatim}

Konstanten in SEL sind im Grunde als Konstanten im
mathematischen Sinne zu verstehen. Ihre Werte k"onnen im gesamten
Programm nicht
ver"andert werden. Im Gegensatz zu anderen Programmiersprachen
besitzt SEL nicht die F"ahigkeit, Konstanten einen Namen zu geben,
\index{Konstante!benannte}
was allerdings durch die \fui{\#define}-Direktive leicht umgangen
werden kann. Weiters gibt es in SEL keine Konstanten von
zusammengesetzten Typen,
\index{Typ!zusammengesetzter}
da diese F"ahigkeit nur sehr selten ben"otigt wird.

\subsubsection{int-Konstanten}
\index{Konstante!\tint |ii}
\typeindex{int}
\fuindexii{INT\protect{\_CONSTANT}}

\begin{verbatim}
INT_CONSTANT: [0-9]+
\end{verbatim}

Eine \tint -Konstante ist eine Folge von Ziffern, die als Zahl
mit der Basis 10, also als Dezimalzahl
\index{Zahl!ganze}
\index{Zahl!dezimale}
interpretiert werden.
Laut obiger Definition gibt es jedoch
keine negativen \tint -Konstanten, was den Grund hat, da"s
dieser Mechanismus in der Grammatik eingebaut ist. Dies wiederum
ist darin begr"undet, da"s es andernfalls zu Mi"sverst"andnissen
bei der Interpretation von Ausdr"ucken kommen w"urde (man denke
an den Ausdruck \fu{7-4}---er w"urde als Konstante \fu{7} gefolgt
von der Konstante \fu{-4} gedeutet werden, da die lexikalische
Analyse "uber keine Informationen "uber den Kontext verf"ugt, in dem
die Konstante auftritt, mit anderen Worten, sie wei"s nicht, da"s
auf eine Konstante keine weitere folgen kann).

Voraussichtlich wird SEL in der Version 0.4 um die F"ahigkeit
erweitert werden, auch hexadezimale
\index{Zahl!hexadezimale}
\tint -Konstanten in der Form
\fu{0(x|X)[0-9a-fA-F]+} zu interpretieren.

\subsubsection{char-Konstanten}
\index{Konstante!\tchar |ii}
\typeindex{char}
\fuindexii{CHAR\protect{\_CONSTANT}}

\begin{verbatim}
CHAR_CONSTANT: '(.|\.)'
\end{verbatim}

\tchar -Konstanten repr"asentieren einzelne \ind{Zeichen}.
Dieses k"onnen entweder "`normale"' Zeichen oder
Steuerzeichen
\index{Zeichen!Steuerzeichen}
sein---letztere werden durch einen vorangestellten
Back"-slash "`\fu{\char92}"'
\index{Backslash (\fu{\char92})}
gekennzeichnet, jedoch als ein einzelnes Zeichen behandelt.

\subsubsection{float-Konstanten}
\index{Konstante!\tfloat |ii}
\typeindex{float}
\fuindexii{FLOAT\protect{\_CONSTANT}}

\begin{verbatim}
FLOAT_CONSTANT: [0-9]*(`.'[0-9]*)?((e|E)(+|-)?[0-9]+)?
\end{verbatim}

Eine \tfloat -Konstante repr"asentiert eine positive, rationale
\index{Zahl!rationale}
Zahl in \ind{Exponentialschreibweise}, wobei der Exponent auch
weggelassen werden kann. Leider ist eine Definition als
regul"arer Ausdruck nicht ohne Probleme m"oglich. So w"are laut
obiger Definition auch eine leere Eingabe oder ein Dezimalpunkt
f"ur sich alleine eine \tfloat -Konstante, was nat"urlich nicht
im Sinn der Sache liegen kann. Die Definition ist also
dahingehend zu erweitern, da"s in der Mantisse mindestens eine Ziffer
vorkommen mu"s. Beachten Sie bitte, da"s eine
\tfloat -Konstante immer positiv sein mu"s, was jedoch aufgrund
der Grammatik von SEL nicht weiter st"ort.

\subsubsection{String-Konstanten}
\index{Konstante!String|ii}
\fuindexii{STRING\protect{\_CONSTANT}}

\begin{verbatim}
STRING_CONSTANT: "([^"\n]|\.)*"
\end{verbatim}

Eine String-Konstante ist eine in zwei Anf"uhrungsstriche
eingeschlossene Zeichenfolge, innerhalb derer
alle \ind{Zeichen} bis auf die Anf"uhrungsstriche und den
Zeilenvorschub erlaubt sind. Eine besondere Bedeutung kommt den
Steuerzeichen
\index{Zeichen!Steuerzeichen}
zu, die durch einen vorangestellten Back"-slash
"`\fu{\char92}"'
\index{Backslash (\fu{\char92})}
gekennzeichnet sind. Jede String-Konstante wird
von SEL automatisch mit einem \indi{Nullzeichen}, also einem Zeichen mit
dem Wert \fu{0} beendet. Die String-Konstante beansprucht also im
Speicher um ein Byte mehr, als sie an Zeichen enth"alt.

Eine String-Konstante wird von SEL wie ein Zeiger auf
\tchar s
\typeindex{char}
behandelt. Sie k"onnen diesen Zeiger also dazu
verwenden, um die Zeichenfolge auszugeben, sollten jedoch den
Speicherbereich, auf den er verweist, nicht ver"andern.

\subsubsection{Steuerzeichen}
\label{specialchar}
\index{Zeichen!Steuerzeichen|ii}

SEL stellt bestimmte Steuerzeichen zur Verf"ugung, die in Strings
\index{Konstante!String}
und \tchar -Konstanten
\index{Konstante!\tchar }
\typeindex{char}
benutzt werden k"onnen, um Zeichen
darzustellen, denen eine bestimmte Bedeutung zukommt, wie etwa
dem Zeichen f"ur Zeilenvorschub. Steuerzeichen
sind daran zu erkennen, da"s ihnen ein Back"-slash "`\fu{\char92}"'
\index{Backslash (\fu{\char92})|ii}
vorangestellt ist. Trotzdem werden sie von SEL als einzelne
Zeichen behandelt. Hier eine Zusammenstellung aller in SEL
vorhandenen Steuerzeichen:

\medskip
\index{Warnton}
\index{R""uckschritt}
\index{Bildvorschub}
\index{Zeilenvorschub}
\index{Wagenr""ucklauf}
\index{Tabulator}
\begin{tabular}{ll}
\fu{\char92 a} & Bell (Warnton) \\
\fu{\char92 b} & Backspace (R"uckschritt) \\
\fu{\char92 f} & Formfeed (Bildvorschub) \\
\fu{\char92 n} & New line (Zeilenvorschub) \\
\fu{\char92 r} & Carriage return (Wagenr"ucklauf) \\
\fu{\char92 t} & Tab (Tabulator) \\
\fu{\char92 v} & Vertical tab (vertikaler Tabulator)
\end{tabular}
\medskip

Wird hinter den Back"-slash ein Zeichen gesetzt, das in obiger
Tabelle nicht eingetragen ist, wird von SEL automatisch dieses
Zeichen verwendet. So ist es zum Beispiel m"oglich, einen
Back"-slash oder ein Anf"uhrungszeichen innerhalb einer
String-Konstante einzugeben.

SEL wird in der Version 0.4 aller Voraussicht nach um das
Steuerzeichen \fu{\char92 x\rm\it hh} erweitert werden, das an seine
Stelle das Zeichen mit dem in hexadezimaler
\index{Zahl!hexadezimale}
Notation angegeben Code {\it hh} stellt.

\subsection{Variablen}
\index{Variable|ii}
\fuindexii{var}
\fuindex{IDENTIFIER}
\fuindex{type}
\fuindex{constant}
\fuindexii{extern}
\rwindexii{extern}

\begin{verbatim}
var:           IDENTIFIER `:' type [`=' constant] `;'
             ;

extern:        `extern' IDENTIFIER `:' type `;'
             ;
\end{verbatim}

Variablen sind Speicherbereiche, die
vom Programm belegt werden und w"ahrend der Ausf"uhrung
verschiendene Werte
\index{Wert}
annehmen k"onnen. Jede Variable besitzt
einen \ind{Typ}, der sowohl ihre Gr"o"se als auch ihre Repr"asentation
im Speicher bestimmt. SEL unterscheidet grunds"atzlich zwischen
drei Typen von Variablen: programmglobale Variablen, modulglobale
Variablen und lokale Variablen.

Programmglobale Variablen werden in einem \rwi{global}-Block
definiert und sind im gesamten Programm bekannt. Ein Modul, das
eine Variable verwenden will, die in einem anderen Modul
definiert ist, mu"s jedoch, um auf die Variable zugreifen zu
k"onnen, dies mit einer \rw{extern}-Deklaration bekanntgeben.
Sie unterscheidet sich von einer Variablendefinition lediglich
dadurch, da"s ihr das \rw{extern}-Schl"usselwort vorangestellt
ist und keine Wertzuweisung erlaubt ist. Beachten Sie, da"s es
unm"oglich ist, innerhalb eines
Programms zwei programmglobale Variablen desselben Namens zu
definieren, da dies zu Konflikten f"uhren w"urde.

Modulglobale Variablen werden in einem \rwi{local}-Block eines
Moduls definiert und sind nur innerhalb dieses Moduls
bekannt---auf sie kann von anderen Modulen auch nicht durch eine
\rw{extern}-Deklaration zugegriffen werden. Deshalb ist es auch
ohne weiteres m"oglich, mehrere modulglobale Variablen gleichen
Namens zu definieren, soferne dies in jeweils verschiedenen
Modulen geschieht.

Lokale Variablen sind nur innerhalb der \ind{Funktion} bekannt, in der
sie definiert wurden. Solche Variablen sind allerdings nicht
statisch, das hei"st, sie werden beim Aufruf der jeweiligen Funktion
erzeugt und existieren so lange, bis die Funktion zu ihrem
Aufrufer zur"uckkehrt. Weiters ist es auch durchaus m"oglich,
da"s lokale Variablen mehrfach existieren, und zwar dann, wenn
sich die Funktion selbst aufruft, also rekursiv ist. Bei jedem
solchen Aufruf werden neue lokale Variablen erzeugt, die zwar
namensgleich, jedoch von den anderen absolut isoliert sind.
Unter die lokalen Variablen fallen auch die Funktionsparameter.
\index{Parameter}
Sie werden im Funktionskopf definiert und sind auch nur existent,
solange die Funktion abgearbeitet wird.

Es ist in SEL durchaus m"oglich, modulglobale, programmglobale
und lokale Variablen desselben Namens zu definieren, wobei
folgende Regel zu beachten ist: lokale Variablen werden
modulglobalen Variablen vorgezogen, welche ihrerseits den
programmglobalen Variablen vorgezogen werden. Existiert also die
programmglobale Variable \fu{example}, so kann ohne weiteres eine
modulglobale Variable desselben Namens definiert werden.
Innerhalb des Moduls, in der letztere definiert wurde, bezieht
sich \fu{example} nicht auf die programmglobale, sondern auf die
modulglobale Variable. Definiert weiters eine Funktion innerhalb
dieses Moduls eine lokale Variable namens \fu{example}, so kann
diese Funktion nur auf die lokale Variable zugreifen, w"ahrend
der Rest des Moduls auf die modulglobale Variable zugreift.

\example
Die Funktion \fuii{Recursive}
\index{Rekursion|uu}
gibt die Zahlen \fu{1} bis \fu{10} in
ihrer nat"urlichen Reihenfolge aus und ruft sich nach der
Ausgabe jeder Zahl selbst auf, was dazu f"uhrt, da"s f"ur jede
Zahl zus"atzlich nocheinmal die Zahlen \fu{1} bis \fu{10}
ausgegeben werden, f"ur die dasselbe gilt. Um zu verhindern, da"s
die Zahl \fu{1} unendlich oft ausgegeben wird, was zweifellos
geschehen w"urde, w"urd dieses Verfahren unkontrolliert
angewendet, wird der Funktion ein Parameter "ubergeben, der
bestimmt, wie oft sie sich noch selbst aufrufen soll.
Beachten Sie bei diesem Beispiel, da"s sich die einzelnen Z"ahler
\fu{i} und die Parameter \fu{depth} der Funktion nicht gegenseitig
st"oren.

\rwindexuu{for}
\fuindexuu{WriteInt}
\fuindexuu{WriteString}
\begin{verbatim}
function Recursive (depth : int) : void

  i : int;

begin
  for (i = 1 to 10)
    WriteInt(i);
    WriteString("  ");
    if (depth > 0)
      Recursive(depth - 1);
      WriteString("  ");
    end
  end
end
\end{verbatim}

\subsection{Funktionen}
\index{Funktion|ii}

Funktionen sind Unterprogramme, die von verschiedenen Stellen des
Programms aufgerufen werden k"onnen. Um den Funktionen
Information mitzugeben, k"onnen ihnen sogenannte {\em \indi{Parameter}}
"ubergeben werden, deren Typen bei der Funktionsdeklaration
bestimmt werden m"ussen. Der Aufruf einer Funktion geschieht
durch Angabe des Funktionsnamens und in Klammern gesetzten
Parametern,
\index{Parameter""ubergabe|ii}
durch Beistriche getrennt. Sind keine Parameter
erforderlich, m"ussen nur die
beiden Klammern angegeben werden. Jede Funktion besitzt einen
R"uckgabetyp, der
ebenfalls bei der Deklaration festgelegt werden mu"s. Um eine
Funktion, die keine R"uckgabe vornimmt, zu realisieren, mu"s als
R"uckgabetyp \tvoid\
\typeindex{void}
gew"ahlt werden.

\example
Dieses Programm benutzt die Funktionen \fuiu{Factorial},
\fu{WriteString} und \fu{WriteInt}, um die Fakult"aten
\index{Fakult""at}
der Zahlen von \fu{1} bis \fu{10} auszugeben.

\rwindexuu{for}
\rwindexuu{global}
\fuindexuu{WriteString}
\fuindexuu{WriteInt}
\begin{verbatim}
#include "conio.sh"

global
  function Factorial (int) : int
  function main      ()    : void
end

function main () : void

  i : int;

begin
  for (i = 1 to 10)
    WriteString("Fakultaet von ");
    WriteInt(i);
    WriteString(" = ");
    WriteInt(Factorial(i));
    WriteString("\n");
  end
end
\end{verbatim}

\subsection{Operatoren}
\index{Operator|ii}

Operatoren werden, wie schon der Name sagt, ben"otigt, um
bestimmte Operationen auszuf"uhren, wie etwa das Addieren
von Zahlen oder das Zuweisen von Werten. Grunds"atzlich kann man
in SEL zwischen {\em un"aren} und {\em bin"aren} Operatoren
unterscheiden.

Un"are Operatoren stehen vor einem Ausdruck und
beeinflussen so dessen \ind{Wert}. Ein Beispiel hierf"ur w"are der
\ind{NOT}-Operator \fu{!}, der den \ind{Ausdruck}, der ihm nachsteht,
negiert. Un"are Operatoren werden grunds"atzlich von rechts nach
links ausgewertet, das hei"st, es wird zuerst der Wert des
Ausdrucks berechnet und danach negiert, nicht zuletzt deshalb,
weil es nicht anders m"oglich ist.

Bin"are Operatoren beziehen sich auf zwei Operanden, sie werden
deshalb zwischen diese beiden gestellt. Ein Beispiel f"ur einen
bin"aren Operator w"are der Additionsoperator \fu{+}. Bin"are
Operatoren k"onnen weiter unterteilt werden in solche, die von
rechts nach links und solche, die von links
nach rechts auswerten.
\index{Operator!Richtung der Auswertung}

Ersteres bedeutet, da"s zuerst
der Wert des rechten Operanden berechnet und dann auf den
linken angewendet wird. Ein Beispiel w"are der Zuweisungsoperator
\index{Zuweisung}
\fu{=}. Was unter der obigen Definition gemeint ist, l"a"st sich
leicht anhand eines Ausdrucks veranschaulichen: \fu{a=b=1}. SEL
beginnt von links nach rechts zu parsen und bleibt beim ersten
Operator stehen. Der linke Operand w"are demnach \fu{a}, der
rechte \fu{b=1}. Nun wird zuerst der rechte Operand ausgewertet.
SEL stoppt beim zweiten Operator und steht vor den Operanden
\fu{b} und \fu{1}. Wieder wird der rechte Operand ausgewertet,
von dem SEL nun wei"s, da"s er den Wert \fu{1} hat. Dieser wird
nun auf den linken Operanden angewendet, also der Variable \fu{b}
zugewiesen. Weiters wird nun der Wert des Ausdrucks \fu{b=1}
ebenfalls auf \fu{1} gesetzt und \fu{a} zugewiesen. Diese Art der
Auswertung ist zweckm"a"sig, da es so m"oglich ist, mehreren
Variablen auf einmal den selben Wert zuzuweisen. Um die
Auswertung deutlich zu machen, kann man den Ausdruck auch als
\fu{a=(b=1)} anschreiben.

Ein Beispiel f"ur einen Operator, der von links nach rechts
auswertet, ist der Subtraktionsoperator \fu{-}.
\index{Subtraktion}
Dabei wird
von zwei Operanden zun"achst der linke ausgewertet und dann auf
den rechten angewendet. Als Beispiel sei der Ausdruck \fu{3-2-1}
angef"uhrt. W"urde der Operator \fu{-} von rechts nach links
auswerten, so k"onnte man diesen Ausdruck auch als \fu{3-(2-1)}
anschreiben, was jedoch zu dem falschen Ergebnis \fu{2} f"uhrte.
Aus diesem Grunde wertet \fu{-} von links nach rechts aus,
was den Ausdruck zu \fu{(3-2)-1} macht. Dieser Ausdruck liefert
das gew"unschte Ergebnis \fu{0}.

Es folgt eine kurze Beschreibung aller Operatoren sowie der
Typen ihrer Operanden und des Typs des Ergebnisses der Operation.
Bei den Typen der Operanden ist immer zu beachten, da"s, falls
diese unterschiedlich sind, SEL immer dem "`gr"o"seren"' Typ den
Vorrang gibt, wobei \hbox{$ \tchar < \tint < \tfloat $} gilt.
\typeindex{char}
\typeindex{int}
\typeindex{float}
Der Ausdruck \fu{3 + 4.123} ergibt also einen
\tfloat\ mit dem Wert \fu{7.123}, wobei allerdings der Ausdruck
\fu{13 \&\& 5.3} einen \tint\ mit einem Wert ungleich \fu{0}
ergibt, da der Operator \fu{\&\&} nur mit \tint s arbeitet.

\begin{operator}
\tchar\ \fu{`+'} \tchar\ & \tchar\ \\
\tint\ \fu{`+'} \tint\ & \tint\ \\
\tfloat\ \fu{`+'} \tfloat\ & \tfloat\ \\
\type{\poi type} \fu{`+'} \tint\ & \type{\poi type}
\end{operator}
\index{Addition|ii}

Sind die beiden Operanden des bin"aren \fu{+}-Operators numerisch, werden
diese addiert. Ist jedoch der linke Operand ein \ind{Zeiger}, ergibt
die Operation einen Zeiger, der um die Anzahl Elemente erh"oht
ist, die der rechte Operand angibt. Beachten Sie bitte, da"s die
Gr"o"se eines Elements nicht unbedingt ein Byte sein mu"s,
sondern so gro"s ist, wie der Typ, den der Zeiger referenziert. So
ist es z.~B. leicht m"oglich, ein \ind{Array} zu durchlaufen,
ohne einen numerischen Z"ahler zu verwenden.

\begin{operator}
\tchar\ \fu{`-'} \tchar\ & \tchar\ \\
\tint\ \fu{`-'} \tint\ & \tint\ \\
\tfloat\ \fu{`-'} \tfloat\ & \tfloat\ \\
\type{\poi type} \fu{`-'} \tint\ & \type{\poi type}
\end{operator}
\index{Subtraktion|ii}

Der bin"are Operator \fu{-} subtrahiert den rechten Operanden vom linken,
falls beide numerisch sind, andernfalls wird der \ind{Zeiger} um die
Anzahl Elemente erniedrigt, die vom rechten Operanden angegeben
wird. Mit Element ist hier der Typ gemeint, den der Zeiger
referenziert.

\begin{operator}
\tchar\ \fu{`*'} \tchar\ & \tchar\ \\
\tint\ \fu{`*'} \tint\ & \tint\ \\
\tfloat\ \fu{`*'} \tfloat\ & \tfloat\
\end{operator}
\index{Multiplikation|ii}

Der bin"are \fu{*}-Operator multipliziert die beiden Operanden.

\begin{operator}
\tchar\ \fu{`/'} \tchar\ & \tchar\ \\
\tint\ \fu{`/'} \tint\ & \tint\ \\
\tfloat\ \fu{`/'} \tfloat\ & \tfloat\
\end{operator}
\index{Division|ii}
\index{Zahl!Bruch}

Der \fu{/}-Operator dividiert den linken durch den rechten
Operanden. Sind beide Operanden keine \tfloat s, so ist das
Ergebnis auch kein \tfloat , selbst wenn der Bruch nicht
ganzzahlig ist.

\begin{operator}
\tchar\ \fu{`\%'} \tchar\ & \tchar\ \\
\tint\ \fu{`\%'} \tint\ & \tint\
\end{operator}
\index{Division!Rest bei ganzzahliger|ii}
\index{Modulo|see{Division}}

Der \fu{\%}-Operator berechnet den Rest, der bei einer
ganzzahligen Division des linken durch den rechten Operanden zu
verzeichnen w"are. Beachten Sie bitte, da"s dieser Operator {\em
nicht} mit \tfloat s funktioniert.

\newpage
\begin{operator}
\tint\ \fu{`\&\&'} \tint\ & \tint\
\end{operator}
\index{AND!logisches|ii}
\index{UND|see{AND}}

Der \fu{\&\&}-Operator verkn"upft die beiden Operanden mit einem
logischen Und (AND), das hei"st, der Ausdruck ergibt einen Wert
ungleich \fu{0}, wenn sowohl der linke als auch der rechte
Operand ungleich \fu{0} sind, andernfalls ergibt der Ausdruck
\fu{0}.

\begin{operator}
\tint\ \fu{`||'} \tint\ & \tint\
\end{operator}
\index{OR!logisches|ii}
\index{ODER|see{OR}}

Der \fu{||}-Operator verkn"upft seine beiden Operanden mit einem
logischen Oder (OR), was bedeutet, da"s der Ausdruck dann ungleich
\fu{0} ist, wenn mindestens einer der beiden Operanden ungleich
\fu{0} ist. Nur wenn sowohl der linke als auch der rechte Operand
\fu{0} sind, ergibt der Ausdruck \fu{0}.

\begin{operator}
\tint\ \fu{`\char94\char94'} \tint\ & \tint\
\end{operator}
\index{XOR!logisches|ii}
\index{ODER!exklusives|see{XOR}}

Der \fu{\char94\char94}-Operator verkn"upft die beiden Operanden
mit einem logischen exklusiven Oder (XOR). Ein Ausdruck
ergibt also nur dann einen Wert ungleich \fu{0}, wenn genau einer
der Operanden ungleich \fu{0} ist. Haben hingegen entweder beide
Operanden den Wert \fu{0} oder sind beide ungleich \fu{0}, so ist der
Ausdruck gleich \fu{0}.

\begin{operator}
\tint\ \fu{`\&'} \tint\ & \tint\
\end{operator}
\index{AND!bitweises|ii}

Der bin"are \fu{\&}-Operator verkn"upft die beiden Operanden mit
einem bitweisen Und (AND), was bedeutet, da"s er ein logisches Und auf
jedes einzelne Bitpaar des linken und rechten Operanden anwendet.
So ergibt zum Beispiel der Ausdruck \fu{93 \& 45} den Wert
\fu{13}. Die Funktionsweise l"a"st sich leicht nachvollziehen,
indem man die bin"are Schreibweise
\index{Zahl!bin""are}
verwendet, also \fu{1011101b}
f"ur \fu{93}, \fu{0101101b} f"ur \fu{45} und \fu{0001101b} f"ur
\fu{13}. Jene Stellen, die in mindestens einem der
Operanden den Wert \fu{0} haben, haben auch im Ergebnis den Wert
\fu{0}. Alle anderen Stellen, also jene, die sowohl im linken
als auch im rechten Operanden den Wert \fu{1} haben, haben auch
im Ergebnis den Wert \fu{1}.

\begin{operator}
\tint\ \fu{`|'} \tint\ & \tint\
\end{operator}
\index{OR!bitweises|ii}

Der Operator \fu{|} verkn"upft beide Operanden mit einem
bitweisen Oder (OR), das hei"st, er wendet auf jedes Bitpaar der
beiden Operanden ein logisches Oder an. Der Ausdruck \fu{93 | 45}
ergibt zum Beispiel den Wert 125.

\begin{operator}
\tint\ \fu{`\char94'} \tint\ & \tint\
\end{operator}
\index{XOR!bitweises|ii}

Der Operator \fu{\char94} verkn"upft seine beiden Operanden mit
einem bitweisen exklusiven Oder (XOR), das hei"st, er wendet auf
jedes Bitpaar des linken und rechten Operanden jeweils ein
logisches exklusives Oder an. So ergibt zum Beispiel der Ausdruck
\fu{93 \char94\ 45} den Wert 112.

\begin{operator}
\tint\ \fu{`<<'} \tint\ & \tint\
\end{operator}
\index{Bitverschiebung!nach links|ii}

Der \fu{<<}-Operator ergibt einen Wert, der dem linken Operanden
entspricht, wobei jedoch die einzelnen Bits desselben um soviel
Stellen nach links geschoben werden, wie der rechte Operand
angibt. W"urde dies in eine Zahl resultieren, die mehr Bits hat,
als der \tint -Typ zul"a"st, so werden die "`"ubersch"ussigen"'
Bits abgeschnitten. Ein Beispiel: Der Ausdruck \fu{93 << 3}
ergibt \fu{744}. Warum dies so ist, l"a"st sich leicht anhand der
Bin"arschreibweise
\index{Zahl!bin""are}
der beiden Zahlen verdeutlichen. Die Zahl
\fu{93} kann auch als \fu{1011101b} geschrieben werden, die Zahl
\fu{744} als \fu{1011101000b}. Wie man sieht, wurden also
lediglich drei Nullstellen am Ende der Zahl eingef"ugt.
Anmerkung: Aufgrund der Eigenschaften des bin"aren Zahlensystems
gilt (ohne Ber"ucksichtigung der negativen Zahlen)
\hbox{$a \>\fu{<<}\> n = a \cdot 2^n.$}

\begin{operator}
\tint\ \fu{`>>'} \tint\ & \tint\
\end{operator}
\index{Bitverschiebung!nach rechts|ii}

Der Operator \fu{>>} ergibt einen Wert, der dem linken Operanden
entspricht, dessen Bits um soviel Stellen nach rechts ger"uckt
werden, wie der rechte Operand angibt. Bits, die dabei rechts
"uber die Einerstelle herausragen w"urden, werden dabei
abgeschnitten. Die sich dabei ergebende Zahl kann (ohne
Ber"ucksichtigung der negativen Zahlen) als
\hbox{$a \>\fu{>>}\> n = \lfloor a / 2^n \rfloor $} angeschrieben werden.
So ergibt z.~B. der Ausdruck \fu{93 >> 3} den Wert \fu{11}.

\begin{operator}
\fu{`-'} \tchar\ & \tchar\ \\
\fu{`-'} \tint\ & \tint\ \\
\fu{`-'} \tfloat\ & \tfloat\
\end{operator}
\index{Negation!numerische|ii}
\index{Negation!logische|see{NOT}}
\index{Negation!bitweise|see{NOT}}

Der un"are Operator \fu{-} negiert den Operanden, das
hei"st, er macht aus einer positiven Zahl eine negative und
umgekehrt. So ergibt zum Beispiel der Ausdruck \fu{- 93} den Wert
\fu{-93}.

\begin{operator}
\fu{`*'} \type{\poi type} & \type{type}
\end{operator}
\index{Zeiger!Dereferenzierung|ii}

Der un"are \fu{*}-Operator dereferenziert den Operanden, das
hei"st, er ergibt jene Variable, auf die der Operand zeigt (ist
der Operand kein Zeiger, ist die Operation illegal). Angenommen,
die Variable \fu{p} ist vom Typ \type{\poi int} und wurde mit
\fu{p = \&i} initialisiert, wobei \fu{i} vom Typ \tint\ ist, so
ist der Ausdruck \fu{*p} mit dem Ausdruck \fu{i} gleichzusetzen.

\begin{operator}
\fu{`\&'} \type{type} & \type{\poi type}
\end{operator}
\index{Zeiger!Ermittlung|ii}
\index{Variable!Adresse}

Der un"are \fu{\&}-Operator ergibt einen Zeiger auf seinen
Operanden, was nat"urlich nur m"oglich ist, wenn dieser keine
Konstante ist. So ist z.~B. der Ausdruck \fu{\&i} mit dem
Ausdruck \fu{p} gleichzusetzen, wenn \fu{*p} gleich \fu{i} ist.
Aus diesem und dem letzten Beispiel wird klar, da"s \fu{\&} das
Gegenst"uck zum \fu{*}-Operator ist und umgekehrt.

\begin{operator}
\fu{`!'} \tint\ & \tint\
\end{operator}
\index{NOT!logisches|ii}

Der Operator \fu{!} ergibt die logische Negation (NOT) seines
Operanden, was bedeutet, da"s der Wert gleich \fu{0} ist, falls
der Operand ungleich \fu{0} ist und umgekehrt. So ergibt z.~B.
der Ausdruck \fu{!93} den Wert \fu{0}.

\begin{operator}
\fu{`\char126'} \tint\ & \tint\
\end{operator}
\index{NOT!bitweises|ii}
\index{NOT!numerisches|see{Negation}}

Der Operator \fu{\char126} errechnet die bitweise Negation seines
Operanden, das hei"st, er wendet auf jedes Bit desselben die
logische Negation an. Das Ergebnis kann je nach Wortl"ange und
Darstellung des Typs \tint\ stark differenzieren. Im Regelfall
kann jedoch mit einer Wortl"ange von 32 Bits und einer
Einerkomplementdarstellung der negativen Zahlen gerechnet werden.
So ergibt zum Beispiel der Ausdruck \fu{\char126 93} auf MS-DOS-
und Windows NT-Rechnern den Wert \fu{-94}.

\begin{operator}
\tchar\ \fu{`=='} \tchar\ & \tint\ \\
\tint\ \fu{`=='} \tint\ & \tint\ \\
\tfloat\ \fu{`=='} \tfloat\ & \tint\
\end{operator}
\index{Vergleich|ii}

Der Operator \fu{==} "uberpr"uft die beiden Operanden auf
Gleichheit, das hei"st, der Ausdruck ergibt einen Wert ungleich
\fu{0}, falls der linke und der rechte Operand den selben Wert
haben, andernfalls \fu{0}.

\begin{operator}
\tchar\ \fu{`<'} \tchar\ & \tint\ \\
\tint\ \fu{`<'} \tint\ & \tint\ \\
\tfloat\ \fu{`<'} \tfloat\ & \tint\
\end{operator}

Der Operator \fu{<} "uberpr"uft, ob der linke Operand kleiner als
der rechte ist. Falls dies zutrifft, ergibt der Ausdruck einen
Wert ungleich \fu{0}, ansonsten \fu{0}.

\begin{operator}
\tchar\ \fu{`>'} \tchar\ & \tint\ \\
\tint\ \fu{`>'} \tint\ & \tint\ \\
\tfloat\ \fu{`>'} \tfloat\ & \tint\
\end{operator}

Der Operator \fu{>} "uberpr"uft, ob der linke Operand gr"o"ser
ist als der rechte. Trifft das zu, ergibt der Ausdruck einen Wert
ungleich \fu{0}, andernfalls \fu{0}.

\begin{operator}
\tchar\ \fu{`<='} \tchar\ & \tint\ \\
\tint\ \fu{`<='} \tint\ & \tint\ \\
\tfloat\ \fu{`<='} \tfloat\ & \tint\
\end{operator}

Der Operator \fu{<=} "uberpr"uft, ob der linke Operand kleiner
oder gleich dem rechten Operanden ist und ergibt in diesem Fall
einen Wert ungleich \fu{0}, ansonsten \fu{0}.

\begin{operator}
\tchar\ \fu{`>='} \tchar\ & \tint\ \\
\tint\ \fu{`>='} \tint\ & \tint\ \\
\tfloat\ \fu{`>='} \tfloat\ & \tint\
\end{operator}

Der Operator \fu{>=} "uberpr"uft, ob der linke Operand gr"o"ser
oder gleich dem rechten Operanden ist und ergibt in diesem Fall
einen Wert ungleich \fu{0}, andernfalls \fu{0}.

\begin{operator}
\tchar\ \fu{`!='} \tchar\ & \tint\ \\
\tint\ \fu{`!='} \tint\ & \tint\ \\
\tfloat\ \fu{`!='} \tfloat\ & \tint\
\end{operator}

Der Operator \fu{!=} "uberpr"uft die beiden Operanden auf
Ungleichheit und ergibt in diesem Fall einen Wert ungleich \fu{0},
bei Gleichheit ergibt sich der Wert \fu{0}.

\begin{operator}
\type{type} \fu{`='} \type{type} & \type{type}
\end{operator}
\index{Zuweisung|ii}

Der Operator \fu{=} weist dem linken Operanden den Wert des
rechten Operanden zu, was aber vorraussetzt, da"s der linke
Operand keine \ind{Konstante}, also entweder eine \ind{Variable} (auch
innerhalb einer \ind{Struktur}), ein
dereferenzierter \ind{Zeiger} oder ein indiziertes \ind{Array} ist. SEL ist
so konzipiert, da"s auch ganze Arrays oder Strukturen mit einer
Zuweisung kopiert werden k"onnen. Beachten Sie aber bei der
Arbeit mit Zeigern, da"s eine Zuweisung nicht bedeutet, da"s die
Werte, auf die die Zeiger zeigen, sondern die Adressen, auf die
die Zeiger verweisen, kopiert werden. Zur Veranschaulichung ein
Beispiel: Angenommen, die Variablen \fu{p} und \fu{q} sind vom
Typ \type{\poi int} und zeigen auf die \tint s \fu{i} und \fu{j},
soda"s \fu{*p} gleich \fu{i} und \fu{*q} gleich \fu{j} ist. In
diesem Fall bewirkt der Ausdruck \fu{p = q} nicht, da"s nun der
Wert von \fu{i} gleich dem von \fu{j} ist, sondern da"s nun
\fu{p}, genauso wie \fu{q}, auf \fu{j} zeigt. Ersterer Effekt
w"are mit dem Ausdruck \fu{*p = *q} zu erzielen.

\begin{operator}
\type{struct} or \type{\poi struct} \fu{`->'} \fu{IDENTIFIER} & \type{type}
\end{operator}
\fuindex{IDENTIFIER}
\index{Struktur!Zugriff auf Element|ii}

Der Operator \fu{->} wird benutzt, um ein Strukturelement zu
erreichen. Um dies zu vereinfachen, kann der Operator sowohl
direkt mit einer Struktur als auch mit einem \ind{Zeiger} auf eine
Struktur arbeiten. Der rechte Operand gibt den Namen des Elements
an, auf das zugegriffen werden soll.

\begin{operator}
\type{\poi type} or \type{\arr{}type} \fu{`['} \fu{expr} \fu{`]'} & \type{type}
\end{operator}
\fuindex{expr}
\index{Array!Zugriff auf Element|ii}

Der \fu{[]}-Operator wird benutzt, um auf ein einzelnes
Element eines Arrays zuzugreifen, wobei dieses Array auch als
\ind{Zeiger} dargestellt werden kann. Auf diese Weise kann auch auf
einen Speicherbereich, der nicht als Array definiert worden ist,
wie auf ein Array zugegriffen werden. Hierzu wird in eckigen
Klammern ein Index angegeben, wobei das erste Element des Arrays
den Index \fu{0} besitzt und negative Indizes nicht erlaubt sind.
Anmerkung: Der Ausdruck \fu{p[i]} kann auch als \fu{*(p + i)}
angeschrieben werden, wobei \fu{p} ein Array oder ein Zeiger
eines beliebigen Typs ist und \fu{i} ein positiver \tint .
Beachten Sie, da"s SEL keine "Uberpr"ufung des Index vornimmt,
das hei"st, da"s es m"oglich ist, auf Elemente zuzugreifen, die
eigentlich nicht mehr zum Array geh"oren, wovon jedoch in den
meisten F"allen abzuraten ist.


\subsubsection{Rangfolge der Operatoren}
\index{Operator!Rangfolge|ii}
\index{Operator!Priorit""at|ii}

Um eine unn"otige Klammerung zu vermeiden, besitzen die
Operatoren in SEL eine Rangfolge. Die arithmetischen Operatoren
sind an die der Mathematik angelehnt, soda"s zum Beispiel der
Ausdruck \fu{a+b*c-d} dem Ausdruck \fu{(a+(b*c))-d} gleichkommt,
was der mathematischen Notation entspricht. Ist allerdings
gew"unscht, da"s \fu{a+b} mit \fu{c-d} multipliziert wird, mu"s
die nat"urliche Operatorenrangfolge mittels Klammerung
umgangen und der Ausdruck zu \fu{(a+b)*(c-d)} umgeformt
werden.

Die folgende Tabelle ist eine Aufstellung aller Operatoren von
SEL, wobei Operatoren mit gleicher Priorit"at in jeweils der
gleichen Zeile stehen. Die Operatoren, die in der Liste weiter
oben stehen, haben niedrigere Priorit"at, das hei"st, sie binden
schw"acher. Die rechte Spalte der Tabelle gibt die Richtung
der Auswertung
\index{Operator!Richtung der Auswertung|ii}
an und bezieht sich auf alle Operatoren in der jeweiligen Zeile.

\medskip
\begin{tabular}{ll}
\verb?=?, \verb?delete? & von rechts nach links \\
\verb?&&?, \verb?||?, \verb?^^? & von links nach rechts \\
\verb?==?, \verb?<?, \verb?>?, \verb?<=?, \verb?>=?, \verb?!=? &
  von links nach rechts\\
\verb?&?, \verb?|?, \verb?^?, \verb?<<?, \verb?>>? &
  von links nach rechts \\
\verb?+?, \verb?-? & von links nach rechts \\
\verb?*?, \verb?/?, \verb?%? & von links nach rechts \\
\verb?!?, \verb?~?, \verb?-? (un"ar), \verb?*? (un"ar), \verb?&? (un"ar) &
  von rechts nach links \\
\verb?->?, \verb?:?, \verb?[]? & von links nach rechts \\
\verb?sizeof?, \verb?new? & ---
\end{tabular}

\section{Typen}
\index{Typ|ii}
\fuindexii{type}
\rwindex{char}
\rwindex{int}
\rwindex{float}
\rwindex{void}
\fuindex{array}
\fuindex{pointer}
\fuindex{IDENTIFIER}

\begin{verbatim}
type:          `char'
             | `int'
             | `float'
             | `void'
             | array
             | pointer
             | IDENTIFIER
             ;
\end{verbatim}

Typen werden in SEL verwendet, um Werte
\index{Wert}
zu klassifizieren. So hat
z.~B. jeder \ind{Ausdruck}, jede \ind{Variable} und jede \ind{Funktion}
einen Typ. Die vier Basistypen von SEL k"onnen durch Kombination
zur Konstruktion weiterer, komplexerer Typen verwendet werden.

\subsection{Basistypen}
\index{Typ!Basistyp}
\typeindexii{int}
\typeindexii{char}
\typeindexii{float}
\typeindexii{void}

Die Programmiersprache SEL kennt vier Basistypen, n"amlich
\tchar , \tint , \tfloat\ und \tvoid . Die ersten beiden Typen,
\tchar\ und \tint\ sind sogenannte Aufz"ahlungstypen,
was sich mit den ganzen Zahlen in der Mathematik vergleichen l"a"st.

\tchar\ wird haupts"achlich verwendet, um \indi{Zeichen} zu
repr"asentieren und hat die Gr"o"se eines Bytes. Obwohl \tchar\
Zeichen repr"asentiert, k"onnen auf ihn alle arithmetischen
Operationen angewendet werden, die auch mit \tint\ funktionieren.

Letzterer wird meistens verwendet, um ganzzahlige Berechnungen
\index{Zahl!ganze|ii}
auszuf"uhren. Aus diesem Grund mu"s er gen"ugend gro"s
dimensioniert sein. In der Regel entspricht ein \tint\ 32 Bits,
was bedeutet, da"s er einen Wertebereich von $-2^{31}$ bis
$+2^{31}-1$ hat. Au"serdem wird \tint\ in SEL zur boolee'schen
Arithmetik
\index{boolee'sche Arithmetik}
herangezogen, also zur Berechnung von logischen
Aussagen. In diesem Fall bedeutet ein Wert von \fu{0} die Aussage
"`logisch falsch"', ein Wert ungleich \fu{0} bedeutet "`logisch
wahr"'.

Der Typ \tfloat\ repr"asentiert rationale Zahlen
\index{Zahl!rationale|ii}
mit begrenzter Genauigkeit. Auf MS-DOS-Rechnern hat er eine Gr"o"se
von 32 Bits.

Eine besondere Stellung unter den Typen nimmt \tvoid\ ein. Er
repr"asentiert nichts und es ist auch nicht m"oglich, einem
\tvoid\ einen Wert zuzuweisen, was auch gar keinen Sinn h"atte.
\tvoid\ wird z.~B. verwendet, um Funktionen zu
deklarieren, die keinen R"uckgabewert
\index{R""uckgabewert}
haben. Au"serdem kann er
in einer Funktion verwendet werden, die einen \ind{Zeiger} unbekannten
Typs verlangt oder zur"uckgibt.

\subsection{Arrays}
\index{Array|ii}
\fuindexii{array}
\fuindex{INT\protect{\_CONSTANT}}
\fuindex{type}

\begin{verbatim}
array:         `[' INT_CONSTANT `]' type
             ;
\end{verbatim}

Ein Array ist eine Zusammenfassung einer bestimmten Anzahl von
Variablen gleichen Typs, die im Speicher als eine Folge von
Variablen
\index{Variable!Folge|ii}
repr"asentiert wird, wobei die Variablen mit kleinerem
Index an einer Stelle geringerer Adresse angesiedelt sind.

In SEL ist es m"oglich, Arrays beliebig zu schachteln, also
mehrdimensionale Array zu erzeugen. So ist etwa
\type{\arr{5}\arr{10}int} ein Array, das aus 5 Arrays besteht,
wobei jedes dieser Arrays aus 10 \tint s besteht. Um auf das
zweite Element aus dem dritten Array zuzugreifen, wird der
Ausdruck \fu{a[3][2]} verwendet. Anmerkung: Dieses Array
beansprucht genausoviel Speicher wie $5 \cdot 10 = 50$ \tint s.

Arrays k"onnen nat"urlich nicht nur aus Elementartypen
konstruiert werden, sondern aus jedem beliebigen Typ, also z.~B.
aus Strukturen. Der Nachteil jedoch ist, da"s die
Gr"o"se des Arrays schon bei der Deklaration feststehen mu"s, sie
kann also nicht bei Bedarf ge"andert werden. Beachten Sie, da"s
SEL beim Zugriff auf Array-Elemente nicht "uberpr"uft, ob das
angegebene Element auch wirklich zum Array geh"ort, also ob das
Array gen"ugend gro"s dimensioniert ist.

\subsection{Zeiger}
\index{Zeiger|ii}
\fuindexii{pointer}
\fuindex{type}

\begin{verbatim}
pointer:       `*' type
             ;
\end{verbatim}

Ein Zeiger ist eine Variable, die die Adresse
\index{Variable!Adresse}
einer anderen
Variablen enth"alt, die von dem Typ ist, der dem \fu{*} in der
Typendeklaration folgt, und hat in der Regel dieselbe Gr"o"se wie
ein \tint . Bis auf die \ind{Zuweisung} sind alle Operationen
funktionsgleich mit der eines Arrays.
\index{Zeiger!""Aquivalenz zu Array}
\index{Array!""Aquivalenz zu Zeiger}

Zeiger k"onnen, wie Arrays,
beliebig geschachtelt werden, das hei"st, da"s es ohne weiteres
m"oglich ist, einen Zeiger auf einen Zeiger zu deklarieren und zu
benutzen. Beachten Sie allerdings, da"s ein zweidimensionales
Array nicht mit einem Zeiger auf einen Zeiger gleichgesetzt
werden kann. So ist z.~B. ein Array \type{\arr{10}int} in
der Verwendung mit einem Zeiger \type{\poi int} gleich (bis auf
die Ausnahme der Zuweisungsoperation), jedoch ist ein Array
\type{\arr{5}\arr{10}int} nicht gleich einem Zeiger \type{\poi
\poi int}, da der Typ, aus dem das Array besteht und auf den der
Zeiger zeigt, zu diesem Zweck gleich sein m"u"sten. In diesem Fall
ist jedoch der Typ, aus dem das Array aufgebaut ist, wieder ein
Array, n"amlich \type{\arr{10}int}. Der Typ, auf den der Zeiger
zeigt, ist jedoch \type{\poi int}. Um ein zweidimensionales Array
wie oben angeschrieben als Zeiger zu repr"asentieren, mu"s dieser
den Typ \type{\poi \arr{10}int} haben.

\subsubsection{Die new- und delete-Operatoren}
\label{newdelete}
\rwindexii{new}
\rwindexii{delete}

Um eine \ind{dynamische Allokation} von Variablen zu erm"oglichen,
existieren in SEL die Operatoren \rw{new} und \rw{delete}. Es
folgt eine kurze Beschreibung:

\begin{operator}
\fu{`new'} \fu{type} & \type{\poi type}
\end{operator}

Das Ergebnis der \rw{new}-Operation ist ein Zeiger auf eine
Variable des Typs, der als Operand angegeben wird. Diese Variable
wird dynamisch allokiert, das hei"st, sie wird dem \indi{Heap}
entnommen. Ist dieser schon so voll, da"s eine Variable des
angegebenen Typs keinen Platz mehr finden w"urde, ist das
Ergebnis ein \ind{Nullzeiger}. Beachten Sie, da"s der Speicherbereich
so lange belegt bleibt, bis \rw{delete} verwendet wird, um ihn
freizugeben, da SEL keine "`garbage collection"' benutzt.
Anmerkung: Um einen Speicherbereich zu allokieren, dessen Gr"o"se
nicht schon vor der Ausf"uhrung des Programms feststeht, ist die Funktion
\fui{MemAlloc} zu benutzen. Weiters ist es mittels der Funktion
\fui{MemReAlloc} m"oglich, einen bereits
allokierten Block zu vergr"o"sern bzw. zu verkleinern, wobei der
Inhalt des Blocks unver"andert bleibt (bei einer Verkleinerung
geht nat"urlich ein Teil des Inhaltes verloren).

\begin{operator}
\fu{`delete'} \type{\poi type} & \tvoid
\end{operator}

Der \rw{delete}-Operator gibt einen Speicherblock frei, der
vorher mit \rw{new} oder \fui{MemAlloc} allokiert wurde. Beachten
Sie, da"s dem Operator genau der Zeiger als Operand angeben
werden mu"s, den \rw{new} oder \fu{MemAlloc} (oder im Falle einer
Ver"anderung der Gr"o"se des Speicherbereichs \fui{MemReAlloc})
geliefert hat, da sonst die Speicherverwaltung ins Wanken kommen
kann, was in einem Programmabsturz resultieren k"onnte.

\subsection{Strukturen}
\index{Struktur|ii}
\fuindexii{struct}
\rwindexii{struct}
\fuindex{IDENTIFIER}
\fuindex{type}
\rwindex{end}

\begin{verbatim}
struct:        `struct' IDENTIFIER
                 {IDENTIFIER `:' type `;'}
               `end'
             ;
\end{verbatim}

Eine Struktur ist eine Zusammenfassung mehrerer Variablen
\index{Variable!Zusammenfassung mehrerer|ii}
beliebigen Typs
\index{Typ}
unter einem gemeinsamen Namen. Mittels des
Operators \fu{->} kann auf ein einzelnes Element der Struktur
zugegriffen werden.
\index{Struktur!Zugriff auf Element}
Dem \rw{struct}-Schl"usselwort folgt der
zuk"unftige Name der Struktur---bei der Deklaration von Variablen
wird dieser Name als Typ verwendet. Danach werden die einzelnen
Elemente der Struktur definiert. Die Namen dieser Elemente d"urfen
sich ohne weiteres mit anderen Variablennamen des Programms
decken, da durch den Operator \fu{->} eine eindeutige
Identifikation gew"ahrt ist, allerdings m"ussen sie innerhalb der
Struktur eindeutig sein. Die Elemente d"urfen beliebige Typen
haben, es ist sogar erlaubt, innerhalb einer Struktur einen
Zeiger auf ebendiese Struktur zu deklarieren, nicht aber ein
Array aus der Struktur und auch nicht die Struktur selbst, da
diese Deklarationen die Gr"o"se der Struktur ins Unendliche
anwachsen lie"sen. Die Gr"o"se einer Struktur ist in der Regel
gleich der Summe der Gr"o"sen ihrer Elemente.

Beachten Sie, da"s die Definition einer Struktur nur in dem \ind{Modul}
bekannt ist, in der sie get"atigt wird, selbst, wenn sie
innerhalb eines \rw{global}-Blocks steht. In der Regel ist es
deshalb zweckm"a"sig, Strukturen, die von mehreren Modulen
verwendet werden, in Include-Dateien
\index{Include-Datei}
zu definieren und diese in den
jeweiligen Modulen zu inkludieren.

\example
Die folgende Struktur ist ein Element einer verketteten \indu{Liste},
deren Daten die Adressen von Personen sind. Das Element \fu{Next}
zeigt auf das jeweils n"achste Element der Liste.

\index{Adresse@{\tt Adresse}~(\rw{struct})|ii}
\begin{verbatim}
struct Adresse
  Name    : char[26];
  Vorname : char[31];
  PLZ     : int;
  Stadt   : char[36];
  Strasse : char[36];
  Next    : *Adresse;
end
\end{verbatim}

\subsection{Zusammengesetzte Datentypen}
\index{Typ!zusammengesetzter|ii}

Nat"urlich ist es in SEL auch m"oglich, Datentypen aus den oben
besprochenen zu kombinieren, also zum Beispiel ein Array aus
Strukturen oder eine Struktur, die Arrays oder Zeiger beinhaltet,
zu definieren. Zu beachten ist nur, da"s die vorhandene
Speicherkapazit"at nicht "uberschritten wird. Unter MS-DOS
kommt weiterhin die 64~KByte-Barriere hinzu, die aus der
Segmentierung des Speichers im Real-Mode der 80x86-Prozessoren
resultiert. Deshalb ist es nicht m"oglich, in MS-DOS Variablen zu
definieren, die gr"o"ser als 64~KByte w"aren (sehr wohl ist es
m"oglich, Typen dieser Gr"o"se zu definieren, solange keine
Variablen dieses Typs verwendet werden).

\subsection{Der sizeof-Operator}
\index{Typ!Gr""o""se}
\rwindexii{sizeof}

\begin{operator}
\fu{`sizeof'} \type{type} & \tint\
\end{operator}

Der \rw{sizeof}-Operator verlangt als einzigen Operanden einen
Typ und gibt dessen Gr"o"se in Bytes zur"uck. Diese kann f"ur
ein und denselben Typen je nach Rechner verschieden sein, in der
Regel ist das jedoch nicht der Fall. Trotzdem sollten sie immer
\rw{sizeof} verwenden, wenn es um die Gr"o"se eines Typs geht und
nicht von einer bestimmten Gr"o"se ausgehen. Anmerkung: Der
einzige Typ, dessen Gr"o"se auf allen Rechnern gleich ist, ist
\tvoid\
\typeindex{void}
(nicht jedoch zum Beispiel \type{\poi void}). Seine
Gr"o"se betr"agt \fu{0}.

\subsection{Typecasting}
\index{Typecasting|ii}
\index{Typ!Umwandlung|see{Typecasting}}
\fuindex{type}

\begin{operator}
\type{\_type} \fu{`:'} \fu{type} & \type{type}
\end{operator}

Um Ausdr"ucke eines Typs in einen anderen umzuwandeln, bietet SEL
Typecasting. Der Operator \fu{:} wandelt seinen linken Operanden
in den Typ um, den der rechte Operand angibt. Dabei k"onnen
folgende Umwandlungen durchgef"uhrt werden:

\medskip
\begin{tabular}{ll}
Ausgangstyp & Zieltyp \\[0.5ex]
\tchar\ & \tint , \tfloat\ \\
\tint\ & \tchar , \tfloat , \type{\poi type}, \type{\arr{}type} \\
\tfloat\ & \tchar , \tint\ \\
\type{\poi type} & \tint , \type{\arr{}type} \\
\type{\arr{}type} & \tint , \type{\poi type}
\end{tabular}
\medskip

Bei der Umwandlung von \tint s
\typeindex{int}
in Arrays
\index{Array}
und \ind{Zeiger} und
umgekehrt ist zu beachten, da"s der \tint\ die Speicheradresse
angibt, auf die der Pointer zeigt, bzw. an der das Array beginnt
(die Adresse des ersten Arrayelements).

Zus"atzlich zum expliziten Typecasting f"uhrt SEL noch implizites
Typecasting durch, falls dies n"otig ist. Verlangt also eine
Funktion einen Parameter vom Typ \tfloat ,
\typeindex{float}
und es wird ein \tint\
"ubergeben, wird dieser automatisch in einen \tfloat\
umgewandelt. Beim impliziten Typecasting gelten wie beim
expliziten Typecasting die oben beschriebenen Regeln.

\section{Bezeichner}
\label{identifiers}
\index{Bezeichner|ii}
\fuindexii{IDENTIFIER}

\begin{verbatim}
IDENTIFIER: [A-Za-z_][A-Za-z0-9_]*
\end{verbatim}

Bezeichner werden in SEL f"ur Funktions-, Variablen- und
Strukturnamen
\index{Funktion!Name|see{Bezeichner}}
\index{Variable!Name|see{Bezeichner}}
\index{Struktur!Name|see{Bezeichner}}
verwendet. Ein Bezeichner mu"s mit einem Buchstaben
oder einem Unterstrich beginnen, danach folgt eine beliebige
Anzahl von Buchstaben, Ziffern oder Unterstrichen. So ist z. B.
\fu{\_xYz1} ein g"ultiger Bezeichner, nicht aber
\fu{1xy}, da er mit einer Ziffer beginnt. Die Bezeichner in SEL
sind {\em case-sensitiv}, das hei"st, es wird zwischen Gro"s-
und Kleinschreibung unterschieden. So ist z. B. \fu{counter} ein
anderer Bezeichner als \fu{Counter} oder \fu{COUNTER}. Beachten
Sie aber, da"s sie keines der reservierten W"orter
\index{reserviertes Wort|ii}
\index{Schl""usselwort|see{reserviertes Wort}}
als einen Bezeichner verwenden d"urfen. Der Vollst"andigkeit halber hier
eine Liste aller reservierten W"orter von SEL:

\medskip
\begin{tabular}{lllllll}
\bf begin & \bf break  & \bf case   & \bf char  & \bf default & \bf delete   & \bf downto \\
\bf else  & \bf end    & \bf extern & \bf float & \bf for     & \bf function & \bf global \\
\bf if    & \bf int    & \bf local  & \bf new   & \bf repeat  & \bf return   & \bf sizeof \\
\bf step  & \bf struct & \bf switch & \bf to    & \bf until   & \bf void     & \bf while  \\
\end{tabular}

\chapter{Der Pr\protect{"aprozessor}}
\index{Pr""aprozessor|ii}

Bevor ein SEL-Programm interpretiert bzw. compiliert wird, wird
es durch den Pr"aprozessor bearbeitet. Seine Hauptaufgaben sind
das Entfernen von Kommentaren, das Einf"ugen von Include-Dateien
und das Expandieren von Makros.

\section{Kommentare}
\index{Kommentar|ii}

Kommentare werden in SEL durch das Zeichen \fu{//} eingeleitet
und dauern bis an das Ende der jeweiligen Zeile an. Sollen also
l"angere Kommentare in den Quelltext eingef"ugt werden, mu"s am
Anfang jeder Kommentarzeile \fu{//} stehen. Beachten Sie bitte,
da"s \fu{//}, eingeschlossen in doppelte Anf"uhrungszeichen, nicht
den Anfang eines Kommentars, sondern eine String-Konstante
\index{Konstante!String}
darstellt.

\example
\fuindexii{Demo}
\fuindexuu{WriteInt}
\fuindexuu{WriteString}
\begin{verbatim}
function Demo (void) : void
begin
  WriteInt(20);        // Das ist ein Kommentar
  WriteString("Das // ist keiner\n");
end
\end{verbatim}

\section{Die \#define-Direktive}
\fuindexii{\#define}

Mittels der \fu{\#define}-Direktive ist es in SEL m"oglich,
Makros
\index{Makro|ii}
zu definieren. Ein Beispiel:

\begin{verbatim}
#define MACRO        3+5
\end{verbatim}

Tritt nun im nachfolgenden Quellcode irgendwo (au"ser innerhalb
einer String-Konstanten
\index{Konstante!String}
oder eines Kommentars) der Bezeichner
\fu{MACRO} auf, wird dieser vom Pr"aprozessor durch \fu{3+5}
ersetzt. Bei Makros wie dem obigen ist zu beachten, da"s ein
Makro nicht automatisch geklammert wird, d.~h., da"s der Ausdruck
\fu{MACRO*2} nicht das Ergebnis \fu{16}, sondern \fu{13}
(\fu{3+5*2}) hat.

Ist ein Makrotext zu lang, um ihn in einer Zeile unterzubringen,
kann er mit \fu{\char92}
\index{Backslash (\fu{\char92})|ii}
abgeteilt werden:

\begin{verbatim}
#define XYZ          if (x == y)     \
                       WriteInt(z);  \
                     end
\end{verbatim}

Beachten Sie, da"s die \fu{\#define}-Direktive immer am Beginn
einer Zeile stehen mu"s, und in dieser Zeile au"ser der Direktive, dem
Makronamen und dem einzusetzenden Text sonst nichts.

Makros mit Parametern, wie etwa in C, sind in der momentanen
Version von SEL nicht vorhanden, sind jedoch f"ur eine kommende
Version geplant.

\section{Die \#include-Direktive}
\index{Include-Datei|ii}
\fuindexii{\#include}

Mit der \fu{\#include}-Direktive ist es m"oglich, in eine
Quelldatei eine andere einzuf"ugen. Dabei mu"s auf \fu{\#include}
ein in Anf"uhrungszeichen gesetzter Dateiname folgen. Ebenso wie
die \fu{\#define}-Direktive mu"s auch die
\fu{\#include}-Direktive alleine am Anfang einer Zeile stehen.

\example
\fuindexuu{WriteString}
\begin{verbatim}
#include "conio.sh"

function main () : void
begin
  WriteString("Test\n");
end
\end{verbatim}

Hier wird an der Stelle der \fu{\#include}-Direktive der Inhalt
der Datei \fu{`conio.sh'} eingef"ugt. Sollte diese Datei selbst
\fu{\#include}-Direktiven enthalten, werden auch diese vom
Pr"aprozessor ausgef"uhrt, es ist also m"oglich,
\fu{\#include}-Direktiven zu verschachteln.

F"ur alle
Bibliotheksfunktionen von SEL gibt es Include-Dateien, die
deren Funktionsdeklarationen enthalten, soda"s diese nicht
jedesmal von Hand eingef"ugt werden m"ussen. Die Namen dieser
Dateien sind f"ur jede Funktion in der Funktionsreferenz
angegeben.

\chapter{Funktionsreferenz}
\label{funcref}

%%%%%%%%%%%%%%%%%%%%%%%%%%%%%%%%%%%%%%%%%%%%%%%%%

\begin{function}{MemAlloc}

\syntax
\#include "memory.sh"
function MemAlloc (s : int) : *void
\endgroup

\parameters
\param{s}{\tint} Gr"o"se des zu allokierenden Speicherbereichs \cr

\usage
Die Funktion \fu{MemAlloc} allokiert
\index{dynamische Allokation}
einen Speicherbereich mit
der Gr"o"se \fu{s} Bytes. Ist eine solche Allokation wegen
Speichermangels nicht m"oglich, gibt die Funktion einen
\ind{Nullzeiger} zur"uck. Im Gegensatz zum Operator \rwi{new} ist es mit
\fu{MemAlloc} auch m"oglich, einen Speicherbereich zu allokieren,
dessen Gr"o"se erst bei Laufzeit bestimmt wird. Der allokierte
Speicherbereich sollte, wenn er nicht mehr ben"otigt wird, mit
der Funktion \fu{MemFree} freigegeben werden.

\fseealso
\fu{MemFree}, \rw{new} (Kapitel \ref{newdelete})
\end{function}

%%%%%%%%%%%%%%%%%%%%%%%%%%%%%%%%%%%%%%%%%%%%%%%%%

\begin{function}{MemAvail}

\syntax
\#include "memory.sh"
function MemAvail () : int
\endgroup

\usage
Die Funktion \fu{MemAvail} gibt die Anzahl der noch freien Bytes
auf dem \ind{Heap} zur"uck, also jenes Speicherbereichs, der vom
Operator \rwi{new} (siehe Kapitel \ref{newdelete}) und der
Funktion \fu{MemAlloc} benutzt wird, um Speicher zu allokieren.

\fseealso
\fu{MemAlloc}, \rw{new} (Kapitel \ref{newdelete})
\end{function}

%%%%%%%%%%%%%%%%%%%%%%%%%%%%%%%%%%%%%%%%%%%%%%%%%
\newpage

\begin{function}{MemFree}

\syntax
\#include "memory.sh"
function MemFree (p : *void) : void
\endgroup

\parameters
\param{p}{\type{\poi void}} freizugebender Speicherblock \cr

\usage
Die Funktion \fu{MemFree} gibt einen Speicherblock, der mit
\fu{MemAlloc}, \fu{MemReAlloc} oder \rwi{new} allokiert worden
ist, frei. Beachten Sie jedoch, da"s Speicherbl"ocke, die mit
\rw{new} allokiert worden sind, besser mit \rwi{delete}
freigegeben werden sollten.

\fseealso
\fu{MemAlloc}, \fu{MemReAlloc}, \rw{delete}, \rw{new} (Kapitel \ref{newdelete})
\end{function}

%%%%%%%%%%%%%%%%%%%%%%%%%%%%%%%%%%%%%%%%%%%%%%%%%

\begin{function}{MemReAlloc}

\syntax
\#include "memory.sh"
function MemReAlloc (p : *void, s : int) : *void
\endgroup

\parameters
\param{p}{\type{\poi void}} allokierter Speicherblock \cr
\param{s}{\tint} neue Gr"o"se \cr

\usage
Die Funktion \fu{MemReAlloc} "andert die Gr"o"se des
Speicherblocks \fu{p},
\index{dynamische Allokation}
der mit \fu{MemAlloc}, \fu{MemReAlloc}
oder \rwi{new} allokiert worden sein mu"s, auf die Gr"o"se \fu{s}
Bytes. Ist \fu{p} ein Nullzeiger, wird ein Speicherbereich der
Gr"o"se \fu{s} wie mit \fu{MemAlloc} neu angelegt. Ist \fu{s}
gleich \fu{0}, dann wird der Speicherblock wie mit \fu{MemFree}
freigegeben. Andernfalls wird seine Gr"o"se ver"andert, wobei
sein Inhalt soweit wie m"oglich bestehen bleibt. Ist es aus
Speichermangel nicht m"oglich, die Gr"o"se des Blocks zu "andern,
gibt die Funktion einen \ind{Nullzeiger} zur"uck, und der alte Block
bleibt bestehen. Andernfalls gibt die Funktion einen Zeiger auf
den neuen Block zur"uck.

\fseealso
\fu{MemAlloc}, \fu{MemFree}, \rw{new} (Kapitel \ref{newdelete})
\end{function}

%%%%%%%%%%%%%%%%%%%%%%%%%%%%%%%%%%%%%%%%%%%%%%%%%

\begin{function}{Random}
\index{Zufallszahl|ii}

\syntax
\#include "stdlib.sh"
function Random (i : int) : int
\endgroup

\parameters
\param{i}{\tint} H"ochstgrenze f"ur die Zufallszahl \cr

\usage
Die Funktion Random gibt eine Zufallszahl
$0 \leq x \leq \fu{i}$ zur"uck. Um zu gew"ahrleisten, da"s
diese Zahl nicht f"ur jeden Aufruf des Programms dieselbe ist,
mu"s der Zufallszahlengenerator mit der Funktion \fu{Randomize}
initialisiert werden.

\fseealso
\fu{Randomize}
\end{function}

%%%%%%%%%%%%%%%%%%%%%%%%%%%%%%%%%%%%%%%%%%%%%%%%%
\newpage

\begin{function}{Randomize}
\index{Zufallszahl!Initialisierung des Generators|ii}

\syntax
\#include "stdlib.sh"
function Randomize () : void
\endgroup

\usage
Die Funktion \fu{Randomize} initialisiert den
Zufallszahlengenerator von SEL. Hierf"ur wird die aktuelle
Uhrzeit verwendet. Das garantiert, da"s bei jedem Aufruf des
Programms eine andere Reihe von Zufallszahlen generiert wird.

\fseealso
\fu{Random}
\end{function}

%%%%%%%%%%%%%%%%%%%%%%%%%%%%%%%%%%%%%%%%%%%%%%%%%

\begin{function}{ReadChar}
\typeindex{char}

\syntax
\#include "conio.sh"
function ReadChar () : char
\endgroup

\usage
Die Funktion \fu{ReadChar} liest ein einzelnes Zeichen von der
Systemkonsole ein, ohne es auf dem Bildschirm auszugeben. Dazu
kann die Funktion \fu{WriteChar} verwendet werden.

\fseealso
\fu{WriteChar}
\end{function}

%%%%%%%%%%%%%%%%%%%%%%%%%%%%%%%%%%%%%%%%%%%%%%%%%

\begin{function}{ReadFloat}
\typeindex{float}

\syntax
\#include "conio.sh"
function ReadFloat () : float
\endgroup

\usage
\fu{ReadFloat} liest einen \tfloat\ in Dezimal- bzw.
\indu{Exponentialschreibweise} (z.~B. \fu{1.23E45}) von der
Systemkonsole ein. Backspaces werden richtig interpretiert,
soda"s der Anwender eine Korrekturm"oglichkeit hat.

\fseealso
\fu{WriteFloat}
\end{function}

%%%%%%%%%%%%%%%%%%%%%%%%%%%%%%%%%%%%%%%%%%%%%%%%%

\begin{function}{ReadInt}
\typeindex{int}

\syntax
\#include "conio.sh"
function ReadInt () : int
\endgroup

\usage
\fu{ReadInt} liest einen \tint\ als Dezimalzahl
\index{Zahl!dezimale}
von der
Systemkonsole ein. Dabei werden Backspaces ber"ucksichtigt,
um dem Anwender eine Korrekturm"oglichkeit zu geben. Zu beachten
ist allerdings, da"s kein positives Vorzeichen (\fu{+}) angegeben
werden darf.

\fseealso
\fu{WriteInt}
\end{function}

%%%%%%%%%%%%%%%%%%%%%%%%%%%%%%%%%%%%%%%%%%%%%%%%%
\newpage

\begin{function}{ReadString}
\index{String}

\syntax
\#include "conio.sh"
function ReadString (s : *char) : int
\endgroup

\parameters
\param{s}{\type{\poi char}} Speicherplatz f"ur den String \cr

\usage
Die Funktion \fu{ReadString} liest einen nullterminierten String
von der Systemkonsole in den Speicherbereich, den der Parameter
\fu{s} angibt, ein. Dieser Speicherbereich mu"s gen"ugend gro"s
dimensioniert sein, da \fu{ReadString} keinerlei Tests vornimmt,
ob er f"ur den String ausreichend ist. Die Stringeingabe wird vom
Anwender durch ein Newline-Zeichen beendet, das jedoch im String
nicht aufscheint. Die Funktion gibt die Anzahl der gelesenen
Zeichen, inklusive dem Newline, also mindestens \fu{1}, zur"uck.

\fseealso
\fu{WriteString}
\end{function}

%%%%%%%%%%%%%%%%%%%%%%%%%%%%%%%%%%%%%%%%%%%%%%%%%

\begin{function}{WriteChar}
\typeindex{char}

\syntax
\#include "conio.sh"
function WriteChar (c : char) : void
\endgroup

\parameters
\param{c}{\tchar} auszugebendes Zeichen \cr

\usage
Die Funktion \fu{WriteChar} gibt das ihr "ubergebene Zeichen
\fu{c} auf der Systemkonsole aus, wobei Steuerzeichen (siehe
Kapitel \ref{specialchar}) ber"ucksichtigt werden.

\fseealso
\fu{ReadChar}
\end{function}

%%%%%%%%%%%%%%%%%%%%%%%%%%%%%%%%%%%%%%%%%%%%%%%%%

\begin{function}{WriteFloat}
\typeindex{float}

\syntax
\#include "conio.sh"
function WriteFloat (f : float) : void
\endgroup

\parameters
\param{f}{\tfloat} auszugebender Wert \cr

\usage
Die Funktion \fu{WriteFloat} gibt einen \tfloat\ in passender
Schreibweise (Dezimal- oder \ind{Exponentialschreibweise}) auf der
Systemkonsole aus.

\fseealso
\fu{ReadFloat}
\end{function}

%%%%%%%%%%%%%%%%%%%%%%%%%%%%%%%%%%%%%%%%%%%%%%%%%
\newpage

\begin{function}{WriteInt}
\typeindex{int}

\syntax
\#include "conio.sh"
function WriteInt (i : int) : void
\endgroup

\parameters
\param{i}{\tint} auszugebender Wert \cr

\usage
Die Funktion \fu{WriteInt} gibt den Parameter \fu{i} als
Dezimalzahl
\index{Zahl!dezimale}
auf der Systemkonsole aus.

\fseealso
\fu{ReadInt}
\end{function}

%%%%%%%%%%%%%%%%%%%%%%%%%%%%%%%%%%%%%%%%%%%%%%%%%

\begin{function}{WriteString}
\index{String}

\syntax
\#include "conio.sh"
function WriteString (s : *char) : void
\endgroup

\parameters
\param{s}{\type{\poi char}} auszugebender String \cr

\usage
Die Funktion \fu{WriteString} gibt einen nullterminierten String
auf der Systemkonsole aus. Steuerzeichen (siehe Kapitel
\ref{specialchar}) werden dabei interpretiert.

\fseealso
\fu{ReadString}
\end{function}

%%%%%%%%%%%%%%%%%%%%%%%%%%%%%%%%%%%%%%%%%%%%%%%%%

\newpage
\index{Index|ii}
% \addcontentsline{toc}{chapter}{Index}
\footnotesize
\printindex

\end{document}


\begin{document}

\begin{titlepage}
\vspace*{2cm}
\begin{flushleft}
\inch SEL\rm \hfill Version 0.3
\end{flushleft}
\rule{\textwidth}{2mm}
\begin{flushright}
\huge{\bf Kurzreferenz}\\[0.5ex]
\end{flushright}
\vspace{13.5cm}
\begin{flushleft}
\large \bf Mark Probst\\
\rule[2mm]{\textwidth}{1mm}
\end{flushleft}
\end{titlepage}

\tableofcontents

\newpage

\chapter{\protect{"Uber} diesen Text}

Dieser Text hat zum Ziel, die Programmiersprache SEL (Schani
Electronics Language) in ihrer momentanen Version 0.3
vollst"andig zu beschreiben. Dabei ist er als Kurzreferenz
aufgebaut, kann aber auch zur Not verwendet werden, um die
Sprache SEL kennenzulernen, vor allem von Programmierern, die
bereits {\sc Pascal}
\index{Pascal@{\sc Pascal}}
oder \ind{C}
kennen, da SEL, verglichen mit diesen Sprachen, kaum etwas Neues
bietet.

Die einzigen Implementationen von SEL, die zur Zeit verf"ugbar
sind, sind die beiden \ind{Interpreter} unter MS-DOS und Windows
NT. Die Funktionsreferenz (Kapitel \ref{funcref}) spiegelt deren
momentanen Stand in Bezug auf verf"ugbare Funktionen wider.

\section{Typographie}
\index{Typographie|ii}

Um Mi"sverst"andnisse zu vermeiden, werden in diesem Text
verschiedene Schriftarten f"ur verschiedene Zwecke verwendet.
Durch eine {\em kursive Schriftart} werden neue Begriffe
gekennzeichnet und wichtige Textabschnitte hervorgehoben.

Schl"usselw"orter der Sprache SEL werden in {\bf fetter Schrift}
geschrieben. Eine Aufstellung aller Schl"usselw"orter finden Sie
in Kapitel \ref{identifiers}. Operatoren hingegen werden in
{\tt Schreibmaschinenschrift} geschrieben.

Typen werden nicht nur in fetter Schrift geschrieben, sondern
zus"atzlich in geschwungene \type{Klammern} gesetzt, um sie von
Schl"usselw"ortern unterscheiden zu k"onnen.

Bezeichner, Makros, Funktionsnamen und Nonterminals der
SEL-Grammatik werden ebenfalls in
{\tt Schreib"-maschi"-nen"-schrift} geschrieben.

Indexeintr"age
\index{Index|ii}
\index{Stichwortverzeichnis|see{Index}}
sind wie folgt zu interpretieren: Kursive
Eintr"age wie etwa {\em 23} deuten auf einen Haupteintrag, wo die
meisten Informationen zu dem jeweiligen Thema zu finden sind.
Unterstrichene Eintr"age wie z.~B. \underline{31} bedeuten, da"s
auf dieser Seite ein Beispiel zum Thema zu finden ist, das jedoch
nicht unbedingt n"aher erl"autert wird.

\section{Die Backus-Naur-Form}
\index{Backus-Naur-Form|ii}
\index{BNF|see{Backus-Naur-Form}}

Die Backus-Naur-Form (kurz BNF) wird verwendet, um Grammatiken
darzustellen. Dabei besteht jede \indi{Grammatik} aus mehreren
Symbolen, die ihrerseits jeweils aus mindestens einer
\indi{Produktion} bestehen. Symbole werden links angeschrieben.
Symbole rechts davon, durch einen Doppelpunkt oder ein anderes
Zeichen vom Grammatiksymbol getrennt, jeweils durch einen
vertikalen Strich (\fu{|}) voneinander getrennt. Auf die letzte
Produktion eines jeden Symbols folgt ein Semikolon (\fu{;}).

Produktionen sind Aneinanderreihungen von Terminal- und
Nichtterminalsymbolen. Nichtterminalsymbole sind jene, die auf
den linken Seiten stehen. Terminalsymbole sind solche, die in
Quelltexten der Sprache direkt vorkommen k"onnen, und solche,
die durch regul"are Ausdr"ucke definiert werden (siehe n"achstes
Unterkapitel). Erstere werden unter Hochkomma angef"uhrt (wie
etwa \fu{'new'}), w"ahrend letztere durch Gro"sschreibung
gekennzeichnet sind.

Ein Satz einer Sprache mu"s nun so konstruiert sein, da"s er
durch Ersetzen von Nichtterminalsymbolen durch bestimmte Produktionen
derselben hergeleitet werden kann.

\newpage
\example
\begin{verbatim}
satz:         subjekt praedikat
            ;
subjekt:      artikel substantiv
            ;
artikel:      `der'
            | `die'
            | `das'
            ;
substantiv:   `frau'
            | `mann'
            | `kind'
            ;
pradikat:     `geht'
            | `schlaeft'
            | `isst'
            ;
\end{verbatim}

Diese Grammatik hat f"unf Nonterminalsymbole und kann einige
primitive deutsche S"atze herleiten wie etwa "`der mann
schlaeft"'. Dabei geht man vom Symbol \fu{satz} aus und setzt
f"ur die Nonterminals \fu{subjekt} und \fu{praedikat} jeweils
eine ihrer Produktionen ein. F"ur \fu{subjekt} wird \fu{artikel}
und \fu{substantiv} eingesetzt und f"ur diese wiederum die beiden
Terminals \fu{`der'} und \fu{`mann'}. Dasselbe geschieht mit
\fu{pradikat}.

Dieses Beispiel demonstriert nicht nur, wie sich S"atze von einer
BNF herleiten lassen, sondern auch, da"s diese nur f"ur den
syntaktischen Teil der Grammatik eingesetzt werden kann. So
l"a"st sich aus obiger BNF auch der Satz "`der kind isst"'
herleiten, der semantisch inkorrekt ist. Deshalb wird der
semantische
\index{Semantik}
Teil in der Regel verbal formuliert, wie etwa in
dieser Kurzreferenz.

Zus"atzlich zu dieser Grundfunktionalit"at beherrscht die BNF
noch zwei zus"atzliche Hilfsmittel zur Formulierung von
Grammatiken: Wird ein Teil einer Produktion in geschwungene Klammern
eingeschlossen (\fu{\{\}}), dann darf dieser Teil im Quelltext
weggelassen oder beliebig oft aufgef"uhrt werden. Durch die Produktion
\fu{`a' \{`b'\} `a'} werden also alle S"atze \fu{aa},
\fu{aba}, \fu{abba}, \fu{abbba}, \ldots\ dargestellt.

Zus"atzlich k"onnen Teile von Produktionen in eckige Klammern
eingeschlossen werden, was bedeutet, da"s sie optional sind, also
im Quelltext weggelassen k"onnen, aber auf keinen Fall "ofter als
einmal auftreten d"urfen. Die Produktion \fu{`a' [`b'] `a'}
beschreibt also die S"atze \fu{aa} und \fu{aba}.

\section{Regul\protect{"are} Ausdr\protect{"ucke}}
\index{regul""arer Ausdruck|ii}

Regul"are Ausdr"ucke werden benutzt, um Zeichenfolgen
wie etwa Bezeichner oder Zahlen zu beschreiben. Dabei werden
aufeinanderfolgende Zeichen hintereinander angeschrieben.
Zus"atzlich gibt es Steuerzeichen, mit denen es m"oglich ist, die
beschriebene Zeichenfolge zu variieren. Es folgt eine
Aufstellung:

\begin{tabular}{lp{13cm}}
\fu{()} & Die beiden runden Klammern werden benutzt, um eine
          Folge von Zeichen zu klammern, sie also nur als ein Zeichen zu
          behandeln. Das macht es m"oglich, Steuerzeichen wie etwa \fu{+}
          auf eine ganze Zeichenfolge anstatt nur auf das letzte Zeichen
          anzuwenden. \\
\fu{[]} & Die beiden eckigen Klammern werden verwendet, um
          Zeichen zu gruppieren. Der ganze Klammerausdruck steht
          f"ur ein beliebiges Zeichen, das innerhalb der Klammer
          steht. Um auch gr"o"sere Gruppen anschreiben zu
          k"onnen, ist es m"oglich, innerhalb der Gruppe mittels
          \fu{-} von einem bestimmten bis zu einem anderen
          bestimmten Zeichen alle Zeichen einzubinden. So gibt
          z.~B. \fu{[0-9]} die Gruppe aller Ziffern an. Tritt
          diese Gruppe irgendwo auf, steht sie f"ur eine
          beliebige Ziffer. Wird als erstes Zeichen einer Gruppe
          \fu{\char94} angegeben, umfa"st die Gruppe all jene
          Zeichen, die nicht innerhalb der Klammern aufgef"uhrt
          sind. \\
\end{tabular}          

\begin{tabular}{lp{13cm}}
\fu{`'} & Sollen Zeichen verwendet werden, die schon als
          Steuerzeichen vergeben sind, werden sie in einfache
          Anf"uhrungszeichen eingeschlossen. \\
\fu{.}  & Der Punkt steht f"ur jedes beliebige Zeichen au"ser dem
          Zeilenwechsel \fu{\char92 n}. \\
\fu{|}  & Ein Ausdruck \fu{a|b} kann entweder durch \fu{a} oder
          durch \fu{b} ersetzt werden. Um zu vermeiden, da"s
          immer der gesamte Ausdruck mit ODER verkn"upft wird, werden
          Oder-Ausdr"ucke in Klammern gesetzt. \\
\fu{?}  & Ein Zeichen oder ein Ausdruck in Klammern, gefolgt von
          einem \fu{?}, ist optional, darf also auftreten, kann
          aber auch weggelassen werden. \\
\fu{+}  & Ein Zeichen oder ein Ausdruck in Klammern, gefolgt von
          einem \fu{+}, steht f"ur ein oder mehrere Vorkommnisse
          dieses Zeichens oder Ausdrucks. \\
\fu{*}  & Der Operator \fu{*} ist mit dem Operator \fu{+}
          identisch, au"ser in dem Fall, da"s das vorhergehende Zeichen
          bzw. der in Klammern stehende Ausdruck auch weggelassen
          werden kann. \\
\end{tabular}

\example
\begin{verbatim}
IDENTIFIER: [a-zA-Z][^\n ]*
\end{verbatim}

Dieser regul"are Ausdruck beschreibt all jene Bezeichner, die mit
einem Buchstaben beginnen, auf den beliebige Zeichen
(au"ser dem Zeilenwechsel \fu{\char92 n} und dem Leerzeichen)
folgen k"onnen. G"ultige \fu{IDENTIFIER} sind etwa \fu{ax*} und
\fu{ZX\#\%}.

\begin{verbatim}
HEX_NUMBER: 0x[0-9a-fA-F]+
\end{verbatim}

Der Ausdruck \fu{HEX\_NUMBER} beschreibt eine hexadezimale
Zahlenkonstante,
\index{Zahl!hexadezimale}
wie sie etwa in \ind{C} verwendet werden kann. Sie beginnt mit
den Zeichen \fu{0x}, auf die mindestens eine Hexadezimalziffer
(Dezimalziffer oder Gro"s- bzw. Kleinbuchstabe von \fu{A} bis
\fu{F}) folgen mu"s. Zu der unendlich gro"sen Menge der
beschriebenen Zahlen z"ahlen \fu{0xff} und \fu{0x123}.

\chapter{Sprachreferenz}

Dieses Kapitel ist eine Referenz "uber die komplette Grammatik
\index{Grammatik!von SEL|ii}
der Programmiersprache SEL. Der Aufbau ist dabei top-down-orientiert,
da es meines Erachtens nat"urlicher ist, vom Gro"sen ins Kleine
zu gehen, als umgekehrt.

Das erste Unterkapitel widmet sich deshalb der groben Struktur
eines SEL-Programms. Es wird die Modularit"at von SEL
besprochen sowie der grundlegende Aufbau einer Funktion. Weiters
wird in diesem Unterkapitel die wichtigste Funktion eines
SEL-Programms beschrieben, n"amlich die \fu{main}-Funktion.

Das zweite Unterkapitel bespricht die Kontrollstrukturen, also
die Verzweigungen und Schleifen, die die strukturierte
Programmiersprache ausmachen.

Das dritte Unterkapitel widmet sich den Ausdr"ucken oder
{\em expressions}, die den Hauptanteil an einem Programm haben. Zu
Ausdr"ucken z"ahlen zum Beispiel Zuweisungen, Vergleiche oder
arithmetische Operationen.

Im vierten Unterkapitel werden die Datentypen besprochen, die
mehr oder weniger Bestandteile der Ausdr"ucke sind.

\section{Struktur eines Programms}
SEL ist eine modulare
\index{Modul}
Programmiersprache, d.h. ein Programm kann sich (mu"s sich aber
nicht) aus mehreren Modulen---oder Quelldateien---zusammensetzen.
Zus"atzlich kann jedes Modul seinen internen Aufbau vor dem Rest
des Programms sch"utzen, um Implementationsdetails zu verbergen
(\indi{Kapselung}), damit diese jederzeit ge"andert werden
k"onnen, ohne da"s der aufrufende Programmteil dazu ver"andert
werden mu"s. Dies setzt jedoch auch ein vern"unftiges
"`\ind{Interface}"' (Schnittstelle zwischen dem Modul und dem
Programm) voraus. Es wird also in der Regel n"otig sein, vor der
Implementation einen Entwurf des Programms zu machen---einfaches
"`Drauflosprogrammieren"' wird sich schnell als Sackgasse
erweisen.

\subsection{Das Modul}
\index{Modul|ii}
\fuindexii{module}
\fuindexii{block}
\fuindex{function}
\fuindex{local}
\fuindex{global}

\begin{verbatim}
module:        {block}
             ;

block:         function
             | local
             | global
             ;
\end{verbatim}

Jedes Modul besteht aus einem oder mehreren Bl"ocken,
\index{Block|ii}
die in Funktionen, \rwi{local}- und \rwi{global}-Bl"ocke
eingeteilt werden (es ist theoretisch auch m"oglich, ein Modul zu
schreiben, das keinen einzigen Block enth"alt, doch dieses Modul
h"atte keinen anderen Zweck, als wertvollen Speicherplatz zu
verschwenden).

Funktionen sind Programmteile---oder Unterprogramme---, die von
bestimmten Stellen im Programm aufgerufen werden k"onnen. Da SEL
eine modulare Programmiersprache ist, ist es m"oglich, Funktionen
zu kapseln, sie also nur in dem Modul verf"ugbar zu machen, in
dem sie definiert wurden.

Die \rw{local}-Bl"ocke dienen dazu, Variablen,
\index{Variable!modulglobale}
Funktionen
\index{Funktion!modulglobale}
und Strukturen
\index{Struktur}
zu definieren, die nur innerhalb des jeweiligen Moduls
erreichbar sein sollen.

Die \rw{global}-Bl"ocke dienen einerseits dem Zweck, Variablen,
\index{Variable!programmglobale}
Funktionen
\index{Funktion!programmglobale}
und Strukturen zu definieren, die auch au"serhalb des Moduls
verf"ugbar sein sollen, und andererseits jene Variablen,
Strukturen und Funktionen zu deklarieren, die nicht in diesem
Modul definiert sind und trotzdem verwendet werden sollen.
Beachten Sie bitte, da"s diese in einem anderen Modul
als global definiert sein m"ussen.

\subsection{Funktionen}
\index{Funktion|ii}
\index{Block}
\fuindexii{function}
\fuindex{type}
\fuindex{statement}
\fuindex{IDENTIFIER}
\rwindex{begin}
\rwindex{end}
\fuindexii{formal\protect{\_params}}
\fuindexii{func\protect{\_var}}
\fuindex{expr}
\fuindexii{func\protect{\_proto}}
\rwindexii{function}

\begin{verbatim}
function:      `function' IDENTIFIER `(' [formal_params] `)' `:' type
                 {func_var}
               `begin'
                 {statement}
               `end'
             ;

formal_params: IDENTIFIER `:' [`&'] type [`,' formal_params]
             ;

func_var:      IDENTIFIER `:' type [`=' expr] `;'
             ;

func_proto:    `function' IDENTIFIER
               `(' [[`&'] type {`,' [`&'] type}] `)' `:' type
             ;
\end{verbatim}

Eine Funktion ist ein Unterprogramm,
\index{Unterprogramm|see{Funktion}}
also ein Programmteil, der
von anderen Stellen des Programms aufgerufen werden kann. Dabei
k"onnen einer Funktion ein oder mehrere \indi{Parameter} (deren
Typen bei der Deklaration der Funktion festgelegt werden)
"ubergeben werden. Innerhalb dieser Funktion k"onnen diese
Parameter als ganz normale Variablen angesprochen werden. Die
Namen dieser Variablen werden im Funktionskopf angegeben.

Funktionsdeklarationen k"onnen an zweierlei Stellen angetroffen
werden, n"amlich dort, wo die Funktionen definiert sind, und
in \rw{local}- und \rw{global}-Bl"ocken, wo allerdings nur der
Funktionskopf ohne Parameternamen angegeben wird. Zu beachten
ist, da"s Funktionen im Programm erst nach dem ersten Vorkommen
ihrer Deklarationen oder Definitionen bekannt sind. Eine Funktion
kann also nicht aufgerufen werden, wenn sie im Quellcode nicht
vorher deklariert oder definiert wurde.

Funktionen sind, solange sie nicht explizit in einem
\rw{global}-Block des jeweiligen Moduls aufgef"uhrt sind,
gekapselt, also auf das Modul beschr"ankt, in dem sie
definiert wurden. Trotzdem ist es m"oglich, Funktionen in einem
\rw{local}-Block des
Moduls aufzuf"uhren, f"ur den Fall, da"s eine Funktion an einer
Stelle aufgerufen werden mu"s, die vor ihrer Definition liegt.

Zur Semantik: Dem \rw{function}-Schl"usselwort folgt der Name der
Funktion, der wie jeder Bezeichner case-sensitiv ist. In Klammern
gesetzt folgen darauf die Namen und Typen der einzelnen Parameter.
Innerhalb der Funktion kann auf die Parameter mit Hilfe ihrer
Namen wie auf normale Variablen zugegriffen werden.

SEL unterst"utzt zwei Arten der Parameter"ubergabe:
\index{Parameter""ubergabe|ii}
\index{call-by-value|see{Parameter""ubergabe}}
\index{call-by-reference|see{Parameter""ubergabe}}
{\em call-by-value} und {\em call-by-reference}. Ein
Parameter, der als call-by-value deklariert ist, wird der
Funktion als Wert "ubergeben, es wird beim Funktionsaufruf also
zuerst der Wert des Parameters berechnet und dieser dann kopiert.
Die jeweilige Funktion greift dann auf die Kopie des Parameters
zu, kann also die Variable, die "ubergeben wurde, nicht
ver"andern. Von einem als call-by-reference deklarierten Parameter
(dieser ist daran zu erkennen, da"s seinem Typ im Funktionskopf
das \fu{\&}-Zeichen
\index{Referenztyp|ii}
vorangestellt ist) hingegen wird der Funktion intern die Adresse
der als Parameter "ubergebenen Variable zug"anglich gemacht,
wovon die Funktion allerdings nichts merkt. Dies wirkt sich
dahingehend aus, da"s die Funktion durch "Anderung des Wertes des
Parameters auch den Wert der Variable ver"andert, die ihr als
Parameter "ubergeben wurde. Beachten Sie allerdings, da"s es
nicht m"oglich ist, eine \ind{Konstante} f"ur einen
call-by-reference-Parameter zu "ubergeben.

Nach der Parameterliste folgt der R"uckgabetyp
\index{R""uckgabetyp|ii}
der Funktion. Das ist der Typ, den der Aufrufer der Funktion
zur"uckgeliefert bekommt. Der R"uckgabewert wird durch die
\rwi{return}-Anweisung bestimmt. Ist keine R"uckgabe n"otig, kann
der Typ \tvoid\
\typeindex{void}
verwendet werden.

Im Anschlu"s an den R"uckgabetyp k"onnen lokale
Variablen
\index{Variable!lokale|ii}
vereinbart werden. Diese Variablen sind---der Name
sagt es bereits---nur innerhalb der Funktion verf"ugbar, in der
sie deklariert wurden. Gleichzeitig k"onnen diesen Variablen
Werte zugewiesen werden. Geschieht dies nicht, sind die Werte
dieser Variablen undefiniert. Beachten Sie, da"s Variablen
auch mit berechneten Werten initialisiert werden k"onnen.

Nach der Vereinbarung der lokalen Variablen folgt in
\rwi{begin}--\rwi{end}
eingeschlossen der eigentliche Code der
Funktion. Von hier aus k"onnen wieder andere Funktionen
aufgerufen werden, die Funktion kann sich sogar selbst aufrufen
(\ind{Rekursion}).

\example
Die Funktion \fuii{Factorial} berechnet die Fakult"at
\index{Fakult""at|ii}
ihres Parameters \fu{i}
(\hbox{$\fu{i}! = \prod_{n=1}^{\fu{\footnotesize i}} n$}).
Hierzu verwendet sie zwei lokale Variablen \fu{c} und \fu{r}.

\rwindexuu{global}
\rwindexuu{for}
\begin{verbatim}
global
  function Factorial (int) : int
end

function Factorial (i : int) : int

  c : int;
  r : int = 1;

begin
  for (c = 2 to i)
    r = r * c;
  end
  return r;
end
\end{verbatim}

\subsubsection{Die main-Funktion}
Die wichtigste \ind{Funktion} in einem SEL-Programm ist die Funktion
\fuii{main}. Sie stellt gewisserma"sen das Hauptprogramm dar und
ist somit die erste Funktion, die in einem SEL-Programm aufgerufen
wird. Hat \fu{main} ihre Arbeit beendet, kehrt der Computer zum
aufrufenden Programm (in der Regel der Kommandozeileninterpreter)
zur"uck. Dabei kann letzterem ein sogenannter {\em Exit-Code}
"ubergeben werden. Dieser \indi{Exit-Code} hat den Typ \tint\
\typeindex{int}
und wird mit der \rwi{return}-Anweisung zur"uckgegeben. Ist der
R"uckgabetyp der Funktion \fu{main} \tvoid ,
\typeindex{void}
so ist der Exit-Code \fu{0}.

Zu beachten ist, da"s die \fu{main}-Funktion immer als global
deklariert werden mu"s, unerheblich, in welchem Modul sie
definiert ist.

\example
Dieses {\em hello-world}-Programm
\index{hello-world|uu}
besteht aus einem
einzigen Modul und hat die Aufgabe, den String "`\fu{hello, world!}"'
auszugeben. Zus"atzlich wird der Exit-Code \fu{1} zur"uckgeliefert.

\fuindexuu{WriteString}
\rwindexuu{global}
\begin{verbatim}
#include "conio.sh"

global
  function main () : int
end

function main () : int
begin
  WriteString("hello, world!\n");
  return 1;
end
\end{verbatim}

\subsubsection{Die return-Anweisung}
Die \rwii{return}-Anweisung hat den Zweck, aus einer \ind{Funktion} in
die "ubergeordnete Funktion (bzw. das Betriebssystem im Falle der
\fu{main}-Funktion) zur"uckzukehren. Ist der R"uckgabetyp der
jeweiligen Funktion nicht \tvoid ,
\typeindex{void}
so mu"s auch ein \ind{Wert}
zur"uckgegeben werden. Dieser wird der \rw{return}-Anweisung
nachgestellt. \rw{return} kann an einer beliebigen Stelle
innerhalb einer Funktion stehen, Schleifen
\index{Schleife} werden---falls n"otig---abgebrochen.

\example
Die Funktion \fuii{Square} gibt
das \ind{Quadrat} eines ihr "ubergebenen \tint\ \fu{i} zur"uck.

\begin{verbatim}
function Square (i : int) : int
begin
  return i * i;
end
\end{verbatim}

\subsection{local-Bl\protect{"ocke}}
\rwindexii{local}
\rwindex{end}
\index{Block}
\fuindexii{local}
\fuindexii{local\protect{\_decl}}
\fuindex{struct}
\fuindex{func\protect{\_proto}}
\fuindex{var}

\begin{verbatim}
local:         `local'
                 {local_decl}
               `end'
             ;

local_decl:    struct
             | func_proto
             | var
             ;
\end{verbatim}

Die \rw{local}-Bl"ocke dienen dazu, Variablen,
\index{Variable!modulglobale|ii}
Strukturen
\index{Struktur}
und Funktionen
\index{Funktion!modulglobale|ii}
zu definieren, die nur in dem jeweiligen Modul
bekannt sein sollen. Dieses Vorgehen nennt sich \ind{Kapselung} und hat den
Vorteil, da"s es m"oglich ist, die Implementationsdetails eines
Moduls vor dem Rest des Programms zu verstecken. Das macht es
einerseits m"oglich, das Modul anders zu implementieren, ohne
da"s das Restprogramm in irgendeiner Weise ge"andert werden mu"s
und andererseits, da"s der Code des Moduls in anderen Programmen
leichter wiederverwendet werden kann. F"ur eine solche Vorgangsweise
ist es jedoch unabdingbar, f"ur jedes Modul ein gut "uberdachtes
\ind{Interface} (Schnittstelle zwischen Modul und Restprogramm) zu
verwenden.

Variablen, Strukturen und Funktionen sind im Source erst ab der
Stelle ihrer Deklaration bekannt. Aus diesem Grunde ist es
ratsam, \rw{local}-Bl"ocke an den Anfang der Datei zu stellen,
was au"serdem die "Ubersichtlichkeit des Codes erh"oht.

\subsection{global-Bl\protect{"ocke}}
\rwindexii{global}
\rwindex{end}
\index{Block}
\fuindexii{global}
\fuindexii{global\protect{\_decl}}
\fuindex{struct}
\fuindex{func\protect{\_proto}}
\fuindex{var}
\fuindex{extern}

\begin{verbatim}
global:        `global'
                 {global_decl}
               `end'
             ;

global_decl:   struct
             | func_proto
             | var
             | extern
             ;
\end{verbatim}

Die in einem \rw{global}-Block definierten Strukturen,
\index{Struktur}
Funktionen
\index{Funktion!programmglobale|ii}
und Variablen
\index{Variable!programmglobale|ii}
sind dazu bestimmt, auch von anderen
Modulen verwendet zu werden. Zus"atzlich k"onnen in einem
\rw{global}-Block Variablen deklariert werden, die in diesem Modul
verwendet werden sollen, jedoch in einem anderen Modul als global
definiert worden sind. Variablen, Strukturen und Funktionen sind
erst ab ihrer Deklaration verf"ugbar, soda"s es am
praktikabelsten ist, die \rw{global}-Bl"ocke an den Beginn der
Datei zu setzen oder in einer \ind{Include-Datei} unterzubringen.

\section{Anweisungen}
\index{Anweisung|ii}
\fuindexii{statement}
\fuindex{expr}
\fuindex{if}
\fuindex{while}
\fuindex{for}
\fuindex{repeat}
\fuindex{switch}
\fuindex{prefix}
\rwindex{break}
\rwindex{return}

\begin{verbatim}
statement:     [expr] `;'
             | if
             | while
             | for
             | repeat
             | switch
             | prefix
             | `break' `;'
             | `return' [expr] `;'
             ;
\end{verbatim}

Anweisungen (oder {\em statements}) k"onnen innerhalb von Funktionen
\index{Funktion}
verwendet werden und machen das eigentliche Programm aus.
Anweisungen k"onnen grunds"atzlich in Ausdr"ucke ({\em expressions}),
\index{Ausdruck}
Kontrollanweisungen
\index{Kontrollanweisung}
und Schleifen
\index{Schleife}
eingeteilt werden. Erstere verrichten die
eigentliche Arbeit, und die beiden letzteren steuern den Programmflu"s. Wie
aus der Grammatik ersichtlich ist, kann eine Anweisung auch
einfach nur ein Semikolon enthalten. Solche Anweisungen haben die
Bezeichnung {\em \indi{Nullanweisung}} und haben---bis auf die
Verbesserung der "Ubersichtlichkeit in bestimmten Situationen---%
keine Aufgabe.

\subsection{Die if-Anweisung}
\rwindexii{if}
\fuindexii{if}
\fuindex{expr}
\fuindex{statement}
\rwindexii{else}
\rwindex{end}

\begin{verbatim}
if:            `if' `(' expr `)'
                 {statement}
               [`else'
                 {statement}]
               `end'
             ;
\end{verbatim}

Die \rw{if}-Anweisung ist eine Kontrollanweisung und dient der
bedingten Ausf"uhrung eines Programmteils. Dem \rw{if}
folgt ein in Klammern eingeschlossener \ind{Ausdruck}, der zuerst
evaluiert (berechnet) und dann nach \tint\
\typeindex{int}
gecastet wird. Ist dieser
Ausdruck ungleich \fu{0}, werden die darauffolgenden
Anweisungen ausgef"uhrt. Ist jedoch der Ausdruck \fu{0}, dann
werden die Anweisungen, die dem \rw{else} folgen, sofern ein
\rw{else} vorhanden ist, ausgef"uhrt, ansonsten geschieht gar
nichts.

\example
Die Funktion \fuii{GetInt} fragt den Benutzer nach einer Zahl.
Ist die Zahl gr"o"ser oder gleich  \fu{0}, so wird der Text
"`\fu{Gratulation, die Zahl ist positiv!}"'
auf dem Bildschirm dargestellt. Andernfalls wird die Zahl ohne
negatives Vorzeichen ausgegeben.

\fuindexuu{ReadInt}
\fuindexuu{WriteString}
\fuindexuu{WriteInt}
\begin{verbatim}
function GetInt () : void

  i : int;

begin
  i = ReadInt();
  if (i >= 0)
    WriteString("Gratulation, die Zahl ist positiv!\n");
  else
    WriteInt(-i);
    WriteString("\n");
  end
end
\end{verbatim}

\subsection{Die switch-Anweisung}
\rwindexii{switch}
\fuindexii{switch}
\fuindex{expr}
\fuindex{statement}
\rwindex{end}
\fuindexii{prefix}
\rwindex{case}
\fuindex{constant}
\rwindex{default}

\begin{verbatim}
switch:        `switch' `(' expr `)'
                 {statement}
               `end'
             ;

prefix:        `case' constant `:'
             | `default' `:'
             ;
\end{verbatim}

Die \rw{switch}-Anweisung wird verwendet, um f"ur mehrere
verschiedene Werte eines Ausdrucks
\index{Ausdruck}
verschiedene Aktionen
auszuf"uhren. Als erstes wird der Ausdruck evaluiert. Danach
wird er mit der \ind{Konstante}, die dem ersten im
\fu{\{statement\}} vorkommenden \rw{case} folgt, verglichen. Ist
der Vergleich positiv, so werden alle der \rw{case}-Anweisung
folgenden Anweisungen ausgef"uhrt, solange, bis ein \rwii{break}
gefunden wird, was den Programflu"s nach dem \rw{end} fortsetzt.
Ist der Vergleich jedoch negativ, wird das n"achste \rw{case}
gesucht u.s.w. Wird bei diesem Vorgang ein \rwii{default} gefunden,
dann wird die Ausf"uhrung {\em in jedem Fall} nach dieser
Anweisung fortgesetzt. Aus diesem Grunde sollte \rw{default}
immer am Schlu"s der \rw{case}-Kette stehen.
Findet sich weder ein \rw{case}, das dem
Wert des Ausdrucks entspricht, noch ein \rw{default}, geschieht
nichts und die Ausf"uhrung wird nach dem \rw{end} fortgesetzt.

\example
Die Funktion \fuii{SimpleMenu} stellt ein Men"u auf
dem Bildschirm dar und wartet auf eine Eingabe. F"ur die Ziffern
\fu{`1'} bis \fu{`3'} werden die entsprechenden Funktionen
\fuii{Choice1} bis \fuii{Choice3}
\fuindexii{Choice2}
ausgef"uhrt. Dieser Vorgang
wiederholt sich so lange, bis der Benutzer \fu{`x'} oder \fu{`X'}
eingibt, was die Funktion beendet. Bet"atigt der Benutzer eine
Taste, die nicht verarbeitet werden kann, wird eine
dementsprechende Meldung ausgegeben. F"ur die Ausgabe des Men"us
auf dem Bildschirm wird die Funktion \fui{DisplayMenu} verwendet.

\rwindexuu{while}
\fuindexuu{ReadChar}
\fuindexuu{WriteString}
\begin{verbatim}
function SimpleMenu () : void
begin
  while (1)
    DisplayMenu();
    switch (ReadChar())
      case '1' :
        Choice1();
        break;
      case '2' :
        Choice2();
        break;
      case '3' :
        Choice3();
        break;
      case 'x' :
      case 'X' :
        return;
      default :
        WriteString("Falsche Eingabe!\n");
    end
  end
end
\end{verbatim}
        
\subsection{Die while-Schleife}
\rwindexii{while}
\fuindexii{while}
\fuindex{expr}
\fuindex{statement}
\rwindex{end}

\begin{verbatim}
while:         `while' `(' expr `)'
                 {statement}
               `end'
             ;
\end{verbatim}

Die \rw{while}-Schleife f"uhrt die zwischen \rw{while} und
\rw{end} aufgef"uhrten Anweisungen sooft aus, solange der in Klammern
eingeschlossene \ind{Ausdruck} ungleich \fu{0} ist. Dabei wird wie
folgt vorgegangen: Als erstes wird der Ausdruck evaluiert und
nach \tint\
\typeindex{int}
gecastet. Ist er
gleich \fu{0}, so wird die Ausf"uhrung nach dem \rw{end}
fortgesetzt, andernfalls werden die Anweisungen, die dem Ausdruck
folgen, ausgef"uhrt. Hierauf wird wieder an den Anfang der
Schleife gesprungen und der Ausdruck abermals gepr"uft u.s.w.
Beachten Sie, da"s es au"ser mit Hilfe der \rwi{return}-Anweisung
nicht m"oglich ist, aus der Schleife auszusteigen, ohne da"s der
Ausdruck gleich \fu{0} ist.

\example
Die Funktion \fuii{OutputList} durchl"auft die ihr "ubergebene
\indu{Liste} des Typs \fu{List}
\index{List@{\tt List}~(\rw{struct})|ii}
und gibt dabei alle Datenelemente---in
diesem Beispiel der Einfachheit halber vom Typ \tint ---aus.
Anmerkung: Das letzte Element der Liste ist dadurch
gekennzeichnet, da"s der Zeiger, der auf das n"achste Element
zeigen sollte, ein \ind{Nullzeiger} ist.

\fuindexuu{WriteInt}
\rwindexuu{local}
\begin{verbatim}
local
  struct List
    i    : int;
    Next : *List;
  end
end

function OutputList (l : *List) : void
begin
  while (l != 0)
    WriteInt(l->i);
    l = l->Next;
  end
end
\end{verbatim}

\subsection{Die repeat-Schleife}
\rwindexii{repeat}
\fuindexii{repeat}
\rwindexii{until}
\fuindex{statement}
\fuindex{expr}

\begin{verbatim}
repeat:        `repeat'
                 {statement}
               `until' `(' expr `)'
             ;
\end{verbatim}

Die \rw{repeat}-Schleife wiederholt die zwischen \rw{repeat} und
\rw{until} aufgef"uhrten Anweisungen so lange, bis der auf
\rw{until} folgende \ind{Ausdruck} ungleich \fu{0} ist. Der Ablauf ist
hierbei folgenderma"sen: Zun"achst werden die Anweisungen
ausgef"uhrt, danach wird der Ausdruck evaluiert, nach \tint\
\typeindex{int}
gecastet und auf
Gleichheit mit \fu{0} gepr"uft. Ist diese gegeben, wird zu
\rw{repeat} zur"uckgesprungen und der Vorgang wiederholt sich.
Ist der Ausdruck hingegen ungleich \fu{0}, ist die Schleife
beendet. Beachten Sie bitte, da"s---im Gegensatz zur
\rwi{while}-Schleife---die Schleife {\em mindestens einmal}
ausgef"uhrt wird, unerheblich, ob der Ausdruck schon vor der
Ausf"uhrung ungleich \fu{0} w"are.

\example
Dieses Beispielprogramm liest so lange Zahlen vom Benutzer ein,
bis dieser \fu{0} eingibt. Anschlie"send wird der Mittelwert der
eingelesenen Zahlen ausgegeben. Wurden au"ser der \fu{0} keine
Zahlen eingegeben, erscheint eine dementsprechende Meldung.
    
\rwindexuu{global}
\fuindexuu{ReadInt}
\rwindexuu{if}
\fuindexuu{WriteInt}
\fuindexuu{WriteString}
\begin{verbatim}
#include "conio.sh"

global
  function main () : void
end

function main () : void

  i : int;
  a : int =  0;
  n : int = -1;

begin
  repeat
    a = a + (i = ReadInt());
    n = n + 1;
  until (i == 0)
  if (n)
    WriteInt(a / n);
  else
    WriteString("Es wurden keine Zahlen eingegeben!");
  end
  WriteString("\n");
end
\end{verbatim}

\subsection{Die for-Schleife}
\rwindexii{for}
\fuindexii{for}
\fuindex{primary}
\fuindex{expr}
\rwindexii{to}
\rwindexii{downto}
\rwindexii{step}
\fuindex{statement}
\rwindex{end}

\begin{verbatim}
for:           `for' `(' primary `=' expr (`to' | `downto') expr
               [`step' expr] `)'
                 {statement}
               `end'
             ;
\end{verbatim}

Die \rw{for}-Schleife ist eine spezielle Schleife, die es
erlaubt, Anweisungen eine bestimmte Anzahl mal zu wiederholen.
Hierzu ist nach der "offnenden Klammer die Variable anzugeben,
die als Z"ahler fungiert. Da hier ein \fu{primary} (siehe Kapitel
\ref{exprs}) erwartet wird,
ist es auch m"oglich, Elemente eines Arrays
\index{Array}
oder einer \ind{Struktur}
als Z"ahler zu verwenden. Nach dem Zuweisungsoperator wird der
Anfangsindex angegeben und danach entweder das Schl"usselwort
\rw{to} oder \rw{downto}, je nachdem, ob nach oben oder nach
unten gez"ahlt werden soll. Darauf folgt der Zielindex, also
jener Wert, den die Indexvariable maximal oder minimal annehmen
darf. Optional kann noch nach dem Schl"usselwort \rw{step} die
Schrittweite angeben werden. Zu beachten ist, da"s diese auch
dann, wenn nach unten gez"ahlt wird, positiv sein mu"s.

Bei der Ausf"uhrung wird folgenderma"sen vorgegangen: Zun"achst
werden alle Ausdr"ucke
\index{Ausdruck}
evaluiert und nach \tint\
\typeindex{int}
gecastet. Danach
wird "uberpr"uft, ob der Wert der Indexvariablen schon au"serhalb
der zul"assigen Grenze ist. Ist das nicht der Fall, werden die
Anweisungen das erste Mal ausgef"uhrt und der Z"ahler
entsprechend erh"oht bzw. erniedrigt. Danach wird wieder
"uberpr"uft, ob die Anweisungen nocheinmal ausgef"uhrt werden
sollen u.s.w.

\example
Die Funktion \fuii{Squares} gibt der Reihe nach die Quadrate
\index{Quadrat}
der ganzen Zahlen $i$ f"ur \hbox{$\fu{a} \le i \le \fu{b}$} aus.
Der Funktion werden die \tint s \fu{a} und \fu{b} "ubergeben.

\fuindexuu{WriteInt}
\fuindexuu{WriteString}
\begin{verbatim}
function Squares (a : int, b : int) : void

  i : int;

begin
  for (i = a to b)
    WriteInt(i * i);
    WriteString("\n");
  end
end
\end{verbatim}

\section{Ausdr\protect{"ucke}}
\label{exprs}
\index{Ausdruck|ii}
\fuindexii{expr}
\rwindex{delete}
\fuindexii{primary}
\fuindex{IDENTIFIER}
\fuindex{INT\protect{\_CONSTANT}}
\fuindex{CHAR\protect{\_CONSTANT}}
\fuindex{FLOAT\protect{\_CONSTANT}}
\fuindex{STRING\protect{\_CONSTANT}}
\fuindex{type}
\rwindex{sizeof}
\rwindex{new}

\begin{verbatim}
expr:          expr (`=' | `&&' | `||' | `^^' | `==' | `<' | `>'
                   | `<=' | `>=' | `!=' | `&' | `|' | `^' | `<<'
                   | `>>' | `+' | `-' | `*' | `/' | `%') expr
             | (`!' | `~' | `-' | `*' | `&') expr
             | `delete' expr
             | primary
             ;

primary:       IDENTIFIER
             | `(' expr `)'
             | INT_CONSTANT
             | CHAR_CONSTANT
             | FLOAT_CONSTANT
             | STRING_CONSTANT
             | primary `->' IDENTIFIER
             | primary `:' type
             | primary `[' expr `]'
             | IDENTIFIER `(' [expr {`,' expr}] `)'
             | `sizeof' type
             | `new' type
             ;
\end{verbatim}

Ausdr"ucke sind sozusagen das "`Herz"' von SEL, denn sie
erledigen die eigentliche Arbeit, w"ahrend Kontrollanweisungen
und Schleifen nur kontrollieren, was passieren soll. Ausdr"ucke k"onnen die
verschiedensten Aufgaben erledigen: Sie k"onnen zum Vergleich
zweier Werte benutzt werden, zur Wertzuweisung oder zum
Funktionsaufruf. Die gemeinsamen Merkmale der Ausdr"ucke sind:
Jeder Ausdruck besitzt sowohl einen \ind{Typ} als auch einen
\ind{Wert} (bis auf die Ausdr"ucke, die vom Typ \tvoid\
\typeindex{void}
sind).

\subsection{Konstanten}
\index{Konstante}
\fuindexii{constant}
\fuindex{INT\protect{\_CONSTANT}}
\fuindex{CHAR\protect{\_CONSTANT}}
\fuindex{FLOAT\protect{\_CONSTANT}}
\fuindex{STRING\protect{\_CONSTANT}}

\begin{verbatim}
constant:      [`-'] INT_CONSTANT
             | CHAR_CONSTANT
             | [`-'] FLOAT_CONSTANT
             | STRING_CONSTANT
             ;
\end{verbatim}

Konstanten in SEL sind im Grunde als Konstanten im
mathematischen Sinne zu verstehen. Ihre Werte k"onnen im gesamten
Programm nicht
ver"andert werden. Im Gegensatz zu anderen Programmiersprachen
besitzt SEL nicht die F"ahigkeit, Konstanten einen Namen zu geben,
\index{Konstante!benannte}
was allerdings durch die \fui{\#define}-Direktive leicht umgangen
werden kann. Weiters gibt es in SEL keine Konstanten von
zusammengesetzten Typen,
\index{Typ!zusammengesetzter}
da diese F"ahigkeit nur sehr selten ben"otigt wird.

\subsubsection{int-Konstanten}
\index{Konstante!\tint |ii}
\typeindex{int}
\fuindexii{INT\protect{\_CONSTANT}}

\begin{verbatim}
INT_CONSTANT: [0-9]+
\end{verbatim}

Eine \tint -Konstante ist eine Folge von Ziffern, die als Zahl
mit der Basis 10, also als Dezimalzahl
\index{Zahl!ganze}
\index{Zahl!dezimale}
interpretiert werden.
Laut obiger Definition gibt es jedoch
keine negativen \tint -Konstanten, was den Grund hat, da"s
dieser Mechanismus in der Grammatik eingebaut ist. Dies wiederum
ist darin begr"undet, da"s es andernfalls zu Mi"sverst"andnissen
bei der Interpretation von Ausdr"ucken kommen w"urde (man denke
an den Ausdruck \fu{7-4}---er w"urde als Konstante \fu{7} gefolgt
von der Konstante \fu{-4} gedeutet werden, da die lexikalische
Analyse "uber keine Informationen "uber den Kontext verf"ugt, in dem
die Konstante auftritt, mit anderen Worten, sie wei"s nicht, da"s
auf eine Konstante keine weitere folgen kann).

Voraussichtlich wird SEL in der Version 0.4 um die F"ahigkeit
erweitert werden, auch hexadezimale
\index{Zahl!hexadezimale}
\tint -Konstanten in der Form
\fu{0(x|X)[0-9a-fA-F]+} zu interpretieren.

\subsubsection{char-Konstanten}
\index{Konstante!\tchar |ii}
\typeindex{char}
\fuindexii{CHAR\protect{\_CONSTANT}}

\begin{verbatim}
CHAR_CONSTANT: '(.|\.)'
\end{verbatim}

\tchar -Konstanten repr"asentieren einzelne \ind{Zeichen}.
Dieses k"onnen entweder "`normale"' Zeichen oder
Steuerzeichen
\index{Zeichen!Steuerzeichen}
sein---letztere werden durch einen vorangestellten
Back"-slash "`\fu{\char92}"'
\index{Backslash (\fu{\char92})}
gekennzeichnet, jedoch als ein einzelnes Zeichen behandelt.

\subsubsection{float-Konstanten}
\index{Konstante!\tfloat |ii}
\typeindex{float}
\fuindexii{FLOAT\protect{\_CONSTANT}}

\begin{verbatim}
FLOAT_CONSTANT: [0-9]*(`.'[0-9]*)?((e|E)(+|-)?[0-9]+)?
\end{verbatim}

Eine \tfloat -Konstante repr"asentiert eine positive, rationale
\index{Zahl!rationale}
Zahl in \ind{Exponentialschreibweise}, wobei der Exponent auch
weggelassen werden kann. Leider ist eine Definition als
regul"arer Ausdruck nicht ohne Probleme m"oglich. So w"are laut
obiger Definition auch eine leere Eingabe oder ein Dezimalpunkt
f"ur sich alleine eine \tfloat -Konstante, was nat"urlich nicht
im Sinn der Sache liegen kann. Die Definition ist also
dahingehend zu erweitern, da"s in der Mantisse mindestens eine Ziffer
vorkommen mu"s. Beachten Sie bitte, da"s eine
\tfloat -Konstante immer positiv sein mu"s, was jedoch aufgrund
der Grammatik von SEL nicht weiter st"ort.

\subsubsection{String-Konstanten}
\index{Konstante!String|ii}
\fuindexii{STRING\protect{\_CONSTANT}}

\begin{verbatim}
STRING_CONSTANT: "([^"\n]|\.)*"
\end{verbatim}

Eine String-Konstante ist eine in zwei Anf"uhrungsstriche
eingeschlossene Zeichenfolge, innerhalb derer
alle \ind{Zeichen} bis auf die Anf"uhrungsstriche und den
Zeilenvorschub erlaubt sind. Eine besondere Bedeutung kommt den
Steuerzeichen
\index{Zeichen!Steuerzeichen}
zu, die durch einen vorangestellten Back"-slash
"`\fu{\char92}"'
\index{Backslash (\fu{\char92})}
gekennzeichnet sind. Jede String-Konstante wird
von SEL automatisch mit einem \indi{Nullzeichen}, also einem Zeichen mit
dem Wert \fu{0} beendet. Die String-Konstante beansprucht also im
Speicher um ein Byte mehr, als sie an Zeichen enth"alt.

Eine String-Konstante wird von SEL wie ein Zeiger auf
\tchar s
\typeindex{char}
behandelt. Sie k"onnen diesen Zeiger also dazu
verwenden, um die Zeichenfolge auszugeben, sollten jedoch den
Speicherbereich, auf den er verweist, nicht ver"andern.

\subsubsection{Steuerzeichen}
\label{specialchar}
\index{Zeichen!Steuerzeichen|ii}

SEL stellt bestimmte Steuerzeichen zur Verf"ugung, die in Strings
\index{Konstante!String}
und \tchar -Konstanten
\index{Konstante!\tchar }
\typeindex{char}
benutzt werden k"onnen, um Zeichen
darzustellen, denen eine bestimmte Bedeutung zukommt, wie etwa
dem Zeichen f"ur Zeilenvorschub. Steuerzeichen
sind daran zu erkennen, da"s ihnen ein Back"-slash "`\fu{\char92}"'
\index{Backslash (\fu{\char92})|ii}
vorangestellt ist. Trotzdem werden sie von SEL als einzelne
Zeichen behandelt. Hier eine Zusammenstellung aller in SEL
vorhandenen Steuerzeichen:

\medskip
\index{Warnton}
\index{R""uckschritt}
\index{Bildvorschub}
\index{Zeilenvorschub}
\index{Wagenr""ucklauf}
\index{Tabulator}
\begin{tabular}{ll}
\fu{\char92 a} & Bell (Warnton) \\
\fu{\char92 b} & Backspace (R"uckschritt) \\
\fu{\char92 f} & Formfeed (Bildvorschub) \\
\fu{\char92 n} & New line (Zeilenvorschub) \\
\fu{\char92 r} & Carriage return (Wagenr"ucklauf) \\
\fu{\char92 t} & Tab (Tabulator) \\
\fu{\char92 v} & Vertical tab (vertikaler Tabulator)
\end{tabular}
\medskip

Wird hinter den Back"-slash ein Zeichen gesetzt, das in obiger
Tabelle nicht eingetragen ist, wird von SEL automatisch dieses
Zeichen verwendet. So ist es zum Beispiel m"oglich, einen
Back"-slash oder ein Anf"uhrungszeichen innerhalb einer
String-Konstante einzugeben.

SEL wird in der Version 0.4 aller Voraussicht nach um das
Steuerzeichen \fu{\char92 x\rm\it hh} erweitert werden, das an seine
Stelle das Zeichen mit dem in hexadezimaler
\index{Zahl!hexadezimale}
Notation angegeben Code {\it hh} stellt.

\subsection{Variablen}
\index{Variable|ii}
\fuindexii{var}
\fuindex{IDENTIFIER}
\fuindex{type}
\fuindex{constant}
\fuindexii{extern}
\rwindexii{extern}

\begin{verbatim}
var:           IDENTIFIER `:' type [`=' constant] `;'
             ;

extern:        `extern' IDENTIFIER `:' type `;'
             ;
\end{verbatim}

Variablen sind Speicherbereiche, die
vom Programm belegt werden und w"ahrend der Ausf"uhrung
verschiendene Werte
\index{Wert}
annehmen k"onnen. Jede Variable besitzt
einen \ind{Typ}, der sowohl ihre Gr"o"se als auch ihre Repr"asentation
im Speicher bestimmt. SEL unterscheidet grunds"atzlich zwischen
drei Typen von Variablen: programmglobale Variablen, modulglobale
Variablen und lokale Variablen.

Programmglobale Variablen werden in einem \rwi{global}-Block
definiert und sind im gesamten Programm bekannt. Ein Modul, das
eine Variable verwenden will, die in einem anderen Modul
definiert ist, mu"s jedoch, um auf die Variable zugreifen zu
k"onnen, dies mit einer \rw{extern}-Deklaration bekanntgeben.
Sie unterscheidet sich von einer Variablendefinition lediglich
dadurch, da"s ihr das \rw{extern}-Schl"usselwort vorangestellt
ist und keine Wertzuweisung erlaubt ist. Beachten Sie, da"s es
unm"oglich ist, innerhalb eines
Programms zwei programmglobale Variablen desselben Namens zu
definieren, da dies zu Konflikten f"uhren w"urde.

Modulglobale Variablen werden in einem \rwi{local}-Block eines
Moduls definiert und sind nur innerhalb dieses Moduls
bekannt---auf sie kann von anderen Modulen auch nicht durch eine
\rw{extern}-Deklaration zugegriffen werden. Deshalb ist es auch
ohne weiteres m"oglich, mehrere modulglobale Variablen gleichen
Namens zu definieren, soferne dies in jeweils verschiedenen
Modulen geschieht.

Lokale Variablen sind nur innerhalb der \ind{Funktion} bekannt, in der
sie definiert wurden. Solche Variablen sind allerdings nicht
statisch, das hei"st, sie werden beim Aufruf der jeweiligen Funktion
erzeugt und existieren so lange, bis die Funktion zu ihrem
Aufrufer zur"uckkehrt. Weiters ist es auch durchaus m"oglich,
da"s lokale Variablen mehrfach existieren, und zwar dann, wenn
sich die Funktion selbst aufruft, also rekursiv ist. Bei jedem
solchen Aufruf werden neue lokale Variablen erzeugt, die zwar
namensgleich, jedoch von den anderen absolut isoliert sind.
Unter die lokalen Variablen fallen auch die Funktionsparameter.
\index{Parameter}
Sie werden im Funktionskopf definiert und sind auch nur existent,
solange die Funktion abgearbeitet wird.

Es ist in SEL durchaus m"oglich, modulglobale, programmglobale
und lokale Variablen desselben Namens zu definieren, wobei
folgende Regel zu beachten ist: lokale Variablen werden
modulglobalen Variablen vorgezogen, welche ihrerseits den
programmglobalen Variablen vorgezogen werden. Existiert also die
programmglobale Variable \fu{example}, so kann ohne weiteres eine
modulglobale Variable desselben Namens definiert werden.
Innerhalb des Moduls, in der letztere definiert wurde, bezieht
sich \fu{example} nicht auf die programmglobale, sondern auf die
modulglobale Variable. Definiert weiters eine Funktion innerhalb
dieses Moduls eine lokale Variable namens \fu{example}, so kann
diese Funktion nur auf die lokale Variable zugreifen, w"ahrend
der Rest des Moduls auf die modulglobale Variable zugreift.

\example
Die Funktion \fuii{Recursive}
\index{Rekursion|uu}
gibt die Zahlen \fu{1} bis \fu{10} in
ihrer nat"urlichen Reihenfolge aus und ruft sich nach der
Ausgabe jeder Zahl selbst auf, was dazu f"uhrt, da"s f"ur jede
Zahl zus"atzlich nocheinmal die Zahlen \fu{1} bis \fu{10}
ausgegeben werden, f"ur die dasselbe gilt. Um zu verhindern, da"s
die Zahl \fu{1} unendlich oft ausgegeben wird, was zweifellos
geschehen w"urde, w"urd dieses Verfahren unkontrolliert
angewendet, wird der Funktion ein Parameter "ubergeben, der
bestimmt, wie oft sie sich noch selbst aufrufen soll.
Beachten Sie bei diesem Beispiel, da"s sich die einzelnen Z"ahler
\fu{i} und die Parameter \fu{depth} der Funktion nicht gegenseitig
st"oren.

\rwindexuu{for}
\fuindexuu{WriteInt}
\fuindexuu{WriteString}
\begin{verbatim}
function Recursive (depth : int) : void

  i : int;

begin
  for (i = 1 to 10)
    WriteInt(i);
    WriteString("  ");
    if (depth > 0)
      Recursive(depth - 1);
      WriteString("  ");
    end
  end
end
\end{verbatim}

\subsection{Funktionen}
\index{Funktion|ii}

Funktionen sind Unterprogramme, die von verschiedenen Stellen des
Programms aufgerufen werden k"onnen. Um den Funktionen
Information mitzugeben, k"onnen ihnen sogenannte {\em \indi{Parameter}}
"ubergeben werden, deren Typen bei der Funktionsdeklaration
bestimmt werden m"ussen. Der Aufruf einer Funktion geschieht
durch Angabe des Funktionsnamens und in Klammern gesetzten
Parametern,
\index{Parameter""ubergabe|ii}
durch Beistriche getrennt. Sind keine Parameter
erforderlich, m"ussen nur die
beiden Klammern angegeben werden. Jede Funktion besitzt einen
R"uckgabetyp, der
ebenfalls bei der Deklaration festgelegt werden mu"s. Um eine
Funktion, die keine R"uckgabe vornimmt, zu realisieren, mu"s als
R"uckgabetyp \tvoid\
\typeindex{void}
gew"ahlt werden.

\example
Dieses Programm benutzt die Funktionen \fuiu{Factorial},
\fu{WriteString} und \fu{WriteInt}, um die Fakult"aten
\index{Fakult""at}
der Zahlen von \fu{1} bis \fu{10} auszugeben.

\rwindexuu{for}
\rwindexuu{global}
\fuindexuu{WriteString}
\fuindexuu{WriteInt}
\begin{verbatim}
#include "conio.sh"

global
  function Factorial (int) : int
  function main      ()    : void
end

function main () : void

  i : int;

begin
  for (i = 1 to 10)
    WriteString("Fakultaet von ");
    WriteInt(i);
    WriteString(" = ");
    WriteInt(Factorial(i));
    WriteString("\n");
  end
end
\end{verbatim}

\subsection{Operatoren}
\index{Operator|ii}

Operatoren werden, wie schon der Name sagt, ben"otigt, um
bestimmte Operationen auszuf"uhren, wie etwa das Addieren
von Zahlen oder das Zuweisen von Werten. Grunds"atzlich kann man
in SEL zwischen {\em un"aren} und {\em bin"aren} Operatoren
unterscheiden.

Un"are Operatoren stehen vor einem Ausdruck und
beeinflussen so dessen \ind{Wert}. Ein Beispiel hierf"ur w"are der
\ind{NOT}-Operator \fu{!}, der den \ind{Ausdruck}, der ihm nachsteht,
negiert. Un"are Operatoren werden grunds"atzlich von rechts nach
links ausgewertet, das hei"st, es wird zuerst der Wert des
Ausdrucks berechnet und danach negiert, nicht zuletzt deshalb,
weil es nicht anders m"oglich ist.

Bin"are Operatoren beziehen sich auf zwei Operanden, sie werden
deshalb zwischen diese beiden gestellt. Ein Beispiel f"ur einen
bin"aren Operator w"are der Additionsoperator \fu{+}. Bin"are
Operatoren k"onnen weiter unterteilt werden in solche, die von
rechts nach links und solche, die von links
nach rechts auswerten.
\index{Operator!Richtung der Auswertung}

Ersteres bedeutet, da"s zuerst
der Wert des rechten Operanden berechnet und dann auf den
linken angewendet wird. Ein Beispiel w"are der Zuweisungsoperator
\index{Zuweisung}
\fu{=}. Was unter der obigen Definition gemeint ist, l"a"st sich
leicht anhand eines Ausdrucks veranschaulichen: \fu{a=b=1}. SEL
beginnt von links nach rechts zu parsen und bleibt beim ersten
Operator stehen. Der linke Operand w"are demnach \fu{a}, der
rechte \fu{b=1}. Nun wird zuerst der rechte Operand ausgewertet.
SEL stoppt beim zweiten Operator und steht vor den Operanden
\fu{b} und \fu{1}. Wieder wird der rechte Operand ausgewertet,
von dem SEL nun wei"s, da"s er den Wert \fu{1} hat. Dieser wird
nun auf den linken Operanden angewendet, also der Variable \fu{b}
zugewiesen. Weiters wird nun der Wert des Ausdrucks \fu{b=1}
ebenfalls auf \fu{1} gesetzt und \fu{a} zugewiesen. Diese Art der
Auswertung ist zweckm"a"sig, da es so m"oglich ist, mehreren
Variablen auf einmal den selben Wert zuzuweisen. Um die
Auswertung deutlich zu machen, kann man den Ausdruck auch als
\fu{a=(b=1)} anschreiben.

Ein Beispiel f"ur einen Operator, der von links nach rechts
auswertet, ist der Subtraktionsoperator \fu{-}.
\index{Subtraktion}
Dabei wird
von zwei Operanden zun"achst der linke ausgewertet und dann auf
den rechten angewendet. Als Beispiel sei der Ausdruck \fu{3-2-1}
angef"uhrt. W"urde der Operator \fu{-} von rechts nach links
auswerten, so k"onnte man diesen Ausdruck auch als \fu{3-(2-1)}
anschreiben, was jedoch zu dem falschen Ergebnis \fu{2} f"uhrte.
Aus diesem Grunde wertet \fu{-} von links nach rechts aus,
was den Ausdruck zu \fu{(3-2)-1} macht. Dieser Ausdruck liefert
das gew"unschte Ergebnis \fu{0}.

Es folgt eine kurze Beschreibung aller Operatoren sowie der
Typen ihrer Operanden und des Typs des Ergebnisses der Operation.
Bei den Typen der Operanden ist immer zu beachten, da"s, falls
diese unterschiedlich sind, SEL immer dem "`gr"o"seren"' Typ den
Vorrang gibt, wobei \hbox{$ \tchar < \tint < \tfloat $} gilt.
\typeindex{char}
\typeindex{int}
\typeindex{float}
Der Ausdruck \fu{3 + 4.123} ergibt also einen
\tfloat\ mit dem Wert \fu{7.123}, wobei allerdings der Ausdruck
\fu{13 \&\& 5.3} einen \tint\ mit einem Wert ungleich \fu{0}
ergibt, da der Operator \fu{\&\&} nur mit \tint s arbeitet.

\begin{operator}
\tchar\ \fu{`+'} \tchar\ & \tchar\ \\
\tint\ \fu{`+'} \tint\ & \tint\ \\
\tfloat\ \fu{`+'} \tfloat\ & \tfloat\ \\
\type{\poi type} \fu{`+'} \tint\ & \type{\poi type}
\end{operator}
\index{Addition|ii}

Sind die beiden Operanden des bin"aren \fu{+}-Operators numerisch, werden
diese addiert. Ist jedoch der linke Operand ein \ind{Zeiger}, ergibt
die Operation einen Zeiger, der um die Anzahl Elemente erh"oht
ist, die der rechte Operand angibt. Beachten Sie bitte, da"s die
Gr"o"se eines Elements nicht unbedingt ein Byte sein mu"s,
sondern so gro"s ist, wie der Typ, den der Zeiger referenziert. So
ist es z.~B. leicht m"oglich, ein \ind{Array} zu durchlaufen,
ohne einen numerischen Z"ahler zu verwenden.

\begin{operator}
\tchar\ \fu{`-'} \tchar\ & \tchar\ \\
\tint\ \fu{`-'} \tint\ & \tint\ \\
\tfloat\ \fu{`-'} \tfloat\ & \tfloat\ \\
\type{\poi type} \fu{`-'} \tint\ & \type{\poi type}
\end{operator}
\index{Subtraktion|ii}

Der bin"are Operator \fu{-} subtrahiert den rechten Operanden vom linken,
falls beide numerisch sind, andernfalls wird der \ind{Zeiger} um die
Anzahl Elemente erniedrigt, die vom rechten Operanden angegeben
wird. Mit Element ist hier der Typ gemeint, den der Zeiger
referenziert.

\begin{operator}
\tchar\ \fu{`*'} \tchar\ & \tchar\ \\
\tint\ \fu{`*'} \tint\ & \tint\ \\
\tfloat\ \fu{`*'} \tfloat\ & \tfloat\
\end{operator}
\index{Multiplikation|ii}

Der bin"are \fu{*}-Operator multipliziert die beiden Operanden.

\begin{operator}
\tchar\ \fu{`/'} \tchar\ & \tchar\ \\
\tint\ \fu{`/'} \tint\ & \tint\ \\
\tfloat\ \fu{`/'} \tfloat\ & \tfloat\
\end{operator}
\index{Division|ii}
\index{Zahl!Bruch}

Der \fu{/}-Operator dividiert den linken durch den rechten
Operanden. Sind beide Operanden keine \tfloat s, so ist das
Ergebnis auch kein \tfloat , selbst wenn der Bruch nicht
ganzzahlig ist.

\begin{operator}
\tchar\ \fu{`\%'} \tchar\ & \tchar\ \\
\tint\ \fu{`\%'} \tint\ & \tint\
\end{operator}
\index{Division!Rest bei ganzzahliger|ii}
\index{Modulo|see{Division}}

Der \fu{\%}-Operator berechnet den Rest, der bei einer
ganzzahligen Division des linken durch den rechten Operanden zu
verzeichnen w"are. Beachten Sie bitte, da"s dieser Operator {\em
nicht} mit \tfloat s funktioniert.

\newpage
\begin{operator}
\tint\ \fu{`\&\&'} \tint\ & \tint\
\end{operator}
\index{AND!logisches|ii}
\index{UND|see{AND}}

Der \fu{\&\&}-Operator verkn"upft die beiden Operanden mit einem
logischen Und (AND), das hei"st, der Ausdruck ergibt einen Wert
ungleich \fu{0}, wenn sowohl der linke als auch der rechte
Operand ungleich \fu{0} sind, andernfalls ergibt der Ausdruck
\fu{0}.

\begin{operator}
\tint\ \fu{`||'} \tint\ & \tint\
\end{operator}
\index{OR!logisches|ii}
\index{ODER|see{OR}}

Der \fu{||}-Operator verkn"upft seine beiden Operanden mit einem
logischen Oder (OR), was bedeutet, da"s der Ausdruck dann ungleich
\fu{0} ist, wenn mindestens einer der beiden Operanden ungleich
\fu{0} ist. Nur wenn sowohl der linke als auch der rechte Operand
\fu{0} sind, ergibt der Ausdruck \fu{0}.

\begin{operator}
\tint\ \fu{`\char94\char94'} \tint\ & \tint\
\end{operator}
\index{XOR!logisches|ii}
\index{ODER!exklusives|see{XOR}}

Der \fu{\char94\char94}-Operator verkn"upft die beiden Operanden
mit einem logischen exklusiven Oder (XOR). Ein Ausdruck
ergibt also nur dann einen Wert ungleich \fu{0}, wenn genau einer
der Operanden ungleich \fu{0} ist. Haben hingegen entweder beide
Operanden den Wert \fu{0} oder sind beide ungleich \fu{0}, so ist der
Ausdruck gleich \fu{0}.

\begin{operator}
\tint\ \fu{`\&'} \tint\ & \tint\
\end{operator}
\index{AND!bitweises|ii}

Der bin"are \fu{\&}-Operator verkn"upft die beiden Operanden mit
einem bitweisen Und (AND), was bedeutet, da"s er ein logisches Und auf
jedes einzelne Bitpaar des linken und rechten Operanden anwendet.
So ergibt zum Beispiel der Ausdruck \fu{93 \& 45} den Wert
\fu{13}. Die Funktionsweise l"a"st sich leicht nachvollziehen,
indem man die bin"are Schreibweise
\index{Zahl!bin""are}
verwendet, also \fu{1011101b}
f"ur \fu{93}, \fu{0101101b} f"ur \fu{45} und \fu{0001101b} f"ur
\fu{13}. Jene Stellen, die in mindestens einem der
Operanden den Wert \fu{0} haben, haben auch im Ergebnis den Wert
\fu{0}. Alle anderen Stellen, also jene, die sowohl im linken
als auch im rechten Operanden den Wert \fu{1} haben, haben auch
im Ergebnis den Wert \fu{1}.

\begin{operator}
\tint\ \fu{`|'} \tint\ & \tint\
\end{operator}
\index{OR!bitweises|ii}

Der Operator \fu{|} verkn"upft beide Operanden mit einem
bitweisen Oder (OR), das hei"st, er wendet auf jedes Bitpaar der
beiden Operanden ein logisches Oder an. Der Ausdruck \fu{93 | 45}
ergibt zum Beispiel den Wert 125.

\begin{operator}
\tint\ \fu{`\char94'} \tint\ & \tint\
\end{operator}
\index{XOR!bitweises|ii}

Der Operator \fu{\char94} verkn"upft seine beiden Operanden mit
einem bitweisen exklusiven Oder (XOR), das hei"st, er wendet auf
jedes Bitpaar des linken und rechten Operanden jeweils ein
logisches exklusives Oder an. So ergibt zum Beispiel der Ausdruck
\fu{93 \char94\ 45} den Wert 112.

\begin{operator}
\tint\ \fu{`<<'} \tint\ & \tint\
\end{operator}
\index{Bitverschiebung!nach links|ii}

Der \fu{<<}-Operator ergibt einen Wert, der dem linken Operanden
entspricht, wobei jedoch die einzelnen Bits desselben um soviel
Stellen nach links geschoben werden, wie der rechte Operand
angibt. W"urde dies in eine Zahl resultieren, die mehr Bits hat,
als der \tint -Typ zul"a"st, so werden die "`"ubersch"ussigen"'
Bits abgeschnitten. Ein Beispiel: Der Ausdruck \fu{93 << 3}
ergibt \fu{744}. Warum dies so ist, l"a"st sich leicht anhand der
Bin"arschreibweise
\index{Zahl!bin""are}
der beiden Zahlen verdeutlichen. Die Zahl
\fu{93} kann auch als \fu{1011101b} geschrieben werden, die Zahl
\fu{744} als \fu{1011101000b}. Wie man sieht, wurden also
lediglich drei Nullstellen am Ende der Zahl eingef"ugt.
Anmerkung: Aufgrund der Eigenschaften des bin"aren Zahlensystems
gilt (ohne Ber"ucksichtigung der negativen Zahlen)
\hbox{$a \>\fu{<<}\> n = a \cdot 2^n.$}

\begin{operator}
\tint\ \fu{`>>'} \tint\ & \tint\
\end{operator}
\index{Bitverschiebung!nach rechts|ii}

Der Operator \fu{>>} ergibt einen Wert, der dem linken Operanden
entspricht, dessen Bits um soviel Stellen nach rechts ger"uckt
werden, wie der rechte Operand angibt. Bits, die dabei rechts
"uber die Einerstelle herausragen w"urden, werden dabei
abgeschnitten. Die sich dabei ergebende Zahl kann (ohne
Ber"ucksichtigung der negativen Zahlen) als
\hbox{$a \>\fu{>>}\> n = \lfloor a / 2^n \rfloor $} angeschrieben werden.
So ergibt z.~B. der Ausdruck \fu{93 >> 3} den Wert \fu{11}.

\begin{operator}
\fu{`-'} \tchar\ & \tchar\ \\
\fu{`-'} \tint\ & \tint\ \\
\fu{`-'} \tfloat\ & \tfloat\
\end{operator}
\index{Negation!numerische|ii}
\index{Negation!logische|see{NOT}}
\index{Negation!bitweise|see{NOT}}

Der un"are Operator \fu{-} negiert den Operanden, das
hei"st, er macht aus einer positiven Zahl eine negative und
umgekehrt. So ergibt zum Beispiel der Ausdruck \fu{- 93} den Wert
\fu{-93}.

\begin{operator}
\fu{`*'} \type{\poi type} & \type{type}
\end{operator}
\index{Zeiger!Dereferenzierung|ii}

Der un"are \fu{*}-Operator dereferenziert den Operanden, das
hei"st, er ergibt jene Variable, auf die der Operand zeigt (ist
der Operand kein Zeiger, ist die Operation illegal). Angenommen,
die Variable \fu{p} ist vom Typ \type{\poi int} und wurde mit
\fu{p = \&i} initialisiert, wobei \fu{i} vom Typ \tint\ ist, so
ist der Ausdruck \fu{*p} mit dem Ausdruck \fu{i} gleichzusetzen.

\begin{operator}
\fu{`\&'} \type{type} & \type{\poi type}
\end{operator}
\index{Zeiger!Ermittlung|ii}
\index{Variable!Adresse}

Der un"are \fu{\&}-Operator ergibt einen Zeiger auf seinen
Operanden, was nat"urlich nur m"oglich ist, wenn dieser keine
Konstante ist. So ist z.~B. der Ausdruck \fu{\&i} mit dem
Ausdruck \fu{p} gleichzusetzen, wenn \fu{*p} gleich \fu{i} ist.
Aus diesem und dem letzten Beispiel wird klar, da"s \fu{\&} das
Gegenst"uck zum \fu{*}-Operator ist und umgekehrt.

\begin{operator}
\fu{`!'} \tint\ & \tint\
\end{operator}
\index{NOT!logisches|ii}

Der Operator \fu{!} ergibt die logische Negation (NOT) seines
Operanden, was bedeutet, da"s der Wert gleich \fu{0} ist, falls
der Operand ungleich \fu{0} ist und umgekehrt. So ergibt z.~B.
der Ausdruck \fu{!93} den Wert \fu{0}.

\begin{operator}
\fu{`\char126'} \tint\ & \tint\
\end{operator}
\index{NOT!bitweises|ii}
\index{NOT!numerisches|see{Negation}}

Der Operator \fu{\char126} errechnet die bitweise Negation seines
Operanden, das hei"st, er wendet auf jedes Bit desselben die
logische Negation an. Das Ergebnis kann je nach Wortl"ange und
Darstellung des Typs \tint\ stark differenzieren. Im Regelfall
kann jedoch mit einer Wortl"ange von 32 Bits und einer
Einerkomplementdarstellung der negativen Zahlen gerechnet werden.
So ergibt zum Beispiel der Ausdruck \fu{\char126 93} auf MS-DOS-
und Windows NT-Rechnern den Wert \fu{-94}.

\begin{operator}
\tchar\ \fu{`=='} \tchar\ & \tint\ \\
\tint\ \fu{`=='} \tint\ & \tint\ \\
\tfloat\ \fu{`=='} \tfloat\ & \tint\
\end{operator}
\index{Vergleich|ii}

Der Operator \fu{==} "uberpr"uft die beiden Operanden auf
Gleichheit, das hei"st, der Ausdruck ergibt einen Wert ungleich
\fu{0}, falls der linke und der rechte Operand den selben Wert
haben, andernfalls \fu{0}.

\begin{operator}
\tchar\ \fu{`<'} \tchar\ & \tint\ \\
\tint\ \fu{`<'} \tint\ & \tint\ \\
\tfloat\ \fu{`<'} \tfloat\ & \tint\
\end{operator}

Der Operator \fu{<} "uberpr"uft, ob der linke Operand kleiner als
der rechte ist. Falls dies zutrifft, ergibt der Ausdruck einen
Wert ungleich \fu{0}, ansonsten \fu{0}.

\begin{operator}
\tchar\ \fu{`>'} \tchar\ & \tint\ \\
\tint\ \fu{`>'} \tint\ & \tint\ \\
\tfloat\ \fu{`>'} \tfloat\ & \tint\
\end{operator}

Der Operator \fu{>} "uberpr"uft, ob der linke Operand gr"o"ser
ist als der rechte. Trifft das zu, ergibt der Ausdruck einen Wert
ungleich \fu{0}, andernfalls \fu{0}.

\begin{operator}
\tchar\ \fu{`<='} \tchar\ & \tint\ \\
\tint\ \fu{`<='} \tint\ & \tint\ \\
\tfloat\ \fu{`<='} \tfloat\ & \tint\
\end{operator}

Der Operator \fu{<=} "uberpr"uft, ob der linke Operand kleiner
oder gleich dem rechten Operanden ist und ergibt in diesem Fall
einen Wert ungleich \fu{0}, ansonsten \fu{0}.

\begin{operator}
\tchar\ \fu{`>='} \tchar\ & \tint\ \\
\tint\ \fu{`>='} \tint\ & \tint\ \\
\tfloat\ \fu{`>='} \tfloat\ & \tint\
\end{operator}

Der Operator \fu{>=} "uberpr"uft, ob der linke Operand gr"o"ser
oder gleich dem rechten Operanden ist und ergibt in diesem Fall
einen Wert ungleich \fu{0}, andernfalls \fu{0}.

\begin{operator}
\tchar\ \fu{`!='} \tchar\ & \tint\ \\
\tint\ \fu{`!='} \tint\ & \tint\ \\
\tfloat\ \fu{`!='} \tfloat\ & \tint\
\end{operator}

Der Operator \fu{!=} "uberpr"uft die beiden Operanden auf
Ungleichheit und ergibt in diesem Fall einen Wert ungleich \fu{0},
bei Gleichheit ergibt sich der Wert \fu{0}.

\begin{operator}
\type{type} \fu{`='} \type{type} & \type{type}
\end{operator}
\index{Zuweisung|ii}

Der Operator \fu{=} weist dem linken Operanden den Wert des
rechten Operanden zu, was aber vorraussetzt, da"s der linke
Operand keine \ind{Konstante}, also entweder eine \ind{Variable} (auch
innerhalb einer \ind{Struktur}), ein
dereferenzierter \ind{Zeiger} oder ein indiziertes \ind{Array} ist. SEL ist
so konzipiert, da"s auch ganze Arrays oder Strukturen mit einer
Zuweisung kopiert werden k"onnen. Beachten Sie aber bei der
Arbeit mit Zeigern, da"s eine Zuweisung nicht bedeutet, da"s die
Werte, auf die die Zeiger zeigen, sondern die Adressen, auf die
die Zeiger verweisen, kopiert werden. Zur Veranschaulichung ein
Beispiel: Angenommen, die Variablen \fu{p} und \fu{q} sind vom
Typ \type{\poi int} und zeigen auf die \tint s \fu{i} und \fu{j},
soda"s \fu{*p} gleich \fu{i} und \fu{*q} gleich \fu{j} ist. In
diesem Fall bewirkt der Ausdruck \fu{p = q} nicht, da"s nun der
Wert von \fu{i} gleich dem von \fu{j} ist, sondern da"s nun
\fu{p}, genauso wie \fu{q}, auf \fu{j} zeigt. Ersterer Effekt
w"are mit dem Ausdruck \fu{*p = *q} zu erzielen.

\begin{operator}
\type{struct} or \type{\poi struct} \fu{`->'} \fu{IDENTIFIER} & \type{type}
\end{operator}
\fuindex{IDENTIFIER}
\index{Struktur!Zugriff auf Element|ii}

Der Operator \fu{->} wird benutzt, um ein Strukturelement zu
erreichen. Um dies zu vereinfachen, kann der Operator sowohl
direkt mit einer Struktur als auch mit einem \ind{Zeiger} auf eine
Struktur arbeiten. Der rechte Operand gibt den Namen des Elements
an, auf das zugegriffen werden soll.

\begin{operator}
\type{\poi type} or \type{\arr{}type} \fu{`['} \fu{expr} \fu{`]'} & \type{type}
\end{operator}
\fuindex{expr}
\index{Array!Zugriff auf Element|ii}

Der \fu{[]}-Operator wird benutzt, um auf ein einzelnes
Element eines Arrays zuzugreifen, wobei dieses Array auch als
\ind{Zeiger} dargestellt werden kann. Auf diese Weise kann auch auf
einen Speicherbereich, der nicht als Array definiert worden ist,
wie auf ein Array zugegriffen werden. Hierzu wird in eckigen
Klammern ein Index angegeben, wobei das erste Element des Arrays
den Index \fu{0} besitzt und negative Indizes nicht erlaubt sind.
Anmerkung: Der Ausdruck \fu{p[i]} kann auch als \fu{*(p + i)}
angeschrieben werden, wobei \fu{p} ein Array oder ein Zeiger
eines beliebigen Typs ist und \fu{i} ein positiver \tint .
Beachten Sie, da"s SEL keine "Uberpr"ufung des Index vornimmt,
das hei"st, da"s es m"oglich ist, auf Elemente zuzugreifen, die
eigentlich nicht mehr zum Array geh"oren, wovon jedoch in den
meisten F"allen abzuraten ist.


\subsubsection{Rangfolge der Operatoren}
\index{Operator!Rangfolge|ii}
\index{Operator!Priorit""at|ii}

Um eine unn"otige Klammerung zu vermeiden, besitzen die
Operatoren in SEL eine Rangfolge. Die arithmetischen Operatoren
sind an die der Mathematik angelehnt, soda"s zum Beispiel der
Ausdruck \fu{a+b*c-d} dem Ausdruck \fu{(a+(b*c))-d} gleichkommt,
was der mathematischen Notation entspricht. Ist allerdings
gew"unscht, da"s \fu{a+b} mit \fu{c-d} multipliziert wird, mu"s
die nat"urliche Operatorenrangfolge mittels Klammerung
umgangen und der Ausdruck zu \fu{(a+b)*(c-d)} umgeformt
werden.

Die folgende Tabelle ist eine Aufstellung aller Operatoren von
SEL, wobei Operatoren mit gleicher Priorit"at in jeweils der
gleichen Zeile stehen. Die Operatoren, die in der Liste weiter
oben stehen, haben niedrigere Priorit"at, das hei"st, sie binden
schw"acher. Die rechte Spalte der Tabelle gibt die Richtung
der Auswertung
\index{Operator!Richtung der Auswertung|ii}
an und bezieht sich auf alle Operatoren in der jeweiligen Zeile.

\medskip
\begin{tabular}{ll}
\verb?=?, \verb?delete? & von rechts nach links \\
\verb?&&?, \verb?||?, \verb?^^? & von links nach rechts \\
\verb?==?, \verb?<?, \verb?>?, \verb?<=?, \verb?>=?, \verb?!=? &
  von links nach rechts\\
\verb?&?, \verb?|?, \verb?^?, \verb?<<?, \verb?>>? &
  von links nach rechts \\
\verb?+?, \verb?-? & von links nach rechts \\
\verb?*?, \verb?/?, \verb?%? & von links nach rechts \\
\verb?!?, \verb?~?, \verb?-? (un"ar), \verb?*? (un"ar), \verb?&? (un"ar) &
  von rechts nach links \\
\verb?->?, \verb?:?, \verb?[]? & von links nach rechts \\
\verb?sizeof?, \verb?new? & ---
\end{tabular}

\section{Typen}
\index{Typ|ii}
\fuindexii{type}
\rwindex{char}
\rwindex{int}
\rwindex{float}
\rwindex{void}
\fuindex{array}
\fuindex{pointer}
\fuindex{IDENTIFIER}

\begin{verbatim}
type:          `char'
             | `int'
             | `float'
             | `void'
             | array
             | pointer
             | IDENTIFIER
             ;
\end{verbatim}

Typen werden in SEL verwendet, um Werte
\index{Wert}
zu klassifizieren. So hat
z.~B. jeder \ind{Ausdruck}, jede \ind{Variable} und jede \ind{Funktion}
einen Typ. Die vier Basistypen von SEL k"onnen durch Kombination
zur Konstruktion weiterer, komplexerer Typen verwendet werden.

\subsection{Basistypen}
\index{Typ!Basistyp}
\typeindexii{int}
\typeindexii{char}
\typeindexii{float}
\typeindexii{void}

Die Programmiersprache SEL kennt vier Basistypen, n"amlich
\tchar , \tint , \tfloat\ und \tvoid . Die ersten beiden Typen,
\tchar\ und \tint\ sind sogenannte Aufz"ahlungstypen,
was sich mit den ganzen Zahlen in der Mathematik vergleichen l"a"st.

\tchar\ wird haupts"achlich verwendet, um \indi{Zeichen} zu
repr"asentieren und hat die Gr"o"se eines Bytes. Obwohl \tchar\
Zeichen repr"asentiert, k"onnen auf ihn alle arithmetischen
Operationen angewendet werden, die auch mit \tint\ funktionieren.

Letzterer wird meistens verwendet, um ganzzahlige Berechnungen
\index{Zahl!ganze|ii}
auszuf"uhren. Aus diesem Grund mu"s er gen"ugend gro"s
dimensioniert sein. In der Regel entspricht ein \tint\ 32 Bits,
was bedeutet, da"s er einen Wertebereich von $-2^{31}$ bis
$+2^{31}-1$ hat. Au"serdem wird \tint\ in SEL zur boolee'schen
Arithmetik
\index{boolee'sche Arithmetik}
herangezogen, also zur Berechnung von logischen
Aussagen. In diesem Fall bedeutet ein Wert von \fu{0} die Aussage
"`logisch falsch"', ein Wert ungleich \fu{0} bedeutet "`logisch
wahr"'.

Der Typ \tfloat\ repr"asentiert rationale Zahlen
\index{Zahl!rationale|ii}
mit begrenzter Genauigkeit. Auf MS-DOS-Rechnern hat er eine Gr"o"se
von 32 Bits.

Eine besondere Stellung unter den Typen nimmt \tvoid\ ein. Er
repr"asentiert nichts und es ist auch nicht m"oglich, einem
\tvoid\ einen Wert zuzuweisen, was auch gar keinen Sinn h"atte.
\tvoid\ wird z.~B. verwendet, um Funktionen zu
deklarieren, die keinen R"uckgabewert
\index{R""uckgabewert}
haben. Au"serdem kann er
in einer Funktion verwendet werden, die einen \ind{Zeiger} unbekannten
Typs verlangt oder zur"uckgibt.

\subsection{Arrays}
\index{Array|ii}
\fuindexii{array}
\fuindex{INT\protect{\_CONSTANT}}
\fuindex{type}

\begin{verbatim}
array:         `[' INT_CONSTANT `]' type
             ;
\end{verbatim}

Ein Array ist eine Zusammenfassung einer bestimmten Anzahl von
Variablen gleichen Typs, die im Speicher als eine Folge von
Variablen
\index{Variable!Folge|ii}
repr"asentiert wird, wobei die Variablen mit kleinerem
Index an einer Stelle geringerer Adresse angesiedelt sind.

In SEL ist es m"oglich, Arrays beliebig zu schachteln, also
mehrdimensionale Array zu erzeugen. So ist etwa
\type{\arr{5}\arr{10}int} ein Array, das aus 5 Arrays besteht,
wobei jedes dieser Arrays aus 10 \tint s besteht. Um auf das
zweite Element aus dem dritten Array zuzugreifen, wird der
Ausdruck \fu{a[3][2]} verwendet. Anmerkung: Dieses Array
beansprucht genausoviel Speicher wie $5 \cdot 10 = 50$ \tint s.

Arrays k"onnen nat"urlich nicht nur aus Elementartypen
konstruiert werden, sondern aus jedem beliebigen Typ, also z.~B.
aus Strukturen. Der Nachteil jedoch ist, da"s die
Gr"o"se des Arrays schon bei der Deklaration feststehen mu"s, sie
kann also nicht bei Bedarf ge"andert werden. Beachten Sie, da"s
SEL beim Zugriff auf Array-Elemente nicht "uberpr"uft, ob das
angegebene Element auch wirklich zum Array geh"ort, also ob das
Array gen"ugend gro"s dimensioniert ist.

\subsection{Zeiger}
\index{Zeiger|ii}
\fuindexii{pointer}
\fuindex{type}

\begin{verbatim}
pointer:       `*' type
             ;
\end{verbatim}

Ein Zeiger ist eine Variable, die die Adresse
\index{Variable!Adresse}
einer anderen
Variablen enth"alt, die von dem Typ ist, der dem \fu{*} in der
Typendeklaration folgt, und hat in der Regel dieselbe Gr"o"se wie
ein \tint . Bis auf die \ind{Zuweisung} sind alle Operationen
funktionsgleich mit der eines Arrays.
\index{Zeiger!""Aquivalenz zu Array}
\index{Array!""Aquivalenz zu Zeiger}

Zeiger k"onnen, wie Arrays,
beliebig geschachtelt werden, das hei"st, da"s es ohne weiteres
m"oglich ist, einen Zeiger auf einen Zeiger zu deklarieren und zu
benutzen. Beachten Sie allerdings, da"s ein zweidimensionales
Array nicht mit einem Zeiger auf einen Zeiger gleichgesetzt
werden kann. So ist z.~B. ein Array \type{\arr{10}int} in
der Verwendung mit einem Zeiger \type{\poi int} gleich (bis auf
die Ausnahme der Zuweisungsoperation), jedoch ist ein Array
\type{\arr{5}\arr{10}int} nicht gleich einem Zeiger \type{\poi
\poi int}, da der Typ, aus dem das Array besteht und auf den der
Zeiger zeigt, zu diesem Zweck gleich sein m"u"sten. In diesem Fall
ist jedoch der Typ, aus dem das Array aufgebaut ist, wieder ein
Array, n"amlich \type{\arr{10}int}. Der Typ, auf den der Zeiger
zeigt, ist jedoch \type{\poi int}. Um ein zweidimensionales Array
wie oben angeschrieben als Zeiger zu repr"asentieren, mu"s dieser
den Typ \type{\poi \arr{10}int} haben.

\subsubsection{Die new- und delete-Operatoren}
\label{newdelete}
\rwindexii{new}
\rwindexii{delete}

Um eine \ind{dynamische Allokation} von Variablen zu erm"oglichen,
existieren in SEL die Operatoren \rw{new} und \rw{delete}. Es
folgt eine kurze Beschreibung:

\begin{operator}
\fu{`new'} \fu{type} & \type{\poi type}
\end{operator}

Das Ergebnis der \rw{new}-Operation ist ein Zeiger auf eine
Variable des Typs, der als Operand angegeben wird. Diese Variable
wird dynamisch allokiert, das hei"st, sie wird dem \indi{Heap}
entnommen. Ist dieser schon so voll, da"s eine Variable des
angegebenen Typs keinen Platz mehr finden w"urde, ist das
Ergebnis ein \ind{Nullzeiger}. Beachten Sie, da"s der Speicherbereich
so lange belegt bleibt, bis \rw{delete} verwendet wird, um ihn
freizugeben, da SEL keine "`garbage collection"' benutzt.
Anmerkung: Um einen Speicherbereich zu allokieren, dessen Gr"o"se
nicht schon vor der Ausf"uhrung des Programms feststeht, ist die Funktion
\fui{MemAlloc} zu benutzen. Weiters ist es mittels der Funktion
\fui{MemReAlloc} m"oglich, einen bereits
allokierten Block zu vergr"o"sern bzw. zu verkleinern, wobei der
Inhalt des Blocks unver"andert bleibt (bei einer Verkleinerung
geht nat"urlich ein Teil des Inhaltes verloren).

\begin{operator}
\fu{`delete'} \type{\poi type} & \tvoid
\end{operator}

Der \rw{delete}-Operator gibt einen Speicherblock frei, der
vorher mit \rw{new} oder \fui{MemAlloc} allokiert wurde. Beachten
Sie, da"s dem Operator genau der Zeiger als Operand angeben
werden mu"s, den \rw{new} oder \fu{MemAlloc} (oder im Falle einer
Ver"anderung der Gr"o"se des Speicherbereichs \fui{MemReAlloc})
geliefert hat, da sonst die Speicherverwaltung ins Wanken kommen
kann, was in einem Programmabsturz resultieren k"onnte.

\subsection{Strukturen}
\index{Struktur|ii}
\fuindexii{struct}
\rwindexii{struct}
\fuindex{IDENTIFIER}
\fuindex{type}
\rwindex{end}

\begin{verbatim}
struct:        `struct' IDENTIFIER
                 {IDENTIFIER `:' type `;'}
               `end'
             ;
\end{verbatim}

Eine Struktur ist eine Zusammenfassung mehrerer Variablen
\index{Variable!Zusammenfassung mehrerer|ii}
beliebigen Typs
\index{Typ}
unter einem gemeinsamen Namen. Mittels des
Operators \fu{->} kann auf ein einzelnes Element der Struktur
zugegriffen werden.
\index{Struktur!Zugriff auf Element}
Dem \rw{struct}-Schl"usselwort folgt der
zuk"unftige Name der Struktur---bei der Deklaration von Variablen
wird dieser Name als Typ verwendet. Danach werden die einzelnen
Elemente der Struktur definiert. Die Namen dieser Elemente d"urfen
sich ohne weiteres mit anderen Variablennamen des Programms
decken, da durch den Operator \fu{->} eine eindeutige
Identifikation gew"ahrt ist, allerdings m"ussen sie innerhalb der
Struktur eindeutig sein. Die Elemente d"urfen beliebige Typen
haben, es ist sogar erlaubt, innerhalb einer Struktur einen
Zeiger auf ebendiese Struktur zu deklarieren, nicht aber ein
Array aus der Struktur und auch nicht die Struktur selbst, da
diese Deklarationen die Gr"o"se der Struktur ins Unendliche
anwachsen lie"sen. Die Gr"o"se einer Struktur ist in der Regel
gleich der Summe der Gr"o"sen ihrer Elemente.

Beachten Sie, da"s die Definition einer Struktur nur in dem \ind{Modul}
bekannt ist, in der sie get"atigt wird, selbst, wenn sie
innerhalb eines \rw{global}-Blocks steht. In der Regel ist es
deshalb zweckm"a"sig, Strukturen, die von mehreren Modulen
verwendet werden, in Include-Dateien
\index{Include-Datei}
zu definieren und diese in den
jeweiligen Modulen zu inkludieren.

\example
Die folgende Struktur ist ein Element einer verketteten \indu{Liste},
deren Daten die Adressen von Personen sind. Das Element \fu{Next}
zeigt auf das jeweils n"achste Element der Liste.

\index{Adresse@{\tt Adresse}~(\rw{struct})|ii}
\begin{verbatim}
struct Adresse
  Name    : char[26];
  Vorname : char[31];
  PLZ     : int;
  Stadt   : char[36];
  Strasse : char[36];
  Next    : *Adresse;
end
\end{verbatim}

\subsection{Zusammengesetzte Datentypen}
\index{Typ!zusammengesetzter|ii}

Nat"urlich ist es in SEL auch m"oglich, Datentypen aus den oben
besprochenen zu kombinieren, also zum Beispiel ein Array aus
Strukturen oder eine Struktur, die Arrays oder Zeiger beinhaltet,
zu definieren. Zu beachten ist nur, da"s die vorhandene
Speicherkapazit"at nicht "uberschritten wird. Unter MS-DOS
kommt weiterhin die 64~KByte-Barriere hinzu, die aus der
Segmentierung des Speichers im Real-Mode der 80x86-Prozessoren
resultiert. Deshalb ist es nicht m"oglich, in MS-DOS Variablen zu
definieren, die gr"o"ser als 64~KByte w"aren (sehr wohl ist es
m"oglich, Typen dieser Gr"o"se zu definieren, solange keine
Variablen dieses Typs verwendet werden).

\subsection{Der sizeof-Operator}
\index{Typ!Gr""o""se}
\rwindexii{sizeof}

\begin{operator}
\fu{`sizeof'} \type{type} & \tint\
\end{operator}

Der \rw{sizeof}-Operator verlangt als einzigen Operanden einen
Typ und gibt dessen Gr"o"se in Bytes zur"uck. Diese kann f"ur
ein und denselben Typen je nach Rechner verschieden sein, in der
Regel ist das jedoch nicht der Fall. Trotzdem sollten sie immer
\rw{sizeof} verwenden, wenn es um die Gr"o"se eines Typs geht und
nicht von einer bestimmten Gr"o"se ausgehen. Anmerkung: Der
einzige Typ, dessen Gr"o"se auf allen Rechnern gleich ist, ist
\tvoid\
\typeindex{void}
(nicht jedoch zum Beispiel \type{\poi void}). Seine
Gr"o"se betr"agt \fu{0}.

\subsection{Typecasting}
\index{Typecasting|ii}
\index{Typ!Umwandlung|see{Typecasting}}
\fuindex{type}

\begin{operator}
\type{\_type} \fu{`:'} \fu{type} & \type{type}
\end{operator}

Um Ausdr"ucke eines Typs in einen anderen umzuwandeln, bietet SEL
Typecasting. Der Operator \fu{:} wandelt seinen linken Operanden
in den Typ um, den der rechte Operand angibt. Dabei k"onnen
folgende Umwandlungen durchgef"uhrt werden:

\medskip
\begin{tabular}{ll}
Ausgangstyp & Zieltyp \\[0.5ex]
\tchar\ & \tint , \tfloat\ \\
\tint\ & \tchar , \tfloat , \type{\poi type}, \type{\arr{}type} \\
\tfloat\ & \tchar , \tint\ \\
\type{\poi type} & \tint , \type{\arr{}type} \\
\type{\arr{}type} & \tint , \type{\poi type}
\end{tabular}
\medskip

Bei der Umwandlung von \tint s
\typeindex{int}
in Arrays
\index{Array}
und \ind{Zeiger} und
umgekehrt ist zu beachten, da"s der \tint\ die Speicheradresse
angibt, auf die der Pointer zeigt, bzw. an der das Array beginnt
(die Adresse des ersten Arrayelements).

Zus"atzlich zum expliziten Typecasting f"uhrt SEL noch implizites
Typecasting durch, falls dies n"otig ist. Verlangt also eine
Funktion einen Parameter vom Typ \tfloat ,
\typeindex{float}
und es wird ein \tint\
"ubergeben, wird dieser automatisch in einen \tfloat\
umgewandelt. Beim impliziten Typecasting gelten wie beim
expliziten Typecasting die oben beschriebenen Regeln.

\section{Bezeichner}
\label{identifiers}
\index{Bezeichner|ii}
\fuindexii{IDENTIFIER}

\begin{verbatim}
IDENTIFIER: [A-Za-z_][A-Za-z0-9_]*
\end{verbatim}

Bezeichner werden in SEL f"ur Funktions-, Variablen- und
Strukturnamen
\index{Funktion!Name|see{Bezeichner}}
\index{Variable!Name|see{Bezeichner}}
\index{Struktur!Name|see{Bezeichner}}
verwendet. Ein Bezeichner mu"s mit einem Buchstaben
oder einem Unterstrich beginnen, danach folgt eine beliebige
Anzahl von Buchstaben, Ziffern oder Unterstrichen. So ist z. B.
\fu{\_xYz1} ein g"ultiger Bezeichner, nicht aber
\fu{1xy}, da er mit einer Ziffer beginnt. Die Bezeichner in SEL
sind {\em case-sensitiv}, das hei"st, es wird zwischen Gro"s-
und Kleinschreibung unterschieden. So ist z. B. \fu{counter} ein
anderer Bezeichner als \fu{Counter} oder \fu{COUNTER}. Beachten
Sie aber, da"s sie keines der reservierten W"orter
\index{reserviertes Wort|ii}
\index{Schl""usselwort|see{reserviertes Wort}}
als einen Bezeichner verwenden d"urfen. Der Vollst"andigkeit halber hier
eine Liste aller reservierten W"orter von SEL:

\medskip
\begin{tabular}{lllllll}
\bf begin & \bf break  & \bf case   & \bf char  & \bf default & \bf delete   & \bf downto \\
\bf else  & \bf end    & \bf extern & \bf float & \bf for     & \bf function & \bf global \\
\bf if    & \bf int    & \bf local  & \bf new   & \bf repeat  & \bf return   & \bf sizeof \\
\bf step  & \bf struct & \bf switch & \bf to    & \bf until   & \bf void     & \bf while  \\
\end{tabular}

\chapter{Der Pr\protect{"aprozessor}}
\index{Pr""aprozessor|ii}

Bevor ein SEL-Programm interpretiert bzw. compiliert wird, wird
es durch den Pr"aprozessor bearbeitet. Seine Hauptaufgaben sind
das Entfernen von Kommentaren, das Einf"ugen von Include-Dateien
und das Expandieren von Makros.

\section{Kommentare}
\index{Kommentar|ii}

Kommentare werden in SEL durch das Zeichen \fu{//} eingeleitet
und dauern bis an das Ende der jeweiligen Zeile an. Sollen also
l"angere Kommentare in den Quelltext eingef"ugt werden, mu"s am
Anfang jeder Kommentarzeile \fu{//} stehen. Beachten Sie bitte,
da"s \fu{//}, eingeschlossen in doppelte Anf"uhrungszeichen, nicht
den Anfang eines Kommentars, sondern eine String-Konstante
\index{Konstante!String}
darstellt.

\example
\fuindexii{Demo}
\fuindexuu{WriteInt}
\fuindexuu{WriteString}
\begin{verbatim}
function Demo (void) : void
begin
  WriteInt(20);        // Das ist ein Kommentar
  WriteString("Das // ist keiner\n");
end
\end{verbatim}

\section{Die \#define-Direktive}
\fuindexii{\#define}

Mittels der \fu{\#define}-Direktive ist es in SEL m"oglich,
Makros
\index{Makro|ii}
zu definieren. Ein Beispiel:

\begin{verbatim}
#define MACRO        3+5
\end{verbatim}

Tritt nun im nachfolgenden Quellcode irgendwo (au"ser innerhalb
einer String-Konstanten
\index{Konstante!String}
oder eines Kommentars) der Bezeichner
\fu{MACRO} auf, wird dieser vom Pr"aprozessor durch \fu{3+5}
ersetzt. Bei Makros wie dem obigen ist zu beachten, da"s ein
Makro nicht automatisch geklammert wird, d.~h., da"s der Ausdruck
\fu{MACRO*2} nicht das Ergebnis \fu{16}, sondern \fu{13}
(\fu{3+5*2}) hat.

Ist ein Makrotext zu lang, um ihn in einer Zeile unterzubringen,
kann er mit \fu{\char92}
\index{Backslash (\fu{\char92})|ii}
abgeteilt werden:

\begin{verbatim}
#define XYZ          if (x == y)     \
                       WriteInt(z);  \
                     end
\end{verbatim}

Beachten Sie, da"s die \fu{\#define}-Direktive immer am Beginn
einer Zeile stehen mu"s, und in dieser Zeile au"ser der Direktive, dem
Makronamen und dem einzusetzenden Text sonst nichts.

Makros mit Parametern, wie etwa in C, sind in der momentanen
Version von SEL nicht vorhanden, sind jedoch f"ur eine kommende
Version geplant.

\section{Die \#include-Direktive}
\index{Include-Datei|ii}
\fuindexii{\#include}

Mit der \fu{\#include}-Direktive ist es m"oglich, in eine
Quelldatei eine andere einzuf"ugen. Dabei mu"s auf \fu{\#include}
ein in Anf"uhrungszeichen gesetzter Dateiname folgen. Ebenso wie
die \fu{\#define}-Direktive mu"s auch die
\fu{\#include}-Direktive alleine am Anfang einer Zeile stehen.

\example
\fuindexuu{WriteString}
\begin{verbatim}
#include "conio.sh"

function main () : void
begin
  WriteString("Test\n");
end
\end{verbatim}

Hier wird an der Stelle der \fu{\#include}-Direktive der Inhalt
der Datei \fu{`conio.sh'} eingef"ugt. Sollte diese Datei selbst
\fu{\#include}-Direktiven enthalten, werden auch diese vom
Pr"aprozessor ausgef"uhrt, es ist also m"oglich,
\fu{\#include}-Direktiven zu verschachteln.

F"ur alle
Bibliotheksfunktionen von SEL gibt es Include-Dateien, die
deren Funktionsdeklarationen enthalten, soda"s diese nicht
jedesmal von Hand eingef"ugt werden m"ussen. Die Namen dieser
Dateien sind f"ur jede Funktion in der Funktionsreferenz
angegeben.

\chapter{Funktionsreferenz}
\label{funcref}

%%%%%%%%%%%%%%%%%%%%%%%%%%%%%%%%%%%%%%%%%%%%%%%%%

\begin{function}{MemAlloc}

\syntax
\#include "memory.sh"
function MemAlloc (s : int) : *void
\endgroup

\parameters
\param{s}{\tint} Gr"o"se des zu allokierenden Speicherbereichs \cr

\usage
Die Funktion \fu{MemAlloc} allokiert
\index{dynamische Allokation}
einen Speicherbereich mit
der Gr"o"se \fu{s} Bytes. Ist eine solche Allokation wegen
Speichermangels nicht m"oglich, gibt die Funktion einen
\ind{Nullzeiger} zur"uck. Im Gegensatz zum Operator \rwi{new} ist es mit
\fu{MemAlloc} auch m"oglich, einen Speicherbereich zu allokieren,
dessen Gr"o"se erst bei Laufzeit bestimmt wird. Der allokierte
Speicherbereich sollte, wenn er nicht mehr ben"otigt wird, mit
der Funktion \fu{MemFree} freigegeben werden.

\fseealso
\fu{MemFree}, \rw{new} (Kapitel \ref{newdelete})
\end{function}

%%%%%%%%%%%%%%%%%%%%%%%%%%%%%%%%%%%%%%%%%%%%%%%%%

\begin{function}{MemAvail}

\syntax
\#include "memory.sh"
function MemAvail () : int
\endgroup

\usage
Die Funktion \fu{MemAvail} gibt die Anzahl der noch freien Bytes
auf dem \ind{Heap} zur"uck, also jenes Speicherbereichs, der vom
Operator \rwi{new} (siehe Kapitel \ref{newdelete}) und der
Funktion \fu{MemAlloc} benutzt wird, um Speicher zu allokieren.

\fseealso
\fu{MemAlloc}, \rw{new} (Kapitel \ref{newdelete})
\end{function}

%%%%%%%%%%%%%%%%%%%%%%%%%%%%%%%%%%%%%%%%%%%%%%%%%
\newpage

\begin{function}{MemFree}

\syntax
\#include "memory.sh"
function MemFree (p : *void) : void
\endgroup

\parameters
\param{p}{\type{\poi void}} freizugebender Speicherblock \cr

\usage
Die Funktion \fu{MemFree} gibt einen Speicherblock, der mit
\fu{MemAlloc}, \fu{MemReAlloc} oder \rwi{new} allokiert worden
ist, frei. Beachten Sie jedoch, da"s Speicherbl"ocke, die mit
\rw{new} allokiert worden sind, besser mit \rwi{delete}
freigegeben werden sollten.

\fseealso
\fu{MemAlloc}, \fu{MemReAlloc}, \rw{delete}, \rw{new} (Kapitel \ref{newdelete})
\end{function}

%%%%%%%%%%%%%%%%%%%%%%%%%%%%%%%%%%%%%%%%%%%%%%%%%

\begin{function}{MemReAlloc}

\syntax
\#include "memory.sh"
function MemReAlloc (p : *void, s : int) : *void
\endgroup

\parameters
\param{p}{\type{\poi void}} allokierter Speicherblock \cr
\param{s}{\tint} neue Gr"o"se \cr

\usage
Die Funktion \fu{MemReAlloc} "andert die Gr"o"se des
Speicherblocks \fu{p},
\index{dynamische Allokation}
der mit \fu{MemAlloc}, \fu{MemReAlloc}
oder \rwi{new} allokiert worden sein mu"s, auf die Gr"o"se \fu{s}
Bytes. Ist \fu{p} ein Nullzeiger, wird ein Speicherbereich der
Gr"o"se \fu{s} wie mit \fu{MemAlloc} neu angelegt. Ist \fu{s}
gleich \fu{0}, dann wird der Speicherblock wie mit \fu{MemFree}
freigegeben. Andernfalls wird seine Gr"o"se ver"andert, wobei
sein Inhalt soweit wie m"oglich bestehen bleibt. Ist es aus
Speichermangel nicht m"oglich, die Gr"o"se des Blocks zu "andern,
gibt die Funktion einen \ind{Nullzeiger} zur"uck, und der alte Block
bleibt bestehen. Andernfalls gibt die Funktion einen Zeiger auf
den neuen Block zur"uck.

\fseealso
\fu{MemAlloc}, \fu{MemFree}, \rw{new} (Kapitel \ref{newdelete})
\end{function}

%%%%%%%%%%%%%%%%%%%%%%%%%%%%%%%%%%%%%%%%%%%%%%%%%

\begin{function}{Random}
\index{Zufallszahl|ii}

\syntax
\#include "stdlib.sh"
function Random (i : int) : int
\endgroup

\parameters
\param{i}{\tint} H"ochstgrenze f"ur die Zufallszahl \cr

\usage
Die Funktion Random gibt eine Zufallszahl
$0 \leq x \leq \fu{i}$ zur"uck. Um zu gew"ahrleisten, da"s
diese Zahl nicht f"ur jeden Aufruf des Programms dieselbe ist,
mu"s der Zufallszahlengenerator mit der Funktion \fu{Randomize}
initialisiert werden.

\fseealso
\fu{Randomize}
\end{function}

%%%%%%%%%%%%%%%%%%%%%%%%%%%%%%%%%%%%%%%%%%%%%%%%%
\newpage

\begin{function}{Randomize}
\index{Zufallszahl!Initialisierung des Generators|ii}

\syntax
\#include "stdlib.sh"
function Randomize () : void
\endgroup

\usage
Die Funktion \fu{Randomize} initialisiert den
Zufallszahlengenerator von SEL. Hierf"ur wird die aktuelle
Uhrzeit verwendet. Das garantiert, da"s bei jedem Aufruf des
Programms eine andere Reihe von Zufallszahlen generiert wird.

\fseealso
\fu{Random}
\end{function}

%%%%%%%%%%%%%%%%%%%%%%%%%%%%%%%%%%%%%%%%%%%%%%%%%

\begin{function}{ReadChar}
\typeindex{char}

\syntax
\#include "conio.sh"
function ReadChar () : char
\endgroup

\usage
Die Funktion \fu{ReadChar} liest ein einzelnes Zeichen von der
Systemkonsole ein, ohne es auf dem Bildschirm auszugeben. Dazu
kann die Funktion \fu{WriteChar} verwendet werden.

\fseealso
\fu{WriteChar}
\end{function}

%%%%%%%%%%%%%%%%%%%%%%%%%%%%%%%%%%%%%%%%%%%%%%%%%

\begin{function}{ReadFloat}
\typeindex{float}

\syntax
\#include "conio.sh"
function ReadFloat () : float
\endgroup

\usage
\fu{ReadFloat} liest einen \tfloat\ in Dezimal- bzw.
\indu{Exponentialschreibweise} (z.~B. \fu{1.23E45}) von der
Systemkonsole ein. Backspaces werden richtig interpretiert,
soda"s der Anwender eine Korrekturm"oglichkeit hat.

\fseealso
\fu{WriteFloat}
\end{function}

%%%%%%%%%%%%%%%%%%%%%%%%%%%%%%%%%%%%%%%%%%%%%%%%%

\begin{function}{ReadInt}
\typeindex{int}

\syntax
\#include "conio.sh"
function ReadInt () : int
\endgroup

\usage
\fu{ReadInt} liest einen \tint\ als Dezimalzahl
\index{Zahl!dezimale}
von der
Systemkonsole ein. Dabei werden Backspaces ber"ucksichtigt,
um dem Anwender eine Korrekturm"oglichkeit zu geben. Zu beachten
ist allerdings, da"s kein positives Vorzeichen (\fu{+}) angegeben
werden darf.

\fseealso
\fu{WriteInt}
\end{function}

%%%%%%%%%%%%%%%%%%%%%%%%%%%%%%%%%%%%%%%%%%%%%%%%%
\newpage

\begin{function}{ReadString}
\index{String}

\syntax
\#include "conio.sh"
function ReadString (s : *char) : int
\endgroup

\parameters
\param{s}{\type{\poi char}} Speicherplatz f"ur den String \cr

\usage
Die Funktion \fu{ReadString} liest einen nullterminierten String
von der Systemkonsole in den Speicherbereich, den der Parameter
\fu{s} angibt, ein. Dieser Speicherbereich mu"s gen"ugend gro"s
dimensioniert sein, da \fu{ReadString} keinerlei Tests vornimmt,
ob er f"ur den String ausreichend ist. Die Stringeingabe wird vom
Anwender durch ein Newline-Zeichen beendet, das jedoch im String
nicht aufscheint. Die Funktion gibt die Anzahl der gelesenen
Zeichen, inklusive dem Newline, also mindestens \fu{1}, zur"uck.

\fseealso
\fu{WriteString}
\end{function}

%%%%%%%%%%%%%%%%%%%%%%%%%%%%%%%%%%%%%%%%%%%%%%%%%

\begin{function}{WriteChar}
\typeindex{char}

\syntax
\#include "conio.sh"
function WriteChar (c : char) : void
\endgroup

\parameters
\param{c}{\tchar} auszugebendes Zeichen \cr

\usage
Die Funktion \fu{WriteChar} gibt das ihr "ubergebene Zeichen
\fu{c} auf der Systemkonsole aus, wobei Steuerzeichen (siehe
Kapitel \ref{specialchar}) ber"ucksichtigt werden.

\fseealso
\fu{ReadChar}
\end{function}

%%%%%%%%%%%%%%%%%%%%%%%%%%%%%%%%%%%%%%%%%%%%%%%%%

\begin{function}{WriteFloat}
\typeindex{float}

\syntax
\#include "conio.sh"
function WriteFloat (f : float) : void
\endgroup

\parameters
\param{f}{\tfloat} auszugebender Wert \cr

\usage
Die Funktion \fu{WriteFloat} gibt einen \tfloat\ in passender
Schreibweise (Dezimal- oder \ind{Exponentialschreibweise}) auf der
Systemkonsole aus.

\fseealso
\fu{ReadFloat}
\end{function}

%%%%%%%%%%%%%%%%%%%%%%%%%%%%%%%%%%%%%%%%%%%%%%%%%
\newpage

\begin{function}{WriteInt}
\typeindex{int}

\syntax
\#include "conio.sh"
function WriteInt (i : int) : void
\endgroup

\parameters
\param{i}{\tint} auszugebender Wert \cr

\usage
Die Funktion \fu{WriteInt} gibt den Parameter \fu{i} als
Dezimalzahl
\index{Zahl!dezimale}
auf der Systemkonsole aus.

\fseealso
\fu{ReadInt}
\end{function}

%%%%%%%%%%%%%%%%%%%%%%%%%%%%%%%%%%%%%%%%%%%%%%%%%

\begin{function}{WriteString}
\index{String}

\syntax
\#include "conio.sh"
function WriteString (s : *char) : void
\endgroup

\parameters
\param{s}{\type{\poi char}} auszugebender String \cr

\usage
Die Funktion \fu{WriteString} gibt einen nullterminierten String
auf der Systemkonsole aus. Steuerzeichen (siehe Kapitel
\ref{specialchar}) werden dabei interpretiert.

\fseealso
\fu{ReadString}
\end{function}

%%%%%%%%%%%%%%%%%%%%%%%%%%%%%%%%%%%%%%%%%%%%%%%%%

\newpage
\index{Index|ii}
% \addcontentsline{toc}{chapter}{Index}
\footnotesize
\printindex

\end{document}


\begin{document}

\begin{titlepage}
\vspace*{2cm}
\begin{flushleft}
\inch SEL\rm \hfill Version 0.3
\end{flushleft}
\rule{\textwidth}{2mm}
\begin{flushright}
\huge{\bf Kurzreferenz}\\[0.5ex]
\end{flushright}
\vspace{13.5cm}
\begin{flushleft}
\large \bf Mark Probst\\
\rule[2mm]{\textwidth}{1mm}
\end{flushleft}
\end{titlepage}

\tableofcontents

\newpage

\chapter{\protect{"Uber} diesen Text}

Dieser Text hat zum Ziel, die Programmiersprache SEL (Schani
Electronics Language) in ihrer momentanen Version 0.3
vollst"andig zu beschreiben. Dabei ist er als Kurzreferenz
aufgebaut, kann aber auch zur Not verwendet werden, um die
Sprache SEL kennenzulernen, vor allem von Programmierern, die
bereits {\sc Pascal}
\index{Pascal@{\sc Pascal}}
oder \ind{C}
kennen, da SEL, verglichen mit diesen Sprachen, kaum etwas Neues
bietet.

Die einzigen Implementationen von SEL, die zur Zeit verf"ugbar
sind, sind die beiden \ind{Interpreter} unter MS-DOS und Windows
NT. Die Funktionsreferenz (Kapitel \ref{funcref}) spiegelt deren
momentanen Stand in Bezug auf verf"ugbare Funktionen wider.

\section{Typographie}
\index{Typographie|ii}

Um Mi"sverst"andnisse zu vermeiden, werden in diesem Text
verschiedene Schriftarten f"ur verschiedene Zwecke verwendet.
Durch eine {\em kursive Schriftart} werden neue Begriffe
gekennzeichnet und wichtige Textabschnitte hervorgehoben.

Schl"usselw"orter der Sprache SEL werden in {\bf fetter Schrift}
geschrieben. Eine Aufstellung aller Schl"usselw"orter finden Sie
in Kapitel \ref{identifiers}. Operatoren hingegen werden in
{\tt Schreibmaschinenschrift} geschrieben.

Typen werden nicht nur in fetter Schrift geschrieben, sondern
zus"atzlich in geschwungene \type{Klammern} gesetzt, um sie von
Schl"usselw"ortern unterscheiden zu k"onnen.

Bezeichner, Makros, Funktionsnamen und Nonterminals der
SEL-Grammatik werden ebenfalls in
{\tt Schreib"-maschi"-nen"-schrift} geschrieben.

Indexeintr"age
\index{Index|ii}
\index{Stichwortverzeichnis|see{Index}}
sind wie folgt zu interpretieren: Kursive
Eintr"age wie etwa {\em 23} deuten auf einen Haupteintrag, wo die
meisten Informationen zu dem jeweiligen Thema zu finden sind.
Unterstrichene Eintr"age wie z.~B. \underline{31} bedeuten, da"s
auf dieser Seite ein Beispiel zum Thema zu finden ist, das jedoch
nicht unbedingt n"aher erl"autert wird.

\section{Die Backus-Naur-Form}
\index{Backus-Naur-Form|ii}
\index{BNF|see{Backus-Naur-Form}}

Die Backus-Naur-Form (kurz BNF) wird verwendet, um Grammatiken
darzustellen. Dabei besteht jede \indi{Grammatik} aus mehreren
Symbolen, die ihrerseits jeweils aus mindestens einer
\indi{Produktion} bestehen. Symbole werden links angeschrieben.
Symbole rechts davon, durch einen Doppelpunkt oder ein anderes
Zeichen vom Grammatiksymbol getrennt, jeweils durch einen
vertikalen Strich (\fu{|}) voneinander getrennt. Auf die letzte
Produktion eines jeden Symbols folgt ein Semikolon (\fu{;}).

Produktionen sind Aneinanderreihungen von Terminal- und
Nichtterminalsymbolen. Nichtterminalsymbole sind jene, die auf
den linken Seiten stehen. Terminalsymbole sind solche, die in
Quelltexten der Sprache direkt vorkommen k"onnen, und solche,
die durch regul"are Ausdr"ucke definiert werden (siehe n"achstes
Unterkapitel). Erstere werden unter Hochkomma angef"uhrt (wie
etwa \fu{'new'}), w"ahrend letztere durch Gro"sschreibung
gekennzeichnet sind.

Ein Satz einer Sprache mu"s nun so konstruiert sein, da"s er
durch Ersetzen von Nichtterminalsymbolen durch bestimmte Produktionen
derselben hergeleitet werden kann.

\newpage
\example
\begin{verbatim}
satz:         subjekt praedikat
            ;
subjekt:      artikel substantiv
            ;
artikel:      `der'
            | `die'
            | `das'
            ;
substantiv:   `frau'
            | `mann'
            | `kind'
            ;
pradikat:     `geht'
            | `schlaeft'
            | `isst'
            ;
\end{verbatim}

Diese Grammatik hat f"unf Nonterminalsymbole und kann einige
primitive deutsche S"atze herleiten wie etwa "`der mann
schlaeft"'. Dabei geht man vom Symbol \fu{satz} aus und setzt
f"ur die Nonterminals \fu{subjekt} und \fu{praedikat} jeweils
eine ihrer Produktionen ein. F"ur \fu{subjekt} wird \fu{artikel}
und \fu{substantiv} eingesetzt und f"ur diese wiederum die beiden
Terminals \fu{`der'} und \fu{`mann'}. Dasselbe geschieht mit
\fu{pradikat}.

Dieses Beispiel demonstriert nicht nur, wie sich S"atze von einer
BNF herleiten lassen, sondern auch, da"s diese nur f"ur den
syntaktischen Teil der Grammatik eingesetzt werden kann. So
l"a"st sich aus obiger BNF auch der Satz "`der kind isst"'
herleiten, der semantisch inkorrekt ist. Deshalb wird der
semantische
\index{Semantik}
Teil in der Regel verbal formuliert, wie etwa in
dieser Kurzreferenz.

Zus"atzlich zu dieser Grundfunktionalit"at beherrscht die BNF
noch zwei zus"atzliche Hilfsmittel zur Formulierung von
Grammatiken: Wird ein Teil einer Produktion in geschwungene Klammern
eingeschlossen (\fu{\{\}}), dann darf dieser Teil im Quelltext
weggelassen oder beliebig oft aufgef"uhrt werden. Durch die Produktion
\fu{`a' \{`b'\} `a'} werden also alle S"atze \fu{aa},
\fu{aba}, \fu{abba}, \fu{abbba}, \ldots\ dargestellt.

Zus"atzlich k"onnen Teile von Produktionen in eckige Klammern
eingeschlossen werden, was bedeutet, da"s sie optional sind, also
im Quelltext weggelassen k"onnen, aber auf keinen Fall "ofter als
einmal auftreten d"urfen. Die Produktion \fu{`a' [`b'] `a'}
beschreibt also die S"atze \fu{aa} und \fu{aba}.

\section{Regul\protect{"are} Ausdr\protect{"ucke}}
\index{regul""arer Ausdruck|ii}

Regul"are Ausdr"ucke werden benutzt, um Zeichenfolgen
wie etwa Bezeichner oder Zahlen zu beschreiben. Dabei werden
aufeinanderfolgende Zeichen hintereinander angeschrieben.
Zus"atzlich gibt es Steuerzeichen, mit denen es m"oglich ist, die
beschriebene Zeichenfolge zu variieren. Es folgt eine
Aufstellung:

\begin{tabular}{lp{13cm}}
\fu{()} & Die beiden runden Klammern werden benutzt, um eine
          Folge von Zeichen zu klammern, sie also nur als ein Zeichen zu
          behandeln. Das macht es m"oglich, Steuerzeichen wie etwa \fu{+}
          auf eine ganze Zeichenfolge anstatt nur auf das letzte Zeichen
          anzuwenden. \\
\fu{[]} & Die beiden eckigen Klammern werden verwendet, um
          Zeichen zu gruppieren. Der ganze Klammerausdruck steht
          f"ur ein beliebiges Zeichen, das innerhalb der Klammer
          steht. Um auch gr"o"sere Gruppen anschreiben zu
          k"onnen, ist es m"oglich, innerhalb der Gruppe mittels
          \fu{-} von einem bestimmten bis zu einem anderen
          bestimmten Zeichen alle Zeichen einzubinden. So gibt
          z.~B. \fu{[0-9]} die Gruppe aller Ziffern an. Tritt
          diese Gruppe irgendwo auf, steht sie f"ur eine
          beliebige Ziffer. Wird als erstes Zeichen einer Gruppe
          \fu{\char94} angegeben, umfa"st die Gruppe all jene
          Zeichen, die nicht innerhalb der Klammern aufgef"uhrt
          sind. \\
\end{tabular}          

\begin{tabular}{lp{13cm}}
\fu{`'} & Sollen Zeichen verwendet werden, die schon als
          Steuerzeichen vergeben sind, werden sie in einfache
          Anf"uhrungszeichen eingeschlossen. \\
\fu{.}  & Der Punkt steht f"ur jedes beliebige Zeichen au"ser dem
          Zeilenwechsel \fu{\char92 n}. \\
\fu{|}  & Ein Ausdruck \fu{a|b} kann entweder durch \fu{a} oder
          durch \fu{b} ersetzt werden. Um zu vermeiden, da"s
          immer der gesamte Ausdruck mit ODER verkn"upft wird, werden
          Oder-Ausdr"ucke in Klammern gesetzt. \\
\fu{?}  & Ein Zeichen oder ein Ausdruck in Klammern, gefolgt von
          einem \fu{?}, ist optional, darf also auftreten, kann
          aber auch weggelassen werden. \\
\fu{+}  & Ein Zeichen oder ein Ausdruck in Klammern, gefolgt von
          einem \fu{+}, steht f"ur ein oder mehrere Vorkommnisse
          dieses Zeichens oder Ausdrucks. \\
\fu{*}  & Der Operator \fu{*} ist mit dem Operator \fu{+}
          identisch, au"ser in dem Fall, da"s das vorhergehende Zeichen
          bzw. der in Klammern stehende Ausdruck auch weggelassen
          werden kann. \\
\end{tabular}

\example
\begin{verbatim}
IDENTIFIER: [a-zA-Z][^\n ]*
\end{verbatim}

Dieser regul"are Ausdruck beschreibt all jene Bezeichner, die mit
einem Buchstaben beginnen, auf den beliebige Zeichen
(au"ser dem Zeilenwechsel \fu{\char92 n} und dem Leerzeichen)
folgen k"onnen. G"ultige \fu{IDENTIFIER} sind etwa \fu{ax*} und
\fu{ZX\#\%}.

\begin{verbatim}
HEX_NUMBER: 0x[0-9a-fA-F]+
\end{verbatim}

Der Ausdruck \fu{HEX\_NUMBER} beschreibt eine hexadezimale
Zahlenkonstante,
\index{Zahl!hexadezimale}
wie sie etwa in \ind{C} verwendet werden kann. Sie beginnt mit
den Zeichen \fu{0x}, auf die mindestens eine Hexadezimalziffer
(Dezimalziffer oder Gro"s- bzw. Kleinbuchstabe von \fu{A} bis
\fu{F}) folgen mu"s. Zu der unendlich gro"sen Menge der
beschriebenen Zahlen z"ahlen \fu{0xff} und \fu{0x123}.

\chapter{Sprachreferenz}

Dieses Kapitel ist eine Referenz "uber die komplette Grammatik
\index{Grammatik!von SEL|ii}
der Programmiersprache SEL. Der Aufbau ist dabei top-down-orientiert,
da es meines Erachtens nat"urlicher ist, vom Gro"sen ins Kleine
zu gehen, als umgekehrt.

Das erste Unterkapitel widmet sich deshalb der groben Struktur
eines SEL-Programms. Es wird die Modularit"at von SEL
besprochen sowie der grundlegende Aufbau einer Funktion. Weiters
wird in diesem Unterkapitel die wichtigste Funktion eines
SEL-Programms beschrieben, n"amlich die \fu{main}-Funktion.

Das zweite Unterkapitel bespricht die Kontrollstrukturen, also
die Verzweigungen und Schleifen, die die strukturierte
Programmiersprache ausmachen.

Das dritte Unterkapitel widmet sich den Ausdr"ucken oder
{\em expressions}, die den Hauptanteil an einem Programm haben. Zu
Ausdr"ucken z"ahlen zum Beispiel Zuweisungen, Vergleiche oder
arithmetische Operationen.

Im vierten Unterkapitel werden die Datentypen besprochen, die
mehr oder weniger Bestandteile der Ausdr"ucke sind.

\section{Struktur eines Programms}
SEL ist eine modulare
\index{Modul}
Programmiersprache, d.h. ein Programm kann sich (mu"s sich aber
nicht) aus mehreren Modulen---oder Quelldateien---zusammensetzen.
Zus"atzlich kann jedes Modul seinen internen Aufbau vor dem Rest
des Programms sch"utzen, um Implementationsdetails zu verbergen
(\indi{Kapselung}), damit diese jederzeit ge"andert werden
k"onnen, ohne da"s der aufrufende Programmteil dazu ver"andert
werden mu"s. Dies setzt jedoch auch ein vern"unftiges
"`\ind{Interface}"' (Schnittstelle zwischen dem Modul und dem
Programm) voraus. Es wird also in der Regel n"otig sein, vor der
Implementation einen Entwurf des Programms zu machen---einfaches
"`Drauflosprogrammieren"' wird sich schnell als Sackgasse
erweisen.

\subsection{Das Modul}
\index{Modul|ii}
\fuindexii{module}
\fuindexii{block}
\fuindex{function}
\fuindex{local}
\fuindex{global}

\begin{verbatim}
module:        {block}
             ;

block:         function
             | local
             | global
             ;
\end{verbatim}

Jedes Modul besteht aus einem oder mehreren Bl"ocken,
\index{Block|ii}
die in Funktionen, \rwi{local}- und \rwi{global}-Bl"ocke
eingeteilt werden (es ist theoretisch auch m"oglich, ein Modul zu
schreiben, das keinen einzigen Block enth"alt, doch dieses Modul
h"atte keinen anderen Zweck, als wertvollen Speicherplatz zu
verschwenden).

Funktionen sind Programmteile---oder Unterprogramme---, die von
bestimmten Stellen im Programm aufgerufen werden k"onnen. Da SEL
eine modulare Programmiersprache ist, ist es m"oglich, Funktionen
zu kapseln, sie also nur in dem Modul verf"ugbar zu machen, in
dem sie definiert wurden.

Die \rw{local}-Bl"ocke dienen dazu, Variablen,
\index{Variable!modulglobale}
Funktionen
\index{Funktion!modulglobale}
und Strukturen
\index{Struktur}
zu definieren, die nur innerhalb des jeweiligen Moduls
erreichbar sein sollen.

Die \rw{global}-Bl"ocke dienen einerseits dem Zweck, Variablen,
\index{Variable!programmglobale}
Funktionen
\index{Funktion!programmglobale}
und Strukturen zu definieren, die auch au"serhalb des Moduls
verf"ugbar sein sollen, und andererseits jene Variablen,
Strukturen und Funktionen zu deklarieren, die nicht in diesem
Modul definiert sind und trotzdem verwendet werden sollen.
Beachten Sie bitte, da"s diese in einem anderen Modul
als global definiert sein m"ussen.

\subsection{Funktionen}
\index{Funktion|ii}
\index{Block}
\fuindexii{function}
\fuindex{type}
\fuindex{statement}
\fuindex{IDENTIFIER}
\rwindex{begin}
\rwindex{end}
\fuindexii{formal\protect{\_params}}
\fuindexii{func\protect{\_var}}
\fuindex{expr}
\fuindexii{func\protect{\_proto}}
\rwindexii{function}

\begin{verbatim}
function:      `function' IDENTIFIER `(' [formal_params] `)' `:' type
                 {func_var}
               `begin'
                 {statement}
               `end'
             ;

formal_params: IDENTIFIER `:' [`&'] type [`,' formal_params]
             ;

func_var:      IDENTIFIER `:' type [`=' expr] `;'
             ;

func_proto:    `function' IDENTIFIER
               `(' [[`&'] type {`,' [`&'] type}] `)' `:' type
             ;
\end{verbatim}

Eine Funktion ist ein Unterprogramm,
\index{Unterprogramm|see{Funktion}}
also ein Programmteil, der
von anderen Stellen des Programms aufgerufen werden kann. Dabei
k"onnen einer Funktion ein oder mehrere \indi{Parameter} (deren
Typen bei der Deklaration der Funktion festgelegt werden)
"ubergeben werden. Innerhalb dieser Funktion k"onnen diese
Parameter als ganz normale Variablen angesprochen werden. Die
Namen dieser Variablen werden im Funktionskopf angegeben.

Funktionsdeklarationen k"onnen an zweierlei Stellen angetroffen
werden, n"amlich dort, wo die Funktionen definiert sind, und
in \rw{local}- und \rw{global}-Bl"ocken, wo allerdings nur der
Funktionskopf ohne Parameternamen angegeben wird. Zu beachten
ist, da"s Funktionen im Programm erst nach dem ersten Vorkommen
ihrer Deklarationen oder Definitionen bekannt sind. Eine Funktion
kann also nicht aufgerufen werden, wenn sie im Quellcode nicht
vorher deklariert oder definiert wurde.

Funktionen sind, solange sie nicht explizit in einem
\rw{global}-Block des jeweiligen Moduls aufgef"uhrt sind,
gekapselt, also auf das Modul beschr"ankt, in dem sie
definiert wurden. Trotzdem ist es m"oglich, Funktionen in einem
\rw{local}-Block des
Moduls aufzuf"uhren, f"ur den Fall, da"s eine Funktion an einer
Stelle aufgerufen werden mu"s, die vor ihrer Definition liegt.

Zur Semantik: Dem \rw{function}-Schl"usselwort folgt der Name der
Funktion, der wie jeder Bezeichner case-sensitiv ist. In Klammern
gesetzt folgen darauf die Namen und Typen der einzelnen Parameter.
Innerhalb der Funktion kann auf die Parameter mit Hilfe ihrer
Namen wie auf normale Variablen zugegriffen werden.

SEL unterst"utzt zwei Arten der Parameter"ubergabe:
\index{Parameter""ubergabe|ii}
\index{call-by-value|see{Parameter""ubergabe}}
\index{call-by-reference|see{Parameter""ubergabe}}
{\em call-by-value} und {\em call-by-reference}. Ein
Parameter, der als call-by-value deklariert ist, wird der
Funktion als Wert "ubergeben, es wird beim Funktionsaufruf also
zuerst der Wert des Parameters berechnet und dieser dann kopiert.
Die jeweilige Funktion greift dann auf die Kopie des Parameters
zu, kann also die Variable, die "ubergeben wurde, nicht
ver"andern. Von einem als call-by-reference deklarierten Parameter
(dieser ist daran zu erkennen, da"s seinem Typ im Funktionskopf
das \fu{\&}-Zeichen
\index{Referenztyp|ii}
vorangestellt ist) hingegen wird der Funktion intern die Adresse
der als Parameter "ubergebenen Variable zug"anglich gemacht,
wovon die Funktion allerdings nichts merkt. Dies wirkt sich
dahingehend aus, da"s die Funktion durch "Anderung des Wertes des
Parameters auch den Wert der Variable ver"andert, die ihr als
Parameter "ubergeben wurde. Beachten Sie allerdings, da"s es
nicht m"oglich ist, eine \ind{Konstante} f"ur einen
call-by-reference-Parameter zu "ubergeben.

Nach der Parameterliste folgt der R"uckgabetyp
\index{R""uckgabetyp|ii}
der Funktion. Das ist der Typ, den der Aufrufer der Funktion
zur"uckgeliefert bekommt. Der R"uckgabewert wird durch die
\rwi{return}-Anweisung bestimmt. Ist keine R"uckgabe n"otig, kann
der Typ \tvoid\
\typeindex{void}
verwendet werden.

Im Anschlu"s an den R"uckgabetyp k"onnen lokale
Variablen
\index{Variable!lokale|ii}
vereinbart werden. Diese Variablen sind---der Name
sagt es bereits---nur innerhalb der Funktion verf"ugbar, in der
sie deklariert wurden. Gleichzeitig k"onnen diesen Variablen
Werte zugewiesen werden. Geschieht dies nicht, sind die Werte
dieser Variablen undefiniert. Beachten Sie, da"s Variablen
auch mit berechneten Werten initialisiert werden k"onnen.

Nach der Vereinbarung der lokalen Variablen folgt in
\rwi{begin}--\rwi{end}
eingeschlossen der eigentliche Code der
Funktion. Von hier aus k"onnen wieder andere Funktionen
aufgerufen werden, die Funktion kann sich sogar selbst aufrufen
(\ind{Rekursion}).

\example
Die Funktion \fuii{Factorial} berechnet die Fakult"at
\index{Fakult""at|ii}
ihres Parameters \fu{i}
(\hbox{$\fu{i}! = \prod_{n=1}^{\fu{\footnotesize i}} n$}).
Hierzu verwendet sie zwei lokale Variablen \fu{c} und \fu{r}.

\rwindexuu{global}
\rwindexuu{for}
\begin{verbatim}
global
  function Factorial (int) : int
end

function Factorial (i : int) : int

  c : int;
  r : int = 1;

begin
  for (c = 2 to i)
    r = r * c;
  end
  return r;
end
\end{verbatim}

\subsubsection{Die main-Funktion}
Die wichtigste \ind{Funktion} in einem SEL-Programm ist die Funktion
\fuii{main}. Sie stellt gewisserma"sen das Hauptprogramm dar und
ist somit die erste Funktion, die in einem SEL-Programm aufgerufen
wird. Hat \fu{main} ihre Arbeit beendet, kehrt der Computer zum
aufrufenden Programm (in der Regel der Kommandozeileninterpreter)
zur"uck. Dabei kann letzterem ein sogenannter {\em Exit-Code}
"ubergeben werden. Dieser \indi{Exit-Code} hat den Typ \tint\
\typeindex{int}
und wird mit der \rwi{return}-Anweisung zur"uckgegeben. Ist der
R"uckgabetyp der Funktion \fu{main} \tvoid ,
\typeindex{void}
so ist der Exit-Code \fu{0}.

Zu beachten ist, da"s die \fu{main}-Funktion immer als global
deklariert werden mu"s, unerheblich, in welchem Modul sie
definiert ist.

\example
Dieses {\em hello-world}-Programm
\index{hello-world|uu}
besteht aus einem
einzigen Modul und hat die Aufgabe, den String "`\fu{hello, world!}"'
auszugeben. Zus"atzlich wird der Exit-Code \fu{1} zur"uckgeliefert.

\fuindexuu{WriteString}
\rwindexuu{global}
\begin{verbatim}
#include "conio.sh"

global
  function main () : int
end

function main () : int
begin
  WriteString("hello, world!\n");
  return 1;
end
\end{verbatim}

\subsubsection{Die return-Anweisung}
Die \rwii{return}-Anweisung hat den Zweck, aus einer \ind{Funktion} in
die "ubergeordnete Funktion (bzw. das Betriebssystem im Falle der
\fu{main}-Funktion) zur"uckzukehren. Ist der R"uckgabetyp der
jeweiligen Funktion nicht \tvoid ,
\typeindex{void}
so mu"s auch ein \ind{Wert}
zur"uckgegeben werden. Dieser wird der \rw{return}-Anweisung
nachgestellt. \rw{return} kann an einer beliebigen Stelle
innerhalb einer Funktion stehen, Schleifen
\index{Schleife} werden---falls n"otig---abgebrochen.

\example
Die Funktion \fuii{Square} gibt
das \ind{Quadrat} eines ihr "ubergebenen \tint\ \fu{i} zur"uck.

\begin{verbatim}
function Square (i : int) : int
begin
  return i * i;
end
\end{verbatim}

\subsection{local-Bl\protect{"ocke}}
\rwindexii{local}
\rwindex{end}
\index{Block}
\fuindexii{local}
\fuindexii{local\protect{\_decl}}
\fuindex{struct}
\fuindex{func\protect{\_proto}}
\fuindex{var}

\begin{verbatim}
local:         `local'
                 {local_decl}
               `end'
             ;

local_decl:    struct
             | func_proto
             | var
             ;
\end{verbatim}

Die \rw{local}-Bl"ocke dienen dazu, Variablen,
\index{Variable!modulglobale|ii}
Strukturen
\index{Struktur}
und Funktionen
\index{Funktion!modulglobale|ii}
zu definieren, die nur in dem jeweiligen Modul
bekannt sein sollen. Dieses Vorgehen nennt sich \ind{Kapselung} und hat den
Vorteil, da"s es m"oglich ist, die Implementationsdetails eines
Moduls vor dem Rest des Programms zu verstecken. Das macht es
einerseits m"oglich, das Modul anders zu implementieren, ohne
da"s das Restprogramm in irgendeiner Weise ge"andert werden mu"s
und andererseits, da"s der Code des Moduls in anderen Programmen
leichter wiederverwendet werden kann. F"ur eine solche Vorgangsweise
ist es jedoch unabdingbar, f"ur jedes Modul ein gut "uberdachtes
\ind{Interface} (Schnittstelle zwischen Modul und Restprogramm) zu
verwenden.

Variablen, Strukturen und Funktionen sind im Source erst ab der
Stelle ihrer Deklaration bekannt. Aus diesem Grunde ist es
ratsam, \rw{local}-Bl"ocke an den Anfang der Datei zu stellen,
was au"serdem die "Ubersichtlichkeit des Codes erh"oht.

\subsection{global-Bl\protect{"ocke}}
\rwindexii{global}
\rwindex{end}
\index{Block}
\fuindexii{global}
\fuindexii{global\protect{\_decl}}
\fuindex{struct}
\fuindex{func\protect{\_proto}}
\fuindex{var}
\fuindex{extern}

\begin{verbatim}
global:        `global'
                 {global_decl}
               `end'
             ;

global_decl:   struct
             | func_proto
             | var
             | extern
             ;
\end{verbatim}

Die in einem \rw{global}-Block definierten Strukturen,
\index{Struktur}
Funktionen
\index{Funktion!programmglobale|ii}
und Variablen
\index{Variable!programmglobale|ii}
sind dazu bestimmt, auch von anderen
Modulen verwendet zu werden. Zus"atzlich k"onnen in einem
\rw{global}-Block Variablen deklariert werden, die in diesem Modul
verwendet werden sollen, jedoch in einem anderen Modul als global
definiert worden sind. Variablen, Strukturen und Funktionen sind
erst ab ihrer Deklaration verf"ugbar, soda"s es am
praktikabelsten ist, die \rw{global}-Bl"ocke an den Beginn der
Datei zu setzen oder in einer \ind{Include-Datei} unterzubringen.

\section{Anweisungen}
\index{Anweisung|ii}
\fuindexii{statement}
\fuindex{expr}
\fuindex{if}
\fuindex{while}
\fuindex{for}
\fuindex{repeat}
\fuindex{switch}
\fuindex{prefix}
\rwindex{break}
\rwindex{return}

\begin{verbatim}
statement:     [expr] `;'
             | if
             | while
             | for
             | repeat
             | switch
             | prefix
             | `break' `;'
             | `return' [expr] `;'
             ;
\end{verbatim}

Anweisungen (oder {\em statements}) k"onnen innerhalb von Funktionen
\index{Funktion}
verwendet werden und machen das eigentliche Programm aus.
Anweisungen k"onnen grunds"atzlich in Ausdr"ucke ({\em expressions}),
\index{Ausdruck}
Kontrollanweisungen
\index{Kontrollanweisung}
und Schleifen
\index{Schleife}
eingeteilt werden. Erstere verrichten die
eigentliche Arbeit, und die beiden letzteren steuern den Programmflu"s. Wie
aus der Grammatik ersichtlich ist, kann eine Anweisung auch
einfach nur ein Semikolon enthalten. Solche Anweisungen haben die
Bezeichnung {\em \indi{Nullanweisung}} und haben---bis auf die
Verbesserung der "Ubersichtlichkeit in bestimmten Situationen---%
keine Aufgabe.

\subsection{Die if-Anweisung}
\rwindexii{if}
\fuindexii{if}
\fuindex{expr}
\fuindex{statement}
\rwindexii{else}
\rwindex{end}

\begin{verbatim}
if:            `if' `(' expr `)'
                 {statement}
               [`else'
                 {statement}]
               `end'
             ;
\end{verbatim}

Die \rw{if}-Anweisung ist eine Kontrollanweisung und dient der
bedingten Ausf"uhrung eines Programmteils. Dem \rw{if}
folgt ein in Klammern eingeschlossener \ind{Ausdruck}, der zuerst
evaluiert (berechnet) und dann nach \tint\
\typeindex{int}
gecastet wird. Ist dieser
Ausdruck ungleich \fu{0}, werden die darauffolgenden
Anweisungen ausgef"uhrt. Ist jedoch der Ausdruck \fu{0}, dann
werden die Anweisungen, die dem \rw{else} folgen, sofern ein
\rw{else} vorhanden ist, ausgef"uhrt, ansonsten geschieht gar
nichts.

\example
Die Funktion \fuii{GetInt} fragt den Benutzer nach einer Zahl.
Ist die Zahl gr"o"ser oder gleich  \fu{0}, so wird der Text
"`\fu{Gratulation, die Zahl ist positiv!}"'
auf dem Bildschirm dargestellt. Andernfalls wird die Zahl ohne
negatives Vorzeichen ausgegeben.

\fuindexuu{ReadInt}
\fuindexuu{WriteString}
\fuindexuu{WriteInt}
\begin{verbatim}
function GetInt () : void

  i : int;

begin
  i = ReadInt();
  if (i >= 0)
    WriteString("Gratulation, die Zahl ist positiv!\n");
  else
    WriteInt(-i);
    WriteString("\n");
  end
end
\end{verbatim}

\subsection{Die switch-Anweisung}
\rwindexii{switch}
\fuindexii{switch}
\fuindex{expr}
\fuindex{statement}
\rwindex{end}
\fuindexii{prefix}
\rwindex{case}
\fuindex{constant}
\rwindex{default}

\begin{verbatim}
switch:        `switch' `(' expr `)'
                 {statement}
               `end'
             ;

prefix:        `case' constant `:'
             | `default' `:'
             ;
\end{verbatim}

Die \rw{switch}-Anweisung wird verwendet, um f"ur mehrere
verschiedene Werte eines Ausdrucks
\index{Ausdruck}
verschiedene Aktionen
auszuf"uhren. Als erstes wird der Ausdruck evaluiert. Danach
wird er mit der \ind{Konstante}, die dem ersten im
\fu{\{statement\}} vorkommenden \rw{case} folgt, verglichen. Ist
der Vergleich positiv, so werden alle der \rw{case}-Anweisung
folgenden Anweisungen ausgef"uhrt, solange, bis ein \rwii{break}
gefunden wird, was den Programflu"s nach dem \rw{end} fortsetzt.
Ist der Vergleich jedoch negativ, wird das n"achste \rw{case}
gesucht u.s.w. Wird bei diesem Vorgang ein \rwii{default} gefunden,
dann wird die Ausf"uhrung {\em in jedem Fall} nach dieser
Anweisung fortgesetzt. Aus diesem Grunde sollte \rw{default}
immer am Schlu"s der \rw{case}-Kette stehen.
Findet sich weder ein \rw{case}, das dem
Wert des Ausdrucks entspricht, noch ein \rw{default}, geschieht
nichts und die Ausf"uhrung wird nach dem \rw{end} fortgesetzt.

\example
Die Funktion \fuii{SimpleMenu} stellt ein Men"u auf
dem Bildschirm dar und wartet auf eine Eingabe. F"ur die Ziffern
\fu{`1'} bis \fu{`3'} werden die entsprechenden Funktionen
\fuii{Choice1} bis \fuii{Choice3}
\fuindexii{Choice2}
ausgef"uhrt. Dieser Vorgang
wiederholt sich so lange, bis der Benutzer \fu{`x'} oder \fu{`X'}
eingibt, was die Funktion beendet. Bet"atigt der Benutzer eine
Taste, die nicht verarbeitet werden kann, wird eine
dementsprechende Meldung ausgegeben. F"ur die Ausgabe des Men"us
auf dem Bildschirm wird die Funktion \fui{DisplayMenu} verwendet.

\rwindexuu{while}
\fuindexuu{ReadChar}
\fuindexuu{WriteString}
\begin{verbatim}
function SimpleMenu () : void
begin
  while (1)
    DisplayMenu();
    switch (ReadChar())
      case '1' :
        Choice1();
        break;
      case '2' :
        Choice2();
        break;
      case '3' :
        Choice3();
        break;
      case 'x' :
      case 'X' :
        return;
      default :
        WriteString("Falsche Eingabe!\n");
    end
  end
end
\end{verbatim}
        
\subsection{Die while-Schleife}
\rwindexii{while}
\fuindexii{while}
\fuindex{expr}
\fuindex{statement}
\rwindex{end}

\begin{verbatim}
while:         `while' `(' expr `)'
                 {statement}
               `end'
             ;
\end{verbatim}

Die \rw{while}-Schleife f"uhrt die zwischen \rw{while} und
\rw{end} aufgef"uhrten Anweisungen sooft aus, solange der in Klammern
eingeschlossene \ind{Ausdruck} ungleich \fu{0} ist. Dabei wird wie
folgt vorgegangen: Als erstes wird der Ausdruck evaluiert und
nach \tint\
\typeindex{int}
gecastet. Ist er
gleich \fu{0}, so wird die Ausf"uhrung nach dem \rw{end}
fortgesetzt, andernfalls werden die Anweisungen, die dem Ausdruck
folgen, ausgef"uhrt. Hierauf wird wieder an den Anfang der
Schleife gesprungen und der Ausdruck abermals gepr"uft u.s.w.
Beachten Sie, da"s es au"ser mit Hilfe der \rwi{return}-Anweisung
nicht m"oglich ist, aus der Schleife auszusteigen, ohne da"s der
Ausdruck gleich \fu{0} ist.

\example
Die Funktion \fuii{OutputList} durchl"auft die ihr "ubergebene
\indu{Liste} des Typs \fu{List}
\index{List@{\tt List}~(\rw{struct})|ii}
und gibt dabei alle Datenelemente---in
diesem Beispiel der Einfachheit halber vom Typ \tint ---aus.
Anmerkung: Das letzte Element der Liste ist dadurch
gekennzeichnet, da"s der Zeiger, der auf das n"achste Element
zeigen sollte, ein \ind{Nullzeiger} ist.

\fuindexuu{WriteInt}
\rwindexuu{local}
\begin{verbatim}
local
  struct List
    i    : int;
    Next : *List;
  end
end

function OutputList (l : *List) : void
begin
  while (l != 0)
    WriteInt(l->i);
    l = l->Next;
  end
end
\end{verbatim}

\subsection{Die repeat-Schleife}
\rwindexii{repeat}
\fuindexii{repeat}
\rwindexii{until}
\fuindex{statement}
\fuindex{expr}

\begin{verbatim}
repeat:        `repeat'
                 {statement}
               `until' `(' expr `)'
             ;
\end{verbatim}

Die \rw{repeat}-Schleife wiederholt die zwischen \rw{repeat} und
\rw{until} aufgef"uhrten Anweisungen so lange, bis der auf
\rw{until} folgende \ind{Ausdruck} ungleich \fu{0} ist. Der Ablauf ist
hierbei folgenderma"sen: Zun"achst werden die Anweisungen
ausgef"uhrt, danach wird der Ausdruck evaluiert, nach \tint\
\typeindex{int}
gecastet und auf
Gleichheit mit \fu{0} gepr"uft. Ist diese gegeben, wird zu
\rw{repeat} zur"uckgesprungen und der Vorgang wiederholt sich.
Ist der Ausdruck hingegen ungleich \fu{0}, ist die Schleife
beendet. Beachten Sie bitte, da"s---im Gegensatz zur
\rwi{while}-Schleife---die Schleife {\em mindestens einmal}
ausgef"uhrt wird, unerheblich, ob der Ausdruck schon vor der
Ausf"uhrung ungleich \fu{0} w"are.

\example
Dieses Beispielprogramm liest so lange Zahlen vom Benutzer ein,
bis dieser \fu{0} eingibt. Anschlie"send wird der Mittelwert der
eingelesenen Zahlen ausgegeben. Wurden au"ser der \fu{0} keine
Zahlen eingegeben, erscheint eine dementsprechende Meldung.
    
\rwindexuu{global}
\fuindexuu{ReadInt}
\rwindexuu{if}
\fuindexuu{WriteInt}
\fuindexuu{WriteString}
\begin{verbatim}
#include "conio.sh"

global
  function main () : void
end

function main () : void

  i : int;
  a : int =  0;
  n : int = -1;

begin
  repeat
    a = a + (i = ReadInt());
    n = n + 1;
  until (i == 0)
  if (n)
    WriteInt(a / n);
  else
    WriteString("Es wurden keine Zahlen eingegeben!");
  end
  WriteString("\n");
end
\end{verbatim}

\subsection{Die for-Schleife}
\rwindexii{for}
\fuindexii{for}
\fuindex{primary}
\fuindex{expr}
\rwindexii{to}
\rwindexii{downto}
\rwindexii{step}
\fuindex{statement}
\rwindex{end}

\begin{verbatim}
for:           `for' `(' primary `=' expr (`to' | `downto') expr
               [`step' expr] `)'
                 {statement}
               `end'
             ;
\end{verbatim}

Die \rw{for}-Schleife ist eine spezielle Schleife, die es
erlaubt, Anweisungen eine bestimmte Anzahl mal zu wiederholen.
Hierzu ist nach der "offnenden Klammer die Variable anzugeben,
die als Z"ahler fungiert. Da hier ein \fu{primary} (siehe Kapitel
\ref{exprs}) erwartet wird,
ist es auch m"oglich, Elemente eines Arrays
\index{Array}
oder einer \ind{Struktur}
als Z"ahler zu verwenden. Nach dem Zuweisungsoperator wird der
Anfangsindex angegeben und danach entweder das Schl"usselwort
\rw{to} oder \rw{downto}, je nachdem, ob nach oben oder nach
unten gez"ahlt werden soll. Darauf folgt der Zielindex, also
jener Wert, den die Indexvariable maximal oder minimal annehmen
darf. Optional kann noch nach dem Schl"usselwort \rw{step} die
Schrittweite angeben werden. Zu beachten ist, da"s diese auch
dann, wenn nach unten gez"ahlt wird, positiv sein mu"s.

Bei der Ausf"uhrung wird folgenderma"sen vorgegangen: Zun"achst
werden alle Ausdr"ucke
\index{Ausdruck}
evaluiert und nach \tint\
\typeindex{int}
gecastet. Danach
wird "uberpr"uft, ob der Wert der Indexvariablen schon au"serhalb
der zul"assigen Grenze ist. Ist das nicht der Fall, werden die
Anweisungen das erste Mal ausgef"uhrt und der Z"ahler
entsprechend erh"oht bzw. erniedrigt. Danach wird wieder
"uberpr"uft, ob die Anweisungen nocheinmal ausgef"uhrt werden
sollen u.s.w.

\example
Die Funktion \fuii{Squares} gibt der Reihe nach die Quadrate
\index{Quadrat}
der ganzen Zahlen $i$ f"ur \hbox{$\fu{a} \le i \le \fu{b}$} aus.
Der Funktion werden die \tint s \fu{a} und \fu{b} "ubergeben.

\fuindexuu{WriteInt}
\fuindexuu{WriteString}
\begin{verbatim}
function Squares (a : int, b : int) : void

  i : int;

begin
  for (i = a to b)
    WriteInt(i * i);
    WriteString("\n");
  end
end
\end{verbatim}

\section{Ausdr\protect{"ucke}}
\label{exprs}
\index{Ausdruck|ii}
\fuindexii{expr}
\rwindex{delete}
\fuindexii{primary}
\fuindex{IDENTIFIER}
\fuindex{INT\protect{\_CONSTANT}}
\fuindex{CHAR\protect{\_CONSTANT}}
\fuindex{FLOAT\protect{\_CONSTANT}}
\fuindex{STRING\protect{\_CONSTANT}}
\fuindex{type}
\rwindex{sizeof}
\rwindex{new}

\begin{verbatim}
expr:          expr (`=' | `&&' | `||' | `^^' | `==' | `<' | `>'
                   | `<=' | `>=' | `!=' | `&' | `|' | `^' | `<<'
                   | `>>' | `+' | `-' | `*' | `/' | `%') expr
             | (`!' | `~' | `-' | `*' | `&') expr
             | `delete' expr
             | primary
             ;

primary:       IDENTIFIER
             | `(' expr `)'
             | INT_CONSTANT
             | CHAR_CONSTANT
             | FLOAT_CONSTANT
             | STRING_CONSTANT
             | primary `->' IDENTIFIER
             | primary `:' type
             | primary `[' expr `]'
             | IDENTIFIER `(' [expr {`,' expr}] `)'
             | `sizeof' type
             | `new' type
             ;
\end{verbatim}

Ausdr"ucke sind sozusagen das "`Herz"' von SEL, denn sie
erledigen die eigentliche Arbeit, w"ahrend Kontrollanweisungen
und Schleifen nur kontrollieren, was passieren soll. Ausdr"ucke k"onnen die
verschiedensten Aufgaben erledigen: Sie k"onnen zum Vergleich
zweier Werte benutzt werden, zur Wertzuweisung oder zum
Funktionsaufruf. Die gemeinsamen Merkmale der Ausdr"ucke sind:
Jeder Ausdruck besitzt sowohl einen \ind{Typ} als auch einen
\ind{Wert} (bis auf die Ausdr"ucke, die vom Typ \tvoid\
\typeindex{void}
sind).

\subsection{Konstanten}
\index{Konstante}
\fuindexii{constant}
\fuindex{INT\protect{\_CONSTANT}}
\fuindex{CHAR\protect{\_CONSTANT}}
\fuindex{FLOAT\protect{\_CONSTANT}}
\fuindex{STRING\protect{\_CONSTANT}}

\begin{verbatim}
constant:      [`-'] INT_CONSTANT
             | CHAR_CONSTANT
             | [`-'] FLOAT_CONSTANT
             | STRING_CONSTANT
             ;
\end{verbatim}

Konstanten in SEL sind im Grunde als Konstanten im
mathematischen Sinne zu verstehen. Ihre Werte k"onnen im gesamten
Programm nicht
ver"andert werden. Im Gegensatz zu anderen Programmiersprachen
besitzt SEL nicht die F"ahigkeit, Konstanten einen Namen zu geben,
\index{Konstante!benannte}
was allerdings durch die \fui{\#define}-Direktive leicht umgangen
werden kann. Weiters gibt es in SEL keine Konstanten von
zusammengesetzten Typen,
\index{Typ!zusammengesetzter}
da diese F"ahigkeit nur sehr selten ben"otigt wird.

\subsubsection{int-Konstanten}
\index{Konstante!\tint |ii}
\typeindex{int}
\fuindexii{INT\protect{\_CONSTANT}}

\begin{verbatim}
INT_CONSTANT: [0-9]+
\end{verbatim}

Eine \tint -Konstante ist eine Folge von Ziffern, die als Zahl
mit der Basis 10, also als Dezimalzahl
\index{Zahl!ganze}
\index{Zahl!dezimale}
interpretiert werden.
Laut obiger Definition gibt es jedoch
keine negativen \tint -Konstanten, was den Grund hat, da"s
dieser Mechanismus in der Grammatik eingebaut ist. Dies wiederum
ist darin begr"undet, da"s es andernfalls zu Mi"sverst"andnissen
bei der Interpretation von Ausdr"ucken kommen w"urde (man denke
an den Ausdruck \fu{7-4}---er w"urde als Konstante \fu{7} gefolgt
von der Konstante \fu{-4} gedeutet werden, da die lexikalische
Analyse "uber keine Informationen "uber den Kontext verf"ugt, in dem
die Konstante auftritt, mit anderen Worten, sie wei"s nicht, da"s
auf eine Konstante keine weitere folgen kann).

Voraussichtlich wird SEL in der Version 0.4 um die F"ahigkeit
erweitert werden, auch hexadezimale
\index{Zahl!hexadezimale}
\tint -Konstanten in der Form
\fu{0(x|X)[0-9a-fA-F]+} zu interpretieren.

\subsubsection{char-Konstanten}
\index{Konstante!\tchar |ii}
\typeindex{char}
\fuindexii{CHAR\protect{\_CONSTANT}}

\begin{verbatim}
CHAR_CONSTANT: '(.|\.)'
\end{verbatim}

\tchar -Konstanten repr"asentieren einzelne \ind{Zeichen}.
Dieses k"onnen entweder "`normale"' Zeichen oder
Steuerzeichen
\index{Zeichen!Steuerzeichen}
sein---letztere werden durch einen vorangestellten
Back"-slash "`\fu{\char92}"'
\index{Backslash (\fu{\char92})}
gekennzeichnet, jedoch als ein einzelnes Zeichen behandelt.

\subsubsection{float-Konstanten}
\index{Konstante!\tfloat |ii}
\typeindex{float}
\fuindexii{FLOAT\protect{\_CONSTANT}}

\begin{verbatim}
FLOAT_CONSTANT: [0-9]*(`.'[0-9]*)?((e|E)(+|-)?[0-9]+)?
\end{verbatim}

Eine \tfloat -Konstante repr"asentiert eine positive, rationale
\index{Zahl!rationale}
Zahl in \ind{Exponentialschreibweise}, wobei der Exponent auch
weggelassen werden kann. Leider ist eine Definition als
regul"arer Ausdruck nicht ohne Probleme m"oglich. So w"are laut
obiger Definition auch eine leere Eingabe oder ein Dezimalpunkt
f"ur sich alleine eine \tfloat -Konstante, was nat"urlich nicht
im Sinn der Sache liegen kann. Die Definition ist also
dahingehend zu erweitern, da"s in der Mantisse mindestens eine Ziffer
vorkommen mu"s. Beachten Sie bitte, da"s eine
\tfloat -Konstante immer positiv sein mu"s, was jedoch aufgrund
der Grammatik von SEL nicht weiter st"ort.

\subsubsection{String-Konstanten}
\index{Konstante!String|ii}
\fuindexii{STRING\protect{\_CONSTANT}}

\begin{verbatim}
STRING_CONSTANT: "([^"\n]|\.)*"
\end{verbatim}

Eine String-Konstante ist eine in zwei Anf"uhrungsstriche
eingeschlossene Zeichenfolge, innerhalb derer
alle \ind{Zeichen} bis auf die Anf"uhrungsstriche und den
Zeilenvorschub erlaubt sind. Eine besondere Bedeutung kommt den
Steuerzeichen
\index{Zeichen!Steuerzeichen}
zu, die durch einen vorangestellten Back"-slash
"`\fu{\char92}"'
\index{Backslash (\fu{\char92})}
gekennzeichnet sind. Jede String-Konstante wird
von SEL automatisch mit einem \indi{Nullzeichen}, also einem Zeichen mit
dem Wert \fu{0} beendet. Die String-Konstante beansprucht also im
Speicher um ein Byte mehr, als sie an Zeichen enth"alt.

Eine String-Konstante wird von SEL wie ein Zeiger auf
\tchar s
\typeindex{char}
behandelt. Sie k"onnen diesen Zeiger also dazu
verwenden, um die Zeichenfolge auszugeben, sollten jedoch den
Speicherbereich, auf den er verweist, nicht ver"andern.

\subsubsection{Steuerzeichen}
\label{specialchar}
\index{Zeichen!Steuerzeichen|ii}

SEL stellt bestimmte Steuerzeichen zur Verf"ugung, die in Strings
\index{Konstante!String}
und \tchar -Konstanten
\index{Konstante!\tchar }
\typeindex{char}
benutzt werden k"onnen, um Zeichen
darzustellen, denen eine bestimmte Bedeutung zukommt, wie etwa
dem Zeichen f"ur Zeilenvorschub. Steuerzeichen
sind daran zu erkennen, da"s ihnen ein Back"-slash "`\fu{\char92}"'
\index{Backslash (\fu{\char92})|ii}
vorangestellt ist. Trotzdem werden sie von SEL als einzelne
Zeichen behandelt. Hier eine Zusammenstellung aller in SEL
vorhandenen Steuerzeichen:

\medskip
\index{Warnton}
\index{R""uckschritt}
\index{Bildvorschub}
\index{Zeilenvorschub}
\index{Wagenr""ucklauf}
\index{Tabulator}
\begin{tabular}{ll}
\fu{\char92 a} & Bell (Warnton) \\
\fu{\char92 b} & Backspace (R"uckschritt) \\
\fu{\char92 f} & Formfeed (Bildvorschub) \\
\fu{\char92 n} & New line (Zeilenvorschub) \\
\fu{\char92 r} & Carriage return (Wagenr"ucklauf) \\
\fu{\char92 t} & Tab (Tabulator) \\
\fu{\char92 v} & Vertical tab (vertikaler Tabulator)
\end{tabular}
\medskip

Wird hinter den Back"-slash ein Zeichen gesetzt, das in obiger
Tabelle nicht eingetragen ist, wird von SEL automatisch dieses
Zeichen verwendet. So ist es zum Beispiel m"oglich, einen
Back"-slash oder ein Anf"uhrungszeichen innerhalb einer
String-Konstante einzugeben.

SEL wird in der Version 0.4 aller Voraussicht nach um das
Steuerzeichen \fu{\char92 x\rm\it hh} erweitert werden, das an seine
Stelle das Zeichen mit dem in hexadezimaler
\index{Zahl!hexadezimale}
Notation angegeben Code {\it hh} stellt.

\subsection{Variablen}
\index{Variable|ii}
\fuindexii{var}
\fuindex{IDENTIFIER}
\fuindex{type}
\fuindex{constant}
\fuindexii{extern}
\rwindexii{extern}

\begin{verbatim}
var:           IDENTIFIER `:' type [`=' constant] `;'
             ;

extern:        `extern' IDENTIFIER `:' type `;'
             ;
\end{verbatim}

Variablen sind Speicherbereiche, die
vom Programm belegt werden und w"ahrend der Ausf"uhrung
verschiendene Werte
\index{Wert}
annehmen k"onnen. Jede Variable besitzt
einen \ind{Typ}, der sowohl ihre Gr"o"se als auch ihre Repr"asentation
im Speicher bestimmt. SEL unterscheidet grunds"atzlich zwischen
drei Typen von Variablen: programmglobale Variablen, modulglobale
Variablen und lokale Variablen.

Programmglobale Variablen werden in einem \rwi{global}-Block
definiert und sind im gesamten Programm bekannt. Ein Modul, das
eine Variable verwenden will, die in einem anderen Modul
definiert ist, mu"s jedoch, um auf die Variable zugreifen zu
k"onnen, dies mit einer \rw{extern}-Deklaration bekanntgeben.
Sie unterscheidet sich von einer Variablendefinition lediglich
dadurch, da"s ihr das \rw{extern}-Schl"usselwort vorangestellt
ist und keine Wertzuweisung erlaubt ist. Beachten Sie, da"s es
unm"oglich ist, innerhalb eines
Programms zwei programmglobale Variablen desselben Namens zu
definieren, da dies zu Konflikten f"uhren w"urde.

Modulglobale Variablen werden in einem \rwi{local}-Block eines
Moduls definiert und sind nur innerhalb dieses Moduls
bekannt---auf sie kann von anderen Modulen auch nicht durch eine
\rw{extern}-Deklaration zugegriffen werden. Deshalb ist es auch
ohne weiteres m"oglich, mehrere modulglobale Variablen gleichen
Namens zu definieren, soferne dies in jeweils verschiedenen
Modulen geschieht.

Lokale Variablen sind nur innerhalb der \ind{Funktion} bekannt, in der
sie definiert wurden. Solche Variablen sind allerdings nicht
statisch, das hei"st, sie werden beim Aufruf der jeweiligen Funktion
erzeugt und existieren so lange, bis die Funktion zu ihrem
Aufrufer zur"uckkehrt. Weiters ist es auch durchaus m"oglich,
da"s lokale Variablen mehrfach existieren, und zwar dann, wenn
sich die Funktion selbst aufruft, also rekursiv ist. Bei jedem
solchen Aufruf werden neue lokale Variablen erzeugt, die zwar
namensgleich, jedoch von den anderen absolut isoliert sind.
Unter die lokalen Variablen fallen auch die Funktionsparameter.
\index{Parameter}
Sie werden im Funktionskopf definiert und sind auch nur existent,
solange die Funktion abgearbeitet wird.

Es ist in SEL durchaus m"oglich, modulglobale, programmglobale
und lokale Variablen desselben Namens zu definieren, wobei
folgende Regel zu beachten ist: lokale Variablen werden
modulglobalen Variablen vorgezogen, welche ihrerseits den
programmglobalen Variablen vorgezogen werden. Existiert also die
programmglobale Variable \fu{example}, so kann ohne weiteres eine
modulglobale Variable desselben Namens definiert werden.
Innerhalb des Moduls, in der letztere definiert wurde, bezieht
sich \fu{example} nicht auf die programmglobale, sondern auf die
modulglobale Variable. Definiert weiters eine Funktion innerhalb
dieses Moduls eine lokale Variable namens \fu{example}, so kann
diese Funktion nur auf die lokale Variable zugreifen, w"ahrend
der Rest des Moduls auf die modulglobale Variable zugreift.

\example
Die Funktion \fuii{Recursive}
\index{Rekursion|uu}
gibt die Zahlen \fu{1} bis \fu{10} in
ihrer nat"urlichen Reihenfolge aus und ruft sich nach der
Ausgabe jeder Zahl selbst auf, was dazu f"uhrt, da"s f"ur jede
Zahl zus"atzlich nocheinmal die Zahlen \fu{1} bis \fu{10}
ausgegeben werden, f"ur die dasselbe gilt. Um zu verhindern, da"s
die Zahl \fu{1} unendlich oft ausgegeben wird, was zweifellos
geschehen w"urde, w"urd dieses Verfahren unkontrolliert
angewendet, wird der Funktion ein Parameter "ubergeben, der
bestimmt, wie oft sie sich noch selbst aufrufen soll.
Beachten Sie bei diesem Beispiel, da"s sich die einzelnen Z"ahler
\fu{i} und die Parameter \fu{depth} der Funktion nicht gegenseitig
st"oren.

\rwindexuu{for}
\fuindexuu{WriteInt}
\fuindexuu{WriteString}
\begin{verbatim}
function Recursive (depth : int) : void

  i : int;

begin
  for (i = 1 to 10)
    WriteInt(i);
    WriteString("  ");
    if (depth > 0)
      Recursive(depth - 1);
      WriteString("  ");
    end
  end
end
\end{verbatim}

\subsection{Funktionen}
\index{Funktion|ii}

Funktionen sind Unterprogramme, die von verschiedenen Stellen des
Programms aufgerufen werden k"onnen. Um den Funktionen
Information mitzugeben, k"onnen ihnen sogenannte {\em \indi{Parameter}}
"ubergeben werden, deren Typen bei der Funktionsdeklaration
bestimmt werden m"ussen. Der Aufruf einer Funktion geschieht
durch Angabe des Funktionsnamens und in Klammern gesetzten
Parametern,
\index{Parameter""ubergabe|ii}
durch Beistriche getrennt. Sind keine Parameter
erforderlich, m"ussen nur die
beiden Klammern angegeben werden. Jede Funktion besitzt einen
R"uckgabetyp, der
ebenfalls bei der Deklaration festgelegt werden mu"s. Um eine
Funktion, die keine R"uckgabe vornimmt, zu realisieren, mu"s als
R"uckgabetyp \tvoid\
\typeindex{void}
gew"ahlt werden.

\example
Dieses Programm benutzt die Funktionen \fuiu{Factorial},
\fu{WriteString} und \fu{WriteInt}, um die Fakult"aten
\index{Fakult""at}
der Zahlen von \fu{1} bis \fu{10} auszugeben.

\rwindexuu{for}
\rwindexuu{global}
\fuindexuu{WriteString}
\fuindexuu{WriteInt}
\begin{verbatim}
#include "conio.sh"

global
  function Factorial (int) : int
  function main      ()    : void
end

function main () : void

  i : int;

begin
  for (i = 1 to 10)
    WriteString("Fakultaet von ");
    WriteInt(i);
    WriteString(" = ");
    WriteInt(Factorial(i));
    WriteString("\n");
  end
end
\end{verbatim}

\subsection{Operatoren}
\index{Operator|ii}

Operatoren werden, wie schon der Name sagt, ben"otigt, um
bestimmte Operationen auszuf"uhren, wie etwa das Addieren
von Zahlen oder das Zuweisen von Werten. Grunds"atzlich kann man
in SEL zwischen {\em un"aren} und {\em bin"aren} Operatoren
unterscheiden.

Un"are Operatoren stehen vor einem Ausdruck und
beeinflussen so dessen \ind{Wert}. Ein Beispiel hierf"ur w"are der
\ind{NOT}-Operator \fu{!}, der den \ind{Ausdruck}, der ihm nachsteht,
negiert. Un"are Operatoren werden grunds"atzlich von rechts nach
links ausgewertet, das hei"st, es wird zuerst der Wert des
Ausdrucks berechnet und danach negiert, nicht zuletzt deshalb,
weil es nicht anders m"oglich ist.

Bin"are Operatoren beziehen sich auf zwei Operanden, sie werden
deshalb zwischen diese beiden gestellt. Ein Beispiel f"ur einen
bin"aren Operator w"are der Additionsoperator \fu{+}. Bin"are
Operatoren k"onnen weiter unterteilt werden in solche, die von
rechts nach links und solche, die von links
nach rechts auswerten.
\index{Operator!Richtung der Auswertung}

Ersteres bedeutet, da"s zuerst
der Wert des rechten Operanden berechnet und dann auf den
linken angewendet wird. Ein Beispiel w"are der Zuweisungsoperator
\index{Zuweisung}
\fu{=}. Was unter der obigen Definition gemeint ist, l"a"st sich
leicht anhand eines Ausdrucks veranschaulichen: \fu{a=b=1}. SEL
beginnt von links nach rechts zu parsen und bleibt beim ersten
Operator stehen. Der linke Operand w"are demnach \fu{a}, der
rechte \fu{b=1}. Nun wird zuerst der rechte Operand ausgewertet.
SEL stoppt beim zweiten Operator und steht vor den Operanden
\fu{b} und \fu{1}. Wieder wird der rechte Operand ausgewertet,
von dem SEL nun wei"s, da"s er den Wert \fu{1} hat. Dieser wird
nun auf den linken Operanden angewendet, also der Variable \fu{b}
zugewiesen. Weiters wird nun der Wert des Ausdrucks \fu{b=1}
ebenfalls auf \fu{1} gesetzt und \fu{a} zugewiesen. Diese Art der
Auswertung ist zweckm"a"sig, da es so m"oglich ist, mehreren
Variablen auf einmal den selben Wert zuzuweisen. Um die
Auswertung deutlich zu machen, kann man den Ausdruck auch als
\fu{a=(b=1)} anschreiben.

Ein Beispiel f"ur einen Operator, der von links nach rechts
auswertet, ist der Subtraktionsoperator \fu{-}.
\index{Subtraktion}
Dabei wird
von zwei Operanden zun"achst der linke ausgewertet und dann auf
den rechten angewendet. Als Beispiel sei der Ausdruck \fu{3-2-1}
angef"uhrt. W"urde der Operator \fu{-} von rechts nach links
auswerten, so k"onnte man diesen Ausdruck auch als \fu{3-(2-1)}
anschreiben, was jedoch zu dem falschen Ergebnis \fu{2} f"uhrte.
Aus diesem Grunde wertet \fu{-} von links nach rechts aus,
was den Ausdruck zu \fu{(3-2)-1} macht. Dieser Ausdruck liefert
das gew"unschte Ergebnis \fu{0}.

Es folgt eine kurze Beschreibung aller Operatoren sowie der
Typen ihrer Operanden und des Typs des Ergebnisses der Operation.
Bei den Typen der Operanden ist immer zu beachten, da"s, falls
diese unterschiedlich sind, SEL immer dem "`gr"o"seren"' Typ den
Vorrang gibt, wobei \hbox{$ \tchar < \tint < \tfloat $} gilt.
\typeindex{char}
\typeindex{int}
\typeindex{float}
Der Ausdruck \fu{3 + 4.123} ergibt also einen
\tfloat\ mit dem Wert \fu{7.123}, wobei allerdings der Ausdruck
\fu{13 \&\& 5.3} einen \tint\ mit einem Wert ungleich \fu{0}
ergibt, da der Operator \fu{\&\&} nur mit \tint s arbeitet.

\begin{operator}
\tchar\ \fu{`+'} \tchar\ & \tchar\ \\
\tint\ \fu{`+'} \tint\ & \tint\ \\
\tfloat\ \fu{`+'} \tfloat\ & \tfloat\ \\
\type{\poi type} \fu{`+'} \tint\ & \type{\poi type}
\end{operator}
\index{Addition|ii}

Sind die beiden Operanden des bin"aren \fu{+}-Operators numerisch, werden
diese addiert. Ist jedoch der linke Operand ein \ind{Zeiger}, ergibt
die Operation einen Zeiger, der um die Anzahl Elemente erh"oht
ist, die der rechte Operand angibt. Beachten Sie bitte, da"s die
Gr"o"se eines Elements nicht unbedingt ein Byte sein mu"s,
sondern so gro"s ist, wie der Typ, den der Zeiger referenziert. So
ist es z.~B. leicht m"oglich, ein \ind{Array} zu durchlaufen,
ohne einen numerischen Z"ahler zu verwenden.

\begin{operator}
\tchar\ \fu{`-'} \tchar\ & \tchar\ \\
\tint\ \fu{`-'} \tint\ & \tint\ \\
\tfloat\ \fu{`-'} \tfloat\ & \tfloat\ \\
\type{\poi type} \fu{`-'} \tint\ & \type{\poi type}
\end{operator}
\index{Subtraktion|ii}

Der bin"are Operator \fu{-} subtrahiert den rechten Operanden vom linken,
falls beide numerisch sind, andernfalls wird der \ind{Zeiger} um die
Anzahl Elemente erniedrigt, die vom rechten Operanden angegeben
wird. Mit Element ist hier der Typ gemeint, den der Zeiger
referenziert.

\begin{operator}
\tchar\ \fu{`*'} \tchar\ & \tchar\ \\
\tint\ \fu{`*'} \tint\ & \tint\ \\
\tfloat\ \fu{`*'} \tfloat\ & \tfloat\
\end{operator}
\index{Multiplikation|ii}

Der bin"are \fu{*}-Operator multipliziert die beiden Operanden.

\begin{operator}
\tchar\ \fu{`/'} \tchar\ & \tchar\ \\
\tint\ \fu{`/'} \tint\ & \tint\ \\
\tfloat\ \fu{`/'} \tfloat\ & \tfloat\
\end{operator}
\index{Division|ii}
\index{Zahl!Bruch}

Der \fu{/}-Operator dividiert den linken durch den rechten
Operanden. Sind beide Operanden keine \tfloat s, so ist das
Ergebnis auch kein \tfloat , selbst wenn der Bruch nicht
ganzzahlig ist.

\begin{operator}
\tchar\ \fu{`\%'} \tchar\ & \tchar\ \\
\tint\ \fu{`\%'} \tint\ & \tint\
\end{operator}
\index{Division!Rest bei ganzzahliger|ii}
\index{Modulo|see{Division}}

Der \fu{\%}-Operator berechnet den Rest, der bei einer
ganzzahligen Division des linken durch den rechten Operanden zu
verzeichnen w"are. Beachten Sie bitte, da"s dieser Operator {\em
nicht} mit \tfloat s funktioniert.

\newpage
\begin{operator}
\tint\ \fu{`\&\&'} \tint\ & \tint\
\end{operator}
\index{AND!logisches|ii}
\index{UND|see{AND}}

Der \fu{\&\&}-Operator verkn"upft die beiden Operanden mit einem
logischen Und (AND), das hei"st, der Ausdruck ergibt einen Wert
ungleich \fu{0}, wenn sowohl der linke als auch der rechte
Operand ungleich \fu{0} sind, andernfalls ergibt der Ausdruck
\fu{0}.

\begin{operator}
\tint\ \fu{`||'} \tint\ & \tint\
\end{operator}
\index{OR!logisches|ii}
\index{ODER|see{OR}}

Der \fu{||}-Operator verkn"upft seine beiden Operanden mit einem
logischen Oder (OR), was bedeutet, da"s der Ausdruck dann ungleich
\fu{0} ist, wenn mindestens einer der beiden Operanden ungleich
\fu{0} ist. Nur wenn sowohl der linke als auch der rechte Operand
\fu{0} sind, ergibt der Ausdruck \fu{0}.

\begin{operator}
\tint\ \fu{`\char94\char94'} \tint\ & \tint\
\end{operator}
\index{XOR!logisches|ii}
\index{ODER!exklusives|see{XOR}}

Der \fu{\char94\char94}-Operator verkn"upft die beiden Operanden
mit einem logischen exklusiven Oder (XOR). Ein Ausdruck
ergibt also nur dann einen Wert ungleich \fu{0}, wenn genau einer
der Operanden ungleich \fu{0} ist. Haben hingegen entweder beide
Operanden den Wert \fu{0} oder sind beide ungleich \fu{0}, so ist der
Ausdruck gleich \fu{0}.

\begin{operator}
\tint\ \fu{`\&'} \tint\ & \tint\
\end{operator}
\index{AND!bitweises|ii}

Der bin"are \fu{\&}-Operator verkn"upft die beiden Operanden mit
einem bitweisen Und (AND), was bedeutet, da"s er ein logisches Und auf
jedes einzelne Bitpaar des linken und rechten Operanden anwendet.
So ergibt zum Beispiel der Ausdruck \fu{93 \& 45} den Wert
\fu{13}. Die Funktionsweise l"a"st sich leicht nachvollziehen,
indem man die bin"are Schreibweise
\index{Zahl!bin""are}
verwendet, also \fu{1011101b}
f"ur \fu{93}, \fu{0101101b} f"ur \fu{45} und \fu{0001101b} f"ur
\fu{13}. Jene Stellen, die in mindestens einem der
Operanden den Wert \fu{0} haben, haben auch im Ergebnis den Wert
\fu{0}. Alle anderen Stellen, also jene, die sowohl im linken
als auch im rechten Operanden den Wert \fu{1} haben, haben auch
im Ergebnis den Wert \fu{1}.

\begin{operator}
\tint\ \fu{`|'} \tint\ & \tint\
\end{operator}
\index{OR!bitweises|ii}

Der Operator \fu{|} verkn"upft beide Operanden mit einem
bitweisen Oder (OR), das hei"st, er wendet auf jedes Bitpaar der
beiden Operanden ein logisches Oder an. Der Ausdruck \fu{93 | 45}
ergibt zum Beispiel den Wert 125.

\begin{operator}
\tint\ \fu{`\char94'} \tint\ & \tint\
\end{operator}
\index{XOR!bitweises|ii}

Der Operator \fu{\char94} verkn"upft seine beiden Operanden mit
einem bitweisen exklusiven Oder (XOR), das hei"st, er wendet auf
jedes Bitpaar des linken und rechten Operanden jeweils ein
logisches exklusives Oder an. So ergibt zum Beispiel der Ausdruck
\fu{93 \char94\ 45} den Wert 112.

\begin{operator}
\tint\ \fu{`<<'} \tint\ & \tint\
\end{operator}
\index{Bitverschiebung!nach links|ii}

Der \fu{<<}-Operator ergibt einen Wert, der dem linken Operanden
entspricht, wobei jedoch die einzelnen Bits desselben um soviel
Stellen nach links geschoben werden, wie der rechte Operand
angibt. W"urde dies in eine Zahl resultieren, die mehr Bits hat,
als der \tint -Typ zul"a"st, so werden die "`"ubersch"ussigen"'
Bits abgeschnitten. Ein Beispiel: Der Ausdruck \fu{93 << 3}
ergibt \fu{744}. Warum dies so ist, l"a"st sich leicht anhand der
Bin"arschreibweise
\index{Zahl!bin""are}
der beiden Zahlen verdeutlichen. Die Zahl
\fu{93} kann auch als \fu{1011101b} geschrieben werden, die Zahl
\fu{744} als \fu{1011101000b}. Wie man sieht, wurden also
lediglich drei Nullstellen am Ende der Zahl eingef"ugt.
Anmerkung: Aufgrund der Eigenschaften des bin"aren Zahlensystems
gilt (ohne Ber"ucksichtigung der negativen Zahlen)
\hbox{$a \>\fu{<<}\> n = a \cdot 2^n.$}

\begin{operator}
\tint\ \fu{`>>'} \tint\ & \tint\
\end{operator}
\index{Bitverschiebung!nach rechts|ii}

Der Operator \fu{>>} ergibt einen Wert, der dem linken Operanden
entspricht, dessen Bits um soviel Stellen nach rechts ger"uckt
werden, wie der rechte Operand angibt. Bits, die dabei rechts
"uber die Einerstelle herausragen w"urden, werden dabei
abgeschnitten. Die sich dabei ergebende Zahl kann (ohne
Ber"ucksichtigung der negativen Zahlen) als
\hbox{$a \>\fu{>>}\> n = \lfloor a / 2^n \rfloor $} angeschrieben werden.
So ergibt z.~B. der Ausdruck \fu{93 >> 3} den Wert \fu{11}.

\begin{operator}
\fu{`-'} \tchar\ & \tchar\ \\
\fu{`-'} \tint\ & \tint\ \\
\fu{`-'} \tfloat\ & \tfloat\
\end{operator}
\index{Negation!numerische|ii}
\index{Negation!logische|see{NOT}}
\index{Negation!bitweise|see{NOT}}

Der un"are Operator \fu{-} negiert den Operanden, das
hei"st, er macht aus einer positiven Zahl eine negative und
umgekehrt. So ergibt zum Beispiel der Ausdruck \fu{- 93} den Wert
\fu{-93}.

\begin{operator}
\fu{`*'} \type{\poi type} & \type{type}
\end{operator}
\index{Zeiger!Dereferenzierung|ii}

Der un"are \fu{*}-Operator dereferenziert den Operanden, das
hei"st, er ergibt jene Variable, auf die der Operand zeigt (ist
der Operand kein Zeiger, ist die Operation illegal). Angenommen,
die Variable \fu{p} ist vom Typ \type{\poi int} und wurde mit
\fu{p = \&i} initialisiert, wobei \fu{i} vom Typ \tint\ ist, so
ist der Ausdruck \fu{*p} mit dem Ausdruck \fu{i} gleichzusetzen.

\begin{operator}
\fu{`\&'} \type{type} & \type{\poi type}
\end{operator}
\index{Zeiger!Ermittlung|ii}
\index{Variable!Adresse}

Der un"are \fu{\&}-Operator ergibt einen Zeiger auf seinen
Operanden, was nat"urlich nur m"oglich ist, wenn dieser keine
Konstante ist. So ist z.~B. der Ausdruck \fu{\&i} mit dem
Ausdruck \fu{p} gleichzusetzen, wenn \fu{*p} gleich \fu{i} ist.
Aus diesem und dem letzten Beispiel wird klar, da"s \fu{\&} das
Gegenst"uck zum \fu{*}-Operator ist und umgekehrt.

\begin{operator}
\fu{`!'} \tint\ & \tint\
\end{operator}
\index{NOT!logisches|ii}

Der Operator \fu{!} ergibt die logische Negation (NOT) seines
Operanden, was bedeutet, da"s der Wert gleich \fu{0} ist, falls
der Operand ungleich \fu{0} ist und umgekehrt. So ergibt z.~B.
der Ausdruck \fu{!93} den Wert \fu{0}.

\begin{operator}
\fu{`\char126'} \tint\ & \tint\
\end{operator}
\index{NOT!bitweises|ii}
\index{NOT!numerisches|see{Negation}}

Der Operator \fu{\char126} errechnet die bitweise Negation seines
Operanden, das hei"st, er wendet auf jedes Bit desselben die
logische Negation an. Das Ergebnis kann je nach Wortl"ange und
Darstellung des Typs \tint\ stark differenzieren. Im Regelfall
kann jedoch mit einer Wortl"ange von 32 Bits und einer
Einerkomplementdarstellung der negativen Zahlen gerechnet werden.
So ergibt zum Beispiel der Ausdruck \fu{\char126 93} auf MS-DOS-
und Windows NT-Rechnern den Wert \fu{-94}.

\begin{operator}
\tchar\ \fu{`=='} \tchar\ & \tint\ \\
\tint\ \fu{`=='} \tint\ & \tint\ \\
\tfloat\ \fu{`=='} \tfloat\ & \tint\
\end{operator}
\index{Vergleich|ii}

Der Operator \fu{==} "uberpr"uft die beiden Operanden auf
Gleichheit, das hei"st, der Ausdruck ergibt einen Wert ungleich
\fu{0}, falls der linke und der rechte Operand den selben Wert
haben, andernfalls \fu{0}.

\begin{operator}
\tchar\ \fu{`<'} \tchar\ & \tint\ \\
\tint\ \fu{`<'} \tint\ & \tint\ \\
\tfloat\ \fu{`<'} \tfloat\ & \tint\
\end{operator}

Der Operator \fu{<} "uberpr"uft, ob der linke Operand kleiner als
der rechte ist. Falls dies zutrifft, ergibt der Ausdruck einen
Wert ungleich \fu{0}, ansonsten \fu{0}.

\begin{operator}
\tchar\ \fu{`>'} \tchar\ & \tint\ \\
\tint\ \fu{`>'} \tint\ & \tint\ \\
\tfloat\ \fu{`>'} \tfloat\ & \tint\
\end{operator}

Der Operator \fu{>} "uberpr"uft, ob der linke Operand gr"o"ser
ist als der rechte. Trifft das zu, ergibt der Ausdruck einen Wert
ungleich \fu{0}, andernfalls \fu{0}.

\begin{operator}
\tchar\ \fu{`<='} \tchar\ & \tint\ \\
\tint\ \fu{`<='} \tint\ & \tint\ \\
\tfloat\ \fu{`<='} \tfloat\ & \tint\
\end{operator}

Der Operator \fu{<=} "uberpr"uft, ob der linke Operand kleiner
oder gleich dem rechten Operanden ist und ergibt in diesem Fall
einen Wert ungleich \fu{0}, ansonsten \fu{0}.

\begin{operator}
\tchar\ \fu{`>='} \tchar\ & \tint\ \\
\tint\ \fu{`>='} \tint\ & \tint\ \\
\tfloat\ \fu{`>='} \tfloat\ & \tint\
\end{operator}

Der Operator \fu{>=} "uberpr"uft, ob der linke Operand gr"o"ser
oder gleich dem rechten Operanden ist und ergibt in diesem Fall
einen Wert ungleich \fu{0}, andernfalls \fu{0}.

\begin{operator}
\tchar\ \fu{`!='} \tchar\ & \tint\ \\
\tint\ \fu{`!='} \tint\ & \tint\ \\
\tfloat\ \fu{`!='} \tfloat\ & \tint\
\end{operator}

Der Operator \fu{!=} "uberpr"uft die beiden Operanden auf
Ungleichheit und ergibt in diesem Fall einen Wert ungleich \fu{0},
bei Gleichheit ergibt sich der Wert \fu{0}.

\begin{operator}
\type{type} \fu{`='} \type{type} & \type{type}
\end{operator}
\index{Zuweisung|ii}

Der Operator \fu{=} weist dem linken Operanden den Wert des
rechten Operanden zu, was aber vorraussetzt, da"s der linke
Operand keine \ind{Konstante}, also entweder eine \ind{Variable} (auch
innerhalb einer \ind{Struktur}), ein
dereferenzierter \ind{Zeiger} oder ein indiziertes \ind{Array} ist. SEL ist
so konzipiert, da"s auch ganze Arrays oder Strukturen mit einer
Zuweisung kopiert werden k"onnen. Beachten Sie aber bei der
Arbeit mit Zeigern, da"s eine Zuweisung nicht bedeutet, da"s die
Werte, auf die die Zeiger zeigen, sondern die Adressen, auf die
die Zeiger verweisen, kopiert werden. Zur Veranschaulichung ein
Beispiel: Angenommen, die Variablen \fu{p} und \fu{q} sind vom
Typ \type{\poi int} und zeigen auf die \tint s \fu{i} und \fu{j},
soda"s \fu{*p} gleich \fu{i} und \fu{*q} gleich \fu{j} ist. In
diesem Fall bewirkt der Ausdruck \fu{p = q} nicht, da"s nun der
Wert von \fu{i} gleich dem von \fu{j} ist, sondern da"s nun
\fu{p}, genauso wie \fu{q}, auf \fu{j} zeigt. Ersterer Effekt
w"are mit dem Ausdruck \fu{*p = *q} zu erzielen.

\begin{operator}
\type{struct} or \type{\poi struct} \fu{`->'} \fu{IDENTIFIER} & \type{type}
\end{operator}
\fuindex{IDENTIFIER}
\index{Struktur!Zugriff auf Element|ii}

Der Operator \fu{->} wird benutzt, um ein Strukturelement zu
erreichen. Um dies zu vereinfachen, kann der Operator sowohl
direkt mit einer Struktur als auch mit einem \ind{Zeiger} auf eine
Struktur arbeiten. Der rechte Operand gibt den Namen des Elements
an, auf das zugegriffen werden soll.

\begin{operator}
\type{\poi type} or \type{\arr{}type} \fu{`['} \fu{expr} \fu{`]'} & \type{type}
\end{operator}
\fuindex{expr}
\index{Array!Zugriff auf Element|ii}

Der \fu{[]}-Operator wird benutzt, um auf ein einzelnes
Element eines Arrays zuzugreifen, wobei dieses Array auch als
\ind{Zeiger} dargestellt werden kann. Auf diese Weise kann auch auf
einen Speicherbereich, der nicht als Array definiert worden ist,
wie auf ein Array zugegriffen werden. Hierzu wird in eckigen
Klammern ein Index angegeben, wobei das erste Element des Arrays
den Index \fu{0} besitzt und negative Indizes nicht erlaubt sind.
Anmerkung: Der Ausdruck \fu{p[i]} kann auch als \fu{*(p + i)}
angeschrieben werden, wobei \fu{p} ein Array oder ein Zeiger
eines beliebigen Typs ist und \fu{i} ein positiver \tint .
Beachten Sie, da"s SEL keine "Uberpr"ufung des Index vornimmt,
das hei"st, da"s es m"oglich ist, auf Elemente zuzugreifen, die
eigentlich nicht mehr zum Array geh"oren, wovon jedoch in den
meisten F"allen abzuraten ist.


\subsubsection{Rangfolge der Operatoren}
\index{Operator!Rangfolge|ii}
\index{Operator!Priorit""at|ii}

Um eine unn"otige Klammerung zu vermeiden, besitzen die
Operatoren in SEL eine Rangfolge. Die arithmetischen Operatoren
sind an die der Mathematik angelehnt, soda"s zum Beispiel der
Ausdruck \fu{a+b*c-d} dem Ausdruck \fu{(a+(b*c))-d} gleichkommt,
was der mathematischen Notation entspricht. Ist allerdings
gew"unscht, da"s \fu{a+b} mit \fu{c-d} multipliziert wird, mu"s
die nat"urliche Operatorenrangfolge mittels Klammerung
umgangen und der Ausdruck zu \fu{(a+b)*(c-d)} umgeformt
werden.

Die folgende Tabelle ist eine Aufstellung aller Operatoren von
SEL, wobei Operatoren mit gleicher Priorit"at in jeweils der
gleichen Zeile stehen. Die Operatoren, die in der Liste weiter
oben stehen, haben niedrigere Priorit"at, das hei"st, sie binden
schw"acher. Die rechte Spalte der Tabelle gibt die Richtung
der Auswertung
\index{Operator!Richtung der Auswertung|ii}
an und bezieht sich auf alle Operatoren in der jeweiligen Zeile.

\medskip
\begin{tabular}{ll}
\verb?=?, \verb?delete? & von rechts nach links \\
\verb?&&?, \verb?||?, \verb?^^? & von links nach rechts \\
\verb?==?, \verb?<?, \verb?>?, \verb?<=?, \verb?>=?, \verb?!=? &
  von links nach rechts\\
\verb?&?, \verb?|?, \verb?^?, \verb?<<?, \verb?>>? &
  von links nach rechts \\
\verb?+?, \verb?-? & von links nach rechts \\
\verb?*?, \verb?/?, \verb?%? & von links nach rechts \\
\verb?!?, \verb?~?, \verb?-? (un"ar), \verb?*? (un"ar), \verb?&? (un"ar) &
  von rechts nach links \\
\verb?->?, \verb?:?, \verb?[]? & von links nach rechts \\
\verb?sizeof?, \verb?new? & ---
\end{tabular}

\section{Typen}
\index{Typ|ii}
\fuindexii{type}
\rwindex{char}
\rwindex{int}
\rwindex{float}
\rwindex{void}
\fuindex{array}
\fuindex{pointer}
\fuindex{IDENTIFIER}

\begin{verbatim}
type:          `char'
             | `int'
             | `float'
             | `void'
             | array
             | pointer
             | IDENTIFIER
             ;
\end{verbatim}

Typen werden in SEL verwendet, um Werte
\index{Wert}
zu klassifizieren. So hat
z.~B. jeder \ind{Ausdruck}, jede \ind{Variable} und jede \ind{Funktion}
einen Typ. Die vier Basistypen von SEL k"onnen durch Kombination
zur Konstruktion weiterer, komplexerer Typen verwendet werden.

\subsection{Basistypen}
\index{Typ!Basistyp}
\typeindexii{int}
\typeindexii{char}
\typeindexii{float}
\typeindexii{void}

Die Programmiersprache SEL kennt vier Basistypen, n"amlich
\tchar , \tint , \tfloat\ und \tvoid . Die ersten beiden Typen,
\tchar\ und \tint\ sind sogenannte Aufz"ahlungstypen,
was sich mit den ganzen Zahlen in der Mathematik vergleichen l"a"st.

\tchar\ wird haupts"achlich verwendet, um \indi{Zeichen} zu
repr"asentieren und hat die Gr"o"se eines Bytes. Obwohl \tchar\
Zeichen repr"asentiert, k"onnen auf ihn alle arithmetischen
Operationen angewendet werden, die auch mit \tint\ funktionieren.

Letzterer wird meistens verwendet, um ganzzahlige Berechnungen
\index{Zahl!ganze|ii}
auszuf"uhren. Aus diesem Grund mu"s er gen"ugend gro"s
dimensioniert sein. In der Regel entspricht ein \tint\ 32 Bits,
was bedeutet, da"s er einen Wertebereich von $-2^{31}$ bis
$+2^{31}-1$ hat. Au"serdem wird \tint\ in SEL zur boolee'schen
Arithmetik
\index{boolee'sche Arithmetik}
herangezogen, also zur Berechnung von logischen
Aussagen. In diesem Fall bedeutet ein Wert von \fu{0} die Aussage
"`logisch falsch"', ein Wert ungleich \fu{0} bedeutet "`logisch
wahr"'.

Der Typ \tfloat\ repr"asentiert rationale Zahlen
\index{Zahl!rationale|ii}
mit begrenzter Genauigkeit. Auf MS-DOS-Rechnern hat er eine Gr"o"se
von 32 Bits.

Eine besondere Stellung unter den Typen nimmt \tvoid\ ein. Er
repr"asentiert nichts und es ist auch nicht m"oglich, einem
\tvoid\ einen Wert zuzuweisen, was auch gar keinen Sinn h"atte.
\tvoid\ wird z.~B. verwendet, um Funktionen zu
deklarieren, die keinen R"uckgabewert
\index{R""uckgabewert}
haben. Au"serdem kann er
in einer Funktion verwendet werden, die einen \ind{Zeiger} unbekannten
Typs verlangt oder zur"uckgibt.

\subsection{Arrays}
\index{Array|ii}
\fuindexii{array}
\fuindex{INT\protect{\_CONSTANT}}
\fuindex{type}

\begin{verbatim}
array:         `[' INT_CONSTANT `]' type
             ;
\end{verbatim}

Ein Array ist eine Zusammenfassung einer bestimmten Anzahl von
Variablen gleichen Typs, die im Speicher als eine Folge von
Variablen
\index{Variable!Folge|ii}
repr"asentiert wird, wobei die Variablen mit kleinerem
Index an einer Stelle geringerer Adresse angesiedelt sind.

In SEL ist es m"oglich, Arrays beliebig zu schachteln, also
mehrdimensionale Array zu erzeugen. So ist etwa
\type{\arr{5}\arr{10}int} ein Array, das aus 5 Arrays besteht,
wobei jedes dieser Arrays aus 10 \tint s besteht. Um auf das
zweite Element aus dem dritten Array zuzugreifen, wird der
Ausdruck \fu{a[3][2]} verwendet. Anmerkung: Dieses Array
beansprucht genausoviel Speicher wie $5 \cdot 10 = 50$ \tint s.

Arrays k"onnen nat"urlich nicht nur aus Elementartypen
konstruiert werden, sondern aus jedem beliebigen Typ, also z.~B.
aus Strukturen. Der Nachteil jedoch ist, da"s die
Gr"o"se des Arrays schon bei der Deklaration feststehen mu"s, sie
kann also nicht bei Bedarf ge"andert werden. Beachten Sie, da"s
SEL beim Zugriff auf Array-Elemente nicht "uberpr"uft, ob das
angegebene Element auch wirklich zum Array geh"ort, also ob das
Array gen"ugend gro"s dimensioniert ist.

\subsection{Zeiger}
\index{Zeiger|ii}
\fuindexii{pointer}
\fuindex{type}

\begin{verbatim}
pointer:       `*' type
             ;
\end{verbatim}

Ein Zeiger ist eine Variable, die die Adresse
\index{Variable!Adresse}
einer anderen
Variablen enth"alt, die von dem Typ ist, der dem \fu{*} in der
Typendeklaration folgt, und hat in der Regel dieselbe Gr"o"se wie
ein \tint . Bis auf die \ind{Zuweisung} sind alle Operationen
funktionsgleich mit der eines Arrays.
\index{Zeiger!""Aquivalenz zu Array}
\index{Array!""Aquivalenz zu Zeiger}

Zeiger k"onnen, wie Arrays,
beliebig geschachtelt werden, das hei"st, da"s es ohne weiteres
m"oglich ist, einen Zeiger auf einen Zeiger zu deklarieren und zu
benutzen. Beachten Sie allerdings, da"s ein zweidimensionales
Array nicht mit einem Zeiger auf einen Zeiger gleichgesetzt
werden kann. So ist z.~B. ein Array \type{\arr{10}int} in
der Verwendung mit einem Zeiger \type{\poi int} gleich (bis auf
die Ausnahme der Zuweisungsoperation), jedoch ist ein Array
\type{\arr{5}\arr{10}int} nicht gleich einem Zeiger \type{\poi
\poi int}, da der Typ, aus dem das Array besteht und auf den der
Zeiger zeigt, zu diesem Zweck gleich sein m"u"sten. In diesem Fall
ist jedoch der Typ, aus dem das Array aufgebaut ist, wieder ein
Array, n"amlich \type{\arr{10}int}. Der Typ, auf den der Zeiger
zeigt, ist jedoch \type{\poi int}. Um ein zweidimensionales Array
wie oben angeschrieben als Zeiger zu repr"asentieren, mu"s dieser
den Typ \type{\poi \arr{10}int} haben.

\subsubsection{Die new- und delete-Operatoren}
\label{newdelete}
\rwindexii{new}
\rwindexii{delete}

Um eine \ind{dynamische Allokation} von Variablen zu erm"oglichen,
existieren in SEL die Operatoren \rw{new} und \rw{delete}. Es
folgt eine kurze Beschreibung:

\begin{operator}
\fu{`new'} \fu{type} & \type{\poi type}
\end{operator}

Das Ergebnis der \rw{new}-Operation ist ein Zeiger auf eine
Variable des Typs, der als Operand angegeben wird. Diese Variable
wird dynamisch allokiert, das hei"st, sie wird dem \indi{Heap}
entnommen. Ist dieser schon so voll, da"s eine Variable des
angegebenen Typs keinen Platz mehr finden w"urde, ist das
Ergebnis ein \ind{Nullzeiger}. Beachten Sie, da"s der Speicherbereich
so lange belegt bleibt, bis \rw{delete} verwendet wird, um ihn
freizugeben, da SEL keine "`garbage collection"' benutzt.
Anmerkung: Um einen Speicherbereich zu allokieren, dessen Gr"o"se
nicht schon vor der Ausf"uhrung des Programms feststeht, ist die Funktion
\fui{MemAlloc} zu benutzen. Weiters ist es mittels der Funktion
\fui{MemReAlloc} m"oglich, einen bereits
allokierten Block zu vergr"o"sern bzw. zu verkleinern, wobei der
Inhalt des Blocks unver"andert bleibt (bei einer Verkleinerung
geht nat"urlich ein Teil des Inhaltes verloren).

\begin{operator}
\fu{`delete'} \type{\poi type} & \tvoid
\end{operator}

Der \rw{delete}-Operator gibt einen Speicherblock frei, der
vorher mit \rw{new} oder \fui{MemAlloc} allokiert wurde. Beachten
Sie, da"s dem Operator genau der Zeiger als Operand angeben
werden mu"s, den \rw{new} oder \fu{MemAlloc} (oder im Falle einer
Ver"anderung der Gr"o"se des Speicherbereichs \fui{MemReAlloc})
geliefert hat, da sonst die Speicherverwaltung ins Wanken kommen
kann, was in einem Programmabsturz resultieren k"onnte.

\subsection{Strukturen}
\index{Struktur|ii}
\fuindexii{struct}
\rwindexii{struct}
\fuindex{IDENTIFIER}
\fuindex{type}
\rwindex{end}

\begin{verbatim}
struct:        `struct' IDENTIFIER
                 {IDENTIFIER `:' type `;'}
               `end'
             ;
\end{verbatim}

Eine Struktur ist eine Zusammenfassung mehrerer Variablen
\index{Variable!Zusammenfassung mehrerer|ii}
beliebigen Typs
\index{Typ}
unter einem gemeinsamen Namen. Mittels des
Operators \fu{->} kann auf ein einzelnes Element der Struktur
zugegriffen werden.
\index{Struktur!Zugriff auf Element}
Dem \rw{struct}-Schl"usselwort folgt der
zuk"unftige Name der Struktur---bei der Deklaration von Variablen
wird dieser Name als Typ verwendet. Danach werden die einzelnen
Elemente der Struktur definiert. Die Namen dieser Elemente d"urfen
sich ohne weiteres mit anderen Variablennamen des Programms
decken, da durch den Operator \fu{->} eine eindeutige
Identifikation gew"ahrt ist, allerdings m"ussen sie innerhalb der
Struktur eindeutig sein. Die Elemente d"urfen beliebige Typen
haben, es ist sogar erlaubt, innerhalb einer Struktur einen
Zeiger auf ebendiese Struktur zu deklarieren, nicht aber ein
Array aus der Struktur und auch nicht die Struktur selbst, da
diese Deklarationen die Gr"o"se der Struktur ins Unendliche
anwachsen lie"sen. Die Gr"o"se einer Struktur ist in der Regel
gleich der Summe der Gr"o"sen ihrer Elemente.

Beachten Sie, da"s die Definition einer Struktur nur in dem \ind{Modul}
bekannt ist, in der sie get"atigt wird, selbst, wenn sie
innerhalb eines \rw{global}-Blocks steht. In der Regel ist es
deshalb zweckm"a"sig, Strukturen, die von mehreren Modulen
verwendet werden, in Include-Dateien
\index{Include-Datei}
zu definieren und diese in den
jeweiligen Modulen zu inkludieren.

\example
Die folgende Struktur ist ein Element einer verketteten \indu{Liste},
deren Daten die Adressen von Personen sind. Das Element \fu{Next}
zeigt auf das jeweils n"achste Element der Liste.

\index{Adresse@{\tt Adresse}~(\rw{struct})|ii}
\begin{verbatim}
struct Adresse
  Name    : char[26];
  Vorname : char[31];
  PLZ     : int;
  Stadt   : char[36];
  Strasse : char[36];
  Next    : *Adresse;
end
\end{verbatim}

\subsection{Zusammengesetzte Datentypen}
\index{Typ!zusammengesetzter|ii}

Nat"urlich ist es in SEL auch m"oglich, Datentypen aus den oben
besprochenen zu kombinieren, also zum Beispiel ein Array aus
Strukturen oder eine Struktur, die Arrays oder Zeiger beinhaltet,
zu definieren. Zu beachten ist nur, da"s die vorhandene
Speicherkapazit"at nicht "uberschritten wird. Unter MS-DOS
kommt weiterhin die 64~KByte-Barriere hinzu, die aus der
Segmentierung des Speichers im Real-Mode der 80x86-Prozessoren
resultiert. Deshalb ist es nicht m"oglich, in MS-DOS Variablen zu
definieren, die gr"o"ser als 64~KByte w"aren (sehr wohl ist es
m"oglich, Typen dieser Gr"o"se zu definieren, solange keine
Variablen dieses Typs verwendet werden).

\subsection{Der sizeof-Operator}
\index{Typ!Gr""o""se}
\rwindexii{sizeof}

\begin{operator}
\fu{`sizeof'} \type{type} & \tint\
\end{operator}

Der \rw{sizeof}-Operator verlangt als einzigen Operanden einen
Typ und gibt dessen Gr"o"se in Bytes zur"uck. Diese kann f"ur
ein und denselben Typen je nach Rechner verschieden sein, in der
Regel ist das jedoch nicht der Fall. Trotzdem sollten sie immer
\rw{sizeof} verwenden, wenn es um die Gr"o"se eines Typs geht und
nicht von einer bestimmten Gr"o"se ausgehen. Anmerkung: Der
einzige Typ, dessen Gr"o"se auf allen Rechnern gleich ist, ist
\tvoid\
\typeindex{void}
(nicht jedoch zum Beispiel \type{\poi void}). Seine
Gr"o"se betr"agt \fu{0}.

\subsection{Typecasting}
\index{Typecasting|ii}
\index{Typ!Umwandlung|see{Typecasting}}
\fuindex{type}

\begin{operator}
\type{\_type} \fu{`:'} \fu{type} & \type{type}
\end{operator}

Um Ausdr"ucke eines Typs in einen anderen umzuwandeln, bietet SEL
Typecasting. Der Operator \fu{:} wandelt seinen linken Operanden
in den Typ um, den der rechte Operand angibt. Dabei k"onnen
folgende Umwandlungen durchgef"uhrt werden:

\medskip
\begin{tabular}{ll}
Ausgangstyp & Zieltyp \\[0.5ex]
\tchar\ & \tint , \tfloat\ \\
\tint\ & \tchar , \tfloat , \type{\poi type}, \type{\arr{}type} \\
\tfloat\ & \tchar , \tint\ \\
\type{\poi type} & \tint , \type{\arr{}type} \\
\type{\arr{}type} & \tint , \type{\poi type}
\end{tabular}
\medskip

Bei der Umwandlung von \tint s
\typeindex{int}
in Arrays
\index{Array}
und \ind{Zeiger} und
umgekehrt ist zu beachten, da"s der \tint\ die Speicheradresse
angibt, auf die der Pointer zeigt, bzw. an der das Array beginnt
(die Adresse des ersten Arrayelements).

Zus"atzlich zum expliziten Typecasting f"uhrt SEL noch implizites
Typecasting durch, falls dies n"otig ist. Verlangt also eine
Funktion einen Parameter vom Typ \tfloat ,
\typeindex{float}
und es wird ein \tint\
"ubergeben, wird dieser automatisch in einen \tfloat\
umgewandelt. Beim impliziten Typecasting gelten wie beim
expliziten Typecasting die oben beschriebenen Regeln.

\section{Bezeichner}
\label{identifiers}
\index{Bezeichner|ii}
\fuindexii{IDENTIFIER}

\begin{verbatim}
IDENTIFIER: [A-Za-z_][A-Za-z0-9_]*
\end{verbatim}

Bezeichner werden in SEL f"ur Funktions-, Variablen- und
Strukturnamen
\index{Funktion!Name|see{Bezeichner}}
\index{Variable!Name|see{Bezeichner}}
\index{Struktur!Name|see{Bezeichner}}
verwendet. Ein Bezeichner mu"s mit einem Buchstaben
oder einem Unterstrich beginnen, danach folgt eine beliebige
Anzahl von Buchstaben, Ziffern oder Unterstrichen. So ist z. B.
\fu{\_xYz1} ein g"ultiger Bezeichner, nicht aber
\fu{1xy}, da er mit einer Ziffer beginnt. Die Bezeichner in SEL
sind {\em case-sensitiv}, das hei"st, es wird zwischen Gro"s-
und Kleinschreibung unterschieden. So ist z. B. \fu{counter} ein
anderer Bezeichner als \fu{Counter} oder \fu{COUNTER}. Beachten
Sie aber, da"s sie keines der reservierten W"orter
\index{reserviertes Wort|ii}
\index{Schl""usselwort|see{reserviertes Wort}}
als einen Bezeichner verwenden d"urfen. Der Vollst"andigkeit halber hier
eine Liste aller reservierten W"orter von SEL:

\medskip
\begin{tabular}{lllllll}
\bf begin & \bf break  & \bf case   & \bf char  & \bf default & \bf delete   & \bf downto \\
\bf else  & \bf end    & \bf extern & \bf float & \bf for     & \bf function & \bf global \\
\bf if    & \bf int    & \bf local  & \bf new   & \bf repeat  & \bf return   & \bf sizeof \\
\bf step  & \bf struct & \bf switch & \bf to    & \bf until   & \bf void     & \bf while  \\
\end{tabular}

\chapter{Der Pr\protect{"aprozessor}}
\index{Pr""aprozessor|ii}

Bevor ein SEL-Programm interpretiert bzw. compiliert wird, wird
es durch den Pr"aprozessor bearbeitet. Seine Hauptaufgaben sind
das Entfernen von Kommentaren, das Einf"ugen von Include-Dateien
und das Expandieren von Makros.

\section{Kommentare}
\index{Kommentar|ii}

Kommentare werden in SEL durch das Zeichen \fu{//} eingeleitet
und dauern bis an das Ende der jeweiligen Zeile an. Sollen also
l"angere Kommentare in den Quelltext eingef"ugt werden, mu"s am
Anfang jeder Kommentarzeile \fu{//} stehen. Beachten Sie bitte,
da"s \fu{//}, eingeschlossen in doppelte Anf"uhrungszeichen, nicht
den Anfang eines Kommentars, sondern eine String-Konstante
\index{Konstante!String}
darstellt.

\example
\fuindexii{Demo}
\fuindexuu{WriteInt}
\fuindexuu{WriteString}
\begin{verbatim}
function Demo (void) : void
begin
  WriteInt(20);        // Das ist ein Kommentar
  WriteString("Das // ist keiner\n");
end
\end{verbatim}

\section{Die \#define-Direktive}
\fuindexii{\#define}

Mittels der \fu{\#define}-Direktive ist es in SEL m"oglich,
Makros
\index{Makro|ii}
zu definieren. Ein Beispiel:

\begin{verbatim}
#define MACRO        3+5
\end{verbatim}

Tritt nun im nachfolgenden Quellcode irgendwo (au"ser innerhalb
einer String-Konstanten
\index{Konstante!String}
oder eines Kommentars) der Bezeichner
\fu{MACRO} auf, wird dieser vom Pr"aprozessor durch \fu{3+5}
ersetzt. Bei Makros wie dem obigen ist zu beachten, da"s ein
Makro nicht automatisch geklammert wird, d.~h., da"s der Ausdruck
\fu{MACRO*2} nicht das Ergebnis \fu{16}, sondern \fu{13}
(\fu{3+5*2}) hat.

Ist ein Makrotext zu lang, um ihn in einer Zeile unterzubringen,
kann er mit \fu{\char92}
\index{Backslash (\fu{\char92})|ii}
abgeteilt werden:

\begin{verbatim}
#define XYZ          if (x == y)     \
                       WriteInt(z);  \
                     end
\end{verbatim}

Beachten Sie, da"s die \fu{\#define}-Direktive immer am Beginn
einer Zeile stehen mu"s, und in dieser Zeile au"ser der Direktive, dem
Makronamen und dem einzusetzenden Text sonst nichts.

Makros mit Parametern, wie etwa in C, sind in der momentanen
Version von SEL nicht vorhanden, sind jedoch f"ur eine kommende
Version geplant.

\section{Die \#include-Direktive}
\index{Include-Datei|ii}
\fuindexii{\#include}

Mit der \fu{\#include}-Direktive ist es m"oglich, in eine
Quelldatei eine andere einzuf"ugen. Dabei mu"s auf \fu{\#include}
ein in Anf"uhrungszeichen gesetzter Dateiname folgen. Ebenso wie
die \fu{\#define}-Direktive mu"s auch die
\fu{\#include}-Direktive alleine am Anfang einer Zeile stehen.

\example
\fuindexuu{WriteString}
\begin{verbatim}
#include "conio.sh"

function main () : void
begin
  WriteString("Test\n");
end
\end{verbatim}

Hier wird an der Stelle der \fu{\#include}-Direktive der Inhalt
der Datei \fu{`conio.sh'} eingef"ugt. Sollte diese Datei selbst
\fu{\#include}-Direktiven enthalten, werden auch diese vom
Pr"aprozessor ausgef"uhrt, es ist also m"oglich,
\fu{\#include}-Direktiven zu verschachteln.

F"ur alle
Bibliotheksfunktionen von SEL gibt es Include-Dateien, die
deren Funktionsdeklarationen enthalten, soda"s diese nicht
jedesmal von Hand eingef"ugt werden m"ussen. Die Namen dieser
Dateien sind f"ur jede Funktion in der Funktionsreferenz
angegeben.

\chapter{Funktionsreferenz}
\label{funcref}

%%%%%%%%%%%%%%%%%%%%%%%%%%%%%%%%%%%%%%%%%%%%%%%%%

\begin{function}{MemAlloc}

\syntax
\#include "memory.sh"
function MemAlloc (s : int) : *void
\endgroup

\parameters
\param{s}{\tint} Gr"o"se des zu allokierenden Speicherbereichs \cr

\usage
Die Funktion \fu{MemAlloc} allokiert
\index{dynamische Allokation}
einen Speicherbereich mit
der Gr"o"se \fu{s} Bytes. Ist eine solche Allokation wegen
Speichermangels nicht m"oglich, gibt die Funktion einen
\ind{Nullzeiger} zur"uck. Im Gegensatz zum Operator \rwi{new} ist es mit
\fu{MemAlloc} auch m"oglich, einen Speicherbereich zu allokieren,
dessen Gr"o"se erst bei Laufzeit bestimmt wird. Der allokierte
Speicherbereich sollte, wenn er nicht mehr ben"otigt wird, mit
der Funktion \fu{MemFree} freigegeben werden.

\fseealso
\fu{MemFree}, \rw{new} (Kapitel \ref{newdelete})
\end{function}

%%%%%%%%%%%%%%%%%%%%%%%%%%%%%%%%%%%%%%%%%%%%%%%%%

\begin{function}{MemAvail}

\syntax
\#include "memory.sh"
function MemAvail () : int
\endgroup

\usage
Die Funktion \fu{MemAvail} gibt die Anzahl der noch freien Bytes
auf dem \ind{Heap} zur"uck, also jenes Speicherbereichs, der vom
Operator \rwi{new} (siehe Kapitel \ref{newdelete}) und der
Funktion \fu{MemAlloc} benutzt wird, um Speicher zu allokieren.

\fseealso
\fu{MemAlloc}, \rw{new} (Kapitel \ref{newdelete})
\end{function}

%%%%%%%%%%%%%%%%%%%%%%%%%%%%%%%%%%%%%%%%%%%%%%%%%
\newpage

\begin{function}{MemFree}

\syntax
\#include "memory.sh"
function MemFree (p : *void) : void
\endgroup

\parameters
\param{p}{\type{\poi void}} freizugebender Speicherblock \cr

\usage
Die Funktion \fu{MemFree} gibt einen Speicherblock, der mit
\fu{MemAlloc}, \fu{MemReAlloc} oder \rwi{new} allokiert worden
ist, frei. Beachten Sie jedoch, da"s Speicherbl"ocke, die mit
\rw{new} allokiert worden sind, besser mit \rwi{delete}
freigegeben werden sollten.

\fseealso
\fu{MemAlloc}, \fu{MemReAlloc}, \rw{delete}, \rw{new} (Kapitel \ref{newdelete})
\end{function}

%%%%%%%%%%%%%%%%%%%%%%%%%%%%%%%%%%%%%%%%%%%%%%%%%

\begin{function}{MemReAlloc}

\syntax
\#include "memory.sh"
function MemReAlloc (p : *void, s : int) : *void
\endgroup

\parameters
\param{p}{\type{\poi void}} allokierter Speicherblock \cr
\param{s}{\tint} neue Gr"o"se \cr

\usage
Die Funktion \fu{MemReAlloc} "andert die Gr"o"se des
Speicherblocks \fu{p},
\index{dynamische Allokation}
der mit \fu{MemAlloc}, \fu{MemReAlloc}
oder \rwi{new} allokiert worden sein mu"s, auf die Gr"o"se \fu{s}
Bytes. Ist \fu{p} ein Nullzeiger, wird ein Speicherbereich der
Gr"o"se \fu{s} wie mit \fu{MemAlloc} neu angelegt. Ist \fu{s}
gleich \fu{0}, dann wird der Speicherblock wie mit \fu{MemFree}
freigegeben. Andernfalls wird seine Gr"o"se ver"andert, wobei
sein Inhalt soweit wie m"oglich bestehen bleibt. Ist es aus
Speichermangel nicht m"oglich, die Gr"o"se des Blocks zu "andern,
gibt die Funktion einen \ind{Nullzeiger} zur"uck, und der alte Block
bleibt bestehen. Andernfalls gibt die Funktion einen Zeiger auf
den neuen Block zur"uck.

\fseealso
\fu{MemAlloc}, \fu{MemFree}, \rw{new} (Kapitel \ref{newdelete})
\end{function}

%%%%%%%%%%%%%%%%%%%%%%%%%%%%%%%%%%%%%%%%%%%%%%%%%

\begin{function}{Random}
\index{Zufallszahl|ii}

\syntax
\#include "stdlib.sh"
function Random (i : int) : int
\endgroup

\parameters
\param{i}{\tint} H"ochstgrenze f"ur die Zufallszahl \cr

\usage
Die Funktion Random gibt eine Zufallszahl
$0 \leq x \leq \fu{i}$ zur"uck. Um zu gew"ahrleisten, da"s
diese Zahl nicht f"ur jeden Aufruf des Programms dieselbe ist,
mu"s der Zufallszahlengenerator mit der Funktion \fu{Randomize}
initialisiert werden.

\fseealso
\fu{Randomize}
\end{function}

%%%%%%%%%%%%%%%%%%%%%%%%%%%%%%%%%%%%%%%%%%%%%%%%%
\newpage

\begin{function}{Randomize}
\index{Zufallszahl!Initialisierung des Generators|ii}

\syntax
\#include "stdlib.sh"
function Randomize () : void
\endgroup

\usage
Die Funktion \fu{Randomize} initialisiert den
Zufallszahlengenerator von SEL. Hierf"ur wird die aktuelle
Uhrzeit verwendet. Das garantiert, da"s bei jedem Aufruf des
Programms eine andere Reihe von Zufallszahlen generiert wird.

\fseealso
\fu{Random}
\end{function}

%%%%%%%%%%%%%%%%%%%%%%%%%%%%%%%%%%%%%%%%%%%%%%%%%

\begin{function}{ReadChar}
\typeindex{char}

\syntax
\#include "conio.sh"
function ReadChar () : char
\endgroup

\usage
Die Funktion \fu{ReadChar} liest ein einzelnes Zeichen von der
Systemkonsole ein, ohne es auf dem Bildschirm auszugeben. Dazu
kann die Funktion \fu{WriteChar} verwendet werden.

\fseealso
\fu{WriteChar}
\end{function}

%%%%%%%%%%%%%%%%%%%%%%%%%%%%%%%%%%%%%%%%%%%%%%%%%

\begin{function}{ReadFloat}
\typeindex{float}

\syntax
\#include "conio.sh"
function ReadFloat () : float
\endgroup

\usage
\fu{ReadFloat} liest einen \tfloat\ in Dezimal- bzw.
\indu{Exponentialschreibweise} (z.~B. \fu{1.23E45}) von der
Systemkonsole ein. Backspaces werden richtig interpretiert,
soda"s der Anwender eine Korrekturm"oglichkeit hat.

\fseealso
\fu{WriteFloat}
\end{function}

%%%%%%%%%%%%%%%%%%%%%%%%%%%%%%%%%%%%%%%%%%%%%%%%%

\begin{function}{ReadInt}
\typeindex{int}

\syntax
\#include "conio.sh"
function ReadInt () : int
\endgroup

\usage
\fu{ReadInt} liest einen \tint\ als Dezimalzahl
\index{Zahl!dezimale}
von der
Systemkonsole ein. Dabei werden Backspaces ber"ucksichtigt,
um dem Anwender eine Korrekturm"oglichkeit zu geben. Zu beachten
ist allerdings, da"s kein positives Vorzeichen (\fu{+}) angegeben
werden darf.

\fseealso
\fu{WriteInt}
\end{function}

%%%%%%%%%%%%%%%%%%%%%%%%%%%%%%%%%%%%%%%%%%%%%%%%%
\newpage

\begin{function}{ReadString}
\index{String}

\syntax
\#include "conio.sh"
function ReadString (s : *char) : int
\endgroup

\parameters
\param{s}{\type{\poi char}} Speicherplatz f"ur den String \cr

\usage
Die Funktion \fu{ReadString} liest einen nullterminierten String
von der Systemkonsole in den Speicherbereich, den der Parameter
\fu{s} angibt, ein. Dieser Speicherbereich mu"s gen"ugend gro"s
dimensioniert sein, da \fu{ReadString} keinerlei Tests vornimmt,
ob er f"ur den String ausreichend ist. Die Stringeingabe wird vom
Anwender durch ein Newline-Zeichen beendet, das jedoch im String
nicht aufscheint. Die Funktion gibt die Anzahl der gelesenen
Zeichen, inklusive dem Newline, also mindestens \fu{1}, zur"uck.

\fseealso
\fu{WriteString}
\end{function}

%%%%%%%%%%%%%%%%%%%%%%%%%%%%%%%%%%%%%%%%%%%%%%%%%

\begin{function}{WriteChar}
\typeindex{char}

\syntax
\#include "conio.sh"
function WriteChar (c : char) : void
\endgroup

\parameters
\param{c}{\tchar} auszugebendes Zeichen \cr

\usage
Die Funktion \fu{WriteChar} gibt das ihr "ubergebene Zeichen
\fu{c} auf der Systemkonsole aus, wobei Steuerzeichen (siehe
Kapitel \ref{specialchar}) ber"ucksichtigt werden.

\fseealso
\fu{ReadChar}
\end{function}

%%%%%%%%%%%%%%%%%%%%%%%%%%%%%%%%%%%%%%%%%%%%%%%%%

\begin{function}{WriteFloat}
\typeindex{float}

\syntax
\#include "conio.sh"
function WriteFloat (f : float) : void
\endgroup

\parameters
\param{f}{\tfloat} auszugebender Wert \cr

\usage
Die Funktion \fu{WriteFloat} gibt einen \tfloat\ in passender
Schreibweise (Dezimal- oder \ind{Exponentialschreibweise}) auf der
Systemkonsole aus.

\fseealso
\fu{ReadFloat}
\end{function}

%%%%%%%%%%%%%%%%%%%%%%%%%%%%%%%%%%%%%%%%%%%%%%%%%
\newpage

\begin{function}{WriteInt}
\typeindex{int}

\syntax
\#include "conio.sh"
function WriteInt (i : int) : void
\endgroup

\parameters
\param{i}{\tint} auszugebender Wert \cr

\usage
Die Funktion \fu{WriteInt} gibt den Parameter \fu{i} als
Dezimalzahl
\index{Zahl!dezimale}
auf der Systemkonsole aus.

\fseealso
\fu{ReadInt}
\end{function}

%%%%%%%%%%%%%%%%%%%%%%%%%%%%%%%%%%%%%%%%%%%%%%%%%

\begin{function}{WriteString}
\index{String}

\syntax
\#include "conio.sh"
function WriteString (s : *char) : void
\endgroup

\parameters
\param{s}{\type{\poi char}} auszugebender String \cr

\usage
Die Funktion \fu{WriteString} gibt einen nullterminierten String
auf der Systemkonsole aus. Steuerzeichen (siehe Kapitel
\ref{specialchar}) werden dabei interpretiert.

\fseealso
\fu{ReadString}
\end{function}

%%%%%%%%%%%%%%%%%%%%%%%%%%%%%%%%%%%%%%%%%%%%%%%%%

\newpage
\index{Index|ii}
% \addcontentsline{toc}{chapter}{Index}
\footnotesize
\printindex

\end{document}


\begin{document}

\begin{titlepage}
\vspace*{2cm}
\begin{flushleft}
\inch SEL\rm \hfill Version 0.3
\end{flushleft}
\rule{\textwidth}{2mm}
\begin{flushright}
\huge{\bf Kurzreferenz}\\[0.5ex]
\end{flushright}
\vspace{13.5cm}
\begin{flushleft}
\large \bf Mark Probst\\
\rule[2mm]{\textwidth}{1mm}
\end{flushleft}
\end{titlepage}

\tableofcontents

\newpage

\chapter{\protect{"Uber} diesen Text}

Dieser Text hat zum Ziel, die Programmiersprache SEL (Schani
Electronics Language) in ihrer momentanen Version 0.3
vollst"andig zu beschreiben. Dabei ist er als Kurzreferenz
aufgebaut, kann aber auch zur Not verwendet werden, um die
Sprache SEL kennenzulernen, vor allem von Programmierern, die
bereits {\sc Pascal}
\index{Pascal@{\sc Pascal}}
oder \ind{C}
kennen, da SEL, verglichen mit diesen Sprachen, kaum etwas Neues
bietet.

Die einzigen Implementationen von SEL, die zur Zeit verf"ugbar
sind, sind die beiden \ind{Interpreter} unter MS-DOS und Windows
NT. Die Funktionsreferenz (Kapitel \ref{funcref}) spiegelt deren
momentanen Stand in Bezug auf verf"ugbare Funktionen wider.

\section{Typographie}
\index{Typographie|ii}

Um Mi"sverst"andnisse zu vermeiden, werden in diesem Text
verschiedene Schriftarten f"ur verschiedene Zwecke verwendet.
Durch eine {\em kursive Schriftart} werden neue Begriffe
gekennzeichnet und wichtige Textabschnitte hervorgehoben.

Schl"usselw"orter der Sprache SEL werden in {\bf fetter Schrift}
geschrieben. Eine Aufstellung aller Schl"usselw"orter finden Sie
in Kapitel \ref{identifiers}. Operatoren hingegen werden in
{\tt Schreibmaschinenschrift} geschrieben.

Typen werden nicht nur in fetter Schrift geschrieben, sondern
zus"atzlich in geschwungene \type{Klammern} gesetzt, um sie von
Schl"usselw"ortern unterscheiden zu k"onnen.

Bezeichner, Makros, Funktionsnamen und Nonterminals der
SEL-Grammatik werden ebenfalls in
{\tt Schreib"-maschi"-nen"-schrift} geschrieben.

Indexeintr"age
\index{Index|ii}
\index{Stichwortverzeichnis|see{Index}}
sind wie folgt zu interpretieren: Kursive
Eintr"age wie etwa {\em 23} deuten auf einen Haupteintrag, wo die
meisten Informationen zu dem jeweiligen Thema zu finden sind.
Unterstrichene Eintr"age wie z.~B. \underline{31} bedeuten, da"s
auf dieser Seite ein Beispiel zum Thema zu finden ist, das jedoch
nicht unbedingt n"aher erl"autert wird.

\section{Die Backus-Naur-Form}
\index{Backus-Naur-Form|ii}
\index{BNF|see{Backus-Naur-Form}}

Die Backus-Naur-Form (kurz BNF) wird verwendet, um Grammatiken
darzustellen. Dabei besteht jede \indi{Grammatik} aus mehreren
Symbolen, die ihrerseits jeweils aus mindestens einer
\indi{Produktion} bestehen. Symbole werden links angeschrieben.
Symbole rechts davon, durch einen Doppelpunkt oder ein anderes
Zeichen vom Grammatiksymbol getrennt, jeweils durch einen
vertikalen Strich (\fu{|}) voneinander getrennt. Auf die letzte
Produktion eines jeden Symbols folgt ein Semikolon (\fu{;}).

Produktionen sind Aneinanderreihungen von Terminal- und
Nichtterminalsymbolen. Nichtterminalsymbole sind jene, die auf
den linken Seiten stehen. Terminalsymbole sind solche, die in
Quelltexten der Sprache direkt vorkommen k"onnen, und solche,
die durch regul"are Ausdr"ucke definiert werden (siehe n"achstes
Unterkapitel). Erstere werden unter Hochkomma angef"uhrt (wie
etwa \fu{'new'}), w"ahrend letztere durch Gro"sschreibung
gekennzeichnet sind.

Ein Satz einer Sprache mu"s nun so konstruiert sein, da"s er
durch Ersetzen von Nichtterminalsymbolen durch bestimmte Produktionen
derselben hergeleitet werden kann.

\newpage
\example
\begin{verbatim}
satz:         subjekt praedikat
            ;
subjekt:      artikel substantiv
            ;
artikel:      `der'
            | `die'
            | `das'
            ;
substantiv:   `frau'
            | `mann'
            | `kind'
            ;
pradikat:     `geht'
            | `schlaeft'
            | `isst'
            ;
\end{verbatim}

Diese Grammatik hat f"unf Nonterminalsymbole und kann einige
primitive deutsche S"atze herleiten wie etwa "`der mann
schlaeft"'. Dabei geht man vom Symbol \fu{satz} aus und setzt
f"ur die Nonterminals \fu{subjekt} und \fu{praedikat} jeweils
eine ihrer Produktionen ein. F"ur \fu{subjekt} wird \fu{artikel}
und \fu{substantiv} eingesetzt und f"ur diese wiederum die beiden
Terminals \fu{`der'} und \fu{`mann'}. Dasselbe geschieht mit
\fu{pradikat}.

Dieses Beispiel demonstriert nicht nur, wie sich S"atze von einer
BNF herleiten lassen, sondern auch, da"s diese nur f"ur den
syntaktischen Teil der Grammatik eingesetzt werden kann. So
l"a"st sich aus obiger BNF auch der Satz "`der kind isst"'
herleiten, der semantisch inkorrekt ist. Deshalb wird der
semantische
\index{Semantik}
Teil in der Regel verbal formuliert, wie etwa in
dieser Kurzreferenz.

Zus"atzlich zu dieser Grundfunktionalit"at beherrscht die BNF
noch zwei zus"atzliche Hilfsmittel zur Formulierung von
Grammatiken: Wird ein Teil einer Produktion in geschwungene Klammern
eingeschlossen (\fu{\{\}}), dann darf dieser Teil im Quelltext
weggelassen oder beliebig oft aufgef"uhrt werden. Durch die Produktion
\fu{`a' \{`b'\} `a'} werden also alle S"atze \fu{aa},
\fu{aba}, \fu{abba}, \fu{abbba}, \ldots\ dargestellt.

Zus"atzlich k"onnen Teile von Produktionen in eckige Klammern
eingeschlossen werden, was bedeutet, da"s sie optional sind, also
im Quelltext weggelassen k"onnen, aber auf keinen Fall "ofter als
einmal auftreten d"urfen. Die Produktion \fu{`a' [`b'] `a'}
beschreibt also die S"atze \fu{aa} und \fu{aba}.

\section{Regul\protect{"are} Ausdr\protect{"ucke}}
\index{regul""arer Ausdruck|ii}

Regul"are Ausdr"ucke werden benutzt, um Zeichenfolgen
wie etwa Bezeichner oder Zahlen zu beschreiben. Dabei werden
aufeinanderfolgende Zeichen hintereinander angeschrieben.
Zus"atzlich gibt es Steuerzeichen, mit denen es m"oglich ist, die
beschriebene Zeichenfolge zu variieren. Es folgt eine
Aufstellung:

\begin{tabular}{lp{13cm}}
\fu{()} & Die beiden runden Klammern werden benutzt, um eine
          Folge von Zeichen zu klammern, sie also nur als ein Zeichen zu
          behandeln. Das macht es m"oglich, Steuerzeichen wie etwa \fu{+}
          auf eine ganze Zeichenfolge anstatt nur auf das letzte Zeichen
          anzuwenden. \\
\fu{[]} & Die beiden eckigen Klammern werden verwendet, um
          Zeichen zu gruppieren. Der ganze Klammerausdruck steht
          f"ur ein beliebiges Zeichen, das innerhalb der Klammer
          steht. Um auch gr"o"sere Gruppen anschreiben zu
          k"onnen, ist es m"oglich, innerhalb der Gruppe mittels
          \fu{-} von einem bestimmten bis zu einem anderen
          bestimmten Zeichen alle Zeichen einzubinden. So gibt
          z.~B. \fu{[0-9]} die Gruppe aller Ziffern an. Tritt
          diese Gruppe irgendwo auf, steht sie f"ur eine
          beliebige Ziffer. Wird als erstes Zeichen einer Gruppe
          \fu{\char94} angegeben, umfa"st die Gruppe all jene
          Zeichen, die nicht innerhalb der Klammern aufgef"uhrt
          sind. \\
\end{tabular}          

\begin{tabular}{lp{13cm}}
\fu{`'} & Sollen Zeichen verwendet werden, die schon als
          Steuerzeichen vergeben sind, werden sie in einfache
          Anf"uhrungszeichen eingeschlossen. \\
\fu{.}  & Der Punkt steht f"ur jedes beliebige Zeichen au"ser dem
          Zeilenwechsel \fu{\char92 n}. \\
\fu{|}  & Ein Ausdruck \fu{a|b} kann entweder durch \fu{a} oder
          durch \fu{b} ersetzt werden. Um zu vermeiden, da"s
          immer der gesamte Ausdruck mit ODER verkn"upft wird, werden
          Oder-Ausdr"ucke in Klammern gesetzt. \\
\fu{?}  & Ein Zeichen oder ein Ausdruck in Klammern, gefolgt von
          einem \fu{?}, ist optional, darf also auftreten, kann
          aber auch weggelassen werden. \\
\fu{+}  & Ein Zeichen oder ein Ausdruck in Klammern, gefolgt von
          einem \fu{+}, steht f"ur ein oder mehrere Vorkommnisse
          dieses Zeichens oder Ausdrucks. \\
\fu{*}  & Der Operator \fu{*} ist mit dem Operator \fu{+}
          identisch, au"ser in dem Fall, da"s das vorhergehende Zeichen
          bzw. der in Klammern stehende Ausdruck auch weggelassen
          werden kann. \\
\end{tabular}

\example
\begin{verbatim}
IDENTIFIER: [a-zA-Z][^\n ]*
\end{verbatim}

Dieser regul"are Ausdruck beschreibt all jene Bezeichner, die mit
einem Buchstaben beginnen, auf den beliebige Zeichen
(au"ser dem Zeilenwechsel \fu{\char92 n} und dem Leerzeichen)
folgen k"onnen. G"ultige \fu{IDENTIFIER} sind etwa \fu{ax*} und
\fu{ZX\#\%}.

\begin{verbatim}
HEX_NUMBER: 0x[0-9a-fA-F]+
\end{verbatim}

Der Ausdruck \fu{HEX\_NUMBER} beschreibt eine hexadezimale
Zahlenkonstante,
\index{Zahl!hexadezimale}
wie sie etwa in \ind{C} verwendet werden kann. Sie beginnt mit
den Zeichen \fu{0x}, auf die mindestens eine Hexadezimalziffer
(Dezimalziffer oder Gro"s- bzw. Kleinbuchstabe von \fu{A} bis
\fu{F}) folgen mu"s. Zu der unendlich gro"sen Menge der
beschriebenen Zahlen z"ahlen \fu{0xff} und \fu{0x123}.

\chapter{Sprachreferenz}

Dieses Kapitel ist eine Referenz "uber die komplette Grammatik
\index{Grammatik!von SEL|ii}
der Programmiersprache SEL. Der Aufbau ist dabei top-down-orientiert,
da es meines Erachtens nat"urlicher ist, vom Gro"sen ins Kleine
zu gehen, als umgekehrt.

Das erste Unterkapitel widmet sich deshalb der groben Struktur
eines SEL-Programms. Es wird die Modularit"at von SEL
besprochen sowie der grundlegende Aufbau einer Funktion. Weiters
wird in diesem Unterkapitel die wichtigste Funktion eines
SEL-Programms beschrieben, n"amlich die \fu{main}-Funktion.

Das zweite Unterkapitel bespricht die Kontrollstrukturen, also
die Verzweigungen und Schleifen, die die strukturierte
Programmiersprache ausmachen.

Das dritte Unterkapitel widmet sich den Ausdr"ucken oder
{\em expressions}, die den Hauptanteil an einem Programm haben. Zu
Ausdr"ucken z"ahlen zum Beispiel Zuweisungen, Vergleiche oder
arithmetische Operationen.

Im vierten Unterkapitel werden die Datentypen besprochen, die
mehr oder weniger Bestandteile der Ausdr"ucke sind.

\section{Struktur eines Programms}
SEL ist eine modulare
\index{Modul}
Programmiersprache, d.h. ein Programm kann sich (mu"s sich aber
nicht) aus mehreren Modulen---oder Quelldateien---zusammensetzen.
Zus"atzlich kann jedes Modul seinen internen Aufbau vor dem Rest
des Programms sch"utzen, um Implementationsdetails zu verbergen
(\indi{Kapselung}), damit diese jederzeit ge"andert werden
k"onnen, ohne da"s der aufrufende Programmteil dazu ver"andert
werden mu"s. Dies setzt jedoch auch ein vern"unftiges
"`\ind{Interface}"' (Schnittstelle zwischen dem Modul und dem
Programm) voraus. Es wird also in der Regel n"otig sein, vor der
Implementation einen Entwurf des Programms zu machen---einfaches
"`Drauflosprogrammieren"' wird sich schnell als Sackgasse
erweisen.

\subsection{Das Modul}
\index{Modul|ii}
\fuindexii{module}
\fuindexii{block}
\fuindex{function}
\fuindex{local}
\fuindex{global}

\begin{verbatim}
module:        {block}
             ;

block:         function
             | local
             | global
             ;
\end{verbatim}

Jedes Modul besteht aus einem oder mehreren Bl"ocken,
\index{Block|ii}
die in Funktionen, \rwi{local}- und \rwi{global}-Bl"ocke
eingeteilt werden (es ist theoretisch auch m"oglich, ein Modul zu
schreiben, das keinen einzigen Block enth"alt, doch dieses Modul
h"atte keinen anderen Zweck, als wertvollen Speicherplatz zu
verschwenden).

Funktionen sind Programmteile---oder Unterprogramme---, die von
bestimmten Stellen im Programm aufgerufen werden k"onnen. Da SEL
eine modulare Programmiersprache ist, ist es m"oglich, Funktionen
zu kapseln, sie also nur in dem Modul verf"ugbar zu machen, in
dem sie definiert wurden.

Die \rw{local}-Bl"ocke dienen dazu, Variablen,
\index{Variable!modulglobale}
Funktionen
\index{Funktion!modulglobale}
und Strukturen
\index{Struktur}
zu definieren, die nur innerhalb des jeweiligen Moduls
erreichbar sein sollen.

Die \rw{global}-Bl"ocke dienen einerseits dem Zweck, Variablen,
\index{Variable!programmglobale}
Funktionen
\index{Funktion!programmglobale}
und Strukturen zu definieren, die auch au"serhalb des Moduls
verf"ugbar sein sollen, und andererseits jene Variablen,
Strukturen und Funktionen zu deklarieren, die nicht in diesem
Modul definiert sind und trotzdem verwendet werden sollen.
Beachten Sie bitte, da"s diese in einem anderen Modul
als global definiert sein m"ussen.

\subsection{Funktionen}
\index{Funktion|ii}
\index{Block}
\fuindexii{function}
\fuindex{type}
\fuindex{statement}
\fuindex{IDENTIFIER}
\rwindex{begin}
\rwindex{end}
\fuindexii{formal\protect{\_params}}
\fuindexii{func\protect{\_var}}
\fuindex{expr}
\fuindexii{func\protect{\_proto}}
\rwindexii{function}

\begin{verbatim}
function:      `function' IDENTIFIER `(' [formal_params] `)' `:' type
                 {func_var}
               `begin'
                 {statement}
               `end'
             ;

formal_params: IDENTIFIER `:' [`&'] type [`,' formal_params]
             ;

func_var:      IDENTIFIER `:' type [`=' expr] `;'
             ;

func_proto:    `function' IDENTIFIER
               `(' [[`&'] type {`,' [`&'] type}] `)' `:' type
             ;
\end{verbatim}

Eine Funktion ist ein Unterprogramm,
\index{Unterprogramm|see{Funktion}}
also ein Programmteil, der
von anderen Stellen des Programms aufgerufen werden kann. Dabei
k"onnen einer Funktion ein oder mehrere \indi{Parameter} (deren
Typen bei der Deklaration der Funktion festgelegt werden)
"ubergeben werden. Innerhalb dieser Funktion k"onnen diese
Parameter als ganz normale Variablen angesprochen werden. Die
Namen dieser Variablen werden im Funktionskopf angegeben.

Funktionsdeklarationen k"onnen an zweierlei Stellen angetroffen
werden, n"amlich dort, wo die Funktionen definiert sind, und
in \rw{local}- und \rw{global}-Bl"ocken, wo allerdings nur der
Funktionskopf ohne Parameternamen angegeben wird. Zu beachten
ist, da"s Funktionen im Programm erst nach dem ersten Vorkommen
ihrer Deklarationen oder Definitionen bekannt sind. Eine Funktion
kann also nicht aufgerufen werden, wenn sie im Quellcode nicht
vorher deklariert oder definiert wurde.

Funktionen sind, solange sie nicht explizit in einem
\rw{global}-Block des jeweiligen Moduls aufgef"uhrt sind,
gekapselt, also auf das Modul beschr"ankt, in dem sie
definiert wurden. Trotzdem ist es m"oglich, Funktionen in einem
\rw{local}-Block des
Moduls aufzuf"uhren, f"ur den Fall, da"s eine Funktion an einer
Stelle aufgerufen werden mu"s, die vor ihrer Definition liegt.

Zur Semantik: Dem \rw{function}-Schl"usselwort folgt der Name der
Funktion, der wie jeder Bezeichner case-sensitiv ist. In Klammern
gesetzt folgen darauf die Namen und Typen der einzelnen Parameter.
Innerhalb der Funktion kann auf die Parameter mit Hilfe ihrer
Namen wie auf normale Variablen zugegriffen werden.

SEL unterst"utzt zwei Arten der Parameter"ubergabe:
\index{Parameter""ubergabe|ii}
\index{call-by-value|see{Parameter""ubergabe}}
\index{call-by-reference|see{Parameter""ubergabe}}
{\em call-by-value} und {\em call-by-reference}. Ein
Parameter, der als call-by-value deklariert ist, wird der
Funktion als Wert "ubergeben, es wird beim Funktionsaufruf also
zuerst der Wert des Parameters berechnet und dieser dann kopiert.
Die jeweilige Funktion greift dann auf die Kopie des Parameters
zu, kann also die Variable, die "ubergeben wurde, nicht
ver"andern. Von einem als call-by-reference deklarierten Parameter
(dieser ist daran zu erkennen, da"s seinem Typ im Funktionskopf
das \fu{\&}-Zeichen
\index{Referenztyp|ii}
vorangestellt ist) hingegen wird der Funktion intern die Adresse
der als Parameter "ubergebenen Variable zug"anglich gemacht,
wovon die Funktion allerdings nichts merkt. Dies wirkt sich
dahingehend aus, da"s die Funktion durch "Anderung des Wertes des
Parameters auch den Wert der Variable ver"andert, die ihr als
Parameter "ubergeben wurde. Beachten Sie allerdings, da"s es
nicht m"oglich ist, eine \ind{Konstante} f"ur einen
call-by-reference-Parameter zu "ubergeben.

Nach der Parameterliste folgt der R"uckgabetyp
\index{R""uckgabetyp|ii}
der Funktion. Das ist der Typ, den der Aufrufer der Funktion
zur"uckgeliefert bekommt. Der R"uckgabewert wird durch die
\rwi{return}-Anweisung bestimmt. Ist keine R"uckgabe n"otig, kann
der Typ \tvoid\
\typeindex{void}
verwendet werden.

Im Anschlu"s an den R"uckgabetyp k"onnen lokale
Variablen
\index{Variable!lokale|ii}
vereinbart werden. Diese Variablen sind---der Name
sagt es bereits---nur innerhalb der Funktion verf"ugbar, in der
sie deklariert wurden. Gleichzeitig k"onnen diesen Variablen
Werte zugewiesen werden. Geschieht dies nicht, sind die Werte
dieser Variablen undefiniert. Beachten Sie, da"s Variablen
auch mit berechneten Werten initialisiert werden k"onnen.

Nach der Vereinbarung der lokalen Variablen folgt in
\rwi{begin}--\rwi{end}
eingeschlossen der eigentliche Code der
Funktion. Von hier aus k"onnen wieder andere Funktionen
aufgerufen werden, die Funktion kann sich sogar selbst aufrufen
(\ind{Rekursion}).

\example
Die Funktion \fuii{Factorial} berechnet die Fakult"at
\index{Fakult""at|ii}
ihres Parameters \fu{i}
(\hbox{$\fu{i}! = \prod_{n=1}^{\fu{\footnotesize i}} n$}).
Hierzu verwendet sie zwei lokale Variablen \fu{c} und \fu{r}.

\rwindexuu{global}
\rwindexuu{for}
\begin{verbatim}
global
  function Factorial (int) : int
end

function Factorial (i : int) : int

  c : int;
  r : int = 1;

begin
  for (c = 2 to i)
    r = r * c;
  end
  return r;
end
\end{verbatim}

\subsubsection{Die main-Funktion}
Die wichtigste \ind{Funktion} in einem SEL-Programm ist die Funktion
\fuii{main}. Sie stellt gewisserma"sen das Hauptprogramm dar und
ist somit die erste Funktion, die in einem SEL-Programm aufgerufen
wird. Hat \fu{main} ihre Arbeit beendet, kehrt der Computer zum
aufrufenden Programm (in der Regel der Kommandozeileninterpreter)
zur"uck. Dabei kann letzterem ein sogenannter {\em Exit-Code}
"ubergeben werden. Dieser \indi{Exit-Code} hat den Typ \tint\
\typeindex{int}
und wird mit der \rwi{return}-Anweisung zur"uckgegeben. Ist der
R"uckgabetyp der Funktion \fu{main} \tvoid ,
\typeindex{void}
so ist der Exit-Code \fu{0}.

Zu beachten ist, da"s die \fu{main}-Funktion immer als global
deklariert werden mu"s, unerheblich, in welchem Modul sie
definiert ist.

\example
Dieses {\em hello-world}-Programm
\index{hello-world|uu}
besteht aus einem
einzigen Modul und hat die Aufgabe, den String "`\fu{hello, world!}"'
auszugeben. Zus"atzlich wird der Exit-Code \fu{1} zur"uckgeliefert.

\fuindexuu{WriteString}
\rwindexuu{global}
\begin{verbatim}
#include "conio.sh"

global
  function main () : int
end

function main () : int
begin
  WriteString("hello, world!\n");
  return 1;
end
\end{verbatim}

\subsubsection{Die return-Anweisung}
Die \rwii{return}-Anweisung hat den Zweck, aus einer \ind{Funktion} in
die "ubergeordnete Funktion (bzw. das Betriebssystem im Falle der
\fu{main}-Funktion) zur"uckzukehren. Ist der R"uckgabetyp der
jeweiligen Funktion nicht \tvoid ,
\typeindex{void}
so mu"s auch ein \ind{Wert}
zur"uckgegeben werden. Dieser wird der \rw{return}-Anweisung
nachgestellt. \rw{return} kann an einer beliebigen Stelle
innerhalb einer Funktion stehen, Schleifen
\index{Schleife} werden---falls n"otig---abgebrochen.

\example
Die Funktion \fuii{Square} gibt
das \ind{Quadrat} eines ihr "ubergebenen \tint\ \fu{i} zur"uck.

\begin{verbatim}
function Square (i : int) : int
begin
  return i * i;
end
\end{verbatim}

\subsection{local-Bl\protect{"ocke}}
\rwindexii{local}
\rwindex{end}
\index{Block}
\fuindexii{local}
\fuindexii{local\protect{\_decl}}
\fuindex{struct}
\fuindex{func\protect{\_proto}}
\fuindex{var}

\begin{verbatim}
local:         `local'
                 {local_decl}
               `end'
             ;

local_decl:    struct
             | func_proto
             | var
             ;
\end{verbatim}

Die \rw{local}-Bl"ocke dienen dazu, Variablen,
\index{Variable!modulglobale|ii}
Strukturen
\index{Struktur}
und Funktionen
\index{Funktion!modulglobale|ii}
zu definieren, die nur in dem jeweiligen Modul
bekannt sein sollen. Dieses Vorgehen nennt sich \ind{Kapselung} und hat den
Vorteil, da"s es m"oglich ist, die Implementationsdetails eines
Moduls vor dem Rest des Programms zu verstecken. Das macht es
einerseits m"oglich, das Modul anders zu implementieren, ohne
da"s das Restprogramm in irgendeiner Weise ge"andert werden mu"s
und andererseits, da"s der Code des Moduls in anderen Programmen
leichter wiederverwendet werden kann. F"ur eine solche Vorgangsweise
ist es jedoch unabdingbar, f"ur jedes Modul ein gut "uberdachtes
\ind{Interface} (Schnittstelle zwischen Modul und Restprogramm) zu
verwenden.

Variablen, Strukturen und Funktionen sind im Source erst ab der
Stelle ihrer Deklaration bekannt. Aus diesem Grunde ist es
ratsam, \rw{local}-Bl"ocke an den Anfang der Datei zu stellen,
was au"serdem die "Ubersichtlichkeit des Codes erh"oht.

\subsection{global-Bl\protect{"ocke}}
\rwindexii{global}
\rwindex{end}
\index{Block}
\fuindexii{global}
\fuindexii{global\protect{\_decl}}
\fuindex{struct}
\fuindex{func\protect{\_proto}}
\fuindex{var}
\fuindex{extern}

\begin{verbatim}
global:        `global'
                 {global_decl}
               `end'
             ;

global_decl:   struct
             | func_proto
             | var
             | extern
             ;
\end{verbatim}

Die in einem \rw{global}-Block definierten Strukturen,
\index{Struktur}
Funktionen
\index{Funktion!programmglobale|ii}
und Variablen
\index{Variable!programmglobale|ii}
sind dazu bestimmt, auch von anderen
Modulen verwendet zu werden. Zus"atzlich k"onnen in einem
\rw{global}-Block Variablen deklariert werden, die in diesem Modul
verwendet werden sollen, jedoch in einem anderen Modul als global
definiert worden sind. Variablen, Strukturen und Funktionen sind
erst ab ihrer Deklaration verf"ugbar, soda"s es am
praktikabelsten ist, die \rw{global}-Bl"ocke an den Beginn der
Datei zu setzen oder in einer \ind{Include-Datei} unterzubringen.

\section{Anweisungen}
\index{Anweisung|ii}
\fuindexii{statement}
\fuindex{expr}
\fuindex{if}
\fuindex{while}
\fuindex{for}
\fuindex{repeat}
\fuindex{switch}
\fuindex{prefix}
\rwindex{break}
\rwindex{return}

\begin{verbatim}
statement:     [expr] `;'
             | if
             | while
             | for
             | repeat
             | switch
             | prefix
             | `break' `;'
             | `return' [expr] `;'
             ;
\end{verbatim}

Anweisungen (oder {\em statements}) k"onnen innerhalb von Funktionen
\index{Funktion}
verwendet werden und machen das eigentliche Programm aus.
Anweisungen k"onnen grunds"atzlich in Ausdr"ucke ({\em expressions}),
\index{Ausdruck}
Kontrollanweisungen
\index{Kontrollanweisung}
und Schleifen
\index{Schleife}
eingeteilt werden. Erstere verrichten die
eigentliche Arbeit, und die beiden letzteren steuern den Programmflu"s. Wie
aus der Grammatik ersichtlich ist, kann eine Anweisung auch
einfach nur ein Semikolon enthalten. Solche Anweisungen haben die
Bezeichnung {\em \indi{Nullanweisung}} und haben---bis auf die
Verbesserung der "Ubersichtlichkeit in bestimmten Situationen---%
keine Aufgabe.

\subsection{Die if-Anweisung}
\rwindexii{if}
\fuindexii{if}
\fuindex{expr}
\fuindex{statement}
\rwindexii{else}
\rwindex{end}

\begin{verbatim}
if:            `if' `(' expr `)'
                 {statement}
               [`else'
                 {statement}]
               `end'
             ;
\end{verbatim}

Die \rw{if}-Anweisung ist eine Kontrollanweisung und dient der
bedingten Ausf"uhrung eines Programmteils. Dem \rw{if}
folgt ein in Klammern eingeschlossener \ind{Ausdruck}, der zuerst
evaluiert (berechnet) und dann nach \tint\
\typeindex{int}
gecastet wird. Ist dieser
Ausdruck ungleich \fu{0}, werden die darauffolgenden
Anweisungen ausgef"uhrt. Ist jedoch der Ausdruck \fu{0}, dann
werden die Anweisungen, die dem \rw{else} folgen, sofern ein
\rw{else} vorhanden ist, ausgef"uhrt, ansonsten geschieht gar
nichts.

\example
Die Funktion \fuii{GetInt} fragt den Benutzer nach einer Zahl.
Ist die Zahl gr"o"ser oder gleich  \fu{0}, so wird der Text
"`\fu{Gratulation, die Zahl ist positiv!}"'
auf dem Bildschirm dargestellt. Andernfalls wird die Zahl ohne
negatives Vorzeichen ausgegeben.

\fuindexuu{ReadInt}
\fuindexuu{WriteString}
\fuindexuu{WriteInt}
\begin{verbatim}
function GetInt () : void

  i : int;

begin
  i = ReadInt();
  if (i >= 0)
    WriteString("Gratulation, die Zahl ist positiv!\n");
  else
    WriteInt(-i);
    WriteString("\n");
  end
end
\end{verbatim}

\subsection{Die switch-Anweisung}
\rwindexii{switch}
\fuindexii{switch}
\fuindex{expr}
\fuindex{statement}
\rwindex{end}
\fuindexii{prefix}
\rwindex{case}
\fuindex{constant}
\rwindex{default}

\begin{verbatim}
switch:        `switch' `(' expr `)'
                 {statement}
               `end'
             ;

prefix:        `case' constant `:'
             | `default' `:'
             ;
\end{verbatim}

Die \rw{switch}-Anweisung wird verwendet, um f"ur mehrere
verschiedene Werte eines Ausdrucks
\index{Ausdruck}
verschiedene Aktionen
auszuf"uhren. Als erstes wird der Ausdruck evaluiert. Danach
wird er mit der \ind{Konstante}, die dem ersten im
\fu{\{statement\}} vorkommenden \rw{case} folgt, verglichen. Ist
der Vergleich positiv, so werden alle der \rw{case}-Anweisung
folgenden Anweisungen ausgef"uhrt, solange, bis ein \rwii{break}
gefunden wird, was den Programflu"s nach dem \rw{end} fortsetzt.
Ist der Vergleich jedoch negativ, wird das n"achste \rw{case}
gesucht u.s.w. Wird bei diesem Vorgang ein \rwii{default} gefunden,
dann wird die Ausf"uhrung {\em in jedem Fall} nach dieser
Anweisung fortgesetzt. Aus diesem Grunde sollte \rw{default}
immer am Schlu"s der \rw{case}-Kette stehen.
Findet sich weder ein \rw{case}, das dem
Wert des Ausdrucks entspricht, noch ein \rw{default}, geschieht
nichts und die Ausf"uhrung wird nach dem \rw{end} fortgesetzt.

\example
Die Funktion \fuii{SimpleMenu} stellt ein Men"u auf
dem Bildschirm dar und wartet auf eine Eingabe. F"ur die Ziffern
\fu{`1'} bis \fu{`3'} werden die entsprechenden Funktionen
\fuii{Choice1} bis \fuii{Choice3}
\fuindexii{Choice2}
ausgef"uhrt. Dieser Vorgang
wiederholt sich so lange, bis der Benutzer \fu{`x'} oder \fu{`X'}
eingibt, was die Funktion beendet. Bet"atigt der Benutzer eine
Taste, die nicht verarbeitet werden kann, wird eine
dementsprechende Meldung ausgegeben. F"ur die Ausgabe des Men"us
auf dem Bildschirm wird die Funktion \fui{DisplayMenu} verwendet.

\rwindexuu{while}
\fuindexuu{ReadChar}
\fuindexuu{WriteString}
\begin{verbatim}
function SimpleMenu () : void
begin
  while (1)
    DisplayMenu();
    switch (ReadChar())
      case '1' :
        Choice1();
        break;
      case '2' :
        Choice2();
        break;
      case '3' :
        Choice3();
        break;
      case 'x' :
      case 'X' :
        return;
      default :
        WriteString("Falsche Eingabe!\n");
    end
  end
end
\end{verbatim}
        
\subsection{Die while-Schleife}
\rwindexii{while}
\fuindexii{while}
\fuindex{expr}
\fuindex{statement}
\rwindex{end}

\begin{verbatim}
while:         `while' `(' expr `)'
                 {statement}
               `end'
             ;
\end{verbatim}

Die \rw{while}-Schleife f"uhrt die zwischen \rw{while} und
\rw{end} aufgef"uhrten Anweisungen sooft aus, solange der in Klammern
eingeschlossene \ind{Ausdruck} ungleich \fu{0} ist. Dabei wird wie
folgt vorgegangen: Als erstes wird der Ausdruck evaluiert und
nach \tint\
\typeindex{int}
gecastet. Ist er
gleich \fu{0}, so wird die Ausf"uhrung nach dem \rw{end}
fortgesetzt, andernfalls werden die Anweisungen, die dem Ausdruck
folgen, ausgef"uhrt. Hierauf wird wieder an den Anfang der
Schleife gesprungen und der Ausdruck abermals gepr"uft u.s.w.
Beachten Sie, da"s es au"ser mit Hilfe der \rwi{return}-Anweisung
nicht m"oglich ist, aus der Schleife auszusteigen, ohne da"s der
Ausdruck gleich \fu{0} ist.

\example
Die Funktion \fuii{OutputList} durchl"auft die ihr "ubergebene
\indu{Liste} des Typs \fu{List}
\index{List@{\tt List}~(\rw{struct})|ii}
und gibt dabei alle Datenelemente---in
diesem Beispiel der Einfachheit halber vom Typ \tint ---aus.
Anmerkung: Das letzte Element der Liste ist dadurch
gekennzeichnet, da"s der Zeiger, der auf das n"achste Element
zeigen sollte, ein \ind{Nullzeiger} ist.

\fuindexuu{WriteInt}
\rwindexuu{local}
\begin{verbatim}
local
  struct List
    i    : int;
    Next : *List;
  end
end

function OutputList (l : *List) : void
begin
  while (l != 0)
    WriteInt(l->i);
    l = l->Next;
  end
end
\end{verbatim}

\subsection{Die repeat-Schleife}
\rwindexii{repeat}
\fuindexii{repeat}
\rwindexii{until}
\fuindex{statement}
\fuindex{expr}

\begin{verbatim}
repeat:        `repeat'
                 {statement}
               `until' `(' expr `)'
             ;
\end{verbatim}

Die \rw{repeat}-Schleife wiederholt die zwischen \rw{repeat} und
\rw{until} aufgef"uhrten Anweisungen so lange, bis der auf
\rw{until} folgende \ind{Ausdruck} ungleich \fu{0} ist. Der Ablauf ist
hierbei folgenderma"sen: Zun"achst werden die Anweisungen
ausgef"uhrt, danach wird der Ausdruck evaluiert, nach \tint\
\typeindex{int}
gecastet und auf
Gleichheit mit \fu{0} gepr"uft. Ist diese gegeben, wird zu
\rw{repeat} zur"uckgesprungen und der Vorgang wiederholt sich.
Ist der Ausdruck hingegen ungleich \fu{0}, ist die Schleife
beendet. Beachten Sie bitte, da"s---im Gegensatz zur
\rwi{while}-Schleife---die Schleife {\em mindestens einmal}
ausgef"uhrt wird, unerheblich, ob der Ausdruck schon vor der
Ausf"uhrung ungleich \fu{0} w"are.

\example
Dieses Beispielprogramm liest so lange Zahlen vom Benutzer ein,
bis dieser \fu{0} eingibt. Anschlie"send wird der Mittelwert der
eingelesenen Zahlen ausgegeben. Wurden au"ser der \fu{0} keine
Zahlen eingegeben, erscheint eine dementsprechende Meldung.
    
\rwindexuu{global}
\fuindexuu{ReadInt}
\rwindexuu{if}
\fuindexuu{WriteInt}
\fuindexuu{WriteString}
\begin{verbatim}
#include "conio.sh"

global
  function main () : void
end

function main () : void

  i : int;
  a : int =  0;
  n : int = -1;

begin
  repeat
    a = a + (i = ReadInt());
    n = n + 1;
  until (i == 0)
  if (n)
    WriteInt(a / n);
  else
    WriteString("Es wurden keine Zahlen eingegeben!");
  end
  WriteString("\n");
end
\end{verbatim}

\subsection{Die for-Schleife}
\rwindexii{for}
\fuindexii{for}
\fuindex{primary}
\fuindex{expr}
\rwindexii{to}
\rwindexii{downto}
\rwindexii{step}
\fuindex{statement}
\rwindex{end}

\begin{verbatim}
for:           `for' `(' primary `=' expr (`to' | `downto') expr
               [`step' expr] `)'
                 {statement}
               `end'
             ;
\end{verbatim}

Die \rw{for}-Schleife ist eine spezielle Schleife, die es
erlaubt, Anweisungen eine bestimmte Anzahl mal zu wiederholen.
Hierzu ist nach der "offnenden Klammer die Variable anzugeben,
die als Z"ahler fungiert. Da hier ein \fu{primary} (siehe Kapitel
\ref{exprs}) erwartet wird,
ist es auch m"oglich, Elemente eines Arrays
\index{Array}
oder einer \ind{Struktur}
als Z"ahler zu verwenden. Nach dem Zuweisungsoperator wird der
Anfangsindex angegeben und danach entweder das Schl"usselwort
\rw{to} oder \rw{downto}, je nachdem, ob nach oben oder nach
unten gez"ahlt werden soll. Darauf folgt der Zielindex, also
jener Wert, den die Indexvariable maximal oder minimal annehmen
darf. Optional kann noch nach dem Schl"usselwort \rw{step} die
Schrittweite angeben werden. Zu beachten ist, da"s diese auch
dann, wenn nach unten gez"ahlt wird, positiv sein mu"s.

Bei der Ausf"uhrung wird folgenderma"sen vorgegangen: Zun"achst
werden alle Ausdr"ucke
\index{Ausdruck}
evaluiert und nach \tint\
\typeindex{int}
gecastet. Danach
wird "uberpr"uft, ob der Wert der Indexvariablen schon au"serhalb
der zul"assigen Grenze ist. Ist das nicht der Fall, werden die
Anweisungen das erste Mal ausgef"uhrt und der Z"ahler
entsprechend erh"oht bzw. erniedrigt. Danach wird wieder
"uberpr"uft, ob die Anweisungen nocheinmal ausgef"uhrt werden
sollen u.s.w.

\example
Die Funktion \fuii{Squares} gibt der Reihe nach die Quadrate
\index{Quadrat}
der ganzen Zahlen $i$ f"ur \hbox{$\fu{a} \le i \le \fu{b}$} aus.
Der Funktion werden die \tint s \fu{a} und \fu{b} "ubergeben.

\fuindexuu{WriteInt}
\fuindexuu{WriteString}
\begin{verbatim}
function Squares (a : int, b : int) : void

  i : int;

begin
  for (i = a to b)
    WriteInt(i * i);
    WriteString("\n");
  end
end
\end{verbatim}

\section{Ausdr\protect{"ucke}}
\label{exprs}
\index{Ausdruck|ii}
\fuindexii{expr}
\rwindex{delete}
\fuindexii{primary}
\fuindex{IDENTIFIER}
\fuindex{INT\protect{\_CONSTANT}}
\fuindex{CHAR\protect{\_CONSTANT}}
\fuindex{FLOAT\protect{\_CONSTANT}}
\fuindex{STRING\protect{\_CONSTANT}}
\fuindex{type}
\rwindex{sizeof}
\rwindex{new}

\begin{verbatim}
expr:          expr (`=' | `&&' | `||' | `^^' | `==' | `<' | `>'
                   | `<=' | `>=' | `!=' | `&' | `|' | `^' | `<<'
                   | `>>' | `+' | `-' | `*' | `/' | `%') expr
             | (`!' | `~' | `-' | `*' | `&') expr
             | `delete' expr
             | primary
             ;

primary:       IDENTIFIER
             | `(' expr `)'
             | INT_CONSTANT
             | CHAR_CONSTANT
             | FLOAT_CONSTANT
             | STRING_CONSTANT
             | primary `->' IDENTIFIER
             | primary `:' type
             | primary `[' expr `]'
             | IDENTIFIER `(' [expr {`,' expr}] `)'
             | `sizeof' type
             | `new' type
             ;
\end{verbatim}

Ausdr"ucke sind sozusagen das "`Herz"' von SEL, denn sie
erledigen die eigentliche Arbeit, w"ahrend Kontrollanweisungen
und Schleifen nur kontrollieren, was passieren soll. Ausdr"ucke k"onnen die
verschiedensten Aufgaben erledigen: Sie k"onnen zum Vergleich
zweier Werte benutzt werden, zur Wertzuweisung oder zum
Funktionsaufruf. Die gemeinsamen Merkmale der Ausdr"ucke sind:
Jeder Ausdruck besitzt sowohl einen \ind{Typ} als auch einen
\ind{Wert} (bis auf die Ausdr"ucke, die vom Typ \tvoid\
\typeindex{void}
sind).

\subsection{Konstanten}
\index{Konstante}
\fuindexii{constant}
\fuindex{INT\protect{\_CONSTANT}}
\fuindex{CHAR\protect{\_CONSTANT}}
\fuindex{FLOAT\protect{\_CONSTANT}}
\fuindex{STRING\protect{\_CONSTANT}}

\begin{verbatim}
constant:      [`-'] INT_CONSTANT
             | CHAR_CONSTANT
             | [`-'] FLOAT_CONSTANT
             | STRING_CONSTANT
             ;
\end{verbatim}

Konstanten in SEL sind im Grunde als Konstanten im
mathematischen Sinne zu verstehen. Ihre Werte k"onnen im gesamten
Programm nicht
ver"andert werden. Im Gegensatz zu anderen Programmiersprachen
besitzt SEL nicht die F"ahigkeit, Konstanten einen Namen zu geben,
\index{Konstante!benannte}
was allerdings durch die \fui{\#define}-Direktive leicht umgangen
werden kann. Weiters gibt es in SEL keine Konstanten von
zusammengesetzten Typen,
\index{Typ!zusammengesetzter}
da diese F"ahigkeit nur sehr selten ben"otigt wird.

\subsubsection{int-Konstanten}
\index{Konstante!\tint |ii}
\typeindex{int}
\fuindexii{INT\protect{\_CONSTANT}}

\begin{verbatim}
INT_CONSTANT: [0-9]+
\end{verbatim}

Eine \tint -Konstante ist eine Folge von Ziffern, die als Zahl
mit der Basis 10, also als Dezimalzahl
\index{Zahl!ganze}
\index{Zahl!dezimale}
interpretiert werden.
Laut obiger Definition gibt es jedoch
keine negativen \tint -Konstanten, was den Grund hat, da"s
dieser Mechanismus in der Grammatik eingebaut ist. Dies wiederum
ist darin begr"undet, da"s es andernfalls zu Mi"sverst"andnissen
bei der Interpretation von Ausdr"ucken kommen w"urde (man denke
an den Ausdruck \fu{7-4}---er w"urde als Konstante \fu{7} gefolgt
von der Konstante \fu{-4} gedeutet werden, da die lexikalische
Analyse "uber keine Informationen "uber den Kontext verf"ugt, in dem
die Konstante auftritt, mit anderen Worten, sie wei"s nicht, da"s
auf eine Konstante keine weitere folgen kann).

Voraussichtlich wird SEL in der Version 0.4 um die F"ahigkeit
erweitert werden, auch hexadezimale
\index{Zahl!hexadezimale}
\tint -Konstanten in der Form
\fu{0(x|X)[0-9a-fA-F]+} zu interpretieren.

\subsubsection{char-Konstanten}
\index{Konstante!\tchar |ii}
\typeindex{char}
\fuindexii{CHAR\protect{\_CONSTANT}}

\begin{verbatim}
CHAR_CONSTANT: '(.|\.)'
\end{verbatim}

\tchar -Konstanten repr"asentieren einzelne \ind{Zeichen}.
Dieses k"onnen entweder "`normale"' Zeichen oder
Steuerzeichen
\index{Zeichen!Steuerzeichen}
sein---letztere werden durch einen vorangestellten
Back"-slash "`\fu{\char92}"'
\index{Backslash (\fu{\char92})}
gekennzeichnet, jedoch als ein einzelnes Zeichen behandelt.

\subsubsection{float-Konstanten}
\index{Konstante!\tfloat |ii}
\typeindex{float}
\fuindexii{FLOAT\protect{\_CONSTANT}}

\begin{verbatim}
FLOAT_CONSTANT: [0-9]*(`.'[0-9]*)?((e|E)(+|-)?[0-9]+)?
\end{verbatim}

Eine \tfloat -Konstante repr"asentiert eine positive, rationale
\index{Zahl!rationale}
Zahl in \ind{Exponentialschreibweise}, wobei der Exponent auch
weggelassen werden kann. Leider ist eine Definition als
regul"arer Ausdruck nicht ohne Probleme m"oglich. So w"are laut
obiger Definition auch eine leere Eingabe oder ein Dezimalpunkt
f"ur sich alleine eine \tfloat -Konstante, was nat"urlich nicht
im Sinn der Sache liegen kann. Die Definition ist also
dahingehend zu erweitern, da"s in der Mantisse mindestens eine Ziffer
vorkommen mu"s. Beachten Sie bitte, da"s eine
\tfloat -Konstante immer positiv sein mu"s, was jedoch aufgrund
der Grammatik von SEL nicht weiter st"ort.

\subsubsection{String-Konstanten}
\index{Konstante!String|ii}
\fuindexii{STRING\protect{\_CONSTANT}}

\begin{verbatim}
STRING_CONSTANT: "([^"\n]|\.)*"
\end{verbatim}

Eine String-Konstante ist eine in zwei Anf"uhrungsstriche
eingeschlossene Zeichenfolge, innerhalb derer
alle \ind{Zeichen} bis auf die Anf"uhrungsstriche und den
Zeilenvorschub erlaubt sind. Eine besondere Bedeutung kommt den
Steuerzeichen
\index{Zeichen!Steuerzeichen}
zu, die durch einen vorangestellten Back"-slash
"`\fu{\char92}"'
\index{Backslash (\fu{\char92})}
gekennzeichnet sind. Jede String-Konstante wird
von SEL automatisch mit einem \indi{Nullzeichen}, also einem Zeichen mit
dem Wert \fu{0} beendet. Die String-Konstante beansprucht also im
Speicher um ein Byte mehr, als sie an Zeichen enth"alt.

Eine String-Konstante wird von SEL wie ein Zeiger auf
\tchar s
\typeindex{char}
behandelt. Sie k"onnen diesen Zeiger also dazu
verwenden, um die Zeichenfolge auszugeben, sollten jedoch den
Speicherbereich, auf den er verweist, nicht ver"andern.

\subsubsection{Steuerzeichen}
\label{specialchar}
\index{Zeichen!Steuerzeichen|ii}

SEL stellt bestimmte Steuerzeichen zur Verf"ugung, die in Strings
\index{Konstante!String}
und \tchar -Konstanten
\index{Konstante!\tchar }
\typeindex{char}
benutzt werden k"onnen, um Zeichen
darzustellen, denen eine bestimmte Bedeutung zukommt, wie etwa
dem Zeichen f"ur Zeilenvorschub. Steuerzeichen
sind daran zu erkennen, da"s ihnen ein Back"-slash "`\fu{\char92}"'
\index{Backslash (\fu{\char92})|ii}
vorangestellt ist. Trotzdem werden sie von SEL als einzelne
Zeichen behandelt. Hier eine Zusammenstellung aller in SEL
vorhandenen Steuerzeichen:

\medskip
\index{Warnton}
\index{R""uckschritt}
\index{Bildvorschub}
\index{Zeilenvorschub}
\index{Wagenr""ucklauf}
\index{Tabulator}
\begin{tabular}{ll}
\fu{\char92 a} & Bell (Warnton) \\
\fu{\char92 b} & Backspace (R"uckschritt) \\
\fu{\char92 f} & Formfeed (Bildvorschub) \\
\fu{\char92 n} & New line (Zeilenvorschub) \\
\fu{\char92 r} & Carriage return (Wagenr"ucklauf) \\
\fu{\char92 t} & Tab (Tabulator) \\
\fu{\char92 v} & Vertical tab (vertikaler Tabulator)
\end{tabular}
\medskip

Wird hinter den Back"-slash ein Zeichen gesetzt, das in obiger
Tabelle nicht eingetragen ist, wird von SEL automatisch dieses
Zeichen verwendet. So ist es zum Beispiel m"oglich, einen
Back"-slash oder ein Anf"uhrungszeichen innerhalb einer
String-Konstante einzugeben.

SEL wird in der Version 0.4 aller Voraussicht nach um das
Steuerzeichen \fu{\char92 x\rm\it hh} erweitert werden, das an seine
Stelle das Zeichen mit dem in hexadezimaler
\index{Zahl!hexadezimale}
Notation angegeben Code {\it hh} stellt.

\subsection{Variablen}
\index{Variable|ii}
\fuindexii{var}
\fuindex{IDENTIFIER}
\fuindex{type}
\fuindex{constant}
\fuindexii{extern}
\rwindexii{extern}

\begin{verbatim}
var:           IDENTIFIER `:' type [`=' constant] `;'
             ;

extern:        `extern' IDENTIFIER `:' type `;'
             ;
\end{verbatim}

Variablen sind Speicherbereiche, die
vom Programm belegt werden und w"ahrend der Ausf"uhrung
verschiendene Werte
\index{Wert}
annehmen k"onnen. Jede Variable besitzt
einen \ind{Typ}, der sowohl ihre Gr"o"se als auch ihre Repr"asentation
im Speicher bestimmt. SEL unterscheidet grunds"atzlich zwischen
drei Typen von Variablen: programmglobale Variablen, modulglobale
Variablen und lokale Variablen.

Programmglobale Variablen werden in einem \rwi{global}-Block
definiert und sind im gesamten Programm bekannt. Ein Modul, das
eine Variable verwenden will, die in einem anderen Modul
definiert ist, mu"s jedoch, um auf die Variable zugreifen zu
k"onnen, dies mit einer \rw{extern}-Deklaration bekanntgeben.
Sie unterscheidet sich von einer Variablendefinition lediglich
dadurch, da"s ihr das \rw{extern}-Schl"usselwort vorangestellt
ist und keine Wertzuweisung erlaubt ist. Beachten Sie, da"s es
unm"oglich ist, innerhalb eines
Programms zwei programmglobale Variablen desselben Namens zu
definieren, da dies zu Konflikten f"uhren w"urde.

Modulglobale Variablen werden in einem \rwi{local}-Block eines
Moduls definiert und sind nur innerhalb dieses Moduls
bekannt---auf sie kann von anderen Modulen auch nicht durch eine
\rw{extern}-Deklaration zugegriffen werden. Deshalb ist es auch
ohne weiteres m"oglich, mehrere modulglobale Variablen gleichen
Namens zu definieren, soferne dies in jeweils verschiedenen
Modulen geschieht.

Lokale Variablen sind nur innerhalb der \ind{Funktion} bekannt, in der
sie definiert wurden. Solche Variablen sind allerdings nicht
statisch, das hei"st, sie werden beim Aufruf der jeweiligen Funktion
erzeugt und existieren so lange, bis die Funktion zu ihrem
Aufrufer zur"uckkehrt. Weiters ist es auch durchaus m"oglich,
da"s lokale Variablen mehrfach existieren, und zwar dann, wenn
sich die Funktion selbst aufruft, also rekursiv ist. Bei jedem
solchen Aufruf werden neue lokale Variablen erzeugt, die zwar
namensgleich, jedoch von den anderen absolut isoliert sind.
Unter die lokalen Variablen fallen auch die Funktionsparameter.
\index{Parameter}
Sie werden im Funktionskopf definiert und sind auch nur existent,
solange die Funktion abgearbeitet wird.

Es ist in SEL durchaus m"oglich, modulglobale, programmglobale
und lokale Variablen desselben Namens zu definieren, wobei
folgende Regel zu beachten ist: lokale Variablen werden
modulglobalen Variablen vorgezogen, welche ihrerseits den
programmglobalen Variablen vorgezogen werden. Existiert also die
programmglobale Variable \fu{example}, so kann ohne weiteres eine
modulglobale Variable desselben Namens definiert werden.
Innerhalb des Moduls, in der letztere definiert wurde, bezieht
sich \fu{example} nicht auf die programmglobale, sondern auf die
modulglobale Variable. Definiert weiters eine Funktion innerhalb
dieses Moduls eine lokale Variable namens \fu{example}, so kann
diese Funktion nur auf die lokale Variable zugreifen, w"ahrend
der Rest des Moduls auf die modulglobale Variable zugreift.

\example
Die Funktion \fuii{Recursive}
\index{Rekursion|uu}
gibt die Zahlen \fu{1} bis \fu{10} in
ihrer nat"urlichen Reihenfolge aus und ruft sich nach der
Ausgabe jeder Zahl selbst auf, was dazu f"uhrt, da"s f"ur jede
Zahl zus"atzlich nocheinmal die Zahlen \fu{1} bis \fu{10}
ausgegeben werden, f"ur die dasselbe gilt. Um zu verhindern, da"s
die Zahl \fu{1} unendlich oft ausgegeben wird, was zweifellos
geschehen w"urde, w"urd dieses Verfahren unkontrolliert
angewendet, wird der Funktion ein Parameter "ubergeben, der
bestimmt, wie oft sie sich noch selbst aufrufen soll.
Beachten Sie bei diesem Beispiel, da"s sich die einzelnen Z"ahler
\fu{i} und die Parameter \fu{depth} der Funktion nicht gegenseitig
st"oren.

\rwindexuu{for}
\fuindexuu{WriteInt}
\fuindexuu{WriteString}
\begin{verbatim}
function Recursive (depth : int) : void

  i : int;

begin
  for (i = 1 to 10)
    WriteInt(i);
    WriteString("  ");
    if (depth > 0)
      Recursive(depth - 1);
      WriteString("  ");
    end
  end
end
\end{verbatim}

\subsection{Funktionen}
\index{Funktion|ii}

Funktionen sind Unterprogramme, die von verschiedenen Stellen des
Programms aufgerufen werden k"onnen. Um den Funktionen
Information mitzugeben, k"onnen ihnen sogenannte {\em \indi{Parameter}}
"ubergeben werden, deren Typen bei der Funktionsdeklaration
bestimmt werden m"ussen. Der Aufruf einer Funktion geschieht
durch Angabe des Funktionsnamens und in Klammern gesetzten
Parametern,
\index{Parameter""ubergabe|ii}
durch Beistriche getrennt. Sind keine Parameter
erforderlich, m"ussen nur die
beiden Klammern angegeben werden. Jede Funktion besitzt einen
R"uckgabetyp, der
ebenfalls bei der Deklaration festgelegt werden mu"s. Um eine
Funktion, die keine R"uckgabe vornimmt, zu realisieren, mu"s als
R"uckgabetyp \tvoid\
\typeindex{void}
gew"ahlt werden.

\example
Dieses Programm benutzt die Funktionen \fuiu{Factorial},
\fu{WriteString} und \fu{WriteInt}, um die Fakult"aten
\index{Fakult""at}
der Zahlen von \fu{1} bis \fu{10} auszugeben.

\rwindexuu{for}
\rwindexuu{global}
\fuindexuu{WriteString}
\fuindexuu{WriteInt}
\begin{verbatim}
#include "conio.sh"

global
  function Factorial (int) : int
  function main      ()    : void
end

function main () : void

  i : int;

begin
  for (i = 1 to 10)
    WriteString("Fakultaet von ");
    WriteInt(i);
    WriteString(" = ");
    WriteInt(Factorial(i));
    WriteString("\n");
  end
end
\end{verbatim}

\subsection{Operatoren}
\index{Operator|ii}

Operatoren werden, wie schon der Name sagt, ben"otigt, um
bestimmte Operationen auszuf"uhren, wie etwa das Addieren
von Zahlen oder das Zuweisen von Werten. Grunds"atzlich kann man
in SEL zwischen {\em un"aren} und {\em bin"aren} Operatoren
unterscheiden.

Un"are Operatoren stehen vor einem Ausdruck und
beeinflussen so dessen \ind{Wert}. Ein Beispiel hierf"ur w"are der
\ind{NOT}-Operator \fu{!}, der den \ind{Ausdruck}, der ihm nachsteht,
negiert. Un"are Operatoren werden grunds"atzlich von rechts nach
links ausgewertet, das hei"st, es wird zuerst der Wert des
Ausdrucks berechnet und danach negiert, nicht zuletzt deshalb,
weil es nicht anders m"oglich ist.

Bin"are Operatoren beziehen sich auf zwei Operanden, sie werden
deshalb zwischen diese beiden gestellt. Ein Beispiel f"ur einen
bin"aren Operator w"are der Additionsoperator \fu{+}. Bin"are
Operatoren k"onnen weiter unterteilt werden in solche, die von
rechts nach links und solche, die von links
nach rechts auswerten.
\index{Operator!Richtung der Auswertung}

Ersteres bedeutet, da"s zuerst
der Wert des rechten Operanden berechnet und dann auf den
linken angewendet wird. Ein Beispiel w"are der Zuweisungsoperator
\index{Zuweisung}
\fu{=}. Was unter der obigen Definition gemeint ist, l"a"st sich
leicht anhand eines Ausdrucks veranschaulichen: \fu{a=b=1}. SEL
beginnt von links nach rechts zu parsen und bleibt beim ersten
Operator stehen. Der linke Operand w"are demnach \fu{a}, der
rechte \fu{b=1}. Nun wird zuerst der rechte Operand ausgewertet.
SEL stoppt beim zweiten Operator und steht vor den Operanden
\fu{b} und \fu{1}. Wieder wird der rechte Operand ausgewertet,
von dem SEL nun wei"s, da"s er den Wert \fu{1} hat. Dieser wird
nun auf den linken Operanden angewendet, also der Variable \fu{b}
zugewiesen. Weiters wird nun der Wert des Ausdrucks \fu{b=1}
ebenfalls auf \fu{1} gesetzt und \fu{a} zugewiesen. Diese Art der
Auswertung ist zweckm"a"sig, da es so m"oglich ist, mehreren
Variablen auf einmal den selben Wert zuzuweisen. Um die
Auswertung deutlich zu machen, kann man den Ausdruck auch als
\fu{a=(b=1)} anschreiben.

Ein Beispiel f"ur einen Operator, der von links nach rechts
auswertet, ist der Subtraktionsoperator \fu{-}.
\index{Subtraktion}
Dabei wird
von zwei Operanden zun"achst der linke ausgewertet und dann auf
den rechten angewendet. Als Beispiel sei der Ausdruck \fu{3-2-1}
angef"uhrt. W"urde der Operator \fu{-} von rechts nach links
auswerten, so k"onnte man diesen Ausdruck auch als \fu{3-(2-1)}
anschreiben, was jedoch zu dem falschen Ergebnis \fu{2} f"uhrte.
Aus diesem Grunde wertet \fu{-} von links nach rechts aus,
was den Ausdruck zu \fu{(3-2)-1} macht. Dieser Ausdruck liefert
das gew"unschte Ergebnis \fu{0}.

Es folgt eine kurze Beschreibung aller Operatoren sowie der
Typen ihrer Operanden und des Typs des Ergebnisses der Operation.
Bei den Typen der Operanden ist immer zu beachten, da"s, falls
diese unterschiedlich sind, SEL immer dem "`gr"o"seren"' Typ den
Vorrang gibt, wobei \hbox{$ \tchar < \tint < \tfloat $} gilt.
\typeindex{char}
\typeindex{int}
\typeindex{float}
Der Ausdruck \fu{3 + 4.123} ergibt also einen
\tfloat\ mit dem Wert \fu{7.123}, wobei allerdings der Ausdruck
\fu{13 \&\& 5.3} einen \tint\ mit einem Wert ungleich \fu{0}
ergibt, da der Operator \fu{\&\&} nur mit \tint s arbeitet.

\begin{operator}
\tchar\ \fu{`+'} \tchar\ & \tchar\ \\
\tint\ \fu{`+'} \tint\ & \tint\ \\
\tfloat\ \fu{`+'} \tfloat\ & \tfloat\ \\
\type{\poi type} \fu{`+'} \tint\ & \type{\poi type}
\end{operator}
\index{Addition|ii}

Sind die beiden Operanden des bin"aren \fu{+}-Operators numerisch, werden
diese addiert. Ist jedoch der linke Operand ein \ind{Zeiger}, ergibt
die Operation einen Zeiger, der um die Anzahl Elemente erh"oht
ist, die der rechte Operand angibt. Beachten Sie bitte, da"s die
Gr"o"se eines Elements nicht unbedingt ein Byte sein mu"s,
sondern so gro"s ist, wie der Typ, den der Zeiger referenziert. So
ist es z.~B. leicht m"oglich, ein \ind{Array} zu durchlaufen,
ohne einen numerischen Z"ahler zu verwenden.

\begin{operator}
\tchar\ \fu{`-'} \tchar\ & \tchar\ \\
\tint\ \fu{`-'} \tint\ & \tint\ \\
\tfloat\ \fu{`-'} \tfloat\ & \tfloat\ \\
\type{\poi type} \fu{`-'} \tint\ & \type{\poi type}
\end{operator}
\index{Subtraktion|ii}

Der bin"are Operator \fu{-} subtrahiert den rechten Operanden vom linken,
falls beide numerisch sind, andernfalls wird der \ind{Zeiger} um die
Anzahl Elemente erniedrigt, die vom rechten Operanden angegeben
wird. Mit Element ist hier der Typ gemeint, den der Zeiger
referenziert.

\begin{operator}
\tchar\ \fu{`*'} \tchar\ & \tchar\ \\
\tint\ \fu{`*'} \tint\ & \tint\ \\
\tfloat\ \fu{`*'} \tfloat\ & \tfloat\
\end{operator}
\index{Multiplikation|ii}

Der bin"are \fu{*}-Operator multipliziert die beiden Operanden.

\begin{operator}
\tchar\ \fu{`/'} \tchar\ & \tchar\ \\
\tint\ \fu{`/'} \tint\ & \tint\ \\
\tfloat\ \fu{`/'} \tfloat\ & \tfloat\
\end{operator}
\index{Division|ii}
\index{Zahl!Bruch}

Der \fu{/}-Operator dividiert den linken durch den rechten
Operanden. Sind beide Operanden keine \tfloat s, so ist das
Ergebnis auch kein \tfloat , selbst wenn der Bruch nicht
ganzzahlig ist.

\begin{operator}
\tchar\ \fu{`\%'} \tchar\ & \tchar\ \\
\tint\ \fu{`\%'} \tint\ & \tint\
\end{operator}
\index{Division!Rest bei ganzzahliger|ii}
\index{Modulo|see{Division}}

Der \fu{\%}-Operator berechnet den Rest, der bei einer
ganzzahligen Division des linken durch den rechten Operanden zu
verzeichnen w"are. Beachten Sie bitte, da"s dieser Operator {\em
nicht} mit \tfloat s funktioniert.

\newpage
\begin{operator}
\tint\ \fu{`\&\&'} \tint\ & \tint\
\end{operator}
\index{AND!logisches|ii}
\index{UND|see{AND}}

Der \fu{\&\&}-Operator verkn"upft die beiden Operanden mit einem
logischen Und (AND), das hei"st, der Ausdruck ergibt einen Wert
ungleich \fu{0}, wenn sowohl der linke als auch der rechte
Operand ungleich \fu{0} sind, andernfalls ergibt der Ausdruck
\fu{0}.

\begin{operator}
\tint\ \fu{`||'} \tint\ & \tint\
\end{operator}
\index{OR!logisches|ii}
\index{ODER|see{OR}}

Der \fu{||}-Operator verkn"upft seine beiden Operanden mit einem
logischen Oder (OR), was bedeutet, da"s der Ausdruck dann ungleich
\fu{0} ist, wenn mindestens einer der beiden Operanden ungleich
\fu{0} ist. Nur wenn sowohl der linke als auch der rechte Operand
\fu{0} sind, ergibt der Ausdruck \fu{0}.

\begin{operator}
\tint\ \fu{`\char94\char94'} \tint\ & \tint\
\end{operator}
\index{XOR!logisches|ii}
\index{ODER!exklusives|see{XOR}}

Der \fu{\char94\char94}-Operator verkn"upft die beiden Operanden
mit einem logischen exklusiven Oder (XOR). Ein Ausdruck
ergibt also nur dann einen Wert ungleich \fu{0}, wenn genau einer
der Operanden ungleich \fu{0} ist. Haben hingegen entweder beide
Operanden den Wert \fu{0} oder sind beide ungleich \fu{0}, so ist der
Ausdruck gleich \fu{0}.

\begin{operator}
\tint\ \fu{`\&'} \tint\ & \tint\
\end{operator}
\index{AND!bitweises|ii}

Der bin"are \fu{\&}-Operator verkn"upft die beiden Operanden mit
einem bitweisen Und (AND), was bedeutet, da"s er ein logisches Und auf
jedes einzelne Bitpaar des linken und rechten Operanden anwendet.
So ergibt zum Beispiel der Ausdruck \fu{93 \& 45} den Wert
\fu{13}. Die Funktionsweise l"a"st sich leicht nachvollziehen,
indem man die bin"are Schreibweise
\index{Zahl!bin""are}
verwendet, also \fu{1011101b}
f"ur \fu{93}, \fu{0101101b} f"ur \fu{45} und \fu{0001101b} f"ur
\fu{13}. Jene Stellen, die in mindestens einem der
Operanden den Wert \fu{0} haben, haben auch im Ergebnis den Wert
\fu{0}. Alle anderen Stellen, also jene, die sowohl im linken
als auch im rechten Operanden den Wert \fu{1} haben, haben auch
im Ergebnis den Wert \fu{1}.

\begin{operator}
\tint\ \fu{`|'} \tint\ & \tint\
\end{operator}
\index{OR!bitweises|ii}

Der Operator \fu{|} verkn"upft beide Operanden mit einem
bitweisen Oder (OR), das hei"st, er wendet auf jedes Bitpaar der
beiden Operanden ein logisches Oder an. Der Ausdruck \fu{93 | 45}
ergibt zum Beispiel den Wert 125.

\begin{operator}
\tint\ \fu{`\char94'} \tint\ & \tint\
\end{operator}
\index{XOR!bitweises|ii}

Der Operator \fu{\char94} verkn"upft seine beiden Operanden mit
einem bitweisen exklusiven Oder (XOR), das hei"st, er wendet auf
jedes Bitpaar des linken und rechten Operanden jeweils ein
logisches exklusives Oder an. So ergibt zum Beispiel der Ausdruck
\fu{93 \char94\ 45} den Wert 112.

\begin{operator}
\tint\ \fu{`<<'} \tint\ & \tint\
\end{operator}
\index{Bitverschiebung!nach links|ii}

Der \fu{<<}-Operator ergibt einen Wert, der dem linken Operanden
entspricht, wobei jedoch die einzelnen Bits desselben um soviel
Stellen nach links geschoben werden, wie der rechte Operand
angibt. W"urde dies in eine Zahl resultieren, die mehr Bits hat,
als der \tint -Typ zul"a"st, so werden die "`"ubersch"ussigen"'
Bits abgeschnitten. Ein Beispiel: Der Ausdruck \fu{93 << 3}
ergibt \fu{744}. Warum dies so ist, l"a"st sich leicht anhand der
Bin"arschreibweise
\index{Zahl!bin""are}
der beiden Zahlen verdeutlichen. Die Zahl
\fu{93} kann auch als \fu{1011101b} geschrieben werden, die Zahl
\fu{744} als \fu{1011101000b}. Wie man sieht, wurden also
lediglich drei Nullstellen am Ende der Zahl eingef"ugt.
Anmerkung: Aufgrund der Eigenschaften des bin"aren Zahlensystems
gilt (ohne Ber"ucksichtigung der negativen Zahlen)
\hbox{$a \>\fu{<<}\> n = a \cdot 2^n.$}

\begin{operator}
\tint\ \fu{`>>'} \tint\ & \tint\
\end{operator}
\index{Bitverschiebung!nach rechts|ii}

Der Operator \fu{>>} ergibt einen Wert, der dem linken Operanden
entspricht, dessen Bits um soviel Stellen nach rechts ger"uckt
werden, wie der rechte Operand angibt. Bits, die dabei rechts
"uber die Einerstelle herausragen w"urden, werden dabei
abgeschnitten. Die sich dabei ergebende Zahl kann (ohne
Ber"ucksichtigung der negativen Zahlen) als
\hbox{$a \>\fu{>>}\> n = \lfloor a / 2^n \rfloor $} angeschrieben werden.
So ergibt z.~B. der Ausdruck \fu{93 >> 3} den Wert \fu{11}.

\begin{operator}
\fu{`-'} \tchar\ & \tchar\ \\
\fu{`-'} \tint\ & \tint\ \\
\fu{`-'} \tfloat\ & \tfloat\
\end{operator}
\index{Negation!numerische|ii}
\index{Negation!logische|see{NOT}}
\index{Negation!bitweise|see{NOT}}

Der un"are Operator \fu{-} negiert den Operanden, das
hei"st, er macht aus einer positiven Zahl eine negative und
umgekehrt. So ergibt zum Beispiel der Ausdruck \fu{- 93} den Wert
\fu{-93}.

\begin{operator}
\fu{`*'} \type{\poi type} & \type{type}
\end{operator}
\index{Zeiger!Dereferenzierung|ii}

Der un"are \fu{*}-Operator dereferenziert den Operanden, das
hei"st, er ergibt jene Variable, auf die der Operand zeigt (ist
der Operand kein Zeiger, ist die Operation illegal). Angenommen,
die Variable \fu{p} ist vom Typ \type{\poi int} und wurde mit
\fu{p = \&i} initialisiert, wobei \fu{i} vom Typ \tint\ ist, so
ist der Ausdruck \fu{*p} mit dem Ausdruck \fu{i} gleichzusetzen.

\begin{operator}
\fu{`\&'} \type{type} & \type{\poi type}
\end{operator}
\index{Zeiger!Ermittlung|ii}
\index{Variable!Adresse}

Der un"are \fu{\&}-Operator ergibt einen Zeiger auf seinen
Operanden, was nat"urlich nur m"oglich ist, wenn dieser keine
Konstante ist. So ist z.~B. der Ausdruck \fu{\&i} mit dem
Ausdruck \fu{p} gleichzusetzen, wenn \fu{*p} gleich \fu{i} ist.
Aus diesem und dem letzten Beispiel wird klar, da"s \fu{\&} das
Gegenst"uck zum \fu{*}-Operator ist und umgekehrt.

\begin{operator}
\fu{`!'} \tint\ & \tint\
\end{operator}
\index{NOT!logisches|ii}

Der Operator \fu{!} ergibt die logische Negation (NOT) seines
Operanden, was bedeutet, da"s der Wert gleich \fu{0} ist, falls
der Operand ungleich \fu{0} ist und umgekehrt. So ergibt z.~B.
der Ausdruck \fu{!93} den Wert \fu{0}.

\begin{operator}
\fu{`\char126'} \tint\ & \tint\
\end{operator}
\index{NOT!bitweises|ii}
\index{NOT!numerisches|see{Negation}}

Der Operator \fu{\char126} errechnet die bitweise Negation seines
Operanden, das hei"st, er wendet auf jedes Bit desselben die
logische Negation an. Das Ergebnis kann je nach Wortl"ange und
Darstellung des Typs \tint\ stark differenzieren. Im Regelfall
kann jedoch mit einer Wortl"ange von 32 Bits und einer
Einerkomplementdarstellung der negativen Zahlen gerechnet werden.
So ergibt zum Beispiel der Ausdruck \fu{\char126 93} auf MS-DOS-
und Windows NT-Rechnern den Wert \fu{-94}.

\begin{operator}
\tchar\ \fu{`=='} \tchar\ & \tint\ \\
\tint\ \fu{`=='} \tint\ & \tint\ \\
\tfloat\ \fu{`=='} \tfloat\ & \tint\
\end{operator}
\index{Vergleich|ii}

Der Operator \fu{==} "uberpr"uft die beiden Operanden auf
Gleichheit, das hei"st, der Ausdruck ergibt einen Wert ungleich
\fu{0}, falls der linke und der rechte Operand den selben Wert
haben, andernfalls \fu{0}.

\begin{operator}
\tchar\ \fu{`<'} \tchar\ & \tint\ \\
\tint\ \fu{`<'} \tint\ & \tint\ \\
\tfloat\ \fu{`<'} \tfloat\ & \tint\
\end{operator}

Der Operator \fu{<} "uberpr"uft, ob der linke Operand kleiner als
der rechte ist. Falls dies zutrifft, ergibt der Ausdruck einen
Wert ungleich \fu{0}, ansonsten \fu{0}.

\begin{operator}
\tchar\ \fu{`>'} \tchar\ & \tint\ \\
\tint\ \fu{`>'} \tint\ & \tint\ \\
\tfloat\ \fu{`>'} \tfloat\ & \tint\
\end{operator}

Der Operator \fu{>} "uberpr"uft, ob der linke Operand gr"o"ser
ist als der rechte. Trifft das zu, ergibt der Ausdruck einen Wert
ungleich \fu{0}, andernfalls \fu{0}.

\begin{operator}
\tchar\ \fu{`<='} \tchar\ & \tint\ \\
\tint\ \fu{`<='} \tint\ & \tint\ \\
\tfloat\ \fu{`<='} \tfloat\ & \tint\
\end{operator}

Der Operator \fu{<=} "uberpr"uft, ob der linke Operand kleiner
oder gleich dem rechten Operanden ist und ergibt in diesem Fall
einen Wert ungleich \fu{0}, ansonsten \fu{0}.

\begin{operator}
\tchar\ \fu{`>='} \tchar\ & \tint\ \\
\tint\ \fu{`>='} \tint\ & \tint\ \\
\tfloat\ \fu{`>='} \tfloat\ & \tint\
\end{operator}

Der Operator \fu{>=} "uberpr"uft, ob der linke Operand gr"o"ser
oder gleich dem rechten Operanden ist und ergibt in diesem Fall
einen Wert ungleich \fu{0}, andernfalls \fu{0}.

\begin{operator}
\tchar\ \fu{`!='} \tchar\ & \tint\ \\
\tint\ \fu{`!='} \tint\ & \tint\ \\
\tfloat\ \fu{`!='} \tfloat\ & \tint\
\end{operator}

Der Operator \fu{!=} "uberpr"uft die beiden Operanden auf
Ungleichheit und ergibt in diesem Fall einen Wert ungleich \fu{0},
bei Gleichheit ergibt sich der Wert \fu{0}.

\begin{operator}
\type{type} \fu{`='} \type{type} & \type{type}
\end{operator}
\index{Zuweisung|ii}

Der Operator \fu{=} weist dem linken Operanden den Wert des
rechten Operanden zu, was aber vorraussetzt, da"s der linke
Operand keine \ind{Konstante}, also entweder eine \ind{Variable} (auch
innerhalb einer \ind{Struktur}), ein
dereferenzierter \ind{Zeiger} oder ein indiziertes \ind{Array} ist. SEL ist
so konzipiert, da"s auch ganze Arrays oder Strukturen mit einer
Zuweisung kopiert werden k"onnen. Beachten Sie aber bei der
Arbeit mit Zeigern, da"s eine Zuweisung nicht bedeutet, da"s die
Werte, auf die die Zeiger zeigen, sondern die Adressen, auf die
die Zeiger verweisen, kopiert werden. Zur Veranschaulichung ein
Beispiel: Angenommen, die Variablen \fu{p} und \fu{q} sind vom
Typ \type{\poi int} und zeigen auf die \tint s \fu{i} und \fu{j},
soda"s \fu{*p} gleich \fu{i} und \fu{*q} gleich \fu{j} ist. In
diesem Fall bewirkt der Ausdruck \fu{p = q} nicht, da"s nun der
Wert von \fu{i} gleich dem von \fu{j} ist, sondern da"s nun
\fu{p}, genauso wie \fu{q}, auf \fu{j} zeigt. Ersterer Effekt
w"are mit dem Ausdruck \fu{*p = *q} zu erzielen.

\begin{operator}
\type{struct} or \type{\poi struct} \fu{`->'} \fu{IDENTIFIER} & \type{type}
\end{operator}
\fuindex{IDENTIFIER}
\index{Struktur!Zugriff auf Element|ii}

Der Operator \fu{->} wird benutzt, um ein Strukturelement zu
erreichen. Um dies zu vereinfachen, kann der Operator sowohl
direkt mit einer Struktur als auch mit einem \ind{Zeiger} auf eine
Struktur arbeiten. Der rechte Operand gibt den Namen des Elements
an, auf das zugegriffen werden soll.

\begin{operator}
\type{\poi type} or \type{\arr{}type} \fu{`['} \fu{expr} \fu{`]'} & \type{type}
\end{operator}
\fuindex{expr}
\index{Array!Zugriff auf Element|ii}

Der \fu{[]}-Operator wird benutzt, um auf ein einzelnes
Element eines Arrays zuzugreifen, wobei dieses Array auch als
\ind{Zeiger} dargestellt werden kann. Auf diese Weise kann auch auf
einen Speicherbereich, der nicht als Array definiert worden ist,
wie auf ein Array zugegriffen werden. Hierzu wird in eckigen
Klammern ein Index angegeben, wobei das erste Element des Arrays
den Index \fu{0} besitzt und negative Indizes nicht erlaubt sind.
Anmerkung: Der Ausdruck \fu{p[i]} kann auch als \fu{*(p + i)}
angeschrieben werden, wobei \fu{p} ein Array oder ein Zeiger
eines beliebigen Typs ist und \fu{i} ein positiver \tint .
Beachten Sie, da"s SEL keine "Uberpr"ufung des Index vornimmt,
das hei"st, da"s es m"oglich ist, auf Elemente zuzugreifen, die
eigentlich nicht mehr zum Array geh"oren, wovon jedoch in den
meisten F"allen abzuraten ist.


\subsubsection{Rangfolge der Operatoren}
\index{Operator!Rangfolge|ii}
\index{Operator!Priorit""at|ii}

Um eine unn"otige Klammerung zu vermeiden, besitzen die
Operatoren in SEL eine Rangfolge. Die arithmetischen Operatoren
sind an die der Mathematik angelehnt, soda"s zum Beispiel der
Ausdruck \fu{a+b*c-d} dem Ausdruck \fu{(a+(b*c))-d} gleichkommt,
was der mathematischen Notation entspricht. Ist allerdings
gew"unscht, da"s \fu{a+b} mit \fu{c-d} multipliziert wird, mu"s
die nat"urliche Operatorenrangfolge mittels Klammerung
umgangen und der Ausdruck zu \fu{(a+b)*(c-d)} umgeformt
werden.

Die folgende Tabelle ist eine Aufstellung aller Operatoren von
SEL, wobei Operatoren mit gleicher Priorit"at in jeweils der
gleichen Zeile stehen. Die Operatoren, die in der Liste weiter
oben stehen, haben niedrigere Priorit"at, das hei"st, sie binden
schw"acher. Die rechte Spalte der Tabelle gibt die Richtung
der Auswertung
\index{Operator!Richtung der Auswertung|ii}
an und bezieht sich auf alle Operatoren in der jeweiligen Zeile.

\medskip
\begin{tabular}{ll}
\verb?=?, \verb?delete? & von rechts nach links \\
\verb?&&?, \verb?||?, \verb?^^? & von links nach rechts \\
\verb?==?, \verb?<?, \verb?>?, \verb?<=?, \verb?>=?, \verb?!=? &
  von links nach rechts\\
\verb?&?, \verb?|?, \verb?^?, \verb?<<?, \verb?>>? &
  von links nach rechts \\
\verb?+?, \verb?-? & von links nach rechts \\
\verb?*?, \verb?/?, \verb?%? & von links nach rechts \\
\verb?!?, \verb?~?, \verb?-? (un"ar), \verb?*? (un"ar), \verb?&? (un"ar) &
  von rechts nach links \\
\verb?->?, \verb?:?, \verb?[]? & von links nach rechts \\
\verb?sizeof?, \verb?new? & ---
\end{tabular}

\section{Typen}
\index{Typ|ii}
\fuindexii{type}
\rwindex{char}
\rwindex{int}
\rwindex{float}
\rwindex{void}
\fuindex{array}
\fuindex{pointer}
\fuindex{IDENTIFIER}

\begin{verbatim}
type:          `char'
             | `int'
             | `float'
             | `void'
             | array
             | pointer
             | IDENTIFIER
             ;
\end{verbatim}

Typen werden in SEL verwendet, um Werte
\index{Wert}
zu klassifizieren. So hat
z.~B. jeder \ind{Ausdruck}, jede \ind{Variable} und jede \ind{Funktion}
einen Typ. Die vier Basistypen von SEL k"onnen durch Kombination
zur Konstruktion weiterer, komplexerer Typen verwendet werden.

\subsection{Basistypen}
\index{Typ!Basistyp}
\typeindexii{int}
\typeindexii{char}
\typeindexii{float}
\typeindexii{void}

Die Programmiersprache SEL kennt vier Basistypen, n"amlich
\tchar , \tint , \tfloat\ und \tvoid . Die ersten beiden Typen,
\tchar\ und \tint\ sind sogenannte Aufz"ahlungstypen,
was sich mit den ganzen Zahlen in der Mathematik vergleichen l"a"st.

\tchar\ wird haupts"achlich verwendet, um \indi{Zeichen} zu
repr"asentieren und hat die Gr"o"se eines Bytes. Obwohl \tchar\
Zeichen repr"asentiert, k"onnen auf ihn alle arithmetischen
Operationen angewendet werden, die auch mit \tint\ funktionieren.

Letzterer wird meistens verwendet, um ganzzahlige Berechnungen
\index{Zahl!ganze|ii}
auszuf"uhren. Aus diesem Grund mu"s er gen"ugend gro"s
dimensioniert sein. In der Regel entspricht ein \tint\ 32 Bits,
was bedeutet, da"s er einen Wertebereich von $-2^{31}$ bis
$+2^{31}-1$ hat. Au"serdem wird \tint\ in SEL zur boolee'schen
Arithmetik
\index{boolee'sche Arithmetik}
herangezogen, also zur Berechnung von logischen
Aussagen. In diesem Fall bedeutet ein Wert von \fu{0} die Aussage
"`logisch falsch"', ein Wert ungleich \fu{0} bedeutet "`logisch
wahr"'.

Der Typ \tfloat\ repr"asentiert rationale Zahlen
\index{Zahl!rationale|ii}
mit begrenzter Genauigkeit. Auf MS-DOS-Rechnern hat er eine Gr"o"se
von 32 Bits.

Eine besondere Stellung unter den Typen nimmt \tvoid\ ein. Er
repr"asentiert nichts und es ist auch nicht m"oglich, einem
\tvoid\ einen Wert zuzuweisen, was auch gar keinen Sinn h"atte.
\tvoid\ wird z.~B. verwendet, um Funktionen zu
deklarieren, die keinen R"uckgabewert
\index{R""uckgabewert}
haben. Au"serdem kann er
in einer Funktion verwendet werden, die einen \ind{Zeiger} unbekannten
Typs verlangt oder zur"uckgibt.

\subsection{Arrays}
\index{Array|ii}
\fuindexii{array}
\fuindex{INT\protect{\_CONSTANT}}
\fuindex{type}

\begin{verbatim}
array:         `[' INT_CONSTANT `]' type
             ;
\end{verbatim}

Ein Array ist eine Zusammenfassung einer bestimmten Anzahl von
Variablen gleichen Typs, die im Speicher als eine Folge von
Variablen
\index{Variable!Folge|ii}
repr"asentiert wird, wobei die Variablen mit kleinerem
Index an einer Stelle geringerer Adresse angesiedelt sind.

In SEL ist es m"oglich, Arrays beliebig zu schachteln, also
mehrdimensionale Array zu erzeugen. So ist etwa
\type{\arr{5}\arr{10}int} ein Array, das aus 5 Arrays besteht,
wobei jedes dieser Arrays aus 10 \tint s besteht. Um auf das
zweite Element aus dem dritten Array zuzugreifen, wird der
Ausdruck \fu{a[3][2]} verwendet. Anmerkung: Dieses Array
beansprucht genausoviel Speicher wie $5 \cdot 10 = 50$ \tint s.

Arrays k"onnen nat"urlich nicht nur aus Elementartypen
konstruiert werden, sondern aus jedem beliebigen Typ, also z.~B.
aus Strukturen. Der Nachteil jedoch ist, da"s die
Gr"o"se des Arrays schon bei der Deklaration feststehen mu"s, sie
kann also nicht bei Bedarf ge"andert werden. Beachten Sie, da"s
SEL beim Zugriff auf Array-Elemente nicht "uberpr"uft, ob das
angegebene Element auch wirklich zum Array geh"ort, also ob das
Array gen"ugend gro"s dimensioniert ist.

\subsection{Zeiger}
\index{Zeiger|ii}
\fuindexii{pointer}
\fuindex{type}

\begin{verbatim}
pointer:       `*' type
             ;
\end{verbatim}

Ein Zeiger ist eine Variable, die die Adresse
\index{Variable!Adresse}
einer anderen
Variablen enth"alt, die von dem Typ ist, der dem \fu{*} in der
Typendeklaration folgt, und hat in der Regel dieselbe Gr"o"se wie
ein \tint . Bis auf die \ind{Zuweisung} sind alle Operationen
funktionsgleich mit der eines Arrays.
\index{Zeiger!""Aquivalenz zu Array}
\index{Array!""Aquivalenz zu Zeiger}

Zeiger k"onnen, wie Arrays,
beliebig geschachtelt werden, das hei"st, da"s es ohne weiteres
m"oglich ist, einen Zeiger auf einen Zeiger zu deklarieren und zu
benutzen. Beachten Sie allerdings, da"s ein zweidimensionales
Array nicht mit einem Zeiger auf einen Zeiger gleichgesetzt
werden kann. So ist z.~B. ein Array \type{\arr{10}int} in
der Verwendung mit einem Zeiger \type{\poi int} gleich (bis auf
die Ausnahme der Zuweisungsoperation), jedoch ist ein Array
\type{\arr{5}\arr{10}int} nicht gleich einem Zeiger \type{\poi
\poi int}, da der Typ, aus dem das Array besteht und auf den der
Zeiger zeigt, zu diesem Zweck gleich sein m"u"sten. In diesem Fall
ist jedoch der Typ, aus dem das Array aufgebaut ist, wieder ein
Array, n"amlich \type{\arr{10}int}. Der Typ, auf den der Zeiger
zeigt, ist jedoch \type{\poi int}. Um ein zweidimensionales Array
wie oben angeschrieben als Zeiger zu repr"asentieren, mu"s dieser
den Typ \type{\poi \arr{10}int} haben.

\subsubsection{Die new- und delete-Operatoren}
\label{newdelete}
\rwindexii{new}
\rwindexii{delete}

Um eine \ind{dynamische Allokation} von Variablen zu erm"oglichen,
existieren in SEL die Operatoren \rw{new} und \rw{delete}. Es
folgt eine kurze Beschreibung:

\begin{operator}
\fu{`new'} \fu{type} & \type{\poi type}
\end{operator}

Das Ergebnis der \rw{new}-Operation ist ein Zeiger auf eine
Variable des Typs, der als Operand angegeben wird. Diese Variable
wird dynamisch allokiert, das hei"st, sie wird dem \indi{Heap}
entnommen. Ist dieser schon so voll, da"s eine Variable des
angegebenen Typs keinen Platz mehr finden w"urde, ist das
Ergebnis ein \ind{Nullzeiger}. Beachten Sie, da"s der Speicherbereich
so lange belegt bleibt, bis \rw{delete} verwendet wird, um ihn
freizugeben, da SEL keine "`garbage collection"' benutzt.
Anmerkung: Um einen Speicherbereich zu allokieren, dessen Gr"o"se
nicht schon vor der Ausf"uhrung des Programms feststeht, ist die Funktion
\fui{MemAlloc} zu benutzen. Weiters ist es mittels der Funktion
\fui{MemReAlloc} m"oglich, einen bereits
allokierten Block zu vergr"o"sern bzw. zu verkleinern, wobei der
Inhalt des Blocks unver"andert bleibt (bei einer Verkleinerung
geht nat"urlich ein Teil des Inhaltes verloren).

\begin{operator}
\fu{`delete'} \type{\poi type} & \tvoid
\end{operator}

Der \rw{delete}-Operator gibt einen Speicherblock frei, der
vorher mit \rw{new} oder \fui{MemAlloc} allokiert wurde. Beachten
Sie, da"s dem Operator genau der Zeiger als Operand angeben
werden mu"s, den \rw{new} oder \fu{MemAlloc} (oder im Falle einer
Ver"anderung der Gr"o"se des Speicherbereichs \fui{MemReAlloc})
geliefert hat, da sonst die Speicherverwaltung ins Wanken kommen
kann, was in einem Programmabsturz resultieren k"onnte.

\subsection{Strukturen}
\index{Struktur|ii}
\fuindexii{struct}
\rwindexii{struct}
\fuindex{IDENTIFIER}
\fuindex{type}
\rwindex{end}

\begin{verbatim}
struct:        `struct' IDENTIFIER
                 {IDENTIFIER `:' type `;'}
               `end'
             ;
\end{verbatim}

Eine Struktur ist eine Zusammenfassung mehrerer Variablen
\index{Variable!Zusammenfassung mehrerer|ii}
beliebigen Typs
\index{Typ}
unter einem gemeinsamen Namen. Mittels des
Operators \fu{->} kann auf ein einzelnes Element der Struktur
zugegriffen werden.
\index{Struktur!Zugriff auf Element}
Dem \rw{struct}-Schl"usselwort folgt der
zuk"unftige Name der Struktur---bei der Deklaration von Variablen
wird dieser Name als Typ verwendet. Danach werden die einzelnen
Elemente der Struktur definiert. Die Namen dieser Elemente d"urfen
sich ohne weiteres mit anderen Variablennamen des Programms
decken, da durch den Operator \fu{->} eine eindeutige
Identifikation gew"ahrt ist, allerdings m"ussen sie innerhalb der
Struktur eindeutig sein. Die Elemente d"urfen beliebige Typen
haben, es ist sogar erlaubt, innerhalb einer Struktur einen
Zeiger auf ebendiese Struktur zu deklarieren, nicht aber ein
Array aus der Struktur und auch nicht die Struktur selbst, da
diese Deklarationen die Gr"o"se der Struktur ins Unendliche
anwachsen lie"sen. Die Gr"o"se einer Struktur ist in der Regel
gleich der Summe der Gr"o"sen ihrer Elemente.

Beachten Sie, da"s die Definition einer Struktur nur in dem \ind{Modul}
bekannt ist, in der sie get"atigt wird, selbst, wenn sie
innerhalb eines \rw{global}-Blocks steht. In der Regel ist es
deshalb zweckm"a"sig, Strukturen, die von mehreren Modulen
verwendet werden, in Include-Dateien
\index{Include-Datei}
zu definieren und diese in den
jeweiligen Modulen zu inkludieren.

\example
Die folgende Struktur ist ein Element einer verketteten \indu{Liste},
deren Daten die Adressen von Personen sind. Das Element \fu{Next}
zeigt auf das jeweils n"achste Element der Liste.

\index{Adresse@{\tt Adresse}~(\rw{struct})|ii}
\begin{verbatim}
struct Adresse
  Name    : char[26];
  Vorname : char[31];
  PLZ     : int;
  Stadt   : char[36];
  Strasse : char[36];
  Next    : *Adresse;
end
\end{verbatim}

\subsection{Zusammengesetzte Datentypen}
\index{Typ!zusammengesetzter|ii}

Nat"urlich ist es in SEL auch m"oglich, Datentypen aus den oben
besprochenen zu kombinieren, also zum Beispiel ein Array aus
Strukturen oder eine Struktur, die Arrays oder Zeiger beinhaltet,
zu definieren. Zu beachten ist nur, da"s die vorhandene
Speicherkapazit"at nicht "uberschritten wird. Unter MS-DOS
kommt weiterhin die 64~KByte-Barriere hinzu, die aus der
Segmentierung des Speichers im Real-Mode der 80x86-Prozessoren
resultiert. Deshalb ist es nicht m"oglich, in MS-DOS Variablen zu
definieren, die gr"o"ser als 64~KByte w"aren (sehr wohl ist es
m"oglich, Typen dieser Gr"o"se zu definieren, solange keine
Variablen dieses Typs verwendet werden).

\subsection{Der sizeof-Operator}
\index{Typ!Gr""o""se}
\rwindexii{sizeof}

\begin{operator}
\fu{`sizeof'} \type{type} & \tint\
\end{operator}

Der \rw{sizeof}-Operator verlangt als einzigen Operanden einen
Typ und gibt dessen Gr"o"se in Bytes zur"uck. Diese kann f"ur
ein und denselben Typen je nach Rechner verschieden sein, in der
Regel ist das jedoch nicht der Fall. Trotzdem sollten sie immer
\rw{sizeof} verwenden, wenn es um die Gr"o"se eines Typs geht und
nicht von einer bestimmten Gr"o"se ausgehen. Anmerkung: Der
einzige Typ, dessen Gr"o"se auf allen Rechnern gleich ist, ist
\tvoid\
\typeindex{void}
(nicht jedoch zum Beispiel \type{\poi void}). Seine
Gr"o"se betr"agt \fu{0}.

\subsection{Typecasting}
\index{Typecasting|ii}
\index{Typ!Umwandlung|see{Typecasting}}
\fuindex{type}

\begin{operator}
\type{\_type} \fu{`:'} \fu{type} & \type{type}
\end{operator}

Um Ausdr"ucke eines Typs in einen anderen umzuwandeln, bietet SEL
Typecasting. Der Operator \fu{:} wandelt seinen linken Operanden
in den Typ um, den der rechte Operand angibt. Dabei k"onnen
folgende Umwandlungen durchgef"uhrt werden:

\medskip
\begin{tabular}{ll}
Ausgangstyp & Zieltyp \\[0.5ex]
\tchar\ & \tint , \tfloat\ \\
\tint\ & \tchar , \tfloat , \type{\poi type}, \type{\arr{}type} \\
\tfloat\ & \tchar , \tint\ \\
\type{\poi type} & \tint , \type{\arr{}type} \\
\type{\arr{}type} & \tint , \type{\poi type}
\end{tabular}
\medskip

Bei der Umwandlung von \tint s
\typeindex{int}
in Arrays
\index{Array}
und \ind{Zeiger} und
umgekehrt ist zu beachten, da"s der \tint\ die Speicheradresse
angibt, auf die der Pointer zeigt, bzw. an der das Array beginnt
(die Adresse des ersten Arrayelements).

Zus"atzlich zum expliziten Typecasting f"uhrt SEL noch implizites
Typecasting durch, falls dies n"otig ist. Verlangt also eine
Funktion einen Parameter vom Typ \tfloat ,
\typeindex{float}
und es wird ein \tint\
"ubergeben, wird dieser automatisch in einen \tfloat\
umgewandelt. Beim impliziten Typecasting gelten wie beim
expliziten Typecasting die oben beschriebenen Regeln.

\section{Bezeichner}
\label{identifiers}
\index{Bezeichner|ii}
\fuindexii{IDENTIFIER}

\begin{verbatim}
IDENTIFIER: [A-Za-z_][A-Za-z0-9_]*
\end{verbatim}

Bezeichner werden in SEL f"ur Funktions-, Variablen- und
Strukturnamen
\index{Funktion!Name|see{Bezeichner}}
\index{Variable!Name|see{Bezeichner}}
\index{Struktur!Name|see{Bezeichner}}
verwendet. Ein Bezeichner mu"s mit einem Buchstaben
oder einem Unterstrich beginnen, danach folgt eine beliebige
Anzahl von Buchstaben, Ziffern oder Unterstrichen. So ist z. B.
\fu{\_xYz1} ein g"ultiger Bezeichner, nicht aber
\fu{1xy}, da er mit einer Ziffer beginnt. Die Bezeichner in SEL
sind {\em case-sensitiv}, das hei"st, es wird zwischen Gro"s-
und Kleinschreibung unterschieden. So ist z. B. \fu{counter} ein
anderer Bezeichner als \fu{Counter} oder \fu{COUNTER}. Beachten
Sie aber, da"s sie keines der reservierten W"orter
\index{reserviertes Wort|ii}
\index{Schl""usselwort|see{reserviertes Wort}}
als einen Bezeichner verwenden d"urfen. Der Vollst"andigkeit halber hier
eine Liste aller reservierten W"orter von SEL:

\medskip
\begin{tabular}{lllllll}
\bf begin & \bf break  & \bf case   & \bf char  & \bf default & \bf delete   & \bf downto \\
\bf else  & \bf end    & \bf extern & \bf float & \bf for     & \bf function & \bf global \\
\bf if    & \bf int    & \bf local  & \bf new   & \bf repeat  & \bf return   & \bf sizeof \\
\bf step  & \bf struct & \bf switch & \bf to    & \bf until   & \bf void     & \bf while  \\
\end{tabular}

\chapter{Der Pr\protect{"aprozessor}}
\index{Pr""aprozessor|ii}

Bevor ein SEL-Programm interpretiert bzw. compiliert wird, wird
es durch den Pr"aprozessor bearbeitet. Seine Hauptaufgaben sind
das Entfernen von Kommentaren, das Einf"ugen von Include-Dateien
und das Expandieren von Makros.

\section{Kommentare}
\index{Kommentar|ii}

Kommentare werden in SEL durch das Zeichen \fu{//} eingeleitet
und dauern bis an das Ende der jeweiligen Zeile an. Sollen also
l"angere Kommentare in den Quelltext eingef"ugt werden, mu"s am
Anfang jeder Kommentarzeile \fu{//} stehen. Beachten Sie bitte,
da"s \fu{//}, eingeschlossen in doppelte Anf"uhrungszeichen, nicht
den Anfang eines Kommentars, sondern eine String-Konstante
\index{Konstante!String}
darstellt.

\example
\fuindexii{Demo}
\fuindexuu{WriteInt}
\fuindexuu{WriteString}
\begin{verbatim}
function Demo (void) : void
begin
  WriteInt(20);        // Das ist ein Kommentar
  WriteString("Das // ist keiner\n");
end
\end{verbatim}

\section{Die \#define-Direktive}
\fuindexii{\#define}

Mittels der \fu{\#define}-Direktive ist es in SEL m"oglich,
Makros
\index{Makro|ii}
zu definieren. Ein Beispiel:

\begin{verbatim}
#define MACRO        3+5
\end{verbatim}

Tritt nun im nachfolgenden Quellcode irgendwo (au"ser innerhalb
einer String-Konstanten
\index{Konstante!String}
oder eines Kommentars) der Bezeichner
\fu{MACRO} auf, wird dieser vom Pr"aprozessor durch \fu{3+5}
ersetzt. Bei Makros wie dem obigen ist zu beachten, da"s ein
Makro nicht automatisch geklammert wird, d.~h., da"s der Ausdruck
\fu{MACRO*2} nicht das Ergebnis \fu{16}, sondern \fu{13}
(\fu{3+5*2}) hat.

Ist ein Makrotext zu lang, um ihn in einer Zeile unterzubringen,
kann er mit \fu{\char92}
\index{Backslash (\fu{\char92})|ii}
abgeteilt werden:

\begin{verbatim}
#define XYZ          if (x == y)     \
                       WriteInt(z);  \
                     end
\end{verbatim}

Beachten Sie, da"s die \fu{\#define}-Direktive immer am Beginn
einer Zeile stehen mu"s, und in dieser Zeile au"ser der Direktive, dem
Makronamen und dem einzusetzenden Text sonst nichts.

Makros mit Parametern, wie etwa in C, sind in der momentanen
Version von SEL nicht vorhanden, sind jedoch f"ur eine kommende
Version geplant.

\section{Die \#include-Direktive}
\index{Include-Datei|ii}
\fuindexii{\#include}

Mit der \fu{\#include}-Direktive ist es m"oglich, in eine
Quelldatei eine andere einzuf"ugen. Dabei mu"s auf \fu{\#include}
ein in Anf"uhrungszeichen gesetzter Dateiname folgen. Ebenso wie
die \fu{\#define}-Direktive mu"s auch die
\fu{\#include}-Direktive alleine am Anfang einer Zeile stehen.

\example
\fuindexuu{WriteString}
\begin{verbatim}
#include "conio.sh"

function main () : void
begin
  WriteString("Test\n");
end
\end{verbatim}

Hier wird an der Stelle der \fu{\#include}-Direktive der Inhalt
der Datei \fu{`conio.sh'} eingef"ugt. Sollte diese Datei selbst
\fu{\#include}-Direktiven enthalten, werden auch diese vom
Pr"aprozessor ausgef"uhrt, es ist also m"oglich,
\fu{\#include}-Direktiven zu verschachteln.

F"ur alle
Bibliotheksfunktionen von SEL gibt es Include-Dateien, die
deren Funktionsdeklarationen enthalten, soda"s diese nicht
jedesmal von Hand eingef"ugt werden m"ussen. Die Namen dieser
Dateien sind f"ur jede Funktion in der Funktionsreferenz
angegeben.

\chapter{Funktionsreferenz}
\label{funcref}

%%%%%%%%%%%%%%%%%%%%%%%%%%%%%%%%%%%%%%%%%%%%%%%%%

\begin{function}{MemAlloc}

\syntax
\#include "memory.sh"
function MemAlloc (s : int) : *void
\endgroup

\parameters
\param{s}{\tint} Gr"o"se des zu allokierenden Speicherbereichs \cr

\usage
Die Funktion \fu{MemAlloc} allokiert
\index{dynamische Allokation}
einen Speicherbereich mit
der Gr"o"se \fu{s} Bytes. Ist eine solche Allokation wegen
Speichermangels nicht m"oglich, gibt die Funktion einen
\ind{Nullzeiger} zur"uck. Im Gegensatz zum Operator \rwi{new} ist es mit
\fu{MemAlloc} auch m"oglich, einen Speicherbereich zu allokieren,
dessen Gr"o"se erst bei Laufzeit bestimmt wird. Der allokierte
Speicherbereich sollte, wenn er nicht mehr ben"otigt wird, mit
der Funktion \fu{MemFree} freigegeben werden.

\fseealso
\fu{MemFree}, \rw{new} (Kapitel \ref{newdelete})
\end{function}

%%%%%%%%%%%%%%%%%%%%%%%%%%%%%%%%%%%%%%%%%%%%%%%%%

\begin{function}{MemAvail}

\syntax
\#include "memory.sh"
function MemAvail () : int
\endgroup

\usage
Die Funktion \fu{MemAvail} gibt die Anzahl der noch freien Bytes
auf dem \ind{Heap} zur"uck, also jenes Speicherbereichs, der vom
Operator \rwi{new} (siehe Kapitel \ref{newdelete}) und der
Funktion \fu{MemAlloc} benutzt wird, um Speicher zu allokieren.

\fseealso
\fu{MemAlloc}, \rw{new} (Kapitel \ref{newdelete})
\end{function}

%%%%%%%%%%%%%%%%%%%%%%%%%%%%%%%%%%%%%%%%%%%%%%%%%
\newpage

\begin{function}{MemFree}

\syntax
\#include "memory.sh"
function MemFree (p : *void) : void
\endgroup

\parameters
\param{p}{\type{\poi void}} freizugebender Speicherblock \cr

\usage
Die Funktion \fu{MemFree} gibt einen Speicherblock, der mit
\fu{MemAlloc}, \fu{MemReAlloc} oder \rwi{new} allokiert worden
ist, frei. Beachten Sie jedoch, da"s Speicherbl"ocke, die mit
\rw{new} allokiert worden sind, besser mit \rwi{delete}
freigegeben werden sollten.

\fseealso
\fu{MemAlloc}, \fu{MemReAlloc}, \rw{delete}, \rw{new} (Kapitel \ref{newdelete})
\end{function}

%%%%%%%%%%%%%%%%%%%%%%%%%%%%%%%%%%%%%%%%%%%%%%%%%

\begin{function}{MemReAlloc}

\syntax
\#include "memory.sh"
function MemReAlloc (p : *void, s : int) : *void
\endgroup

\parameters
\param{p}{\type{\poi void}} allokierter Speicherblock \cr
\param{s}{\tint} neue Gr"o"se \cr

\usage
Die Funktion \fu{MemReAlloc} "andert die Gr"o"se des
Speicherblocks \fu{p},
\index{dynamische Allokation}
der mit \fu{MemAlloc}, \fu{MemReAlloc}
oder \rwi{new} allokiert worden sein mu"s, auf die Gr"o"se \fu{s}
Bytes. Ist \fu{p} ein Nullzeiger, wird ein Speicherbereich der
Gr"o"se \fu{s} wie mit \fu{MemAlloc} neu angelegt. Ist \fu{s}
gleich \fu{0}, dann wird der Speicherblock wie mit \fu{MemFree}
freigegeben. Andernfalls wird seine Gr"o"se ver"andert, wobei
sein Inhalt soweit wie m"oglich bestehen bleibt. Ist es aus
Speichermangel nicht m"oglich, die Gr"o"se des Blocks zu "andern,
gibt die Funktion einen \ind{Nullzeiger} zur"uck, und der alte Block
bleibt bestehen. Andernfalls gibt die Funktion einen Zeiger auf
den neuen Block zur"uck.

\fseealso
\fu{MemAlloc}, \fu{MemFree}, \rw{new} (Kapitel \ref{newdelete})
\end{function}

%%%%%%%%%%%%%%%%%%%%%%%%%%%%%%%%%%%%%%%%%%%%%%%%%

\begin{function}{Random}
\index{Zufallszahl|ii}

\syntax
\#include "stdlib.sh"
function Random (i : int) : int
\endgroup

\parameters
\param{i}{\tint} H"ochstgrenze f"ur die Zufallszahl \cr

\usage
Die Funktion Random gibt eine Zufallszahl
$0 \leq x \leq \fu{i}$ zur"uck. Um zu gew"ahrleisten, da"s
diese Zahl nicht f"ur jeden Aufruf des Programms dieselbe ist,
mu"s der Zufallszahlengenerator mit der Funktion \fu{Randomize}
initialisiert werden.

\fseealso
\fu{Randomize}
\end{function}

%%%%%%%%%%%%%%%%%%%%%%%%%%%%%%%%%%%%%%%%%%%%%%%%%
\newpage

\begin{function}{Randomize}
\index{Zufallszahl!Initialisierung des Generators|ii}

\syntax
\#include "stdlib.sh"
function Randomize () : void
\endgroup

\usage
Die Funktion \fu{Randomize} initialisiert den
Zufallszahlengenerator von SEL. Hierf"ur wird die aktuelle
Uhrzeit verwendet. Das garantiert, da"s bei jedem Aufruf des
Programms eine andere Reihe von Zufallszahlen generiert wird.

\fseealso
\fu{Random}
\end{function}

%%%%%%%%%%%%%%%%%%%%%%%%%%%%%%%%%%%%%%%%%%%%%%%%%

\begin{function}{ReadChar}
\typeindex{char}

\syntax
\#include "conio.sh"
function ReadChar () : char
\endgroup

\usage
Die Funktion \fu{ReadChar} liest ein einzelnes Zeichen von der
Systemkonsole ein, ohne es auf dem Bildschirm auszugeben. Dazu
kann die Funktion \fu{WriteChar} verwendet werden.

\fseealso
\fu{WriteChar}
\end{function}

%%%%%%%%%%%%%%%%%%%%%%%%%%%%%%%%%%%%%%%%%%%%%%%%%

\begin{function}{ReadFloat}
\typeindex{float}

\syntax
\#include "conio.sh"
function ReadFloat () : float
\endgroup

\usage
\fu{ReadFloat} liest einen \tfloat\ in Dezimal- bzw.
\indu{Exponentialschreibweise} (z.~B. \fu{1.23E45}) von der
Systemkonsole ein. Backspaces werden richtig interpretiert,
soda"s der Anwender eine Korrekturm"oglichkeit hat.

\fseealso
\fu{WriteFloat}
\end{function}

%%%%%%%%%%%%%%%%%%%%%%%%%%%%%%%%%%%%%%%%%%%%%%%%%

\begin{function}{ReadInt}
\typeindex{int}

\syntax
\#include "conio.sh"
function ReadInt () : int
\endgroup

\usage
\fu{ReadInt} liest einen \tint\ als Dezimalzahl
\index{Zahl!dezimale}
von der
Systemkonsole ein. Dabei werden Backspaces ber"ucksichtigt,
um dem Anwender eine Korrekturm"oglichkeit zu geben. Zu beachten
ist allerdings, da"s kein positives Vorzeichen (\fu{+}) angegeben
werden darf.

\fseealso
\fu{WriteInt}
\end{function}

%%%%%%%%%%%%%%%%%%%%%%%%%%%%%%%%%%%%%%%%%%%%%%%%%
\newpage

\begin{function}{ReadString}
\index{String}

\syntax
\#include "conio.sh"
function ReadString (s : *char) : int
\endgroup

\parameters
\param{s}{\type{\poi char}} Speicherplatz f"ur den String \cr

\usage
Die Funktion \fu{ReadString} liest einen nullterminierten String
von der Systemkonsole in den Speicherbereich, den der Parameter
\fu{s} angibt, ein. Dieser Speicherbereich mu"s gen"ugend gro"s
dimensioniert sein, da \fu{ReadString} keinerlei Tests vornimmt,
ob er f"ur den String ausreichend ist. Die Stringeingabe wird vom
Anwender durch ein Newline-Zeichen beendet, das jedoch im String
nicht aufscheint. Die Funktion gibt die Anzahl der gelesenen
Zeichen, inklusive dem Newline, also mindestens \fu{1}, zur"uck.

\fseealso
\fu{WriteString}
\end{function}

%%%%%%%%%%%%%%%%%%%%%%%%%%%%%%%%%%%%%%%%%%%%%%%%%

\begin{function}{WriteChar}
\typeindex{char}

\syntax
\#include "conio.sh"
function WriteChar (c : char) : void
\endgroup

\parameters
\param{c}{\tchar} auszugebendes Zeichen \cr

\usage
Die Funktion \fu{WriteChar} gibt das ihr "ubergebene Zeichen
\fu{c} auf der Systemkonsole aus, wobei Steuerzeichen (siehe
Kapitel \ref{specialchar}) ber"ucksichtigt werden.

\fseealso
\fu{ReadChar}
\end{function}

%%%%%%%%%%%%%%%%%%%%%%%%%%%%%%%%%%%%%%%%%%%%%%%%%

\begin{function}{WriteFloat}
\typeindex{float}

\syntax
\#include "conio.sh"
function WriteFloat (f : float) : void
\endgroup

\parameters
\param{f}{\tfloat} auszugebender Wert \cr

\usage
Die Funktion \fu{WriteFloat} gibt einen \tfloat\ in passender
Schreibweise (Dezimal- oder \ind{Exponentialschreibweise}) auf der
Systemkonsole aus.

\fseealso
\fu{ReadFloat}
\end{function}

%%%%%%%%%%%%%%%%%%%%%%%%%%%%%%%%%%%%%%%%%%%%%%%%%
\newpage

\begin{function}{WriteInt}
\typeindex{int}

\syntax
\#include "conio.sh"
function WriteInt (i : int) : void
\endgroup

\parameters
\param{i}{\tint} auszugebender Wert \cr

\usage
Die Funktion \fu{WriteInt} gibt den Parameter \fu{i} als
Dezimalzahl
\index{Zahl!dezimale}
auf der Systemkonsole aus.

\fseealso
\fu{ReadInt}
\end{function}

%%%%%%%%%%%%%%%%%%%%%%%%%%%%%%%%%%%%%%%%%%%%%%%%%

\begin{function}{WriteString}
\index{String}

\syntax
\#include "conio.sh"
function WriteString (s : *char) : void
\endgroup

\parameters
\param{s}{\type{\poi char}} auszugebender String \cr

\usage
Die Funktion \fu{WriteString} gibt einen nullterminierten String
auf der Systemkonsole aus. Steuerzeichen (siehe Kapitel
\ref{specialchar}) werden dabei interpretiert.

\fseealso
\fu{ReadString}
\end{function}

%%%%%%%%%%%%%%%%%%%%%%%%%%%%%%%%%%%%%%%%%%%%%%%%%

\newpage
\index{Index|ii}
% \addcontentsline{toc}{chapter}{Index}
\footnotesize
\printindex

\end{document}
